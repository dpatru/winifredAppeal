They are reproduced below:

\begin{quotation}
To begin, there is nothing in MCL 211.27(2) requiring ``before repair'' and ``after repair'' appraisals when determining whether an assessment includes the true cash value of the normal repairs. [See footnote 7 below] Rather, in a proceeding before the Tribunal, Petitioner bears the burden of proof. [footnote 8 omitted] Further, the Tribunal cannot conclude that the FOJ erred when it concluded that Petitioner's contentions concerning the purchase price, i.e. the true cash value before repairs, was entitled to no weight or credibility. The selling price of a property is not is presumptive true cash value. [footnote 9 omitted] Despite Petitioner's assertions that the marketing efforts for the subject show that the sale was a ``market sale,'' the home was being sold by the U.S. Department of Housing and Urban Development (``HUD''). Because the subject was being sold by a government entity, that entity's motivation may not have been to receive market value for the property.

The Tribunal also concludes that it was not a palpable error to conclude that the assessment did not consider the ``normal repairs.'' This is supported by the fact that the property record card indicates that Respondent believed the true cash value of the subject to be \$48,000 \textit{before Petitioner purchased the property} [footnote 10 omitted] and \$50,400 as of December 31, 2015, after the repairs were complete. An increase of \$2,400 in true cash value (5\%) is easily attributable to inflation and increases in the market. In addition, as stated by the FOJ, the interior photographs depict a property in average condition before Petitioner acquired it. Although not necessarily evidence of true cash value, this evidence supports the property's assessment as a property in average condition both at the time Petitioner acquired it and after he completed the normal repairs. In other words, the record evidence supports the conclusion that the assessment did not consider the increase in true cash value that was the result of normal repairs.

\textit{footnote 7} MCL 211.27(2) allows an assessor to increase ``construction quality classification or reduce the effective age for depreciation purposes'' if the ``appraisal of the property was erroneous before non-consideration of the normal repair.'' It also prevents an assessor from assigning an economic condition factor to the property that differs from the economic condition factor assigned to similar properties as defined by appraisal procedures applied in the jurisdiction. Neither situation is at issue here. Although State Tax Commission Bulletin No. 7 of 2014 requires ``before'' and ``after'' appraisals, such appraisals are only required ``[i]f the true cash value of non-consideration items is shown on the assessment roll. .~.~.'' As described herein, the true cash value of the non-consideration items is not shown on the assessment roll and the STC requirement does not apply. In addition, STC guidance lacks the force of law. \pincite{Rovas}{103; 267}.

\reconsiderationDenied[2].
\end{quotation}

Appellant summarizes the Tribunal's reasoning in its denial of reconsideration, p 2, as:

\begin{enumerate}
  
\item The Tax Tribunal believes that it did not have to make a before-repair appraisal because Petitioner has the burden of proof and the Tribunal disagrees with Petitioner's proof on this point. (The subject's sale is given no weight or credibility.)
  
\item The Tax Tribunal believes that \mathieuGast\ is not violated if there is some evidence that the assessor did not need to consider repairs to justify the true cash value. Here the true cash value was just 5\% more than the previous year's true cash value. The 5\% change was due to inflation, not repairs. The assessor's property record card indicated that the property had been in average condition before its sale. Also, the pictures on the \MLS\ showed the property in average condition.
  
\item The Tax Tribunal believes that \mathieuGast\ does not require before-repair and after-repair appraisals because:
  \begin{enumerate}
  \item The text of the statute does not plainly require the appraisals.
  \item The STC guidance requires repairs if the value of the repairs are on the assessment roll, and here they are not.
  \item Even if the STC required the repairs, the Tax Tribunal is not bound by STC guidance.
  \end{enumerate}

\item The Tax Tribunal gave the property's sale ``no weight or credibility'' because the seller was a government entity (HUD) who may not have been motivated to receive market value.
  
\end{enumerate}

% The Appellant believes that these arguments were not put forth by the Appellee. The \foj[4], records Appellee's argument this way:

% \begin{quotation}
% 	Respondent argues that the change in the property assessment was not based on Petitioner's repairs but based on the purchase of the property in 2015 which uncapped the assessment for 2016. Respondent assessed the property as ``average'' condition as of December 31, 2015. The changes were not substantial and were based on the rate of inflation.
	
% 	On rebuttal, Respondent argues mass appraisal does not account for properties on-by-on. In other words, properties are assessed uniformly and Respondent assumes properties are in average condition.
% \end{quotation}

% Appellant believes that the Tax Tribunal's characterization of Appellee's argument in the first paragraph quoted above is a little nonsensical. It appears to confuse the assessed value (which is determined every year independent of sales) with the taxable value (which uncaps based on sales or transfers). What was probably meant in the first paragraph was something like: 

% \begin{quote}
% 	Respondent argues that the increase in property taxes was caused by an uncapping because of the transfer of the property in 2015. The actual assessed value (or true cash value which is twice the assessed value) changed minimally from 2015 to 2016 based on the rate of inflation. The assessed value did not changed because of repairs done to the property. 
% \end{quote}

% In any case, Appellant believes that however Appellee's arguments are characterized, because the repairs had not yet been ruled to be normal repairs, Appellee's arguments were not as wrong as the arguments put forth by the Tax Tribunal in the \fojAbbr\ and \reconsiderationDeniedAbbr. The Tax Tribunal appears to argue that the assessor may simply ignore normal repairs under certain circumstances. Appellee has not asserted this so far in this case.

Perhaps the root of the Tax Tribunal's error regarding \mathieuGast\ stems from its misunderstanding of what the statute means by ``not consider''. In everyday usage, ``to consider'' implies activity and to ``to not consider'' implies passivity. But in \mathieuGast, non-consideration is active. The statute says that ``the increase in value attributable to [normal repairs] ... shall be indicated on the assessment roll.'' The \cite[s]{STC Bulletin}\ uses the term ``non-consideration treatment''. Passively ignoring normal repairs is a violation of the statute as this Court noted the first time it heard this case:

\begin{quote}
... contrary to MCL 211.27(2), the referee \textit{considered} the increase in value attributed to the repairs when determining the property's TCV. Stated differently, the referee's finding that the property's TCV was \$50,400 was based on its assessment of the property's value after it had been repaired. This was improper because MCL 211.27(2) expressly provides that certain repairs constitute normal repairs so long as they are not part of a structural addition or completion. \pincite{Patru}{5}.
\end{quote}

There is good reason for the STC's nonconsideration treatment. Before-repair and after-repair appraisals force the assessor to *account* for the repairs. Such an accounting would prevent this situation where the assessor has assumed in the past that the property was not in need of repairs and is now continuing the assumption. A before-repair appraisal forces to assessor to ask, What really was the value of the property before the repairs?  The answer is not: the prior year's TCV. The answer requires the assessor to determine the actual condition and value of the property before repairs. In some cases, the previous year's property's record card may accurately reflect the condition of the property and its value. But in other cases, and in this case, the previous year's record card may be wrong. 

In this situation, the assessor testified that she assessed the property using mass appraisal which assumes the property is in average condition. It is undisputed that the City's inspectors had inspected the property and determined that it needed repairs. But the City's assessor was assuming the property did not need repairs. The before-repair appraisal requirement would determine who was correct: the inspector's inspections or the assessor's assumption. 

The Tax Tribunal ruled that because the assessor had assumed the property did not need the repairs, and because the MLS pictures did not explicitly show the need for the repairs, that the repairs could be ignored and the property could be assessed as if the repairs had never happened. The STC's guidance to the contrary could also be ignored. 




Take 1:
Introduction MG requires that the "assessor shall not consider the increase in true cash value that is a result of expenditures for normal repairs . . . in determining the true cash value of property for assessment purposes until the property is sold. 

1. The assessor shall not consider the increase in true cash value that is a result of expenditures for normal repairs, replacement, and maintenance in determining the true cash value of property for assessment purposes until the property is sold. 

2. For the purpose of implementing this subsection, the assessor shall not increase the construction quality classification or reduce the effective age for depreciation purposes, except if the appraisal of the property was erroneous before nonconsideration of the normal repair, replacement, or maintenance, and shall not assign an economic condition factor to the property that differs from the economic condition factor assigned to similar properties as defined by appraisal procedures applied in the jurisdiction. 

3. The increase in value attributable to the items included in subdivisions (a) to (o) that is known to the assessor and excluded from true cash value shall be indicated on the assessment roll. 

4. This subsection applies only to residential property. 

5. The following repairs are considered normal maintenance if they are not part of a structural addition or completion:

Take 2:
The first sentence of MCL 211.27(2) requires that the assessor "not consider" the increase in true cash value that is the result of expenditures for normal repairs. If one were to stop reading here, it might seem that an assessor could comply with the statute merely by passively refusing to actively "consider" normal repairs. This is how the Tax Tribunal seems to have interpreted the statute. However, the third sentence of the statute commands the assessor to indicate on the assessment roll "[t]he increase in value attributable to the [normal repairs]".  Thus compliance with the statute requires that the assessor determine the value of the repairs. 

The STC's guidelines on Mathieu-Gast (MCL 211.27(2)) nonconsideration requires that the value of the repairs be determined by conducting before-repair and after-repair appraisals. 

The Tax Tribunal did not obey the statute. It found that normal repairs had been done, but instead of determining the value of the repairs and not considering the value of the repairs in the assessed true cash value, it simply set the true cash value of the repairs to the after-repair value of the property. This is the same result as if the property had never needed the repairs and the repairs had never happened. The effect of the MTT's decision was to ignore the repairs and ignore the statute. This is obviously the wrong result. The rest of this section, indeed this whole brief, is devoted to determining the exact mistakes made by the Tax Tribunal. 

MCL 211.27(2) is not satisfied by the assessor's mere testimony that she did not consider normal repairs.
The FOJ says at page 3 that that the assessor testified that she did not consider the repairs when determining the assessed value. At page 5, in the findings of fact, the FOJ says that the assessor did not consider the repairs when determining the assessed value. At page 6 in the discussion of law, the FOJ says that the assessor did not consider the repairs. At page 7, the FOJ says that the comparison study contains no line item for repairs. On page one of the denial of the motion for reconsideration, Judge Marmon writes that "the 2016 assessment was not based on . . . repairs to the subject property." The Tax Tribunal writes this as if this were relevant to a Mathieu-Gast analysis. It is not. The Tax Tribunal seems to believe that the statute is satisfied or does not apply, it is not clear which, if the assessor merely testifies that she did not consider repairs and if there exists some evidence that the assessor could determine the true cash value without reference to repairs. This is not relevant to the analysis.

The first question in a Mathieu-Gast analysis is whether normal repairs were done. If normal repairs exist, then the assessor must give them a value. The STC says that before-repair and after-repair appraisals should be used. Then the true cash value for assessment purposes is set to the before-repair value and the value of the repairs is indicated in the assessment roll. That is it. There is no exception in the statute that normal repairs may be ignored if the assessor subjectively did not consider the repairs. Once the Tribunal has determined that normal repairs exists, it must find a value for the repairs (using before and after appraisals), and make sure that the true cash value for assessment purposes is set to the before-repair value and the value of the repairs is noted on the assessment roll. 

Besides misinterpretation of the Mathieu-Gast statute, the Tax Tribunal makes two mistakes. 1) The Tax Tribunal does not make its own independent determination of the true cash value because it finds the Petitioner's evidence unconvincing; and 2) The Tax Tribunal cursorily dismisses the subject's sale price as evidence. Both of these errors were made by the Tax Tribunal in Jones & Laughlin. 

In Jones & Laughlin, the Tax Tribunal believed that it did not need to make an independent determination of the true cash value of the property because the petitioner had failed to meet its burden of proof and the burden of proof was on the petitioner. This Court reversed, comparing this to the entry of a directed verdict upon the failure of a plaintiff's proofs. "To the extent this analogy may be accurate in this case, the entry of judgment against petitioner for failure to provide sufficient evidence was erroneous because, while the petitioner may not have met its burden of persuasion, it did meet its burden of going forward with evidence. Even if the tribunal had correctly concluded that petitioner's proofs had failed, the tribunal still would be required to make an independent determination of the true cash value of the property. The tribunal may not automatically accept a respondent's assessment, but must make its own findings of fact and arrive at a legally supportable true cash value. . . . On remand, the tribunal shall make an independent determination of true cash value." 

In this case the Tax Tribunal did not apply MCL 211.27(2) by determining before-repair and after-repair values partly because it disagreed with Appellant's evidence. (The Tribunal gave no weight or credibility to the sale price.) The Tax Tribunal is repeating the mistake it made in Jones & Laughlin. Even if it disagrees with Appellant's proofs, the Tax Tribunal must make an independent determination of the true cash value. 

The second mistake made by the Tax Tribunal in this case is that it has cursorily rejected the subject's sale price. In Jones and Laughlin, the Tax Tribunal rejected the sale price because the sale occurred after the tax date. This court wrote, "Evidence of the price at which an item of property actually sold is most
certainly relevant evidence of its value at an earlier time within the meaning
of the term ’relevant evidence.’ MRE 401. Although the sale . . . occurred
approximately nine months after the tax date, the lapse in time is important
only with respect to the weight that should be given the evidence, not to the
relevance of the evidence. While the tribunal correctly noted that the sale
price of a particular piece of property does not control its determination of
the value of that property, the tribunal’s opinion that the evidence “has little
or no bearing” on the property’s earlier value suggests that the evidence
was rejected out of hand. Such cursory rejection would be erroneous. Id
(cleaned up).

In this case the Tax Tribunal has ruled that the contention that the sale price was the before-repair value of the property "is entitled to no weight or credibility" because the seller was a government agency, HUD, and because the seller may not have been motivated to receive market value. Denial of Motion p 2. The Tax Tribunal cites no evidence, no facts, that prove that the sale was not a market sale. Nor does does the Tax Tribunal merely discount the weight of the evidence; it rules that the sale is entitled to *no* weight or credibility. Such a cursory dismissal of the sale price based on speculation in contrary to the teaching of Jones & Laughlin. This Court should reverse and require that the Tax Tribunal base its rulings on evidence and not cursorily reject the sale price. 

Take 3
STC form 865, Request for Nonconsideration, allows property owners to list expenditures for repairs qualifying as normal repairs under \mathieuGast. The second page of that form has a section entitled "Instructions To Assessor for Processing Form 865." The first instruction in that section says: "1. The assessor is required to estimate the true cash value of the property both before and after the expenditures." 

Despite the clear instruction of the STC requiring before-repair and after-repair appraisals, the Tax Tribunal states that "there is nothing in MCL 211.27(2) requiring 'before repair' or 'after repair' appraisals when determining whether an assessment includes the true cash value of normal repairs." Denial p 2. To support its open rebellion against the STC,  the Tax Tribunal in footnote 7 of Denial of Reconsideration has three arguments: 1) The second sentence of the statute does not apply. 2) Bulletin 7 says that before and after appraisals are not required if the non consideration items are not shown in the assessment roll. 3) The STC does not have the force of law. 

None of these reasons withstand scrutiny. First, the fact that the second sentence of the statute is not relevant is itself irrelevant to whether before-repair and after-repair appraisals are required. 

Second, the third sentence of the statute requires that the value of repairs be indicated on the assessment roll, so before-repair and after-repair appraisal are required to determine the value of the repairs so that the value can be placed on the assessment roll. The Tax Tribunal misunderstood Bulletin 7. . . . 

Third, the Rojas case is about the power of the judicial-branch courts to interpret laws independent of executive agency interpretations. In the Rojas, the Michigan Supreme Court reversed part of a decision of the Michigan Court of Appeals because the Court of Appeals had given too much deference to an executive agency's interpretation of a law. The Tax Tribunal is itself an executive agency as is the State Tax Commission. The Rojas decision does not stand for the proposition that the Tax Tribunal, one executive agency, can ignore the guidance of the STC, another executive agency. 

Therefore, the Tax Tribunal has not given a good reason for disobeying the guidance of the STC on before-repair and after-repair appraisals.





The third sentence of \mathieuGast\ requires that the assessor indicate on the assessment roll the increase in value attributable to the repairs. Before the assessor can comply with the statute and indicate a value on the assessment roll, the assessor must first determine the value.
Thus, nonconsideration under \mathieuGast\ requires something more that just a passive omission by the assessor. To comply with the statute, the assessor must determine the value of the repairs, make sure that this value was not considered in the true cash value for assessment purposes, and indicate the value of the repairs on the assessment roll.

% \mathieuGast\ requires that once normal repairs are found, the value of the normal repairs must be separately indicated on the assessment roll and the true cash value for assessment purposes must not include the value of the repairs.
% Proof? The statute itself requires that the assessor shall not consider the value of normal repairs . . . and shall indicate the value of the [repairs] on the assessment roll. Also the STC Form 865 indicates that non-consideration treatment is required when the taxpayer submits the form indicating that normal repairs were completed.



MCL 211.27(2) is not satisfied by the assessor's mere testimony that she did not consider normal repairs.
The FOJ says at page 3 that that the assessor testified that she did not consider the repairs when determining the assessed value. At page 5, in the findings of fact, the FOJ says that the assessor did not consider the repairs when determining the assessed value. At page 6 in the discussion of law, the FOJ says that the assessor did not consider the repairs. At page 7, the FOJ says that the comparison study contains no line item for repairs. On page one of the denial of the motion for reconsideration, Judge Marmon writes that "the 2016 assessment was not based on . . . repairs to the subject property." The Tax Tribunal writes this as if this were relevant to a Mathieu-Gast analysis. It is not. The Tax Tribunal seems to believe that the statute is satisfied or does not apply, it is not clear which, if the assessor merely testifies that she did not consider repairs and if there exists some evidence that the assessor could determine the true cash value without reference to repairs. This is not relevant to the analysis.

The first question in a Mathieu-Gast analysis is whether normal repairs were done. If normal repairs exist, then the assessor must give them a value. The STC says that before-repair and after-repair appraisals should be used. Then the true cash value for assessment purposes is set to the before-repair value and the value of the repairs is indicated in the assessment roll. That is it. There is no exception in the statute that normal repairs may be ignored if the assessor subjectively did not consider the repairs. Once the Tribunal has determined that normal repairs exists, it must find a value for the repairs (using before and after appraisals), and make sure that the true cash value for assessment purposes is set to the before-repair value and the value of the repairs is noted on the assessment roll. 


The way to ensure that the value of the repairs is not included in the true cash value is to first find the value of the repairs and then subtract the value of the repairs from the value of the after-repaired value. This is what the STC terms ``non-consideration treatment'' and it requires before-repair and after-repair appraisals. \pincite{STC Bulletin}{2}. 
