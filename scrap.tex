
\section{Introduction}

Appellant bought a house that needed repairs. He repaired it. He wants to be taxed based on the value of the house as it was when he bought it (before repairs) because the \mathieuGast instructs the assessor not to consider normal repairs. 

Appellee, the City of Wayne, wants to tax based on the value of the house after repairs. The Tax Tribunal agreed with the Appellee. It refused to offer Appellant the benefits of \mathieuGast. It set the assessed value to the value of the house after repairs.
 It gave four reasons for not applying \pincite[l]{MCL}{211.27(2)}:

\begin{enumerate}
\item The selling price of the house was not its true cash value.

\item The repairs were not normal repairs under \pincite[l]{MCL}{211.27(2)}.

\item Appellant did not submit enough evidence and the Tribunal will not accept any more evidence.

\item Even if the repairs were normal, because the value of the house before repairs was not proven, the Tribunal cannot apply \pincite[l]{MCL}{211.27(2)} because determining the value of the house before the repairs is the first step.

\end{enumerate}

Appellant argues that each of the reasons given by the Tribunal is wrong:

\begin{enumerate}

\item There is substantial evidence that the selling price was the fair market price and there is no contrary evidence.

\item The Tribunal used non-statutory criteria for determining if the repairs were normal. The repairs were not challenged based on the statutory criteria.

\item Appellant believes that he submitted enough evidence, but even if Appellant did not submit enough evidence, the Tribunal may not bar additional evidence if it is needed to determine the correct assessed value.

\item Appallant believes that the value of the house before repairs was proven to be its sale price, but even if it was not, determining the value of the property before repairs is not a necessary first step to applying \pincite[l]{MCL}{211.27(2)}.

\end{enumerate}



\section{Facts}
The subject of this case is a house, 5073 Winifred, in the City of Wayne. The marketing and sale activity of this house is shown in the MLS listing history which Appellant submitted as evidence to the Tax Tribunal:


% see http://mirrors.rit.edu/CTAN/macros/latex/contrib/enumitem/enumitem.pdf for documentation
\begin{description}[style=multiline,leftmargin=3.5cm,font=\normalfont,itemsep=.5\baselineskip,align=right]
\singlespacing
\item[3/16/2005] HUD buys the house for \$119,271, presumably by foreclosure.
\item[4/3/2013] Listed on the MLS (the Multiple Listing Service, the database used by real estate brokers to market properties) for \$29,900.
\item[5/3/2013] An offer was accepted and the status was ``Pending''.
\item[10/24/2013] On 10/24/2014 the property was withdrawn from the MLS because the listing contract with the real estate broker had expired.
\item[6/17/2015] On 6/17/2015 the property was listed by another broker on the MLS for \$32,000.
\item[6/29/2015] An offer was accepted and the MLS status was ``Pending''.
\item[7/3/2015] The status was changed to ``Active''. (The contract was terminated without sale.)
\item[7/6/2015] The status was changed to ``Pending'' (because the seller accepted the Appellant's offer).
\item[8/19/2015] Appellant closed on the property.
\end{description}

At the time of sale (8/19/2015), the property had two-page list of repairs required by the City. (Proposed Opinion at 3) The City had allowed the seller to sell the property, but it required the repairs before it would grant a Certificate of Compliance.

After the sale, Petitioner repaired the house and rented it. As of tax day, 12/31/2015, the house was repaired and rented.

The City of Wayne, assessed the property for 2016 at \$50,400 True Cash Value (\$25,200 Taxable Value). Appellant appealed to the Board of Review in March 2016. The Board of Review found that the property was properly assessed. Appellant then appealed to the Michigan Tax Tribunal.

A hearing was held 10/27/2016 before a Referee at the Michigan Tax Tribunal. At the hearing, Appellant argued that \pincite[l]{MCL}{211.27(2)} required that the taxable value of the house be determined based on the condition of the house before the repairs. Appellant testified that all of the repairs to the house were ``normal repairs'' under \pincite[l]{MCL}{211.27(2)} and as such could not be used to increase the assessment. 

Appellant also argued that because the house was sold by HUD, he believed that it would have been a federal crime for the house to sell in a non-arm's-length transaction.

Appellant also argued that the sale price of the subject property was consistent with the sale history of the comparable properties submitted by Respondent. Three of the five comparable properties submitted by Respondent sold for \$20,500, \$21,000, and \$30,000, before they sold for \$49,750, \$57,000, and \$65,000.

When asked of the value of the repairs, Appellant estimated less than \$10,000.

Appellee did not contest that the property had been in need of repairs, nor that the repairs needed fell under the specific categories of normal repairs listed in the statute, but argued that \pincite[l]{MCL}{211.27(2)} should be applied only to properties that were kept up.
He also argued that the \$32,000 purchase price plus \$10,000 in repairs, plus the value of Appellant's labor supported an after-repair valuation of \$50,400.

The Referee issued her a Proposed Opinion on December 1, 2016. The Referee refused to apply \pincite[l]{MCL}{211.27(2)} nonconsideration because:
The fair market value of the property before repairs was not its selling price because it  was sold by a bank, the seller may have been under financial duress, the property may have been sold for less than its market value, the property needed repairs at the time of purchase, and the city required that the repairs be completed before the house could be occupied; and
The repairs were not normal repairs under the statute because the house was in substandard condition and the repairs were required by the city. 

After refusing to apply \pincite[l]{MCL}{211.27(2)}, the Referee found that as of tax day, 12/31/2015, the repairs had been completed and the house was in average condition. The Referee used the comparables submitted by the respondent/appellee and concluded the True Cash Value of the house should be \$50,400. 

On 12/21/2016 Appellant filed Exceptions to the Proposed Opinion. He argued the STC (State Tax Commission) Bulletin No. 7 of 2014 (June 11, 2014) suggested that in a case involving 211.27(2), before the before-repair value of the property was determined, that the issue of whether the repairs were normal should be determined. So appellant addressed this issue first. He argued that the Referee's criteria for determining whether repairs were normal were unsupported by the statute's language. \pincite[l]{MCL}{211.27(2)} does not restrict repairs based on the condition of the house or the requirements of the city. On the contrary, the statute explicitly says that normal repairs include replacing the siding, roof, porches, steps, sidewalks, or drives, complete rewiring, and other expensive repairs that are normally only done on houses that need them, i.e., house that are ``substandard'' at least with respect to the repairs needed. Appellant also submitted a brochure published by the City of Wayne (the Appellee) which showed that the normal repairs contemplated by the statute were also some of the most common repairs required by the City. 

Regarding the before-repair fair market value of the property, Appellant argued that its selling price should be used because the property had spent over two years on the market, it had had multiple offers, it sold for its asking price, and the price was consistent with the selling price of three of the comparable properties submitted by Appellee. Further, the Referee's speculation that the bank-seller was under financial duress and sold the property for less than market value had no evidentiary support and was unreasonable in light of the fact that the seller had raised the price, not lowered it in the two years the house was on the market.

Appellant did not contest that the Referee's finding that the after-repair value of the house was \$50,400, but argued that it should merely be used to help value the repairs according to the procedure given in the STC Bulletin. The True Cash Value would calculated as: 
$TrueCashValue = Value_{AfterRepair} - RepairValue$
where $RepairValue = Value_{AfterRepair} - Value_{BeforeRepair}$. After substituting the formula for $RepairValue$, the after-repair values cancel and the formula becomes $TrueCashValue = Value_{BeforeRepair}$. In other words, the True Cash Value is simply the value of the house before the repairs were done, because the difference between the before and after repair values is the value of the repairs, which under \pincite[l]{MCL}{211.27(2)} cannot be considered.

Appellant also included in his Exceptions a list of all the repairs he had done along with an estimated value for each. This was done because his estimate of \$10,000 for the repairs was given off the cuff.

Appellee did not file exceptions to the Proposed Opinion or respond to Appellant's Exceptions.

On 1/26/2017, Tribunal Judge Lasher issued a Final Opinion and Judgment saying that the Referee properly considered the testimony and evidence provided. Then he says,

\begin{quote}%
Though the Tribunal is not persuaded that [the property's] substandard condition necessarily renders all subsequent improvements outside the scope of normal repairs and maintenance as indicated by the Referee, Petitioner has failed to present timely evidence establishing the improvements qualified as normal repairs and maintenance so as to properly be excluded from the assessment. Notably, at the time of hearing, Petitioner had presented for the Tribunal's review only the Board of Review Decision and market listings and listing histories for the subject and comparable properties provided by Respondent. Though additional documentation was filed with Petitioner's exceptions, the parties are required to submit any and all documentation they wish to have considered to both the Tribunal and the opposing party at least 21 days prior to the hearing, or it may not be considered [cite to TTR 287].
\end{quote}

On 2/16/2017, Appellant filed a motion requesting reconsideration. Appellant reiterated his contention that Proposed Opinion used the wrong legal standard for determining whether repairs were normal under \pincite[l]{MCL}{211.27(2)} and that therefore the Final Opinion was flawed. 

Also, Appellant pointed out that the sentiment expressed in quote above from the Final Opinion (a little more evidence by the Appellant might swing the result his way, but that more evidence will not be allowed) contradicted the holding of at least two appellate cases, one of which involved the very statute at issue here. In both Jones \& Laughlin Steel Corporation v. City of Warren, 193 Mich App 348 (1992), 483 NW2nd 416, and Fisher v. Sunfield Township, 163 Mich App 735 (1987), 415 NW2nd 297, when the submitted evidence was not enough to resolve the issues in the case, the Court of Appeals ordered the Tax Tribunal to give the taxpayer plaintiffs more opportunity to present proofs. 

Appellant also argued that enough evidence had already been presented by Appellant's sworn testimony at the hearing, and that Appellee had opportunity to rebut additional evidence presented in the Exceptions but chose not to.
Appellant also contrasted the repairs done to his property with the repairs done in Fisher. The court commented on repairs that were “far in excess of normal repairs. . . . the old porch was extremely narrow, of wood construction and without stairs. The porch following its improvement has been widened and restyled and brick stairs of a highly attractive design of superior quality have been installed.” (Fisher at 742). In contrast, Appellant's repairs merely brought the property back to good condition (roof, sidewalk leveling, painting, replace broken window glass, etc.) or updated it to meet current codes (electrical). Most of the repairs were required by the city when it conducted its standard presale inspection to prevent blight.

Appellant concluded his motion by comparing the increase in value of his own property with the three properties presented by Appellee as comparable. This showed that far from being an abnormal or uncommon situation, most of Appellee's own evidence demonstrated the same swing in value. 

On 3/6/2017, Judge Lasher issued an order denying Appellant's motion for reconsideration. The Judge wrote that \pincite[l]{MCL}{211.27(2)} cannot apply because the value of the house before the repairs is not been proved. Specifically, the seller was a bank, Appellant had not proved that the purchase price was an arm's length sale, and the Referee found that bank sales were not common in the jurisdiction. 

The Judge wrote that the valuation is supported by both the cost and sales comparison approaches as submitted by Respondent. 

Regarding his refusal to admit more evidence, the Judge said he was in compliance with the Tribunal's rules. He did not distinguish the two cases cited by Appellant. 



A hearing was held 10/27/2016 before a Referee at the Michigan Tax Tribunal. At the hearing, Appellant argued that \pincite[l]{MCL}{211.27(2)} required that the taxable value of the house be determined based on the condition of the house before the repairs. Appellant testified that all of the repairs to the house were ``normal repairs'' under \pincite[l]{MCL}{211.27(2)} and as such could not be used to increase the assessment. 

Appellant also argued that because the house was sold by HUD, he believed that it would have been a federal crime for the house to sell in a non-arm's-length transaction.

Appellant also argued that the sale price of the subject property was consistent with the sale history of the comparable properties submitted by Respondent. Three of the five comparable properties submitted by Respondent sold for \$20,500, \$21,000, and \$30,000, before they sold for \$49,750, \$57,000, and \$65,000.


Appellee did not contest that the property had been in need of repairs, nor that the repairs needed fell under the specific categories of normal repairs listed in the statute, but argued that \pincite[l]{MCL}{211.27(2)} should be applied only to properties that were kept up.
He also argued that the \$32,000 purchase price plus \$10,000 in repairs, plus the value of Appellant's labor supported an after-repair valuation of \$50,400.

The repairs were not normal repairs under the statute because the house was in substandard condition and the repairs were required by the city. 



\section{Questions Involved}
\begin{itemize}

\item Does \pincite[l]{MCL}{211.27(2)} allow consideration of otherwise normal repairs when a) the house was sold by a bank, b) the house needed repairs at the time of sale, c) the repairs were required by the city, and 4) the value of the repairs was more than \$10,000?

Tax Tribunal answers yes. Appellant answers no.

\item Can the sale of a bank-owned property be dismissed as evidence of the property's true cash value based solely on speculation that the bank-seller may have been under financial duress and may have sold the property for less than market value? 

 Tax Tribunal answers yes. Appellant answers no.

\item  When more evidence is needed to resolve nonconsideration under \pincite[l]{MCL}{211.27(2)}, Is it proper for the Tax Tribunal to refuse to offer the taxpayer an opportunity to provide it?

 Tax Tribunal answers yes. Appellant answers no.

\item If the written evidence submitted to the Tax Tribunal does not support an essential element of the petitioner's case, does the petitioner lose?

 Tax Tribunal answers yes. Appellant answers no.

\item To benefit from nonconsideration under \pincite[l]{MCL}{211.27(2)}, must the taxpayer prove the value of the house before the repairs?

 Tax Tribunal answers yes. Appellant answers no.

\item  Where Appellant has testified that he believes that the sale of HUD-owned property in a non-arm's-length transaction is a federal crime, . . . did the Tribunal err in ruling that Appellant had failed to prove an arm's length transaction?

 Tax Tribunal did not answer. Appellant answers yes.

\item Given the facts and proofs of this case, was it error for the Tribunal to conclude that the house did not sell for its fair market value?

 Tax Tribunal answers no. Appellant answers yes.

\end{itemize}

\section{New Draft}

Facts: 
Appellant bought a house that needed repairs. He repaired it. The City/Appellee is seeking to tax him on the value of the house after repairs. Appellant would like value of the repairs to be excluded per MCL 211.27(2). The Tax Tribunal agreed with the City. 

The Tax Tribunal issued three rulings: a Proposed Opinion and Judgment, a Final Opinion and Judgment upholding and adopting the Proposed Opinion, and a denial of Appellant's Motion to Reconsider. 

Appellant appeals from three rulings of the Tax Tribunal: 
1. The Tax Tribunal ruled that the statute (MCL 211.27(2)) did not apply because the house needed repairs, it was owned by a bank, the amount of the repairs was \$10,000, the city required the repairs to be done.
2. The Tax Tribunal ruled that even though it was not persuaded that all of the repairs necessarily fell outside normal repairs under the statute, Appellant had not presented sufficient evidence citing the written evidence submitted before the hearing.  
3. The Tax Tribunal ruled that its rules (specifically rule 275(2) requiring submission of written evidence within 21 days of a scheduled hearing) permitted it to exclude later-submitted evidence and to deny reconsideration despite questions on the applicability of MCL 211.27(2). 

Questions presented:

1. Did the Tax Tribunal err by ruling that the MCL 211.27(2) did not apply because ...?
2. Did the Tax Tribunal err by ruling that the Petitioner/Appellant did not submit sufficient evidence when it referred exclusively to written evidence?
3. Did the Tax Tribunal err by failing to accept additional written evidence or ordering a new hearing when questions remained whether repairs qualified under MCL 211.27(2)?

Argument
1. The Tax Tribunal erred in its analysis under MCL 211.27(2).

MCL 211.27(2) reads: The assessor shall not consider the value of normal repairs and replacement in determining the assessed value of residential property. Normal repairs are not new additional structure. The following shall be considered normal repairs: 1. repair or replacement of the roof, siding, ...

STC Bulletin instructs assessors to apply MCL 211.27(2) by first determining whether repairs were normal, then determining their value by determining, if possible, the value of the property before repairs and the value of the property after repairs. Under this approach the value of the repairs is the difference between the after repair value of the property and the before repair value. Where the value of the property before repairs cannot be determined, the State Tax Commission recommends that the value of the repairs be estimated directly. 

The Michigan Supreme Court endorsed the latter approach in xxxx where it found that the actual value of the repairs should be used. But there is no indication that the before and after approach was an alternative option in that case. 

The statute does not define "normal repair and replacement" except to say that structural additions are not normal repairs and that 15 kinds of repairs are normal. Both the State Tax Commission and the Supreme Court have said that the word "normal" in normal repairs should be given its usual meaning. CITE TO STC AND CASE

In case XXXX, the Court of Appeals looked to xxx to determine whether a repair was normal.

In case YYY, the Court of Appeals looked to yyy to determine whether a repair was normal.

Appellant has not been able to find case or statutory authority suggesting that the issue of whether repairs were normal should be determined by the factors used by the Tax Tribunal (the fact that the previous owner or the seller of the property was a bank or a government agency, the fact that the repairs were required, or that they cost the owner a specific sum). At best, the factors used by the Tax Tribunal are irrelevant to whether repairs were normal. 

The factors used by the Tax Tribunal are not just irrelevant, they would actually nullify the statute in many cases. For example, bank-owned houses typically need a lot of repairs because by the time these houses are sold, they have often spent years under owners who have been neglectful or under financial strain. So bank-owned or formerly bank-owned houses are some of the major beneficiaries of MCL 211.26(2). Similarly, cities typically have ordinances that enforce the building code and require property upkeep. (This was the case in this case.) Houses that need repair would be in violation of these ordinances and would be required by the city to be repaired. Under the Tax Tribunal's reasoning, houses in violation of ordinances would not be eligible to benefit from MCL 211.26.(2). Finally, many of the repairs explicitly listed in the statute as normal repairs are costly. Excluding otherwise normal repairs because they are costly would, again, nullify the statute in the cases where it is most needed. 



that have been foreclosed on are typically among the houses that have the most need for repairs and maintenance because their owners 

 evaluating whether a repair is normal, the state
In this case, instead of first considering whether the repairs 

Tax Tribunal considered the value of the house before the repairs first. Finding that it was a bank sale, it concluded that the repairs made were not normal.


Irrelevant
The Tax Tribunal found that the repairs had been made, and that nothing about them was remarkable. The only mention the Proposed opinion makes about the repairs was that they involved carpentry, electrical, and cement. (Proposed Opinion at 3) Nor was the result of the repairs remarkable. The house, after it was repaired, was not considered to be in superior condition compared to the other houses in the neighborhood which were used as comparables. 

The Tax Tribunal erred when it refused to admit additional evidence that would have allowed it to make an independent evaluation of the true cash value of the property. 

The Tax Tribunal erred when it did not consider in person testimony as evidence.



Brief 
 
Questions presented 
 
Does \pincite[l]{MCL}{211.27(2)} allow consideration of otherwise normal repairs when a) the house was sold by a bank, b) the house needed repairs at the time of sale, c) the repairs were required by the city, and 4) the value of the repairs was more than \$10,000? 
	
	Tax Tribunal answers yes. Appellant answers no. 
 
Under the facts of this case, did the Tax Tribunal have any evidence to support its reasoning that the selling bank may have been under financial duress and may have sold the property for less than market value? 
Can the sale of a bank-owned property be dismissed as evidence of the property's true cash value based solely on speculation that the bank-seller may have been under financial duress and may have sold the property for less than market value? 
May the Tax Tribunal reject late evidence that is submitted to 
 
The seller may have been under financial duress and sold the property for less than the market value. Also, Petitioner testified that the subject property was in need of repair at the time of purchase. 
 
 
Applying the legal standard 
Repairs cannot be rejected as normal repairs solely on the bases of standards not in the statute while ignoring the statute's explicit standards. 
 
Rejecting evidence solely on unreasonable, unsupported speculation 
The Referee erred in concluding that a house that sold after having been on the market for more than two years with multiple offers, did not sell for market value based on unreasonable, unsupported speculation that the bank was getting desperate to sell. 
 
Using discretion to reject evidence that would decide the case for the taxpayer 


The purpose of procedural rules are to enable the Tribunal to efficiently work justice. They should not be used to prevent justice from being done.





Our review of the Tax Tribunal's decision in Wheeler is limited. In the absence of fraud, we review a Tax Tribunal decision for "misapplication of the law or adoption of a wrong principle."[8] We consider the Tax Tribunal's factual findings conclusive if they are "supported by competent, material, and substantial evidence on the whole record."[9] However, we review issues of statutory interpretation de novo.[10]\pincite{Malpass}{\_\_\_; 276}

