They are reproduced below:

\begin{quotation}
To begin, there is nothing in MCL 211.27(2) requiring ``before repair'' and ``after repair'' appraisals when determining whether an assessment includes the true cash value of the normal repairs. [See footnote 7 below] Rather, in a proceeding before the Tribunal, Petitioner bears the burden of proof. [footnote 8 omitted] Further, the Tribunal cannot conclude that the FOJ erred when it concluded that Petitioner's contentions concerning the purchase price, i.e. the true cash value before repairs, was entitled to no weight or credibility. The selling price of a property is not is presumptive true cash value. [footnote 9 omitted] Despite Petitioner's assertions that the marketing efforts for the subject show that the sale was a ``market sale,'' the home was being sold by the U.S. Department of Housing and Urban Development (``HUD''). Because the subject was being sold by a government entity, that entity's motivation may not have been to receive market value for the property.

The Tribunal also concludes that it was not a palpable error to conclude that the assessment did not consider the ``normal repairs.'' This is supported by the fact that the property record card indicates that Respondent believed the true cash value of the subject to be \$48,000 \textit{before Petitioner purchased the property} [footnote 10 omitted] and \$50,400 as of December 31, 2015, after the repairs were complete. An increase of \$2,400 in true cash value (5\%) is easily attributable to inflation and increases in the market. In addition, as stated by the FOJ, the interior photographs depict a property in average condition before Petitioner acquired it. Although not necessarily evidence of true cash value, this evidence supports the property's assessment as a property in average condition both at the time Petitioner acquired it and after he completed the normal repairs. In other words, the record evidence supports the conclusion that the assessment did not consider the increase in true cash value that was the result of normal repairs.

\textit{footnote 7} MCL 211.27(2) allows an assessor to increase ``construction quality classification or reduce the effective age for depreciation purposes'' if the ``appraisal of the property was erroneous before non-consideration of the normal repair.'' It also prevents an assessor from assigning an economic condition factor to the property that differs from the economic condition factor assigned to similar properties as defined by appraisal procedures applied in the jurisdiction. Neither situation is at issue here. Although State Tax Commission Bulletin No. 7 of 2014 requires ``before'' and ``after'' appraisals, such appraisals are only required ``[i]f the true cash value of non-consideration items is shown on the assessment roll. .~.~.'' As described herein, the true cash value of the non-consideration items is not shown on the assessment roll and the STC requirement does not apply. In addition, STC guidance lacks the force of law. \pincite{Rovas}{103; 267}.

\reconsiderationDenied[2].
\end{quotation}

Appellant summarizes the Tribunal's reasoning in its denial of reconsideration, p 2, as:

\begin{enumerate}
  
\item The Tax Tribunal believes that it did not have to make a before-repair appraisal because Petitioner has the burden of proof and the Tribunal disagrees with Petitioner's proof on this point. (The subject's sale is given no weight or credibility.)
  
\item The Tax Tribunal believes that \mathieuGast\ is not violated if there is some evidence that the assessor did not need to consider repairs to justify the true cash value. Here the true cash value was just 5\% more than the previous year's true cash value. The 5\% change was due to inflation, not repairs. The assessor's property record card indicated that the property had been in average condition before its sale. Also, the pictures on the \MLS\ showed the property in average condition.
  
\item The Tax Tribunal believes that \mathieuGast\ does not require before-repair and after-repair appraisals because:
  \begin{enumerate}
  \item The text of the statute does not plainly require the appraisals.
  \item The STC guidance requires repairs if the value of the repairs are on the assessment roll, and here they are not.
  \item Even if the STC required the repairs, the Tax Tribunal is not bound by STC guidance.
  \end{enumerate}

\item The Tax Tribunal gave the property's sale ``no weight or credibility'' because the seller was a government entity (HUD) who may not have been motivated to receive market value.
  
\end{enumerate}

% The Appellant believes that these arguments were not put forth by the Appellee. The \foj[4], records Appellee's argument this way:

% \begin{quotation}
% 	Respondent argues that the change in the property assessment was not based on Petitioner's repairs but based on the purchase of the property in 2015 which uncapped the assessment for 2016. Respondent assessed the property as ``average'' condition as of December 31, 2015. The changes were not substantial and were based on the rate of inflation.
	
% 	On rebuttal, Respondent argues mass appraisal does not account for properties on-by-on. In other words, properties are assessed uniformly and Respondent assumes properties are in average condition.
% \end{quotation}

% Appellant believes that the Tax Tribunal's characterization of Appellee's argument in the first paragraph quoted above is a little nonsensical. It appears to confuse the assessed value (which is determined every year independent of sales) with the taxable value (which uncaps based on sales or transfers). What was probably meant in the first paragraph was something like: 

% \begin{quote}
% 	Respondent argues that the increase in property taxes was caused by an uncapping because of the transfer of the property in 2015. The actual assessed value (or true cash value which is twice the assessed value) changed minimally from 2015 to 2016 based on the rate of inflation. The assessed value did not changed because of repairs done to the property. 
% \end{quote}

% In any case, Appellant believes that however Appellee's arguments are characterized, because the repairs had not yet been ruled to be normal repairs, Appellee's arguments were not as wrong as the arguments put forth by the Tax Tribunal in the \fojAbbr\ and \reconsiderationDeniedAbbr. The Tax Tribunal appears to argue that the assessor may simply ignore normal repairs under certain circumstances. Appellee has not asserted this so far in this case.

Perhaps the root of the Tax Tribunal's error regarding \mathieuGast\ stems from its misunderstanding of what the statute means by ``not consider''. In everyday usage, ``to consider'' implies activity and to ``to not consider'' implies passivity. But in \mathieuGast, non-consideration is active. The statute says that ``the increase in value attributable to [normal repairs] ... shall be indicated on the assessment roll.'' The \cite[s]{STC Bulletin}\ uses the term ``non-consideration treatment''. Passively ignoring normal repairs is a violation of the statute as this Court noted the first time it heard this case:

\begin{quote}
... contrary to MCL 211.27(2), the referee \textit{considered} the increase in value attributed to the repairs when determining the property's TCV. Stated differently, the referee's finding that the property's TCV was \$50,400 was based on its assessment of the property's value after it had been repaired. This was improper because MCL 211.27(2) expressly provides that certain repairs constitute normal repairs so long as they are not part of a structural addition or completion. \pincite{Patru}{5}.
\end{quote}

There is good reason for the STC's nonconsideration treatment. Before-repair and after-repair appraisals force the assessor to *account* for the repairs. Such an accounting would prevent this situation where the assessor has assumed in the past that the property was not in need of repairs and is now continuing the assumption. A before-repair appraisal forces to assessor to ask, What really was the value of the property before the repairs?  The answer is not: the prior year's TCV. The answer requires the assessor to determine the actual condition and value of the property before repairs. In some cases, the previous year's property's record card may accurately reflect the condition of the property and its value. But in other cases, and in this case, the previous year's record card may be wrong. 

In this situation, the assessor testified that she assessed the property using mass appraisal which assumes the property is in average condition. It is undisputed that the City's inspectors had inspected the property and determined that it needed repairs. But the City's assessor was assuming the property did not need repairs. The before-repair appraisal requirement would determine who was correct: the inspector's inspections or the assessor's assumption. 

The Tax Tribunal ruled that because the assessor had assumed the property did not need the repairs, and because the MLS pictures did not explicitly show the need for the repairs, that the repairs could be ignored and the property could be assessed as if the repairs had never happened. The STC's guidance to the contrary could also be ignored. 

