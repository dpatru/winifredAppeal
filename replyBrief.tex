%
% This is one of the samples from the lawtex package:
% http://lawtex.sourceforge.net/
% LawTeX is licensed under the GNU General Public License 
%
\providecommand{\documentclassflag}{}
\documentclass[12pt,\documentclassflag]{michiganCourtOfAppealsBrief} 
\usepackage{etoolbox}
\usepackage[margin=1in]{geometry}
\usepackage{newcent,microtype}
\usepackage{setspace,xcolor}
\usepackage{trace}
\usepackage[T1]{fontenc}

% From tex.stackexchange.com (https://tex.stackexchange.com/a/142258/135718)
\usepackage{xcolor}
\usepackage{todonotes}
% horizontal rule with text, https://tex.stackexchange.com/a/15122/135718
\newcommand*\ruleline[1]{%
  \par\noindent%
  \raisebox{.8ex}{\makebox[\linewidth]{\hrulefill\hspace{1ex}%
      \raisebox{-.8ex}{#1}%
      \hspace{1ex}\hrulefill}}}%

% https://www.overleaf.com/learn/latex/Environments
\newenvironment{draft}{%
  \color{blue}%
  \ruleline{Begin Draft}
}{%
  \ruleline{End Draft}
  \color{black}%
}
% 
% 

% \usepackage{xr-hyper}
\usepackage{xr}
\externaldocument{appendix}
\usepackage[hyperindex=false,linkbordercolor=white]{hyperref}


\makeandletter% use \makeandtab to turn off

% Use this to show a line grid-
% \usepackage[fontsize=12pt,baseline=24pt,lines=27]{grid}
% \usepackage{atbegshi,picture,xcolor} % https://tex.stackexchange.com/a/191004/135718
% \AtBeginShipout{%
%   \AtBeginShipoutUpperLeft{%
%     {\color{red}%
%     \put(\dimexpr -1in-\oddsidemargin,%
%          -\dimexpr 1in+\topmargin+\headheight+\headsep+\topskip)%
%       {%
%        \vtop to\dimexpr\vsize+\baselineskip{%
%          \hrule%
%          \leaders\vbox to\baselineskip{\hrule width\hsize\vfill}\vfill%
%        }%
%       }%
%   }}%
% }
%   \linespread{1}

\usepackage[modulo]{lineno}% use \linenumbers to show line numbers, see https://texblog.org/2012/02/08/adding-line-numbers-to-documents/

\chardef\_=`_% https://tex.stackexchange.com/a/301984/135718 

%%Citations
 
%The command \makeandletter turns the ampersand into a printable character, rather than a special alignment tab \makeandletter

% %Set the information for the title page (later produced by \makefrontmatter)
% \docket{No. 10-553} 
% \appellant{Daniel Patru}
% \appellee{City of Wayne}
% \court{Michigan Court of Appeals}
% \circuit{Sixth}
% \brieffor{Appellant}
% \author{Daniel Patru , P74387\\{\em Petitioner}}
% \address{3309 Solway \\ Knoxville, TN 37931\\ (734) 274-9624}

\begin{document}
\singlespacing%

\citecase[Mich Ed]{Mich Ed Ass’n v Secretary of State
  (On Rehearing), 489 Mich 194, 218; 801 NW2d 35 (2011)}
% (stating that nothing will be read into a clear statute that is not within the manifest intention of the Legislature as derived from the language of the statute itself).

\citecase[Patru]{Patru v City of Wayne, unpublished per curiam opinion of the Court of Appeals, issued May 8, 2018 (Docket No. 337547)}%
\addReference{Patru}{patruvwayne}% Associate appendix with case

\citecase[Antisdale]{Antisdale v City of Galesburg, 420 Mich 265, 362 NW2d 632 (1984)}
\citecase[Briggs]{Briggs Tax Service, LLC v Detroit Pub. Schools, 485 Mich 69; 780 NW2d 753 (2010)}
\citecase[Jones & Laughlin]{Jones & Laughlin Steel Corporation v. City of Warren, 193 Mich App 348; 483 NW2nd 416 (1992)}
\citecase[Rovas]{In re Complaint of Rovas Against SBC Michigan, 482 Mich 90; 754 NW2d 259 (2008)}

% \citecase[Fisher]{Fisher v. Sunfield Township, 163 Mich App 735; 415 NW2nd 297 (1987)}
% \citecase{Arbaugh v. Y&H Corp., 546 U.S. 500 (2006)}
% \citecase[CAF]{CAF Investment Co. v Saginaw Twp, 410 Mich 428; 302 NW2d 164 (1981)}
% \citecase[Coyne]{Coyne v Highland Twp, 169 Mich. App. 401; 425 NW2d 567 (1988)}
% \citecase[Malpass]{Malpass v Dep't of Treasury, 494 Mich. 237; 833 NW2d 272 (2013)}
% \citecase[Meadowlanes]{Meadowlanes Ltd Dividend Housing Ass'n v City of Holland, 437 Mich 473; 473 NW2d 636 (1991)}
% \citecase[Plymouth]{Plymouth Twp. v Wayne County Board of Commissioners, 137 Mich App 738; 359 NW2d 547 (1984)}
% \citecase[Professional Plaza]{Professional Plaza v City of Detroit, 647 NW2d 529; 250 Mich. App. 473  (2002)}
% \citecase[Randolf]{Dep't of Transportation v Randolph, 461 Mich. 757; 610 NW2d 893 (2000)}
% %\citecase[SBC Health Midwest]{SBC Health Midwest, Inc. v City of Kentwood,  \_\_  Mich \_\_ (decided May 1, 2017)}
% \citecase[Stevens]{Stevens v Bangor Twp, 150 Mich App 756; 389 NW2d 176 (1986)}
 
{\makeatletter % needed for optional argument to newstatute.
  % \newstatute[1@]{MCL}{}% place MCL first
  % \newstatute[2@]{MCR}{}
  % \newstatute[3@]{TTR}{}
  % \newstatute[4@]{Dearborn Ordinance}{}% place this fourth
  % \newstatute[5@]{Wayne Ordinance}{}
  \newstatute{MCL 211.27(1)}{}% true cash value
%  (1) As used in this act, "true cash value" means the usual selling price at the place where the property to which the term is applied is at the time of assessment, being the price that could be obtained for the property at private sale, and not at auction sale except as otherwise provided in this section, or at forced sale. The usual selling price may include sales at public auction held by a nongovernmental agency or person if those sales have become a common method of acquisition in the jurisdiction for the class of property being valued. The usual selling price does not include sales at public auction if the sale is part of a liquidation of the seller's assets in a bankruptcy proceeding or if the seller is unable to use common marketing techniques to obtain the usual selling price for the property. A sale or other disposition by this state or an agency or political subdivision of this state of land acquired for delinquent taxes or an appraisal made in connection with the sale or other disposition or the value attributed to the property of regulated public utilities by a governmental regulatory agency for rate-making purposes is not controlling evidence of true cash value for assessment purposes. In determining the true cash value, the assessor shall also consider the advantages and disadvantages of location; quality of soil; zoning; existing use; present economic income of structures, including farm structures; present economic income of land if the land is being farmed or otherwise put to income producing use; quantity and value of standing timber; water power and privileges; minerals, quarries, or other valuable deposits not otherwise exempt under this act known to be available in the land and their value. In determining the true cash value of personal property owned by an electric utility cooperative, the assessor shall consider the number of kilowatt hours of electricity sold per mile of distribution line compared to the average number of kilowatt hours of electricity sold per mile of distribution line for all electric utilities.
  
  \newstatute{MCL 211.27(2)}{}% MathieuGast
%   (2) The assessor shall not consider the increase in true cash value that is a result of expenditures for normal repairs, replacement, and maintenance in determining the true cash value of property for assessment purposes until the property is sold. For the purpose of implementing this subsection, the assessor shall not increase the construction quality classification or reduce the effective age for depreciation purposes, except if the appraisal of the property was erroneous before nonconsideration of the normal repair, replacement, or maintenance, and shall not assign an economic condition factor to the property that differs from the economic condition factor assigned to similar properties as defined by appraisal procedures applied in the jurisdiction. The increase in value attributable to the items included in subdivisions (a) to (o) that is known to the assessor and excluded from true cash value shall be indicated on the assessment roll. This subsection applies only to residential property. The following repairs are considered normal maintenance if they are not part of a structural addition or completion:
% (a) Outside painting.
% (b) Repairing or replacing siding, roof, porches, steps, sidewalks, or drives.
% (c) Repainting, repairing, or replacing existing masonry.
% (d) Replacing awnings.
% (e) Adding or replacing gutters and downspouts.
% (f) Replacing storm windows or doors.
% (g) Insulating or weatherstripping.
% (h) Complete rewiring.
% (i) Replacing plumbing and light fixtures.
% (j) Replacing a furnace with a new furnace of the same type or replacing an oil or gas burner.
% (k) Repairing plaster, inside painting, or other redecorating.
% (l) New ceiling, wall, or floor surfacing.
% (m) Removing partitions to enlarge rooms.
% (n) Replacing an automatic hot water heater.
% (o) Replacing dated interior woodwork.
  
  \newstatute{MCL 211.10d(7)}{}%
%  (7) Every lawful assessment roll shall have a certificate attached signed by the certified assessor who prepared or supervised the preparation of the roll. The certificate shall be in the form prescribed by the state tax commission. If after completing the assessment roll the certified assessor for the local assessing district dies or otherwise becomes incapable of certifying the assessment roll, the county equalization director or the state tax commission shall certify the completed assessment roll at no cost to the local assessing district.


  
  \newstatute{MCL 205.753(2)}{}% allows appeals from a final order of the Tax Tribunal

  \newstatute{MCR 7.204(A)(1)(b)}{}% allows appeals within 21 days of an order on a motion for reconsideration
  
}% end makeatletter block
% \def\mathieuGast{\pincite[l]{MCL}{211.27(2)}}
\def\mathieuGast{\cite[s]{MCL 211.27(2)}}%
%\def\ttr287{\pincite[s]{TTR}{287}}
%\def\inspectionOrdinance{\pincite{Wayne Ordinance}{\S1484.04}}
% \long\def\inspectionOrdinanceText{\begin{quote}
% 1484.04  CERTIFICATE REQUIRED PRIOR TO SALE. 
%    It shall be unlawful to sell, convey or transfer an ownership interest, or act as a broker or agent for the sale, conveyance or transfer of an ownership interest, in any residential dwelling unless and until a valid Certificate of Compliance is first issued. 
% (Ord. 1991-10.  Passed 7-16-91.) 
% \end{quote}}


%\newmisc{STC Bulletin 6 of 2007}{Michigan State Tax Commission (STC) Bulletin No. 6 of 2007 (Foreclosure Guidelines)}
\newmisc{STC Bulletin}{Michigan State Tax Commission (STC) Bulletin No. 7 of 2014 (Mathieu Gast Act)}
\addReference{STC Bulletin}{bulletin7}% Associate appendix with case

\newcommand{\makeAbbreviation}[3]{% ensure that the frsit time an abbreviated word is used, it is presented in long form, and after that in short form. 1: command name, 2: short name, 3: long name
  \IfBeginWith{#3}{#2}{%
    \newcommand{#1}[0]{#3\renewcommand{#1}[0]{#2}}}{%
    \newcommand{#1}[0]{#3 (#2)\renewcommand{#1}[0]{#2}}}}

\makeAbbreviation{\MLS}{MLS}{Multiple Listing Service}
\makeAbbreviation{\MTT}{MTT}{Michigan Tax Tribunal}
\makeAbbreviation{\STC}{STC}{State Tax Commission}
%\makeAbbreviation{\FOJ}{FOJ}{First Final Opinion and Judgment (2017)}
% \makeAbbreviation{\explanatoryLetterAbbr}{Explanatory Letter}{Explanatory Letter submitted by Appellant to the Tax Tribunal on 9/6/2018}
% \newcommand{\explanatoryLetter}[1][]{\explanatoryLetterAbbr\if\relax#1\relax\empty, Appendix at \pageref{explanatoryLetter}\else, page #1, Appendix at \pageref{explanatoryLetter.#1}\fi}

% Note that the command/handle must match the appendix
% labels. Minimize the variations of names.  \makeAbbreviationToRecord
% creates a simple abbreviation that ends in Abbr if you don't want to
% refer to the record.
\newcommand{\makeAbbreviationToRecord}[3]{% 1: command/handle, 2: shortname, 3: longname
  % makeAbbreviationToRecord: #1, #2, #3\par%
  \expandafter\makeAbbreviation\csname #1Abbr\endcsname{#2}{#3}%
  \expandafter\newcommand\csname #1\endcsname[1][]{%
    \Call{#1Abbr}%
    \if\relax##1\relax\empty\ (Appendix at \pageref{#1})%
    \else, p ##1 (Appendix at %
    % Check for page range
    \IfSubStr{##1}{-}{%
     \def\pageRefRange####1-####2XXX{\pageref{#1.####1}--\pageref{#1.####2}}%
      \pageRefRange##1XXX}%
    {\pageref{#1.##1}}%
    )\fi%
  }%
}%

\makeAbbreviationToRecord{explanatoryLetter}{Explanatory Letter}{Explanatory Letter (MTT Docket Line 38)}
% explanatory Letter: (abbr) \explanatoryLetterAbbr\ (to record) \explanatoryLetter[2] \par

\makeAbbreviationToRecord{foj}{FOJ}{Second Final Opinion and Judgment (MTT Docket Line 48)}
% FOJ: (abbr) \fojAbbr (to record) \foj[]\par

\makeAbbreviationToRecord{reconsiderationDenied}{Order Denying Reconsideration}{Order Denying Reconsideration (MTT Docket Line 51)}
% reconsiderationDenied: (abbr) \reconsiderationDeniedAbbr[] (to record) \reconsiderationDenied[] \par

\makeAbbreviationToRecord{repairs}{List of Repairs}{List of Repairs (MTT Docket Line 36)}
% \par repairs: (abbr) \repairsAbbr\ (to record) \repairs[] (to appendix) \pageref{repairs}\par

\makeAbbreviationToRecord{stcform}{STC Form 865}{STC Form 865 Request for Nonconsideration (MTT Docket Line 35)}

\makeAbbreviationToRecord{mlsListing}{MLS Listing}{MLS Listing (MTT Docket Line 32)}
% mlsListing: (abbr) \mlsListingAbbr\ (to record) \mlsListing[]\par

\makeAbbreviationToRecord{mlsHistory}{MLS History}{MLS History (MTT Docket Line 33)}
% mlsHistory: (abbr) \mlsHistoryAbbr\ (to record) \mlsHistory[]\par

\makeAbbreviationToRecord{boardOfReviewDecision}{Board of Review Decision}{Board of Review Decision (MTT Docket Line 2)}

\makeAbbreviationToRecord{cityEvidence}{City's Evidence}{City's Evidence (MTT Docket Line 11)}

\makeAbbreviationToRecord{motionForReconsideration}{Motion for Reconsideration}{Motion for Reconsideration (MTT Docket Line 52)}

\begin{centering}
\bf\scshape State of Michigan\\In the Court of Appeals\\Detroit Office\\~\\ 
\rm 

\makeandtab
\setlength{\tabcolsep}{20pt}%
\begin{tabular}{p{.4\textwidth} p{.4\textwidth}}
\cline{1-2}
  {~

  \raggedright Daniel Patru,\par
  \hfill\textit{Petitioner/Appellant,}
  \vspace{.5\baselineskip}\par
  vs\par
  \vspace{.5\baselineskip}
  \raggedright City of Wayne,\par
  \hfill\textit{Respondent/Appellee.}
  
  ~} &  {~
       \par\par
       \hfill Court of Appeals No. 346894\par
       \hfill Lower Court No. 16-001828-TT\par\vspace{\baselineskip}
       \hfill \textbf{Appellant's Reply Brief}\par
       \hfill \textbf{Proof of Service}       
  ~}
  \\ \cline{1-2}\vspace{2mm}
  {~ \par
  Daniel Patru, P74387, Appellant\newline%
  3309 Solway\newline%
  Knoxville, TN 37931\newline%
  (734) 274-9624\newline%
  dpatru@gmail.com\newline\newline%
  ~} & {~ \par~\par
       
       City of Wayne, Appellee\newline%
       3355 South Wayne Rd,\newline%
       Wayne, MI 48184\newline%
       (734) 722-2000\newline\newline%

       Stephen J. Hitchcock (P15005)\newline%
       John C. Clark (P51356)\newline%
       Attorneys for Respondent/Appellee\newline%
       Giarmarco, Mullins \& Horton, P.C.\newline%
       101 W. Big Beaver Road, 10th floor\newline%
       Troy, MI 48084\newline%
       (248) 457-7024\newline%
       sjh@gmhlaw.com
  ~}
\end{tabular}
\makeandletter
\par\vspace{\baselineskip}\vspace{\baselineskip}\vspace{\baselineskip}
%\textbf{ORAL ARGUMENT NOT REQUESTED}

\end{centering}

\pagestyle{romanparen}
\pagenumbering{roman}
\newpage 

\section*{Table of Contents}

\tableofcontents

%\newpage
\tableofauthorities

\pagestyle{plain}
\pagenumbering{arabic}

%This commands creates the title page, table of contents, and table of authorities
% \makefrontmatter{Brief\\Proof of Service}


%Sets the formatting for the entire document after the front matter
\parindent=2.5em
% \setlength{\parskip}{1.25ex plus 2ex minus .5ex} 
% \setstretch{1.45}
\doublespacing
% \linenumbers


% \section{Mathieu-Gast Statute -- MCL 211.27(2)}
% \begin{quotation}
% The assessor shall not consider the increase in true cash value that is a result of expenditures for normal repairs, replacement, and maintenance in determining the true cash value of property for assessment purposes until the property is sold.

% For the purpose of implementing this subsection, the assessor shall not increase the construction quality classification or reduce the effective age for depreciation purposes, except if the appraisal of the property was erroneous before nonconsideration of the normal repair, replacement, or maintenance, and shall not assign an economic condition factor to the property that differs from the economic condition factor assigned to similar properties as defined by appraisal procedures applied in the jurisdiction.

% The increase in value attributable to the items included in subdivisions (a) to (o) that is known to the assessor and excluded from true cash value shall be indicated on the assessment roll.

% This subsection applies only to residential property.

% The following repairs are considered normal maintenance if they are not part of a structural addition or completion: [repairs (a)-(o) omitted]

% % (a) Outside painting.
% % (b) Repairing or replacing siding, roof, porches, steps, sidewalks, or drives.
% % (c) Repainting, repairing, or replacing existing masonry.
% % (d) Replacing awnings.
% % (e) Adding or replacing gutters and downspouts.
% % (f) Replacing storm windows or doors.
% % (g) Insulating or weatherstripping.
% % (h) Complete rewiring.
% % (i) Replacing plumbing and light fixtures.
% % (j) Replacing a furnace with a new furnace of the same type or replacing an oil or gas burner.
% % (k) Repairing plaster, inside painting, or other redecorating.
% % (l) New ceiling, wall, or floor surfacing.
% % (m) Removing partitions to enlarge rooms.
% % (n) Replacing an automatic hot water heater.
% % (o) Replacing dated interior woodwork.
% \end{quotation}

\section{Reply}

Appellee's brief does not dispute Appellant's issues, but instead raises two different issues which do not affect the issues of this case.
Also, Appellee's Counter-Statement of Facts makes mistakes.
Appellant discusses each of these points below.

\subsection{Appellee does not dispute Appellant's issues}

Appellant raised three issues on appeal. He argued that the Tribunal erred when it:

\begin{enumerate}
\item refused to apply nonconsideration treatment to normal repairs contrary to the clear language of \mathieuGast,
\item refused to determine the before-repair value per the STC's guidance and the Tribunal's duty to independently determine the true cash value, and
\item cursorily rejected the subject's sale as evidence of the before-repair value contrary to the teaching of this Court in \cite{Jones & Laughlin}.
\end{enumerate}


%Appellant did not know if Appellee disagreed with any of these issues. In the ``Questions Involved'' section of his appeal brief, Appellant marked the answer to each question as ``Appellee's answer is unknown.'' Appellant's brief at 3.
Appellee's brief does not dispute any of these issues with Appellant. This Court need not shoulder the burden of arguing Appellee's case when Appellee refuses to argue for itself. Therefore this Court should reverse the Tribunal's decision according to the arguments made in Appellant's brief.

\subsection{Appellee's argument for its first issue is not well-defined nor well-explained nor responsive to Appellant's arguments regarding the Tribunal's rejection of the subject's sale}
  %not defined or supported its claim that Appellant has not met his burden of proof}
 
Instead of disputing the Appellant's points, Appellee raises two of its own issues. First, the Appellee claims that Appellant failed to satisfy his burden of proof because ``The Subject's Purchase Price Is Not a Reliable Indicator of Value Due to the Home at the Time of Purchase, it Being a Bank Sale.'' Appellee's brief at 4.

Appellant argued the subject's sale, when it was not repaired, is evidence of the before-repair value. The Tribunal has rejected both the need to determine the before-repair value and the sufficiency of the subject's sale to prove the before-repair value. Appellant has appealed these issues as the second and third question presented in his brief.

Appellee's brief does not address the arguments presented in Appellant's brief, but merely repeats a variation of the Tribunal's objection. The Tribunal rejected the sale because the seller was HUD. Appellee rejects the sale because it was a ``bank sale''.

\subsubsection{The subject's sale is not disfavored by \protect\cite{MCL 211.27(1)}}

%Appellee claims that the subject's sale was a bank sale, even though the seller was HUD, a government entity, not a bank. \mlsListing.

Appellee does not explain what it means by ``bank sale''. Also the term does not appear in a full text search of Michigan's laws via www.legislature.mi.gov.

The law does refer to certain types of sales which are not thought to set the usual selling price. \cite[s]{MCL 211.27(1)}\ defines ``true cash value'' in relevant part as:

\begin{quotation}
  (1) As used in this act, "true cash value" means the usual selling price at the place where the property to which the term is applied is at the time of assessment, being the price that could be obtained for the property at private sale, and not at auction sale except as otherwise provided in this section, or at forced sale.

  The usual selling price may include sales at public auction held by a nongovernmental agency or person if those sales have become a common method of acquisition in the jurisdiction for the class of property being valued.

  The usual selling price does not include sales at public auction if the sale is part of a liquidation of the seller's assets in a bankruptcy proceeding or if the seller is unable to use common marketing techniques to obtain the usual selling price for the property.

  A sale or other disposition by this state or an agency or political subdivision of this state of land acquired for delinquent taxes or an appraisal made in connection with the sale or other disposition or the value attributed to the property of regulated public utilities by a governmental regulatory agency for rate-making purposes is not controlling evidence of true cash value for assessment purposes.

%  The rest of MCL 211.27(1) discusses other factors of true cash value which are  not relevant here]
  %In determining the true cash value, the assessor shall also consider the advantages and disadvantages of location; quality of soil; zoning; existing use; present economic income of structures, including farm structures; present economic income of land if the land is being farmed or otherwise put to income producing use; quantity and value of standing timber; water power and privileges; minerals, quarries, or other valuable deposits not otherwise exempt under this act known to be available in the land and their value. In determining the true cash value of personal property owned by an electric utility cooperative, the assessor shall consider the number of kilowatt hours of electricity sold per mile of distribution line compared to the average number of kilowatt hours of electricity sold per mile of distribution line for all electric utilities.
\end{quotation}

 The subject did not sell at a foreclosure sale or other kind of sale disfavored by \cite[s]{MCL 211.27(1)}. Rather, it was sold more than two months after being listed on the MLS by Century 21 Castelli, a real estate broker. Before being listed by Century 21 Castelli, the subject was listed on the MLS for more than six months by Jeffery Packer, another real estate broker. Also, there were two at least two accepted offers that failed to close before the property was bought by Appellant. \mlsHistory.

 Also, the subject was not sold under circumstances that would require a buyer to pay cash immediately on sale such as in a foreclosure sale.
 The listing specifically says that the home is ``203K elig[able]'' and ``203B with Repair Escrow financing availability is subject to buyer's appraisal.'' These are FHA mortgages that allow for the purchase of homes that need to be repaired. 

\subsubsection{Appellee does not explain why the sale was not a reliable indicator of value}

%Appellee does not give any argument as to why the sale is not a reliable indicator of value.
%The reasons that Appellee uses are irrelevant to its contention that the
In support of its argument that Appellant did not meet his burden of proof because the sale was not a reliable indicator of value, Appellee's brief on page 4 offers bare out-of-context quotes from the \foj[5] about uncapping and MLS pictures. Appellee offers no explanation in its brief as to why these quotes support its argument.

Nor does the Tribunal itself use the passages quoted by Appellee to support Appellee's argument. These quotes come from a single paragraph in which the Tribunal makes four separate arguments.
Appellant addressed these arguments in the section ``Conclusions of Law'' in his \motionForReconsideration[4-5].
In response, the Tribunal clarified its reasoning in the denial of the motion. The Tribunal made clear that it was rejecting the subject's sale based on the fact that the seller was HUD, not because of uncapping or MLS pictures:

\begin{quote}
Further, the Tribunal cannot conclude that the FOJ erred when it
concluded that Petitioner's contentions concerning the purchase price, i.e. the true cash value before repairs, was entitled to no weight or credibility. The selling price of a property is not is presumptive true cash value. Despite Petitioner's assertions that the marketing efforts for the subject show that the sale was a ``market sale,'' the home was
being sold by the U.S. Department of Housing and Urban Development (``HUD''). Because the subject was being sold by a government entity, that entity's motivation may not have been to receive market value for the property. \reconsiderationDenied[2]\ (cleaned up).
\end{quote}

The Tribunal dropped its argument about uncapping. ``Although the Tribunal concludes that the plain language of MCL 211.27(2) . . . does not speak to whether a property has been uncapped, the Tribunal nonetheless concludes that the FOJ did not commit palpable error in its final conclusion.'' \reconsiderationDenied[1-2]. After writing this, the Tribunal rejected the motion on other grounds. 

The Tribunal considered the MLS pictures as evidence of a different point, that the property did not need repairs when it was bought:

\begin{quote}
  The Tribunal also concludes that it was not a palpable error to conclude that the assessment did not consider the ``normal repairs.'' . . .
  % This is supported by the fact that the property record card indicates that Respondent believed the true cash value of the subject to be \$48,000 before Petitioner purchased the property and \$50,400 as of December 31, 2015, after the repairs were complete. An increase of \$2,400 in true cash value (5\%) is easily attributable to inflation and increases in the market.
  {\em In addition, as stated by the FOJ, the interior photographs depict a property in average condition before Petitioner acquired it.} Although not necessarily evidence of true cash value, this evidence supports the property's assessment as a property in average condition both at the time Petitioner acquired it and after he completed the normal repairs. In other words, the record evidence supports the conclusion that the assessment did not consider the increase in true cash value that was the result of normal repairs. \reconsiderationDenied[2]\ (cleaned up, emphasis added).
\end{quote}

Appellant addresses the Tribunal's argument above in his brief, Argument section IA and IB at 7-13. But the point here is that the Tribunal itself did not cite the MLS pictures to support the Appellee's argument.

\subsubsection{Appellee does not address Appellant's arguments to the Tribunal's rejection of the subject's sale}

The Tribunal rejected the subject's sale because the seller was HUD, a government entity, which may not have been motivated to receive market value for the property. \reconsiderationDenied[2]. Appellant argues against the Tribunal's rejection in his brief at 18-19, relying on this Court's teaching in \cite{Jones & Laughlin}. Appellee does not dispute Appellant's argument or distinguish this case. 

Therefore this Court should reject Appellee's first issue.


% Therefore, Appellee's brief does not justify the Tribunal's rejection
% of the subject sale.  Appellee's other arguments

% step back: Appellant implicitly concedes that the before-repair value
% should be determined. It just disagrees as to the sufficiency of the
% evidence.

% step back: even if true, this does not excuse the Tribunal from making
% a reasoned determination of the before-repair value as argued in
% Appellant's brief.

\subsection{Appellee's second issue does not address a dispute}

Appellee's second issue asserts an uncontested fact: that the Tribunal's determination of the {\em after-repair} value, the value on tax day, is correct. Appellee correctly notes on page one of its brief that Appellant does not disagree: ``Appellant acknowledges that he does not dispute the house is worth the assessed TCV of \$50,400 on tax day December 31, 2015."

Appellant's contention is \mathieuGast\ requires that the {\em before-repair} value should be used for assessment purposes because the after-repair value includes the value of the repairs. Appellee does dispute this issue or even address it.

Therefore this Court should reject Appellee's second issue as a support for its case.

% \subsection{Appellee's brief does not address Mathieu-Gast nonconsideration}

% Appellee's brief does not address the main issues of the case, whether the Tribunal erred in regarding Mathieu-Gast nonconsideration, \mathieuGast.

% \mathieuGast\ requires nonconsideration treatment for normal repairs. Nonconconsideration treatment involves three steps:
% \begin{enumerate}
% \item determining the value of the repairs using before-repair and after-repair appraisals (``[T]he true cash value of the item shall be calculated by performing `before' and `after' appraisals and then deducting the `before' true cash value from the `after' true cash value.'' \pincite{STC Bulletin}{2}; ``The assessor is required to estimate the true cash value of the property both before and after the expenditures.'' \stcform[2]),

% \item excluding the value of the value of the repairs from the true cash value for assessment purposes (``The assessor shall not consider the increase in true cash value that is a result of expenditures for normal repairs, replacement, and maintenance in determining the true cash value of property for assessment purposes until the property is sold.'' \mathieuGast, first sentence), and

% \item indicating the value of the repairs in the assessment roll (``The increase in value attributable to the items included in subdivisions (a) to (o) that is known to the assessor and excluded from true cash value shall be indicated on the assessment roll.'' \mathieuGast, third sentence).
% \end{enumerate}

% Except for determining the after-repair value, the Tribunal did none of these things and most of Appellant's brief is devoted to explaining why the Tribunal erred by not doing them. Appellee's brief does not add anything to the discussion, either explicitly (it does not mention \mathieuGast\ at all other than in the Statement of Facts) nor implicitly.

% Instead Appellee in its second issue asserts an uncontested fact: that the Tribunal's determination of the after-repair value is correct. Appellee correctly notes on page one of its brief that Appellant does not disagree: ``Appellant acknowledges that he does not dispute the house is worth the assessed TCV of \$50,400 on tax day December 31, 2015."

% Appellee's other issue, that Appellant failed to satisfy his burden of proof, is likewise unhelpful to this Court. Appellant has pointed out in his brief (Argument, section IIB, pages 15-18) that this Court has ruled that even if the Tribunal disagrees with Petitioner's proofs, it is still required to independently determine the true cash value, here by applying Mathieu-Gast nonconsideration. Appellee has not contradicted this or even addressed the issue. Thus, even if Appellee were correct, that Appellant has not satisfied his burden of proof, this does not excuse the Tribunal from applying Mathieu-Gast nonconsideration. (Note that Appellant does not concede that his proofs were not sufficient, only that if they were insufficient, the Tribunal must still apply nonconsideration.)

% Thus Appellee does not provide this Court with any arguments as to why Appellee is wrong to insist that the Tribunal must apply \mathieuGast\ nonconsideration treatment.

% \subsection{Appellee's brief does not justify the Tribunal's cursory rejection of the subject's sale}

% Besides issues concerning the interpretation and application of Mathieu-Gast nonconsideration, the other issue raised by Appellant was the Tribunal's cursory rejection of the subject's sale. The Tribunal ruled that the sale ``was entitled to no weight or credibility'' because:

% \begin{enumerate}
% \item ``The selling price of a property is not its presumptive true cash value'',
% \item ``the home was being sold by the U.S. Department of Housing and Urban Development'', and
% \item speculation that ``[b]ecause the subject was being sold by a government entity, that entity's motivation may not have been to receive market value for the property.''
% \end{enumerate}
% \reconsiderationDenied[2].

% Appellee's brief does not directly address Appellant's argument regarding this issue. In particular, it does not seek to distinguish this case from \cite{Jones & Laughlin}\ where this Court ruled that it is an error as a matter of law to cursorily reject the sale of the subject property:

% \begin{quote}
%   The Tax Tribunal, however, erred as a matter of law in its treatment of petitioner's evidence regarding the sale. The tribunal held: "A sale that occurs {\em after} the tax date has little or no bearing on the assessment made prior to the sale." (Emphasis in original.) We disagree. Unlike some situations involving assessments of industrial property for which no ready market exists and a hypothetical buyer must be posited, in this case the equipment was actually sold in a commercial transaction, albeit after the tax date. We believe that evidence of the price at which an item of property actually sold is most certainly relevant evidence of its value at an earlier time within the meaning of the term "relevant evidence." MRE 401. \pincite{Jones & Laughlin}{353-354}.
% \end{quote}

% In its subsection entitled ``The Subject's Purchase Price Is Not a
% Reliable Indicator of Value Due to the Home at the Time of Purchase,
% it Being a Bank Sale'', page 4, Appellee's brief gives two quotes without additional explanation from the \foj[5]. The first quote
% (``Petitioner's property assessment did not change by virtue of the
% repairs Petitioner made to subject property. To the contrary, the
% subject property's assessment for 2016 changed based on the sale
% transaction of the subject property in August 2015.'') seems to be a discussion of uncapping and does not justify the Tribunal's cursory rejection of the subject's sale.

% The second quote (``Petitioner's MLS print-out information for the subject and its comparable sales illustrate properties in ``average'' condition.
% Specifically, the subject and comparable sales' interior photographs
% do not depict neglected or vandalized p``Petitioner's property assessment did not change by virtue of the
% repairs Petitioner made to subject property. To the contrary, the
% subject property's assessment for 2016 changed based on the sale
% transaction of the subject property in August 2015.'') seems to be a discussion of uncapping and does not justify the Tribunal's cursory rejection of the subject's sale.

% The second quote (``Petitioner's MLS print-out information for the subject and its comparable sales illustrate properties in ``average'' condition.
% Specifically, the subject and comparable sales' interior photographs
% do not depict neglected or vandalized p``Petitioner's property assessment did not change by virtue of the
% repairs Petitioner made to subject property. To the contrary, the
% subject property's assessment for 2016 changed based on the sale
% transaction of the subject property in August 2015.'') seems to be a discussion of uncapping and does not justify the Tribunal's cursory rejection of the subject's sale.

% The second quote (``Petitioner's MLS print-out information for the subject and its comparable sales illustrate properties in ``average'' condition.
% Specifically, the subject and comparable sales' interior photographs
% do not depict neglected or vandalized properties.'') is about the condition of the property as depicted in the MLS listing.
% The FOJ is confusing here because it packs into one paragraph four reasons for denying ``Petitioner's contentions related to his purchase price and 'normal' repairs.'' Appellant addressed these issues individually in the \motionForReconsideration[4-5]. The Tribunal clarified its reasoning in the denial of the motion. The Tribunal made clear that it was rejecting the subject's sale based on the fact that the seller was HUD:

% \begin{quote}
% Further, the Tribunal cannot conclude that the FOJ erred when it
% concluded that Petitioner's contentions concerning the purchase price, i.e. the true cash value before repairs, was entitled to no weight or credibility. The selling price of a property is not is presumptive true cash value. Despite Petitioner's assertions that the marketing efforts for the subject show that the sale was a ``market sale,'' the home was
% being sold by the U.S. Department of Housing and Urban Development (``HUD''). Because the subject was being sold by a government entity, that entity's motivation may not have been to receive market value for the property. \reconsiderationDenied[2]\ (cleaned up).
% \end{quote}

% The Tribunal considered the MLS pictures as evidence of a different point, that the assessor could have assessed the property without considering the repairs:

% \begin{quote}
%   The Tribunal also concludes that it was not a palpable error to conclude that the assessment did not consider the ``normal repairs.'' This is supported by the fact that the property record card indicates that Respondent believed the true cash value of the subject to be \$48,000 before Petitioner purchased the property and \$50,400 as of December 31, 2015, after the repairs were complete. An increase of \$2,400 in true cash value (5\%) is easily attributable to inflation and increases in the market. {\em In addition, as stated by the FOJ, the interior photographs depict a property in average condition before Petitioner acquired it.} Although not necessarily evidence of true cash value, this evidence supports the property's assessment as a property in average condition both at the time Petitioner acquired it and after he completed the normal repairs. In other words, the record evidence supports the conclusion that the assessment did not consider the increase in true cash value that was the result of normal repairs. \reconsiderationDenied[2]\ (cleaned up, emphasis added).
% \end{quote}

% Thus the MLS pictures were not used by the Tribunal to reject the sale of the subject. Instead, the pictures were used to support the Tribunal's theory that the assessor did not violate \mathieuGast\ because she thought the property did not need repairs. (Appellant's brief, Argument section IA and IB at 7-13 rebuts this theory.)

% Not only did the Tribunal not actually reject the subject's sale based on the MLS pictures, but the evidence in this case does not support the implication that the property did not need repairs when it sold. Prior to the Tribunal's suggestion in the second FOJ, the consensus among Petitioner, Respondent, and the Tribunal was, as this Court summarized in its first ruling on this case, ``It is undisputed that, when he purchased the property, it was in substandard condition and required numerous repairs to make it livable.'' \pincite{Patru}{1}. (See Appellant's brief Argument section IB at 12-13 for more argument on this point.)

% Therefore, Appellee's brief does not justify the Tribunal's rejection of the subject sale.

\subsection{Appellee makes mistakes in its Counter-Statement of Facts}

Appellee has made several mistakes in its Counter-Statement of Fact. Appellant discusses them here to help keep the record clean and to alert this Court to statements in the Appellee's version of the facts which may be confusing.

\subsubsection{Appellee does not mention that Appellant's claim is based on Mathieu-Gast nonconsideration}


The second paragraph on page 1 of Appellee's brief says:

\begin{quote}
Appellant is contesting the true cash value (TCV) of the subject property, alleging that his
purchase price of \$32,000 is the TCV, with an SEV of \$16,000. Appellant acknowledges that he
does not dispute the house is worth the assessed TCV of \$50,400 on tax day December 31, 2015. (See Appellant's Brief, p. 4)
\end{quote}

This paragraph is confusing because it fails to explain why Appellant could be claiming a true cash value of \$32,000 when he admits that the property was worth \$50,400 on tax day. Appellant's version of the facts clear this up. Here is the passage from Appellant's brief page 4, with references to the record removed:

\begin{quotation}
  % Appellant appealed to the Board of Review and then to the Tax Tribunal.
  Appellant does not dispute that the house was worth \$50,400 on tax day in its repaired condition. But he contends that under \mathieuGast\ the repairs were normal repairs and that the true cash value for assessment purposes cannot include the value of the repairs. He contends that the correct true cash value is therefore the before-repair value. 

Appellant contends that the best evidence of the house's before-repair value is its sale price of \$32,000 when it was unrepaired. The house was marketed in the normal way and for a sufficient time. Licensed real estate brokers listed the house on the MLS, initially for \$29,900 on 4/3/2013 and later for \$32,000 on 6/17/2015. Before Appellant bought the property there had been at least two accepted offers on the property that failed to close.
\end{quotation}

Appellant's main contentions on this appeal are that the Tribunal violated \mathieuGast. To leave out this important fact risks misleading this Court.

\subsubsection{Appellee wrongly says that the comparables required at most one adjustment}

Appellee states in the third paragraph on page 1 of its brief ``Appellee provided five comparable sales . . . with only one adjustment, if any, made per comparable.'' All five comparables are adjusted and most (2, 4, and 5) are adjusted for both air conditioning and garage/carport. \cityEvidence[2].

\subsubsection{Appellee does not explain that the subject's listing by electronic means was not unusual or significant}

%Appellee makes at least two other mistakes in its presentation of the facts. They are listed here to keep the record correct.

Appellee states in the third paragraph on page 1 of its brief ``that the subject property was a HUD home, listed only by electronic means.'' This may be misleading because all listings now are electronic, in the sense that the MLS is a computer database accessed via the internet. The subject was listed as any other property on the MLS by local, licensed real estate brokers. See \mlsListing\ and \mlsHistory.

Perhaps what Appellee meant to say is that bids or offers on the subject had to be placed electronically. The 2nd paragraph of the remarks, meant for real estate agents, tells the selling agent to place offers electronically. ``Sold AS IS by elec bid only. . . . Avail 7-3-15. Bids dues daily by 11:59 PM Central Time til sold. . . . For info visit www.HUDHomestore.com.'' \mlsListing. There is no evidence that this affected the property's price or restricted its marketing. The property had at least two accepted offers that failed to close before it sold to Appellant. \mlsHistory.

% \begin{quote}
%   HUD Home. Sold AS IS by elec bid only. Up to \$1250 comm. Avail 7-3-15. Bids due daily by 11:59 PM Central Time til sold. FHA Case#264-004175. Insured w/escrow repair. 203K elig. IE-FHA 203B with Repair Escrow financing availability is subject to buyer's appraisal. Keys in LO \$2/ea or \$12/8-key set. For info visit www.HUDHomestore.com. For add forms, updates, step­by­step video & free photo list, please visit www.BLBResources.com. BLB Resources makes no warranty as to condition of prop. BVAI. LBP Adden
% \end{quote}

\subsubsection{Appellee does not explain that the record card's valuation was based on the assumption that the subject was in average condition}

Appellee states in the third paragraph on page 1 of its brief that ``Appellee's property record card indicates that Appellee believes that TCV of the subject property to be \$48,000 before Appellant purchased the property.'' This is misleading because Appellee does not point out that the assessor testified that her assessments are done using mass appraisal which ``does not account for properties one-by-one'', but rather she ``{\em assumes} that properties are in `average' condition.'' \foj[4] (emphasis added). Thus the assessment reflected in the record card was not a fact-based belief that took into consideration the repairs called for by the City's own inspectors, but rather an assumption that the property was in average condition.

\subsubsection{Appellee's characterizations of the Tribunal's rulings are confusing}

Appellee in the first paragraph on page 2 of its brief, refers to a ``Proposed Opinion and Judgment.'' This should be ``Final Opinion and Judgement'' (FOJ).

Also on page 2, Appellant has a large block quote ending with ``(Exhibit 1, Final Opinion and Judgment, p 6).'' Block quotes normally indicate a quotation, but here, Appellant appears to have extracted from the opinion four points and listed them numbered. Points one and two were taken from page 5. Points three and four are from page 6. Point four concatenates excerpts from two paragraphs discussing different points.

Appellant's quotations on pages 2 and 3 of its brief from the \reconsiderationDenied[2], also concatenates the beginning of one paragraph, discussing the Tribunal's refusal to follow the STC's requirement of before-repair and after-repair appraisals, with the tail of another paragraph, discussing why the assessment did not consider normal repairs. The implication is that the Tribunal used the property record card and MLS pictures to support its conclusion that \mathieuGast\ does not require before-repair and after-repair appraisals.

Appellee's unusual use of block quotation and its presentation of out-of-context quotes is at best confusing. Not only does the formatting not follow convention, but Appellee does not inform the reader of the flow of the argument. For example, Appellee fails to inform the reader that its first point from the FOJ (``Petitioner's property assessment did not change based on the sale transaction of August 2015.'') is immediately followed by ``To the contrary, the subject property's assessment for 2016 changed based on he sale transaction of the subject property in August 2015.'' (See the last paragraph of \foj[5].) Thus the reader is not alerted to the fact that the Tribunal is referring to uncapping here. Nor does Appellee inform the reader that Appellant rebutted the Tribunal on this point in his \motionForReconsideration[4], and that the Tribunal conceded this point and did not rely on it in rejecting the motion to reconsider. ``Although the Tribunal concludes that the plain language of MCL 211.27(2) . . . does not speak to whether a property has been uncapped, the Tribunal nonetheless concludes that the FOJ did not commit palpable error in its final conclusion.'' \reconsiderationDenied[1-2]. This example illustrates that this Court should not assume that Appellee's characterization of the Tribunal's writings is enlightening.

\section{Relief Requested}

Therefore, because Appellee has failed to rebut or even dispute Appellant's issues, Appellant respectfully asks this Court to reverse the ruling of the Tribunal.

\section{Note on Corrected Appendix}

Appellant has discovered that the appendix submitted with his brief has a flawed footer on pages 37-42. The footer does not reflect that the document on those pages is this Court's decision in \cite{Patru}. Appellant includes the corrected appendix with this reply brief.

\section{Proof of Service}

On 4/28/2019, I served a copy of this Brief and Appendix on Appellee's counsel by electronic service.

\vspace{1\baselineskip}

{ \setlength{\leftskip}{3.5in}

  Respectfully Submitted,

  /s/ Daniel Patru, P74387

  4/28/2019

  \setlength{\leftskip}{0pt}}

\newpage\empty% we need a new page so that the index entries on the last
        % page get written out to the right file.
\end{document}


%%% Local Variables:
%%% mode: latex
%%% TeX-master: t
%%% End:
