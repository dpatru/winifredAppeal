%
% This is one of the samples from the lawtex package:
% http://lawtex.sourceforge.net/
% LawTeX is licensed under the GNU General Public License 
%
\providecommand{\documentclassflag}{}
\documentclass[12pt,\documentclassflag]{michiganCourtOfAppealsBrief} 
\usepackage{etoolbox}
\usepackage[margin=1in]{geometry}
\usepackage{newcent,microtype}
\usepackage{setspace,xcolor}
\usepackage{trace}
\usepackage[T1]{fontenc}

% From tex.stackexchange.com (https://tex.stackexchange.com/a/142258/135718)
\usepackage{xcolor}
\usepackage{todonotes}
% horizontal rule with text, https://tex.stackexchange.com/a/15122/135718
\newcommand*\ruleline[1]{%
  \par\noindent%
  \raisebox{.8ex}{\makebox[\linewidth]{\hrulefill\hspace{1ex}%
      \raisebox{-.8ex}{#1}%
      \hspace{1ex}\hrulefill}}}%

% https://www.overleaf.com/learn/latex/Environments
\newenvironment{draft}{%
  \color{blue}%
  \ruleline{Begin Draft}
}{%
  \ruleline{End Draft}
  \color{black}%
}
% 
% 

% \usepackage{xr-hyper}
\usepackage{xr}
\externaldocument{appendix}
\usepackage[hyperindex=false,linkbordercolor=white]{hyperref}


\makeandletter% use \makeandtab to turn off

% Use this to show a line grid-
% \usepackage[fontsize=12pt,baseline=24pt,lines=27]{grid}
% \usepackage{atbegshi,picture,xcolor} % https://tex.stackexchange.com/a/191004/135718
% \AtBeginShipout{%
%   \AtBeginShipoutUpperLeft{%
%     {\color{red}%
%     \put(\dimexpr -1in-\oddsidemargin,%
%          -\dimexpr 1in+\topmargin+\headheight+\headsep+\topskip)%
%       {%
%        \vtop to\dimexpr\vsize+\baselineskip{%
%          \hrule%
%          \leaders\vbox to\baselineskip{\hrule width\hsize\vfill}\vfill%
%        }%
%       }%
%   }}%
% }
%   \linespread{1}

\usepackage[modulo]{lineno}% use \linenumbers to show line numbers, see https://texblog.org/2012/02/08/adding-line-numbers-to-documents/

\chardef\_=`_% https://tex.stackexchange.com/a/301984/135718 

%%Citations
 
%The command \makeandletter turns the ampersand into a printable character, rather than a special alignment tab \makeandletter

% %Set the information for the title page (later produced by \makefrontmatter)
% \docket{No. 10-553} 
% \appellant{Daniel Patru}
% \appellee{City of Wayne}
% \court{Michigan Court of Appeals}
% \circuit{Sixth}
% \brieffor{Appellant}
% \author{Daniel Patru , P74387\\{\em Petitioner}}
% \address{3309 Solway \\ Knoxville, TN 37931\\ (734) 274-9624}

\begin{document}
\singlespacing%

\citecase[Mich Ed]{Mich Ed Ass’n v Secretary of State
  (On Rehearing), 489 Mich 194, 218; 801 NW2d 35 (2011)}
% (stating that nothing will be read into a clear statute that is not within the manifest intention of the Legislature as derived from the language of the statute itself).

\citecase[Patru]{Patru v City of Wayne, unpublished per curiam opinion of the Court of Appeals, issued May 8, 2018 (Docket No. 337547)}%
\addReference{Patru}{patruvwayne}% Associate appendix with case

\citecase[Antisdale]{Antisdale v City of Galesburg, 420 Mich 265, 362 NW2d 632 (1984)}
\citecase[Briggs]{Briggs Tax Service, LLC v Detroit Pub. Schools, 485 Mich 69; 780 NW2d 753 (2010)}
\citecase[Jones & Laughlin]{Jones & Laughlin Steel Corporation v. City of Warren, 193 Mich App 348; 483 NW2nd 416 (1992)}
\citecase[Rovas]{In re Complaint of Rovas Against SBC Michigan, 482 Mich 90; 754 NW2d 259 (2008)}

% \citecase[Fisher]{Fisher v. Sunfield Township, 163 Mich App 735; 415 NW2nd 297 (1987)}
% \citecase{Arbaugh v. Y&H Corp., 546 U.S. 500 (2006)}
% \citecase[CAF]{CAF Investment Co. v Saginaw Twp, 410 Mich 428; 302 NW2d 164 (1981)}
% \citecase[Coyne]{Coyne v Highland Twp, 169 Mich. App. 401; 425 NW2d 567 (1988)}
% \citecase[Malpass]{Malpass v Dep't of Treasury, 494 Mich. 237; 833 NW2d 272 (2013)}
% \citecase[Meadowlanes]{Meadowlanes Ltd Dividend Housing Ass'n v City of Holland, 437 Mich 473; 473 NW2d 636 (1991)}
% \citecase[Plymouth]{Plymouth Twp. v Wayne County Board of Commissioners, 137 Mich App 738; 359 NW2d 547 (1984)}
% \citecase[Professional Plaza]{Professional Plaza v City of Detroit, 647 NW2d 529; 250 Mich. App. 473  (2002)}
% \citecase[Randolf]{Dep't of Transportation v Randolph, 461 Mich. 757; 610 NW2d 893 (2000)}
% %\citecase[SBC Health Midwest]{SBC Health Midwest, Inc. v City of Kentwood,  \_\_  Mich \_\_ (decided May 1, 2017)}
% \citecase[Stevens]{Stevens v Bangor Twp, 150 Mich App 756; 389 NW2d 176 (1986)}
 
{\makeatletter % needed for optional argument to newstatute.
  % \newstatute[1@]{MCL}{}% place MCL first
  % \newstatute[2@]{MCR}{}
  % \newstatute[3@]{TTR}{}
  % \newstatute[4@]{Dearborn Ordinance}{}% place this fourth
  % \newstatute[5@]{Wayne Ordinance}{}
  \newstatute{MCL 211.27(2)}{}% MathieuGast
%   (2) The assessor shall not consider the increase in true cash value that is a result of expenditures for normal repairs, replacement, and maintenance in determining the true cash value of property for assessment purposes until the property is sold. For the purpose of implementing this subsection, the assessor shall not increase the construction quality classification or reduce the effective age for depreciation purposes, except if the appraisal of the property was erroneous before nonconsideration of the normal repair, replacement, or maintenance, and shall not assign an economic condition factor to the property that differs from the economic condition factor assigned to similar properties as defined by appraisal procedures applied in the jurisdiction. The increase in value attributable to the items included in subdivisions (a) to (o) that is known to the assessor and excluded from true cash value shall be indicated on the assessment roll. This subsection applies only to residential property. The following repairs are considered normal maintenance if they are not part of a structural addition or completion:
% (a) Outside painting.
% (b) Repairing or replacing siding, roof, porches, steps, sidewalks, or drives.
% (c) Repainting, repairing, or replacing existing masonry.
% (d) Replacing awnings.
% (e) Adding or replacing gutters and downspouts.
% (f) Replacing storm windows or doors.
% (g) Insulating or weatherstripping.
% (h) Complete rewiring.
% (i) Replacing plumbing and light fixtures.
% (j) Replacing a furnace with a new furnace of the same type or replacing an oil or gas burner.
% (k) Repairing plaster, inside painting, or other redecorating.
% (l) New ceiling, wall, or floor surfacing.
% (m) Removing partitions to enlarge rooms.
% (n) Replacing an automatic hot water heater.
% (o) Replacing dated interior woodwork.
  
  \newstatute{MCL 211.10d(7)}{}%
%  (7) Every lawful assessment roll shall have a certificate attached signed by the certified assessor who prepared or supervised the preparation of the roll. The certificate shall be in the form prescribed by the state tax commission. If after completing the assessment roll the certified assessor for the local assessing district dies or otherwise becomes incapable of certifying the assessment roll, the county equalization director or the state tax commission shall certify the completed assessment roll at no cost to the local assessing district.


  
  \newstatute{MCL 205.753(2)}{}% allows appeals from a final order of the Tax Tribunal

  \newstatute{MCR 7.204(A)(1)(b)}{}% allows appeals within 21 days of an order on a motion for reconsideration
  
}% end makeatletter block
% \def\mathieuGast{\pincite[l]{MCL}{211.27(2)}}
\def\mathieuGast{\cite[s]{MCL 211.27(2)}}%
%\def\ttr287{\pincite[s]{TTR}{287}}
%\def\inspectionOrdinance{\pincite{Wayne Ordinance}{\S1484.04}}
% \long\def\inspectionOrdinanceText{\begin{quote}
% 1484.04  CERTIFICATE REQUIRED PRIOR TO SALE. 
%    It shall be unlawful to sell, convey or transfer an ownership interest, or act as a broker or agent for the sale, conveyance or transfer of an ownership interest, in any residential dwelling unless and until a valid Certificate of Compliance is first issued. 
% (Ord. 1991-10.  Passed 7-16-91.) 
% \end{quote}}


%\newmisc{STC Bulletin 6 of 2007}{Michigan State Tax Commission (STC) Bulletin No. 6 of 2007 (Foreclosure Guidelines)}
\newmisc{STC Bulletin}{Michigan State Tax Commission (STC) Bulletin No. 7 of 2014 (Mathieu Gast Act)}
\addReference{STC Bulletin}{bulletin7}% Associate appendix with case

\newcommand{\makeAbbreviation}[3]{% ensure that the frsit time an abbreviated word is used, it is presented in long form, and after that in short form. 1: command name, 2: short name, 3: long name
  \IfBeginWith{#3}{#2}{%
    \newcommand{#1}[0]{#3\renewcommand{#1}[0]{#2}}}{%
    \newcommand{#1}[0]{#3 (#2)\renewcommand{#1}[0]{#2}}}}

\makeAbbreviation{\MLS}{MLS}{Multiple Listing Service}
\makeAbbreviation{\MTT}{MTT}{Michigan Tax Tribunal}
\makeAbbreviation{\STC}{STC}{State Tax Commission}
%\makeAbbreviation{\FOJ}{FOJ}{First Final Opinion and Judgment (2017)}
% \makeAbbreviation{\explanatoryLetterAbbr}{Explanatory Letter}{Explanatory Letter submitted by Appellant to the Tax Tribunal on 9/6/2018}
% \newcommand{\explanatoryLetter}[1][]{\explanatoryLetterAbbr\if\relax#1\relax\empty, Appendix at \pageref{explanatoryLetter}\else, page #1, Appendix at \pageref{explanatoryLetter.#1}\fi}

% Note that the command/handle must match the appendix
% labels. Minimize the variations of names.  \makeAbbreviationToRecord
% creates a simple abbreviation that ends in Abbr if you don't want to
% refer to the record.
\newcommand{\makeAbbreviationToRecord}[3]{% 1: command/handle, 2: shortname, 3: longname
  % makeAbbreviationToRecord: #1, #2, #3\par%
  \expandafter\makeAbbreviation\csname #1Abbr\endcsname{#2}{#3}%
  \expandafter\newcommand\csname #1\endcsname[1][]{%
    \Call{#1Abbr}%
    \if\relax##1\relax\empty\ (Appendix at \pageref{#1})%
    \else, p ##1 (Appendix at %
    % Check for page range
    \IfSubStr{##1}{-}{%
      \def\pageRefRange####1-####2XXX{\pageref{#1.####1}--\pageref{#1.####2}}%
      \pageRefRange##1XXX}%
    {\pageref{#1.##1}}%
    )\fi%
  }%
}%

\makeAbbreviationToRecord{explanatoryLetter}{Explanatory Letter}{Explanatory Letter (MTT Docket Line 38)}
% explanatory Letter: (abbr) \explanatoryLetterAbbr\ (to record) \explanatoryLetter[2] \par

\makeAbbreviationToRecord{foj}{FOJ}{Second Final Opinion and Judgment (MTT Docket Line 48)}
% FOJ: (abbr) \fojAbbr (to record) \foj[]\par

\makeAbbreviationToRecord{reconsiderationDenied}{Order Denying Reconsideration}{Order Denying Reconsideration (MTT Docket Line 51)}
% reconsiderationDenied: (abbr) \reconsiderationDeniedAbbr[] (to record) \reconsiderationDenied[] \par

\makeAbbreviationToRecord{repairs}{List of Repairs}{List of Repairs (MTT Docket Line 36)}
% \par repairs: (abbr) \repairsAbbr\ (to record) \repairs[] (to appendix) \pageref{repairs}\par

\makeAbbreviationToRecord{stcform}{STC Form 865}{STC Form 865 Request for Nonconsideration (MTT Docket Line 35)}

\makeAbbreviationToRecord{mlsListing}{MLS Listing}{MLS Listing (MTT Docket Line 32)}
% mlsListing: (abbr) \mlsListingAbbr\ (to record) \mlsListing[]\par

\makeAbbreviationToRecord{mlsHistory}{MLS History}{MLS History (MTT Docket Line 33)}
% mlsHistory: (abbr) \mlsHistoryAbbr\ (to record) \mlsHistory[]\par

\makeAbbreviationToRecord{boardOfReviewDecision}{Board of Review Decision}{Board of Review Decision (MTT Docket Line 2)}

\makeAbbreviationToRecord{cityEvidence}{City's Evidence}{City's Evidence (MTT Docket Line 11)}

\makeAbbreviationToRecord{motionForReconsideration}{Motion for Reconsideration}{Motion for Reconsideration (MTT Docket Line 52)}

\begin{centering}
\bf\scshape State of Michigan\\In the Court of Appeals\\Detroit Office\\~\\ 
\rm 

\makeandtab
\setlength{\tabcolsep}{20pt}%
\begin{tabular}{p{.4\textwidth} p{.4\textwidth}}
\cline{1-2}
  {~

  \raggedright Daniel Patru,\par
  \hfill\textit{Petitioner/Appellant,}
  \vspace{.5\baselineskip}\par
  vs\par
  \vspace{.5\baselineskip}
  \raggedright City of Wayne,\par
  \hfill\textit{Respondent/Appellee.}
  
  ~} &  {~
       \par\par
       \hfill Court of Appeals No. 346894\par
       \hfill Lower Court No. 16-001828-TT\par\vspace{\baselineskip}
       \hfill \textbf{Appellant's Reply Brief}\par
       \hfill \textbf{Proof of Service}       
  ~}
  \\ \cline{1-2}\vspace{2mm}
  {~ \par
  Daniel Patru, P74387, Appellant\newline%
  3309 Solway\newline%
  Knoxville, TN 37931\newline%
  (734) 274-9624\newline%
  dpatru@gmail.com\newline\newline%
  ~} & {~ \par~\par
       
       City of Wayne, Appellee\newline%
       3355 South Wayne Rd,\newline%
       Wayne, MI 48184\newline%
       (734) 722-2000\newline\newline%

       Stephen J. Hitchcock (P15005)\newline%
       John C. Clark (P51356)\newline%
       Attorneys for Respondent/Appellee\newline%
       Giarmarco, Mullins \& Horton, P.C.\newline%
       101 W. Big Beaver Road, 10th floor\newline%
       Troy, MI 48084\newline%
       (248) 457-7024\newline%
       sjh@gmhlaw.com
  ~}
\end{tabular}
\makeandletter
\par\vspace{\baselineskip}\vspace{\baselineskip}\vspace{\baselineskip}
%\textbf{ORAL ARGUMENT NOT REQUESTED}

\end{centering}

\pagestyle{romanparen}
\pagenumbering{roman}
\newpage 

\section*{Table of Contents}

\tableofcontents

\newpage
\tableofauthorities

\pagestyle{plain}
\pagenumbering{arabic}

%This commands creates the title page, table of contents, and table of authorities
% \makefrontmatter{Brief\\Proof of Service}


%Sets the formatting for the entire document after the front matter
\parindent=2.5em
% \setlength{\parskip}{1.25ex plus 2ex minus .5ex} 
% \setstretch{1.45}
\doublespacing
% \linenumbers


\section{Mathieu-Gast Statute -- MCL 211.27(2)}
\begin{quotation}
The assessor shall not consider the increase in true cash value that is a result of expenditures for normal repairs, replacement, and maintenance in determining the true cash value of property for assessment purposes until the property is sold.

For the purpose of implementing this subsection, the assessor shall not increase the construction quality classification or reduce the effective age for depreciation purposes, except if the appraisal of the property was erroneous before nonconsideration of the normal repair, replacement, or maintenance, and shall not assign an economic condition factor to the property that differs from the economic condition factor assigned to similar properties as defined by appraisal procedures applied in the jurisdiction.

The increase in value attributable to the items included in subdivisions (a) to (o) that is known to the assessor and excluded from true cash value shall be indicated on the assessment roll.

This subsection applies only to residential property.

The following repairs are considered normal maintenance if they are not part of a structural addition or completion: [repairs (a)-(o) omitted]

% (a) Outside painting.
% (b) Repairing or replacing siding, roof, porches, steps, sidewalks, or drives.
% (c) Repainting, repairing, or replacing existing masonry.
% (d) Replacing awnings.
% (e) Adding or replacing gutters and downspouts.
% (f) Replacing storm windows or doors.
% (g) Insulating or weatherstripping.
% (h) Complete rewiring.
% (i) Replacing plumbing and light fixtures.
% (j) Replacing a furnace with a new furnace of the same type or replacing an oil or gas burner.
% (k) Repairing plaster, inside painting, or other redecorating.
% (l) New ceiling, wall, or floor surfacing.
% (m) Removing partitions to enlarge rooms.
% (n) Replacing an automatic hot water heater.
% (o) Replacing dated interior woodwork.
\end{quotation}

\section{Argument}

\subsection{Appellee's brief does not address Mathieu-Gast nonconsideration}

Appellee's brief does not address the main issues of the case, whether the Tribunal erred in regarding Mathieu-Gast nonconsideration (\mathieuGast).

\mathieuGast\ requires nonconsideration treatment for normal repairs. Nonconconsideration treatment involves three steps:
\begin{enumerate}
\item determining the value of the repairs using before-repair and after-repair appraisals (``[T]he true cash value of the item shall be calculated by performing `before' and `after' appraisals and then deducting the `before' true cash value from the `after' true cash value.'' \pincite{STC Bulletin}{2}; ``The assessor is required to estimate the true cash value of the property both before and after the expenditures.'' \stcform[2]),

\item excluding the value of the value of the repairs from the true cash value for assessment purposes (The assessor shall not consider the increase in true cash value that is a result of expenditures for normal repairs, replacement, and maintenance in determining the true cash value of property for assessment purposes until the property is sold.
\mathieuGast, first sentence), and

\item indicating the value of the repairs in the assessment roll (The increase in value attributable to the items included in subdivisions (a) to (o) that is known to the assessor and excluded from true cash value shall be indicated on the assessment roll. \mathieuGast, third sentence).
\end{enumerate}

Except for determining the after-repair value, the Tribunal did none of these things and most of Appellant's brief is devoted to explaining why the Tribunal erred by not doing them. Appellee's brief does not add anything to the discussion, either explicitly (it does not mention \mathieuGast\ at all other than in the Statement of Facts) nor implicitly.

Instead Appellee in its second issue asserts an uncontested fact: that the Tribunal's determination of the after-repair value is correct. Appellee correctly notes on page one of its brief that Appellant does not disagree: ``Appellant acknowledges that he does not dispute the house is worth the assessed TCV of \$50,400 on tax day December 31, 2015."

Appellee's other issue, that Appellant failed to satisfy his burden of proof, is likewise unhelpful to this Court. Appellant has pointed out in his brief (Argument, section IIB, pages 15-18) that this Court has ruled that even if the Tribunal disagrees with Petitioner's proofs, it is still required to independently determine the true cash value, here by applying Mathieu-Gast nonconsideration. Appellee has not contradicted this or even addressed the issue. Thus, even if Appellee were correct, that Appellant has not satisfied his burden of proof, this does not excuse the Tribunal from applying Mathieu-Gast nonconsideration. (Note that Appellant does not concede that his proofs were not sufficient, only that if they were insufficient, the Tribunal must still apply nonconsideration.)

Thus Appellee does not provide this Court with any arguments as to why Appellee is wrong to insist that the Tribunal must apply \mathieuGast\ nonconsideration treatment.

\subsection{Appellee's brief does not justify the Tribunal's cursory rejection of the subject's sale}

Besides issues concerning the interpretation and application of Mathieu-Gast nonconsideration, the other issue raised by Appellant was the Tribunal's cursory rejection of the subject's sale. The Tribunal ruled that the sale ``was entitled to no weight or credibility'' because:

\begin{enumerate}
\item ``The selling price of a property is not its presumptive true cash value'',
\item ``the home was being sold by the U.S. Department of Housing and Urban Development'', and
\item speculation that ``[b]ecause the subject was being sold by a government entity, that entity's motivation may not have been to receive market value for the property.''
\end{enumerate}
\reconsiderationDenied[2].

Appellee's brief does not address Appellant's argument regarding this issues head on. In particular, it does not seek to distinguish this case from \cite{Jones & Laughlin}\ where this Court ruled that it is an error as a matter of law to cursorily reject the sale of the subject property:

\begin{quote}
  The Tax Tribunal, however, erred as a matter of law in its treatment of petitioner's evidence regarding the sale. The tribunal held: "A sale that occurs after the tax date has little or no bearing on the assessment made
prior to the sale." (Emphasis in original.) We disagree. Unlike some situations involving assessments of industrial property for which no ready market exists and a hypothetical buyer must be posited, in this case the equipment was actually sold in a commercial transaction, albeit after the tax date. We believe that evidence of the price at which an item of property actually sold is most certainly relevant evidence of its value at an earlier time within the meaning of the term "relevant evidence." MRE 401. \pincite{Jones & Laughlin}{353-354}.
\end{quote}

In its subsection entitled ``The Subject's Purchase Price Is Not a
Reliable Indicator of Value Due to the Home at the Time of Purchase,
it Being a Bank Sale'', page 4, Appellee's brief gives two quotes without additional explanation from the \foj[5]. The first quote
(``Petitioner's property assessment did not change by virtue of the
repairs Petitioner made to subject property. To the contrary, the
subject property's assessment for 2016 changed based on the sale
transaction of the subject property in August 2015.'') seems to be a discussion of uncapping and is irrelevant to whether the subject property sold for its market value.

The second quote (``Petitioner's MLS print-out information for the subject and its comparable sales illustrate properties in ``average'' condition.
Specifically, the subject and comparable sales' interior photographs
do not depict neglected or vandalized properties.'') is about the condition of the property as depicted in the MLS listing.
The FOJ is confusing here because it packs into one paragraph four reasons for denying ``Petitioner's contentions related to his purchase price and 'normal' repairs.'' Appellant addressed these issues individually in the \motionForReconsideration{4-5}. The Tribunal clarified its reasoning in the denial of the motion. The Tribunal made clear that it was rejecting the subject's sale based on the fact that the seller was HUD:

\begin{quote}
Further, the Tribunal cannot conclude that the FOJ erred when it
concluded that Petitioner's contentions concerning the purchase price, i.e. the true cash value before repairs, was entitled to no weight or credibility. The selling price of a property is not is presumptive true cash value. Despite Petitioner's assertions that the marketing efforts for the subject show that the sale was a ``market sale,'' the home was
being sold by the U.S. Department of Housing and Urban Development (``HUD''). Because the subject was being sold by a government entity, that entity's motivation may not have been to receive market value for the property. \reconsiderationDenied[2]\ (cleaned up).
\end{quote}

The Tribunal considered the MLS pictures as evidence of a different point, that the assessor could have assessed the property without considering the repairs:

\begin{quote}
  The Tribunal also concludes that it was not a palpable error to conclude that the assessment did not consider the ``normal repairs.'' This is supported by the fact that the property record card indicates that Respondent believed the true cash value of the subject to be \$48,000 before Petitioner purchased the property and \$50,400 as of December 31, 2015, after the repairs were complete. An increase of \$2,400 in true cash value (5\%) is easily attributable to inflation and increases in the market. {\em In addition, as stated by the FOJ, the interior photographs depict a property in average condition before Petitioner acquired it.} Although not necessarily evidence of true cash value, this evidence supports the property's assessment as a property in average condition both at the time Petitioner acquired it and after he completed the normal repairs. In other words, the record evidence supports the conclusion that the assessment did not consider the increase in true cash value that was the result of normal repairs. \reconsiderationDenied[2]\ (cleaned up, emphasis added).
\end{quote}

Thus the MLS pictures were not used by the Tribunal to reject the sale of the subject. Instead, the pictures were used to support the Tribunal's theory that the assessor did not violate \mathieuGast\ because she thought the property did not need repairs. (Appellant's brief, Argument section IA and IB at 7-13 rebuts this theory.)

Not only did the Tribunal not actually reject the subject's sale based on the MLS pictures, but the evidence in this case does not support the implication that the property did not need repairs when it sold. Prior to the Tribunal's suggestion in the second FOJ, the consensus among Petitioner, Respondent, and the Tribunal was, as this Court summarized in its first ruling on this case, ``It is undisputed that, when he purchased the property, it was in substandard condition and required numerous repairs to make it livable.'' \pincite{Patru}{1}. (See Appellant's brief Argument section IB at 12-13 for more argument on this point.)

Therefore, Appellee's brief does not justify the Tribunal's rejection of the subject sale.

\section{Relief Requested}

Applicant respectfully asks this Court to reverse the ruling of the Tribunal. 

\section{Proof of Service}

On 4/23/2019, I served a copy of this Brief on Appellee's counsel by electronic service.

\vspace{1\baselineskip}

{ \setlength{\leftskip}{3.5in}

  Respectfully Submitted,

  /s/ Daniel Patru, P74387

  4/23/2019

  \setlength{\leftskip}{0pt}}

\newpage\empty% we need a new page so that the index entries on the last
        % page get written out to the right file.
\end{document}


%%% Local Variables:
%%% mode: latex
%%% TeX-master: t
%%% End:
