%
% This is one of the samples from the lawtex package:
% http://lawtex.sourceforge.net/
% LawTeX is licensed under the GNU General Public License 
%
\providecommand{\documentclassflag}{}
\documentclass[12pt,\documentclassflag]{michiganCourtOfAppealsBrief}
%\documentclass{article}

% for striking a row in a table, see https://tex.stackexchange.com/a/265728/135718
\usepackage{tikz}
\usetikzlibrary{tikzmark}

\makeandletter% use \makeandtab to turn off

% Use this to show a line grid-
% \usepackage[fontsize=12pt,baseline=24pt,lines=27]{grid}
% \usepackage{atbegshi,picture,xcolor} % https://tex.stackexchange.com/a/191004/135718
% \AtBeginShipout{%
%   \AtBeginShipoutUpperLeft{%
%     {\color{red}%
%     \put(\dimexpr -1in-\oddsidemargin,%
%          -\dimexpr 1in+\topmargin+\headheight+\headsep+\topskip)%
%       {%
%        \vtop to\dimexpr\vsize+\baselineskip{%
%          \hrule%
%          \leaders\vbox to\baselineskip{\hrule width\hsize\vfill}\vfill%
%        }%
%       }%
%   }}%
% }
%   \linespread{1}

\usepackage[modulo]{lineno}% use \linenumbers to show line numbers, see https://texblog.org/2012/02/08/adding-line-numbers-to-documents/

% from https://tex.stackexchange.com/a/313337/135718
\usepackage{enumitem,amssymb}
\newlist{todolist}{itemize}{2}
\setlist[todolist]{label=$\square$}
\usepackage{pifont}
\newcommand{\cmark}{\ding{51}}%
\newcommand{\xmark}{\ding{55}}%
\newcommand{\done}{\rlap{$\square$}{\raisebox{2pt}{\large\hspace{1pt}\cmark}}%
\hspace{-2.5pt}}
\newcommand{\wontfix}{\rlap{$\square$}{\large\hspace{1pt}\xmark}}

% allow underscores in words
\chardef\_=`_% https://tex.stackexchange.com/a/301984/135718 

%%Citations
 
%The command \makeandletter turns the ampersand into a printable character, rather than a special alignment tab \makeandletter

\begin{document}
\singlespacing%

\citecase[Antisdale]{Antisdale v City of Galesburg, 420 Mich 265, 362 NW2d 632 (1984)}
\citecase[Berger]{Berger v. Berger, 747 N.W.2d 336; 277 Mich. App. 700 (2008)}
\citecase[Briggs]{Briggs Tax Service, LLC v Detroit Pub. Schools, 485 Mich 69; 780 NW2d 753 (2010)}
\citecase[CAF]{CAF Investment Company v. State Tax Commission, 392 Mich. 442; 221 N.W.2d 588 (1974)}
\citecase[Clark]{Clark Equipment Company v. Township Of Leoni, 113 Mich. App. 778; 318 N.W.2d 586 (1982)}
\citecase[Kar]{Kar v. Hogan, 399 Mich. 529; 251 N.W.2d 77  (1976)}
\citecase[Jones & Laughlin]{Jones & Laughlin Steel Corporation v. City of Warren, 193 Mich App 348; 483 NW2nd 416 (1992)}
\citecase[Menard]{Menard, Inc. v Escanaba, 315 Mich. App. 512; 891 N.W.2d 1 (2016)}
\citecase[Mich Ed]{Mich Ed Ass'n v Secretary of State
  (On Rehearing), 489 Mich 194, 218; 801 NW2d 35 (2011)}
% (stating that nothing will be read into a clear statute that is not within the manifest intention of the Legislature as derived from the language of the statute itself).

\citecase[Patru 1]{Patru v City of Wayne, unpublished per curiam opinion of the Court of Appeals, issued May 8, 2018 (Docket No. 337547)}%
\addReference{Patru 1}{patruvwayne}% Associate appendix with case
\citecase[Patru 2]{Patru v City of Wayne, unpublished per curiam opinion of the Court of Appeals, issued February 18, 2020 (Docket No. 346894)}%
\addReference{Patru 2}{patruvwayne}% Associate appendix with case
\citecase[Pontiac Country Club]{Pontiac Country Club v Waterford Twp, 299 Mich App 427; 830 NW2d 785 (2013)}
\citecase[Luckow]{Luckow Estate v Luckow, 291 Mich App 417; 805 NW2d 453 (2011)}
\citecase[Stamp]{Stamp v Mill Street Inn, 152 Mich App 290; 393 NW2d 614 (1986)}
\citecase[Macomb County Department of Human Services]{Macomb County Department of Human Services v Anderson, 304 Mich App 750; 849 NW2d 408 (2014)}
\citecase[Walters]{People v Walters, 266 Mich App 341; 700 NW2d 424 (2005)}

{\makeatletter % needed for optional argument to newstatute.
  % \newstatute[1@]{MCL}{}% place MCL first
  % \newstatute[2@]{MCR}{}
  % \newstatute[3@]{TTR}{}
  % \newstatute[4@]{Dearborn Ordinance}{}% place this fourth
  % \newstatute[5@]{Wayne Ordinance}{}
  \newstatute{MCL 211.27(2)}{}% MathieuGast
%   (2) The assessor shall not consider the increase in true cash value that is a result of expenditures for normal repairs, replacement, and maintenance in determining the true cash value of property for assessment purposes until the property is sold. For the purpose of implementing this subsection, the assessor shall not increase the construction quality classification or reduce the effective age for depreciation purposes, except if the appraisal of the property was erroneous before nonconsideration of the normal repair, replacement, or maintenance, and shall not assign an economic condition factor to the property that differs from the economic condition factor assigned to similar properties as defined by appraisal procedures applied in the jurisdiction. The increase in value attributable to the items included in subdivisions (a) to (o) that is known to the assessor and excluded from true cash value shall be indicated on the assessment roll. This subsection applies only to residential property. The following repairs are considered normal maintenance if they are not part of a structural addition or completion:
% (a) Outside painting.
% (b) Repairing or replacing siding, roof, porches, steps, sidewalks, or drives.
% (c) Repainting, repairing, or replacing existing masonry.
% (d) Replacing awnings.
% (e) Adding or replacing gutters and downspouts.
% (f) Replacing storm windows or doors.
% (g) Insulating or weatherstripping.
% (h) Complete rewiring.
% (i) Replacing plumbing and light fixtures.
% (j) Replacing a furnace with a new furnace of the same type or replacing an oil or gas burner.
% (k) Repairing plaster, inside painting, or other redecorating.
% (l) New ceiling, wall, or floor surfacing.
% (m) Removing partitions to enlarge rooms.
% (n) Replacing an automatic hot water heater.
% (o) Replacing dated interior woodwork.

  \newstatute{MCL 211.27(6)}{}%
% (6) Except as otherwise provided in subsection (7), the purchase price paid in a transfer of property is not the presumptive true cash value of the property transferred. In determining the true cash value of transferred property, an assessing officer shall assess that property using the same valuation method used to value all other property of that same classification in the assessing jurisdiction. As used in this subsection and subsection (7), "purchase price" means the total consideration agreed to in an arms-length transaction and not at a forced sale paid by the purchaser of the property, stated in dollars, whether or not paid in dollars.
  
  \newstatute{MCL 211.10d(7)}{}%
%  (7) Every lawful assessment roll shall have a certificate attached signed by the certified assessor who prepared or supervised the preparation of the roll. The certificate shall be in the form prescribed by the state tax commission. If after completing the assessment roll the certified assessor for the local assessing district dies or otherwise becomes incapable of certifying the assessment roll, the county equalization director or the state tax commission shall certify the completed assessment roll at no cost to the local assessing district.


  
  \newstatute{MCL 205.753(2)}{}% allows appeals from a final order of the Tax Tribunal

  \newstatute{MCR 7.204(A)(1)(b)}{}% allows appeals within 21 days of an order on a motion for reconsideration
  
} % end makeatletter block

\def\mathieuGast{\cite[s]{MCL 211.27(2)}}%

%\newmisc{STC Bulletin 6 of 2007}{Michigan State Tax Commission (STC) Bulletin No. 6 of 2007 (Foreclosure Guidelines)}
\newmisc{STC Bulletin}{Michigan State Tax Commission (STC) Bulletin No. 7 of 2014 (Mathieu Gast Act)}
\addReference{STC Bulletin}{bulletin7}% Associate appendix with case

\newbook{The Appraisal of Real Estate}{Appraisal Institute}{The Appraisal of Real Estate}{(14th ed, Chicago: Appraisal Institute, 2013)}
\newbook{Appraising the Tough Ones}{Harrison}{Appraising the Tough Ones : Creative Ways to Value Residential Properties}{(Chicago: Appraisal Institute, 1996)}
\newbook{McCormick}{McCormick}{Evidence}{(2d ed)}
\newbook{Assessor's Manual}{Michigan State Tax Commission}{\href{https://www.michigan.gov/documents/treasury/1.__2014_Michigan_Assessors_Manual_Volume_I_Introduction_575738_7.pdf}{Assessor's Manual}}{(Vol 1 Residential, 2014)} 




\newbook{Wex}{Legal Information Institute}{\href{https://www.law.cornell.edu/wex}{Wex}}{<https://www.law.cornell.edu/wex> (accessed October 7, 2019)}

\newmisc{Respondent's Evidence}{Respondent's Evidence (entry line 16)}
\SetIndexType{Respondent's Evidence}{}
\newmisc{Chart of Median Price}{Chart of Ann Arbor Township Median Price Per Square Foot, Petitioner's Evidence (entry line 15)}
\SetIndexType{Chart of Median Price}{}
\newmisc{June 24 letter}{Petitioner's June 24 letter (entry line 11)}
\SetIndexType{June 24 letter}{}
\newmisc{July 10 letter}{Petitioner's July 10 letter (entry line 21)}
\SetIndexType{July 10 letter}{}
\newmisc{Form 865}{STC Form 865 (Mathieu-Gast Nonconsideration)}
\newmisc{Affidavit}{Affidavit of Carolyn Lepard (entry line 10)}
\SetIndexType{Affidavit}{}

\newmisc{POJ}{Proposed Opinion and Judgment (POJ)}
\SetIndexType{POJ}{}
\newmisc{Exceptions}{Exceptions (entry line 25)}
\SetIndexType{Exceptions}{}
\newmisc{FOJ}{Final Opinion and Judgment (FOJ)}
\SetIndexType{FOJ}{}
\newcommand{\makeAbbreviation}[3]{% ensure that the frsit time an abbreviated word is used, it is presented in long form, and after that in short form. 1: command name, 2: short name, 3: long name
  \IfBeginWith{#3}{#2}{%
    \newcommand{#1}[0]{#3\renewcommand{#1}[0]{#2}}}{%
    \newcommand{#1}[0]{#3 (#2)\renewcommand{#1}[0]{#2}}}}

\makeAbbreviation{\MLS}{MLS}{Multiple Listing Service}
\makeAbbreviation{\MTT}{MTT}{Michigan Tax Tribunal}
\makeAbbreviation{\STC}{STC}{State Tax Commission}
  
\newcommand{\makeAbbreviationToRecord}[3]{% 1: command/handle, 2: shortname, 3: longname
  % makeAbbreviationToRecord: #1, #2, #3\par%
  \expandafter\makeAbbreviation\csname #1Abbr\endcsname{#2}{#3}%
  \expandafter\newcommand\csname #1\endcsname[1][]{%
    \Call{#1Abbr}%
    \if\relax##1\relax\empty\ (Appendix at \pageref{#1})%
    \else, p ##1 (Appendix at %
    % Check for page range
    \IfSubStr{##1}{-}{%
      \def\pageRefRange####1-####2XXX{\pageref{#1.####1}--\pageref{#1.####2}}%
      \pageRefRange##1XXX}%
    {\pageref{#1.##1}}%
    )\fi%
  }%
}%

\makeAbbreviationToRecord{explanatoryLetter}{Explanatory Letter}{Explanatory Letter (MTT Docket Line 38)}
% explanatory Letter: (abbr) \explanatoryLetterAbbr\ (to record) \explanatoryLetter[2] \par

\makeAbbreviationToRecord{foj}{FOJ}{Second Final Opinion and Judgment (MTT Docket Line 48)}
% FOJ: (abbr) \fojAbbr (to record) \foj[]\par

\makeAbbreviationToRecord{reconsiderationDenied}{Order Denying Reconsideration}{Order Denying Reconsideration (MTT Docket Line 51)}
% reconsiderationDenied: (abbr) \reconsiderationDeniedAbbr[] (to record) \reconsiderationDenied[] \par

\makeAbbreviationToRecord{repairs}{List of Repairs}{List of Repairs (MTT Docket Line 36)}
% \par repairs: (abbr) \repairsAbbr\ (to record) \repairs[] (to appendix) \pageref{repairs}\par

\makeAbbreviationToRecord{stcform}{STC Form 865}{STC Form 865 Request for Nonconsideration (MTT Docket Line 35)}

\makeAbbreviationToRecord{mlsListing}{MLS Listing}{MLS Listing (MTT Docket Line 32)}
% mlsListing: (abbr) \mlsListingAbbr\ (to record) \mlsListing[]\par

\makeAbbreviationToRecord{mlsHistory}{MLS History}{MLS History (MTT Docket Line 33)}
% mlsHistory: (abbr) \mlsHistoryAbbr\ (to record) \mlsHistory[]\par

\makeAbbreviationToRecord{boardOfReviewDecision}{Board of Review Decision}{Board of Review Decision (MTT Docket Line 2)}

\makeAbbreviationToRecord{cityEvidence}{City's Evidence}{City's Evidence (MTT Docket Line 11)}

\makeAbbreviationToRecord{motionForReconsideration}{Motion for Reconsideration}{Motion for Reconsideration (MTT Docket Line 52)}
%  \newstatute{MCL 211.10d(7)}{}%
%  (7) Every lawful assessment roll shall have a certificate attached signed by the certified assessor who prepared or supervised the preparation of the roll. The certificate shall be in the form prescribed by the state tax commission. If after completing the assessment roll the certified assessor for the local assessing district dies or otherwise becomes incapable of certifying the assessment roll, the county equalization director or the state tax commission shall certify the completed assessment roll at no cost to the local assessing district.


  
  % \newstatute{MCL 205.753(2)}{}% allows appeals from a final order of the Tax Tribunal

  % \newstatute{MCR 7.204(A)(1)(b)}{}% allows appeals within 21 days of an order on a motion for reconsideration
  



%\newmisc{STC Bulletin 6 of 2007}{Michigan State Tax Commission (STC) Bulletin No. 6 of 2007 (Foreclosure Guidelines)}
% \newmisc{STC Bulletin}{Michigan State Tax Commission (STC) Bulletin No. 7 of 2014 (Mathieu Gast Act)}
% \addReference{STC Bulletin}{bulletin7}% Associate appendix with case

% \newcommand{\makeAbbreviation}[3]{% ensure that the frsit time an abbreviated word is used, it is presented in long form, and after that in short form. 1: command name, 2: short name, 3: long name
%   \IfBeginWith{#3}{#2}{%
%     \newcommand{#1}[0]{#3\renewcommand{#1}[0]{#2}}}{%
%     \newcommand{#1}[0]{#3 (#2)\renewcommand{#1}[0]{#2}}}}

% \makeAbbreviation{\MLS}{MLS}{Multiple Listing Service}
% \makeAbbreviation{\MTT}{MTT}{Michigan Tax Tribunal}
% \makeAbbreviation{\STC}{STC}{State Tax Commission}
%\makeAbbreviation{\FOJ}{FOJ}{First Final Opinion and Judgment (2017)}
% \makeAbbreviation{\explanatoryLetterAbbr}{Explanatory Letter}{Explanatory Letter submitted by Appellant to the Tax Tribunal on 9/6/2018}
% \newcommand{\explanatoryLetter}[1][]{\explanatoryLetterAbbr\if\relax#1\relax\empty, Appendix at \pageref{explanatoryLetter}\else, page #1, Appendix at \pageref{explanatoryLetter.#1}\fi}

% Note that the command/handle must match the appendix
% labels. Minimize the variations of names.  \makeAbbreviationToRecord
% creates a simple abbreviation that ends in Abbr if you don't want to
% refer to the record.
% \newcommand{\makeAbbreviationToRecord}[3]{% 1: command/handle, 2: shortname, 3: longname
%   % makeAbbreviationToRecord: #1, #2, #3\par%
%   \expandafter\makeAbbreviation\csname #1Abbr\endcsname{#2}{#3}%
%   \expandafter\newcommand\csname #1\endcsname[1][]{%
%     \Call{#1Abbr}%
%     \if\relax##1\relax\empty\ (Appendix at \pageref{#1})%
%     \else, p ##1 (Appendix at %
%     % Check for page range
%     \IfSubStr{##1}{-}{%
%       \def\pageRefRange####1-####2XXX{\pageref{#1.####1}--\pageref{#1.####2}}%
%       \pageRefRange##1XXX}%
%     {\pageref{#1.##1}}%
%     )\fi%
%   }%
% }%

% \makeAbbreviationToRecord{explanatoryLetter}{Explanatory Letter}{Explanatory Letter (MTT Docket Line 38)}
% % explanatory Letter: (abbr) \explanatoryLetterAbbr\ (to record) \explanatoryLetter[2] \par

% \makeAbbreviationToRecord{foj}{FOJ}{Second Final Opinion and Judgment (MTT Docket Line 48)}
% % FOJ: (abbr) \fojAbbr (to record) \foj[]\par

% \makeAbbreviationToRecord{reconsiderationDenied}{Order Denying Reconsideration}{Order Denying Reconsideration (MTT Docket Line 51)}
% % reconsiderationDenied: (abbr) \reconsiderationDeniedAbbr[] (to record) \reconsiderationDenied[] \par

% \makeAbbreviationToRecord{repairs}{List of Repairs}{List of Repairs (MTT Docket Line 36)}
% % \par repairs: (abbr) \repairsAbbr\ (to record) \repairs[] (to appendix) \pageref{repairs}\par

% \makeAbbreviationToRecord{stcform}{STC Form 865}{STC Form 865 Request for Nonconsideration (MTT Docket Line 35)}

% \makeAbbreviationToRecord{mlsListing}{MLS Listing}{MLS Listing (MTT Docket Line 32)}
% % mlsListing: (abbr) \mlsListingAbbr\ (to record) \mlsListing[]\par

% \makeAbbreviationToRecord{mlsHistory}{MLS History}{MLS History (MTT Docket Line 33)}
% % mlsHistory: (abbr) \mlsHistoryAbbr\ (to record) \mlsHistory[]\par

% \makeAbbreviationToRecord{boardOfReviewDecision}{Board of Review Decision}{Board of Review Decision (MTT Docket Line 2)}

% \makeAbbreviationToRecord{cityEvidence}{City's Evidence}{City's Evidence (MTT Docket Line 11)}

% \makeAbbreviationToRecord{motionForReconsideration}{Motion for Reconsideration}{Motion for Reconsideration (MTT Docket Line 52)}

% \makeAbbreviationToRecord{explanatoryLetter}{Explanatory Letter}{Explanatory Letter (MTT Docket Line 38)}
% % explanatory Letter: (abbr) \explanatoryLetterAbbr\ (to record) \explanatoryLetter[2] \par

% \makeAbbreviationToRecord{foj}{FOJ}{Second Final Opinion and Judgment (MTT Docket Line 48)}
% % FOJ: (abbr) \fojAbbr (to record) \foj[]\par

% \makeAbbreviationToRecord{reconsiderationDenied}{Order Denying Reconsideration}{Order Denying Reconsideration (MTT Docket Line 51)}
% % reconsiderationDenied: (abbr) \reconsiderationDeniedAbbr[] (to record) \reconsiderationDenied[] \par

% \makeAbbreviationToRecord{repairs}{List of Repairs}{List of Repairs (MTT Docket Line 36)}
% % \par repairs: (abbr) \repairsAbbr\ (to record) \repairs[] (to appendix) \pageref{repairs}\par

% \makeAbbreviationToRecord{stcform}{STC Form 865}{STC Form 865 Request for Nonconsideration (MTT Docket Line 35)}

% \makeAbbreviationToRecord{mlsListing}{MLS Listing}{MLS Listing (MTT Docket Line 32)}
% % mlsListing: (abbr) \mlsListingAbbr\ (to record) \mlsListing[]\par

% \makeAbbreviationToRecord{mlsHistory}{MLS History}{MLS History (MTT Docket Line 33)}
% % mlsHistory: (abbr) \mlsHistoryAbbr\ (to record) \mlsHistory[]\par

% \makeAbbreviationToRecord{boardOfReviewDecision}{Board of Review Decision}{Board of Review Decision (MTT Docket Line 2)}

% \makeAbbreviationToRecord{cityEvidence}{City's Evidence}{City's Evidence (MTT Docket Line 11)}

% \makeAbbreviationToRecord{motionForReconsideration}{Motion for Reconsideration}{Motion for Reconsideration (MTT Docket Line 52)}


\begin{centering}
\bf\scshape State of Michigan\\In the Court of Appeals\\Detroit Office\\~\\ 
\rm 

\makeandtab
\setlength{\tabcolsep}{20pt}%
\begin{tabular}{p{.4\textwidth} p{.4\textwidth}}
\cline{1-2}
  {~

  \raggedright Daniel Patru,\par
  \hfill\textit{Petitioner/Appellant,}
  \vspace{.5\baselineskip}\par
  vs\par
  \vspace{.5\baselineskip}
  \raggedright City of Wayne,\par
  \hfill\textit{Respondent/Appellee.}
  
  ~} &  {~
       \par\par
       \hfill Court of Appeals No. 346894\par
       \hfill Lower Court No. 16-001828-TT\par\vspace{\baselineskip}

       \hfill \raggedleft\textbf{Motion and Brief for Reconsideration}\vspace{.5\baselineskip}\par
       \hfill \textbf{Proof of Service}\newline      
  ~}
  \\ \cline{1-2}\vspace{2mm}
  % {~ \par
  % Andre Haerian, Petitioner\newline
  % 390 Meadow Creek,\newline
  % Ann Arbor Twp, MI 48105\newline \newline
  
  % Daniel Patru, P74387, \newline%
  % 3309 Solway\newline%
  % Knoxville, TN 37931\newline%
  % (734) 274-9624\newline%
  % dpatru@gmail.com\newline\newline%
  % ~} & {~ \par~\par
       
  %      Ann Arbor Charter Twp, Appellee\newline%
  %      3792 Pontiac Trail,\newline%
  %      Ann Arbor, MI 48105-9656\newline%
  %      (734) 663-3418\newline\newline%

  %      Emily Pizzo, Ann Arbor Twp Deputy Assessor\newline%
  %      3792 Pontiac Trail,\newline%
  %      Ann Arbor, MI 48105-9656\newline%
  % (734) 663-8540\newline
  % assessor@aatwp.org
       
  % ~}
\end{tabular}
\makeandletter
\par\vspace{\baselineskip}\vspace{\baselineskip}\vspace{\baselineskip}
%\textbf{ORAL ARGUMENT NOT REQUESTED}

\end{centering}

\pagestyle{romanparen}
\pagenumbering{roman}

\section{Checklist}
\begin{itemize}
  \item Prep
  \begin{todolist}
  \item[\done] Copy files in new directory
  \item[\done] Clean up files
  \item Read the rules, copying them and applying them
  \item[\done] Reconsideration in Court of Appeals.
      
  \end{todolist}

  \item Write
  \begin{todolist}
  \item Write solution
  \end{todolist}

  \item Proof
  \begin{todolist}
  \item A motion for reconsideration may be filed within 21 days after the date of the order or the date stamped on an opinion.
  \item %The motion shall include all facts, arguments, and citations to authorities in a single document and
    shall not exceed 10 double-spaced pages.
  \item A copy of the order or opinion of which reconsideration is sought must be included with the motion.
  \item Motions for reconsideration are subject to the restrictions contained in MCR 2.119(F)(3).

  \end{todolist}

  
\item example
  \begin{todolist}
  \item[\done] Frame the problem
  \item Write solution
  \item[\wontfix] profit
  \end{todolist}
\end{itemize}


\newpage 

\section*{Table of Contents}

\tableofcontents


\newpage
\tableofauthorities

\pagestyle{plain}
\pagenumbering{arabic}

%Sets the formatting for the entire document after the front matter
\parindent=2.5em
% \setlength{\parskip}{1.25ex plus 2ex minus .5ex} 
% \setstretch{1.45}
\doublespacing
% \linenumbers


% \section{Mathieu-Gast Statute -- MCL 211.27(2)}
% \begin{quotation}
% The assessor shall not consider the increase in true cash value that is a result of expenditures for normal repairs, replacement, and maintenance in determining the true cash value of property for assessment purposes until the property is sold.

% For the purpose of implementing this subsection, the assessor shall not increase the construction quality classification or reduce the effective age for depreciation purposes, except if the appraisal of the property was erroneous before nonconsideration of the normal repair, replacement, or maintenance, and shall not assign an economic condition factor to the property that differs from the economic condition factor assigned to similar properties as defined by appraisal procedures applied in the jurisdiction.

% The increase in value attributable to the items included in subdivisions (a) to (o) that is known to the assessor and excluded from true cash value shall be indicated on the assessment roll.

% This subsection applies only to residential property.

% The following repairs are considered normal maintenance if they are not part of a structural addition or completion: [repairs (a)-(o) omitted]

% % (a) Outside painting.
% % (b) Repairing or replacing siding, roof, porches, steps, sidewalks, or drives.
% % (c) Repainting, repairing, or replacing existing masonry.
% % (d) Replacing awnings.
% % (e) Adding or replacing gutters and downspouts.
% % (f) Replacing storm windows or doors.
% % (g) Insulating or weatherstripping.
% % (h) Complete rewiring.
% % (i) Replacing plumbing and light fixtures.
% % (j) Replacing a furnace with a new furnace of the same type or replacing an oil or gas burner.
% % (k) Repairing plaster, inside painting, or other redecorating.
% % (l) New ceiling, wall, or floor surfacing.
% % (m) Removing partitions to enlarge rooms.
% % (n) Replacing an automatic hot water heater.
% % (o) Replacing dated interior woodwork.
% \end{quotation}

\section{Motion}

% \begin{quotation}
%   Rule 7.215 Opinions, Orders, Judgments, and Final Process for Court of Appeals
% (I) Reconsideration.

% (1) A motion for reconsideration may be filed within 21 days after the date of the order or the date stamped on an opinion. The motion shall include all facts, arguments, and citations to authorities in a single document and shall not exceed 10 double-spaced pages. A copy of the order or opinion of which reconsideration is sought must be included with the motion. Motions for reconsideration are subject to the restrictions contained in MCR 2.119(F)(3).

% (2) A party may answer a motion for reconsideration within 14 days after the motion is served on the party. An answer to a motion for reconsideration shall be a single document and shall not exceed 7 double-spaced pages.

% (3) The clerk will not accept for filing a motion for reconsideration of an order denying a motion for reconsideration.

% (4) The clerk will not accept for filing a late motion for reconsideration.
% \end{quotation}

Petioner moves for reconsideration of the \cite{FOJ}\ under MCR 7.215(I).


\section{opinion}
This Court's decision is reasoned as follows:

\begin{enumerate}
  \item MCL 211.27(2) did not prohibit the assessor from considering the impact
of any ``normal repairs'' on the property's TCV for purposes of the 2016 tax year because there was a transfer of ownership of the
subject property in 2015. \pincite{Patru 2}{3}.

\item Initially, we agree with the Tax Tribunal that because there was a transfer of ownership of the
subject property in 2015, MCL 211.27(2) did not prohibit the assessor from considering the impact
of any ``normal repairs'' on the property's TCV for purposes of the 2016 tax year.
The taxable status of real property for a given tax year is ``determined as of each December
31 of the immediately preceding year.'' MCL 211.2(2). Generally, property is assessed at 50% of
its true cash value[.]'' MCL 211.27a(1). However, the ``taxable value'' of property is subject to
``capping,'' meaning it is limited to the property's taxable value in the immediately preceding year,
subject to certain allowable adjustments. MCL 211.27a(2)(a); Const 1963, art 9, \S 3.2
 However,
upon a transfer of ownership of property, the taxable value is ``uncapped,'' meaning ``the property's
taxable value for the calendar year following the year of the transfer is the property's state
equalized valuation for the calendar year following the transfer.'' MCL 211.27a(3). The taxable
value does not become capped again until the end of the calendar year following the transfer of
ownership. MCL 211.27a(4); Michigan Props, LLC v Meridian Twp, 491 Mich 518, 530; 817
NW2d 548 (2012). Thus, because petitioner purchased the property in 2015 (i.e., there was a
transfer of ownership), the property's taxable value for the 2016 tax year was to be determined by
its actual assessed value as of December 31, 2015, without regard to any capping limitations.
While MCL 211.27(2) does not expressly provide that it does not apply to ``normal repairs''
performed during a year when ownership of property is transferred (i.e., the taxable value becomes
uncapped), the statute must be read in conjunction with other provisions of the General Property

2
``[B]efore ownership of property is transferred, its taxable value may increase no more than the
lesser of the rate of inflation or five percent.'' Lyle Schmidt Farms, LLC v Mendon Twp, 315 Mich
App 824, 831; 891 NW2d 43 (2016). Once ownership of the property is transferred, its taxable
value is ``uncapped,'' see id., and its taxable value for the year following the transfer is determined
by the property's actual value, MCL 211.27a(3).
-4-
Tax Act, MCL 211.1 et seq., and Const 1963, art 9, \S 3. See Bloomfield Twp v Kane, 302 Mich
App 170, 176; 839 NW2d 505 (2013) (statutes that relate to the same matter are considered to be
in pari materia and must be read together as a whole). It is apparent that MCL 211.27(2) was
adopted to distinguish ``normal repairs'' or property maintenance from other improvements (e.g.,
``additions'') that increase the value of property as long as the property is owned by the same party.
In Toll Northville LTD v Twp of Northville, 480 Mich 6, 12; 743 NW2d 902 (2008), our Supreme
Court, quoting WPW Acquisition Co v City of Troy, 466 Mich 117, 121-122; 643 NW2d 692
(2001), observed that the purpose of Const 1963, art 9, \S 3, as amended in 1994 by Proposal A, is
to generally limit increases in property taxes on a parcel of property, as long as it
remains owned by the same party, by capping the amount that the ``taxable value''
of the property may increase each year , even if the ``true cash value,'' that is, the
actual market value, of the property rises at a greater rate. However, a qualification
is made to allow adjustments for ``additions.''
As indicated, however, when property is sold, it becomes ``uncapped'' and the property's value for
the following tax year is determined by its value as of December 31 of the preceding year. In other
words, the property's TCV for the tax year following a transfer of ownership is determined by its
value as of December 31 of the calendar year in which the transfer of ownership occurred.
Accordingly, the restriction on consideration of ``normal repairs'' for purposes of calculating
increases in TCV is intended to apply only while property is owned by the same party, and thus
would not apply to repairs performed during a year in which ownership of the property is
transferred. Rather, the taxable value of property for the tax year following a transfer of ownership
would be 50% of the property's TCV as of December 31 of the immediately preceding year (i.e.,
the year there was a transfer of ownership). MCL 211.2(2); MCL 211.27a(1) and (3).


\item This case is not governed by \cite{Patru 1}\ because ``this Court did not resolve whether petitioner's 2015 repairs to the property could or could not be considered in determining the property's TCV for the 2016 tax year, but instead
determined that ``further proceedings are necessary to determine whether the repairs were normal
repairs within the meaning of MCL 211.27(2).'' . . . More significantly, this Court
did not address the effect of the property's transfer of ownership in 2015 on the tribunal's
consideration of ``normal repairs'' under MCL 211.27(2) for purposes of the 2016 tax year.
Because this issue was not actually addressed and decided in the prior appeal, the law-of-the-case
doctrine does not apply.'' \pincite{Patru 2}{5}.

\item Petitioner further argues that the tribunal erred by also finding that, regardless of the proper
construction of MCL 211.27(2), petitioner's repairs did not have any bearing on the property's
TCV, which was determined to be \$50,400 as of December 31, 2015. We again disagree. The
tribunal did not credit petitioner's argument that the property was in substandard condition when
he purchased it. The tribunal reviewed petitioner's Multiple Listing Service (MLS) printouts and
photographs for both the subject property and comparable properties. The tribunal found that
petitioner's MLS listing for the subject property showed a property in ``average'' condition, and
that petitioner's photographs of the property, before any repairs, showed ``a property that is livable
and habitable with reasonable marketability and appeal.'' The tribunal noted that the purpose of
petitioner's repairs was ``to ready the property as a tenant rental.'' It is undisputed that the assessed
TCV of the property for the 2015 tax year was \$48,000. Given this evidence, it was appropriate
for the tribunal to draw conclusions about the value of the property before and after petitioner's
purchase, specifically, that the prior year's assessment of \$48,000 was reflective of the property's
TCV before petitioner purchased it, that an increase of \$2,400 could be attributed to inflation and 
increases in the market, and that petitioner's ``normal repairs'' were not attributable to the
property's substandard condition, but rather were intended primarily to prepare the property as
rental property. The tribunal concluded that the ``evidence supports the property's assessment as
a property in average condition both at the time Petitioner acquired it and after he completed the
normal repairs,''and that ``the assessment did not consider the increase in true cash value that was
the result of normal repairs.''

\item Petitioner also argues that the tribunal erred by failing to make its own independent
determination of the property's TCV. See Jones \& Laughlin Steel Corp v City of Warren, 193
Mich App 348, 354-356; 483 NW2d 416 (1992). The record does not support this argument. The
tribunal evaluated the evidence and proposed valuation methods offered by both parties. The
tribunal found that ``[r]espondent's sales comparison approach is the most reliable and credible
valuation evidence which also supports the assessment and 2016 uncapping of the subject
property.'' The tribunal did not just automatically accept respondent's valuation. Rather, it
analyzed respondent's sales-comparison data and found that ``a reasoned and reconciled
determination of market value is obtainable from Respondent's sales,'' which it concluded
``supports the assessment and 2016 uncapping of the subject property.''

\item Petitioner further argues that the tribunal erred by rejecting his 2015 purchase price of the
property as determinative of its TCV. MCL 211.27(6) provides:
Except as otherwise provided in subsection (7), the purchase price paid in a
transfer of property is not the presumptive true cash value of the property
transferred. In determining the true cash value of transferred property, an assessing
officer shall assess that property using the same valuation method used to value all
other property of that same classification in the assessing jurisdiction. As used in
this subsection and subsection (7), ``purchase price'' means the total consideration
agreed to in an arms-length transaction and not at a forced sale paid by the purchaser
of the property, stated in dollars, whether or not paid in dollars.
-7-
In this case, the tribunal considered petitioner's evidence of the 2015 purchase price for the
property, but it also considered the nature of the sale, which involved a bank sale in which the
grantor was the United States Department of Housing and Urban Development. The tribunal found
that the seller was not necessarily motivated to receive market value for the property and that the
property's purchase price was not presumptive of its TCV. The tribunal instead gave greater
weight to the evidence submitted by respondent in support of its sales-comparison approach to
valuation, which was based on sales of five properties similar in age and with comparable square
footage, style, siding, and condition. The tribunal found that respondent's evidence provided ``the
most reliable and credible valuation evidence'' and supported respondent's assessment of the
subject property. 

\end{enumerate}

\section{Issues}

\begin{enumerate}
\item Does the subject's sale comform to the sales comparision approach, one of the three recognized methods of valuation?
  
\item Does \cite[s]{MCL 211.27(6)}\ bar the Tribnual from using the subject's sale to set its true cash value?
  
\item Where the Tribunal based its valuation on the subject's sale along with two other comparable sales, did the Tribunal err when did not discount the subject's sale price because of market decline even though it discounted the comparables?

\item Did the Tribunal err in relying on admittedly flawed analysis to reach its valuation?
% \item
%   Must the Tribunal show how it arrived at its valuation beyond stating that it is between the parties' contentions?
\end{enumerate}  

\section{Facts}
\label{facts}
\subsection{Appellant buys the house, repairs it, and appeals its assessment}

Appellant purchased the subject house in August 2015 for \$32,000. The house was sold by the Department of Housing and Urban Development (HUD) through a real estate broker who had listed the house on the MLS. \mlsListing[]. At the time of purchase, the house needed numerous repairs, most of which were required by the City to obtain a Certificate of Occupancy. \repairs[]. By tax day, 12/31/2015, Appellant had repaired the house and rented it. \foj[4-5].

The City of Wayne assessed the house on tax day at \$50,400 true cash value, rather than the \$32,000 purchase price. \boardOfReviewDecision.

Appellant appealed to the Board of Review and then to the Tax Tribunal. Appellant does not dispute that the house was worth \$50,400 on tax day in its repaired condition. But he contends that under \mathieuGast\ the repairs were normal repairs and that the true cash value for assessment purposes cannot include the value of the repairs. He contends that the correct true cash value is therefore the before-repair value. \explanatoryLetter[2].

Appellant contends that the best evidence of the house's before-repair value is its sale price of \$32,000 when it was unrepaired. Id. The house was marketed in the normal way and for a sufficient time. Licensed real estate brokers listed the house on the MLS, initially for \$29,900 on 4/3/2013 and later for \$32,000 on 6/17/2015. Before Appellant bought the property there had been at least two accepted offers on the property that failed to close. \mlsHistory[]. 

The City contends that the true cash value should be the after-repair value. \cityEvidence.

\subsection{The Tribunal refuses to apply nonconsideration but this Court reverses}

The Referee who first heard the case at the Tribunal refused to apply nonconsideration. She held that the repairs were not normal repairs because the house was in substandard condition. ``Thus, the referee determined that if a property is purchased in substandard condition, any repairs done on the property to bring it into good repair do not constitute normal repairs, maintenance, or replacement within the meaning of MCL 211.27(2), so the increase in TCV resulting from those repairs can be immediately considered in determining the TCV for assessment purposes.'' \pincite{Patru 1}{2}.

This Court reversed. ``Nothing in MCL 211.27(2) provides that the repairs .~.~. are not normal repairs in the event that they are performed on a substandard property. Thus, by reading a requirement into the statute that was not stated by the legislature, the trial court erred .~.~.'' \pincite{Patru 1}{5}.

This Court remanded for a rehearing to determine if the repairs were normal repairs because ``[t]he referee did not fully evaluate that evidence---which included testimony---because [she] misapprehended how to properly apply MCL 211.27(2).'' \pincite{Patru 1}{5}.

\subsection{On rehearing, the Tribunal again refuses to apply nonconsideration}

The rehearing was held on 10/18/2018 before Tribunal Judge Marcus L. Abood. He ruled that the repairs were normal repairs, worth approximately \$10,000. \foj[4]. However, he went on to rule on pages 5 and 6 of the FOJ that ``Petitioner's contentions related to his purchase price and `normal' repairs are given no weight or credibility in the determination of market value'' because:
\begin{enumerate}
\item The property's assessment for 2016 changed based on the sale transaction in August 2015 and not based on Petitioner's repairs. 
\item The assessment considered the property in average condition as it was on 12/31/2015, tax day.
\item The MLS photographs submitted by Petitioner show the property in average condition, not neglected or vandalized.
\item The property's sale was not ``an arm's length sale transaction'' because the Petitioner has not claimed it so and because the seller was HUD.
\end{enumerate}

Judge Abood did not determine the property's before-repair value. Instead, he ruled that the property must be valued in its repaired condition because it was repaired before tax day. \foj[5]. He accepted the City's comparative sales analysis, writing: ``Respondent's sales adjustment grid does not included a line-item entry for repairs. This comparative analysis is devoid of any relationship to Petitioner's ``normal'' repairs to subject property which occurred before the issuance of a certificate of occupancy and the December 31, 2015 tax day.'' \foj[6].

\subsection{Appellant files a Motion for Reconsideration}

Appellant responded to the FOJ with a motion for reconsideration. The motion points out that this Court reversed the previous judgment of the Tribunal because  ``the referee's finding that the property's TCV was \$50,400 was based on its assessment of the property's value after it had been repaired.'' \motionForReconsideration[1]. The Tribunal was repeating the mistake, except this time after admitting that the repairs were normal repairs. Id.\ p 2. 

The Motion for Reconsideration also pointed out that the Tribunal had not done before-repair and after-repair appraisals as required by \mathieuGast. Id.\ p 4.

The Motion for Reconsideration also pointed out flaws in the four reasons given in the FOJ for giving no weight or credibility to Appellant's contentions related to his purchase price and normal repairs. Specifically:

\begin{enumerate}
\item The FOJ claimed the property's assessment changed based on the sale transaction. Appellant pointed out that assessments are not based on sales but rather are done yearly. The property \emph{uncapped} as a result of the sale, but Appellant had no issue with uncapping. Id.\ p 4.

\item The FOJ claimed the assessment considered the property in average condition as it was on 12/31/2015, tax day. Appellant pointed out that this does not invalidate \mathieuGast\ which requires the removal of the contribution of normal repairs to the assessed value. Id.\ p 5. 

\item The FOJ claimed that the MLS photographs submitted by Petitioner show the property in average condition, not neglected or vandalized. Appellant pointed out that photos do not excuse performance of \mathieuGast; nor do they contradict the fact that the City itself inspected the house and required repairs; nor do the photos, put in the MLS by the real estate brokers who listed and sold the house, show that the house was listed and sold at a non-market price. Id. p 5.
  
\item The FOJ claimed that the property's sale was not an arm's length sale transaction because the Petitioner has not claimed it so and because the seller was HUD. Appellant pointed out that he had included MLS data to show that the property's sale was a market sale and that merely mentioning that the seller was HUD is not evidence that the sale was not a market sale. Id.\ p 6.
\end{enumerate}

\subsection{The Tribunal denies the Motion for Reconsideration}

Tribunal Judge David B. Marmon, instead of Judge Abood, denied Motion for Reconsideration. He did not specifically rebut Appellant's points made in the motion. Instead he clarified why the Tribunal was not applying \mathieuGast. The \reconsiderationDenied[2] contains the Tribunal's reasoning. In summary:

\begin{enumerate}
  
\item The Tribunal did not determine the before-repair value because:
  \begin{enumerate}
  \item the text of the statute does not plainly require a before-repair appraisal;
  \item the situations in the second sentence of MCL 211.27(2) are not at issue;
  \item the STC guidance requires appraisal if the value of the repairs are on the assessment roll, and here they are not;
  \item STC guidance lacks the force of law; and
  \item Petitioner has the burden of proof and the Tribunal disagrees with Petitioner's evidence (the sale price is given no weight or credibility).
  \end{enumerate}
  
\item The Tribunal gave the property's sale price ``no weight or credibility'' because the seller was a government entity (HUD) who may not have been motivated to receive market value.
  
\item The Tribunal did not give nonconsideration treatment to the normal repairs because the assessment did not consider the repairs. The property's true cash value assessment was \$48,000 in the year before the repairs and \$50,400 after the repairs. The 5\% change was due to inflation not repairs. Also, the pictures on the MLS showed the property in average condition.

\end{enumerate}

% Points 1, 2, and 4 are addressed in order in the Arguments. Point 3 is addressed in the discussion of point 2.  

\section{Mathieu-Gast Statute -- MCL 211.27(2)}
\begin{quotation}
The assessor shall not consider the increase in true cash value that is a result of expenditures for normal repairs, replacement, and maintenance in determining the true cash value of property for assessment purposes until the property is sold.

For the purpose of implementing this subsection, the assessor shall not increase the construction quality classification or reduce the effective age for depreciation purposes, except if the appraisal of the property was erroneous before nonconsideration of the normal repair, replacement, or maintenance, and shall not assign an economic condition factor to the property that differs from the economic condition factor assigned to similar properties as defined by appraisal procedures applied in the jurisdiction.

The increase in value attributable to the items included in subdivisions (a) to (o) that is known to the assessor and excluded from true cash value shall be indicated on the assessment roll.

This subsection applies only to residential property.

The following repairs are considered normal maintenance if they are not part of a structural addition or completion: [repairs (a)-(o) omitted]

% (a) Outside painting.
% (b) Repairing or replacing siding, roof, porches, steps, sidewalks, or drives.
% (c) Repainting, repairing, or replacing existing masonry.
% (d) Replacing awnings.
% (e) Adding or replacing gutters and downspouts.
% (f) Replacing storm windows or doors.
% (g) Insulating or weatherstripping.
% (h) Complete rewiring.
% (i) Replacing plumbing and light fixtures.
% (j) Replacing a furnace with a new furnace of the same type or replacing an oil or gas burner.
% (k) Repairing plaster, inside painting, or other redecorating.
% (l) New ceiling, wall, or floor surfacing.
% (m) Removing partitions to enlarge rooms.
% (n) Replacing an automatic hot water heater.
% (o) Replacing dated interior woodwork.
\end{quotation}



\section{Argument}
\subsection{This Court creates an unsupported exception to Mathieu-Gast}

\makeandtab
\ctable[
  caption = {Relationship of Capped value and nonconsideration},
  label = {tab:cappedVsNonconsideration},
  pos = !htbp,
  %maxwidth = \textwidth, 
  width = \textwidth, 
  doinside = {\setstretch{1}}
  ]{lp{6.5cm}p{6.5cm}}{
    % \tnote[a]{TCV is as of tax day, 12/31/2018. Per Respondent, the market dropped 8\% from the subject's sale to Tax Day.}
    % \tnote[b]{Condition assumes repairs are complete. Deductions for repairs are explicitly considered under repairs.}
    % \tnote[c]{Mathieu-Gast requires that repairs not be considered. Comparables already repaired must be adjusted to down to reflect the fact that the TCV is the before-repair value.}
  }{ \FL
    & \textbf{Capped Value} & \textbf{Nonconsidered Value} \ML
    Law & MCL 211.27a & MCL 211.27(2)\NN
    Affects & Taxable Value & Assessed Value \NN
    Based On & Minimum AV since property was purchased & Change in market value due to owner's normal repairs \NN
    Normalizes & The year after a sale the TV is set to the AV. & The year after a sale the seller's nonconsideration value is considered.\NN
    Purpose & Shields existing owners' property taxes from rising market values. & Encourages home repairs and maintenance. \LL
}
\makeandletter

Previous Year's Assessed True Cash Value and
HUD's motivation to sell.


This Court's main error was that it incorporated into the Mathieu-Gast statute, MCL 211.27(2), the exception that the statute does not apply in the year that the property is purchased. This exception is not in the statute itself, has no support in the rest of the Tax Act, is contrary to the State Tax Commission guidance and is irrational. Namely, this Court's interpretation of Mathieu-Gast would reward all owners for normal repairs except the industrious purchaser of a rundown house who immediately repairs the property rather than waits to repair the property in the next calendar year.

\subsubsection{Unsupported by statute}
\subsubsection{Contrary to STC guidelines}
\subsubsection{Irrational}
 \subsection{This Court misapplies the law of the case doctrine}

Besides misinterpreting Mathieu-Gast, this Court has also misapplied the law of the case doctrine in two ways. First, it misapplied the doctrine in finding that this Court had left open the question in \cite{Patru 1}\ as to whether Mathieu-Gast applies for repairs done by a purchaser in the year of purchase. This Court had not left open this question. Instead, this Court, and the Tribunal before it, had assumed that there was no such exception to Mathieu-Gast. This point was not brought up in the first go-around.
Instead, this Court had remanded solely on the question of whether normal repairs had been performed. 
It was assumed by all, the Petitioner/Appellant, the Respondend/Appellee, the Tribunal, and this Court, that once normal repairs were found, that the assessed value would be set at the pre-repair value.

\begin{quotation}
  In its final opinion and judgment, the Tribunal recognized that the referee erred in its
interpretation of MCL 211.27(2); however, it nevertheless upheld the determination of TCV.
The Tribunal reasoned that because the spreadsheet detailing the repairs completed on the
property had not been submitted before the hearing, it had no obligation to consider that
evidence, so it concluded that Patru failed to establish that the repairs constituted normal repairs.
However, as stated above, Patru did present evidence at the hearing in support of his claim that
MCL 211.27(2) applied. The referee did not fully evaluate that evidence—which included
testimony—because it misapprehended how to properly apply MCL 211.27(2).
Further, because the hearing was not transcribed, we cannot determine whether the
evidence Patru provided at the hearing was reflective of the information on the spreadsheet
submitted with his exceptions. If the testimony provided was an oral recitation of the
information included on the spreadsheet, then Patru presented testimony sufficient to establish
that at least some of the repairs constituted normal repairs under MCL 211.27(2), and so the
increase in TCV attributed to those repairs should not be considered in the property's TCV for
assessment purposes until such time as Patru sells the property. However, if Patru merely
testified that he did some carpentry, electrical, and masonry repairs and no further explanation of
the work that was provided, then he would have arguably failed to support his claim. Either way,
on the record before this Court, we cannot evaluate the sufficiency of the evidence presented at
the hearing. Thus, we conclude that further proceedings are necessary in order to determine
whether the repairs were normal repairs within the meaning of MCL 211.27(2). Accordingly, we
remand to the Tax Tribunal for a rehearing. Further, because the existing record is insufficient to
resolve whether the repairs are normal repairs within the meaning of the statute, the parties shall
be afforded further opportunity to submit additional proofs. See Fisher v Sunfield Township, 163

walk and broken treads on front steps, which is a normal repair under MCL 211.27(2)(b), and
repainted the interior, which is a normal repair under MCL 211.27(2)(k).
-6-
Mich App 735, 743; 415 n 297 (1987) (requiring rehearing when it was not clear whether the
proofs submitted were sufficient to establish that repair expenditures were normal repairs).
(footnote 3: We note that, on reconsideration, the Tribunal faulted Patru for failing to establish a pre-repair
TCV. However, as the Tribunal must make its own, independent determination of TCV, Great
Lakes Div of Nat'l Steel Corp v City of Ecorse, 227 Mich App 379, 389; 576 NW2d 667 (1998),
we conclude that Patru's failure to persuade the Tribunal that the property's purchase price
reflected the pre-repair TCV is irrelevant. The Tribunal independently had to evaluate all the
evidence presented and, properly applying MCL 211.27(2), arrive at the property's TCV.)
\end{quotation}

Secondly, this Court failed to apply the law of the case doctrine in upholding the Tribunal's second ruling which was based on the second Tribunal's opinion that the subject property was is average condition at the time of purchase. The first Tribunal found the opposite, and indeed based its decision in part on the fact that the repairs were not normal because the  property's condition was too poor.

\begin{quote}
Thus, the referee determined that if a property is purchased in substandard condition, any repairs
done on the property to bring it into good repair do not constitute normal repairs, maintenance, or
replacement within the meaning of MCL 211.27(2), so the increase in TCV resulting from those
repairs can be immediately considered in determining the TCV for assessment purposes. The
referee then determined that the TCV for the property was \$50,400.
\end{quote}

In this case both the law and the facts were changed by the Tribunal in the 2nd go-around, contrary to the law of the case.

\subsection{This Court allows the Tribunal to improperly assume facts}

This Court also erred when it endorsed the idea that Tribunal could use the previous year's uncontested assessed TCV to override the actual evidence presented in the case as to the subject property's pre-repair TCV. ``It is undisputed that the assessed
TCV of the property for the 2015 tax year was \$48,000. Given this evidence, it was appropriate
for the tribunal to draw conclusions about the value of the property before and after petitioner's
purchase, specifically, that the prior year's assessment of \$48,000 was reflective of the property's
TCV before petitioner purchased it, that an increase of \$2,400 could be attributed to inflation and 
-6-
increases in the market, and that petitioner's “normal repairs” were not attributable to the
property's substandard condition, but rather were intended primarily to prepare the property as
rental property.'' \pincite{Patru 2}{5-6}.

The previous year's assessed value was not at issue, the time for its appeal having passed before Petitioner/Appellant had purchased the property.\footnote{If the previous year's assessed TCV is presumed to be the actual TCV of the property on the previous tax day, tax appeals could be settled simply by looking at property appreciation. But \cite{Jones & Laughlin}\ teaches that the Tax Tribunal must independently determine the TCV at issue using primary evidence, sales studies and replacement cost estimates.} At issue was the pre-repair value of the property which under \cite{Jones & Laughlin}\ requires the Tribunal to make an independent  determination. The Tribunal cannot assume that City's previous-year's assessed TCV was correct.


\begin{quotation}
   It is undisputed that the assessed
TCV of the property for the 2015 tax year was \$48,000. Given this evidence, it was appropriate
for the tribunal to draw conclusions about the value of the property before and after petitioner's
purchase, specifically, that the prior year's assessment of \$48,000 was reflective of the property's
TCV before petitioner purchased it, that an increase of \$2,400 could be attributed to inflation and 
-6-
increases in the market, and that petitioner's “normal repairs” were not attributable to the
property's substandard condition, but rather were intended primarily to prepare the property as
rental property. The tribunal concluded that the “evidence supports the property's assessment as
a property in average condition both at the time Petitioner acquired it and after he completed the
normal repairs,” and that “the assessment did not consider the increase in true cash value that was
the result of normal repairs.”
\end{quotation}

The same error is present when this Court endorses the Tribunal's substitution of opinion for fact regarding the seller's motivation.


normal repairs must be considered when when an owner performs the repair in the same year as when the owner purchases the property.

they are performed in the same year as  nonconsidered only when performed in

\subsection{This opinion meets the standards for publications, so it should be correct}

\begin{quotation}
  Rule 7.215 Opinions, Orders, Judgments, and Final Process for Court of Appeals

(A) Opinions of Court. An opinion must be written and bear the writer's name or the label ``per curiam'' or ``memorandum'' opinion. An opinion of the court that bears the writer's name shall be published by the Supreme Court reporter of decisions. A memorandum opinion shall not be published. A per curiam opinion shall not be published unless one of the judges deciding the case directs the reporter to do so at the time it is filed with the clerk. A copy of an opinion to be published must be delivered to the reporter no later than when it is filed with the clerk. The reporter is responsible for having those opinions published as are opinions of the Supreme Court, but in separate volumes containing opinions of the Court of Appeals only, in a form and under a contract approved by the Supreme Court. An opinion not designated for publication shall be deemed ``unpublished.''

(B) Standards for Publication. A court opinion must be published if it:
n
(1) establishes a new rule of law;

(2) construes as a matter of first impression a provision of a constitution, statute, regulation, ordinance, or court rule;

(3) alters, modifies, or reverses an existing rule of law;

(4) reaffirms a principle of law or construction of a constitution, statute, regulation, ordinance, or court rule not applied in a reported decision since November 1, 1990;

(5) involves a legal issue of significant public interest;

(6) criticizes existing law; or

(7) resolves a conflict among unpublished Court of Appeals opinions brought to the Court's attention; or

(8) decides an appeal from a lower court order ruling that a provision of the Michigan Constitution, a Michigan Statute, a rule or regulation included in the Michigan Administrative Code, or any other action of the legislative or executive branch of state government is invalid.
\end{quotation}


\section{Standard of Review}

MCL 2.119(F)(3) says:

\begin{quote}
  Generally, and without restricting the discretion of the court, a motion for rehearing or reconsideration which merely presents the same issues ruled on by the court, either expressly or by reasonable implication, will not be granted. The moving party must demonstrate a palpable error by which the court and the parties have been misled and show that a different disposition of the motion must result from correction of the error.
\end{quote}

 A palpable error is a clear error ``easily perceptible, plain, obvious, readily
 visible, noticeable, patent, distinct, manifest.'' \pincite{Luckow}{426; 453}\ (cleaned up).
 
``The palpable error provision in MCL 2.119(F)(3)  is not mandatory and only provides guidance to
a court about when it may be appropriate to consider a motion for rehearing or reconsideration.''
\pincite{Walters}{350; 430}.

``The rule [MCL 2.119(F)(3)] does not categorically prevent a trial court from revisiting an issue even when the motion for reconsideration presents the same issue already ruled upon; in fact, it allows considerable discretion to correct mistakes.'' \pincite{Macomb County Department of Human Services}{754}. 








\section{Relief}

Therefore because


\needspace{6\baselineskip}
\section{Proof of Service}

I certify that I served a copy of these Exceptions on Respondent's representative, Emily Pizzo, by email on the same day I emailed them to the Tribunal.

\vspace{1\baselineskip}

{ \setlength{\leftskip}{3.5in}

  Respectfully Submitted,

  /s/ Daniel Patru, P74387

\today

  \setlength{\leftskip}{0pt}}

\newpage\empty% we need a new page so that the index entries on the last
        % page get written out to the right file.
\end{document}


%%% Local Variables:
%%% mode: latex
%%% TeX-master: t
%%% End:
