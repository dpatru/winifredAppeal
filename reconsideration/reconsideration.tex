%
% This is one of the samples from the lawtex package:
% http://lawtex.sourceforge.net/
% LawTeX is licensed under the GNU General Public License 
%
\providecommand{\documentclassflag}{}
\documentclass[12pt,\documentclassflag]{michiganCourtOfAppealsBrief}
%\documentclass{article}

% for striking a row in a table, see https://tex.stackexchange.com/a/265728/135718
\usepackage{tikz}
\usetikzlibrary{tikzmark}

\makeandletter% use \makeandtab to turn off

% Use this to show a line grid-
% \usepackage[fontsize=12pt,baseline=24pt,lines=27]{grid}
% \usepackage{atbegshi,picture,xcolor} % https://tex.stackexchange.com/a/191004/135718
% \AtBeginShipout{%
%   \AtBeginShipoutUpperLeft{%
%     {\color{red}%
%     \put(\dimexpr -1in-\oddsidemargin,%
%          -\dimexpr 1in+\topmargin+\headheight+\headsep+\topskip)%
%       {%
%        \vtop to\dimexpr\vsize+\baselineskip{%
%          \hrule%
%          \leaders\vbox to\baselineskip{\hrule width\hsize\vfill}\vfill%
%        }%
%       }%
%   }}%
% }
%   \linespread{1}

\usepackage[modulo]{lineno}% use \linenumbers to show line numbers, see https://texblog.org/2012/02/08/adding-line-numbers-to-documents/

% from https://tex.stackexchange.com/a/313337/135718
\usepackage{enumitem,amssymb}
\newlist{todolist}{itemize}{2}
\setlist[todolist]{label=$\square$}
\usepackage{pifont}
\newcommand{\cmark}{\ding{51}}%
\newcommand{\xmark}{\ding{55}}%
\newcommand{\done}{\rlap{$\square$}{\raisebox{2pt}{\large\hspace{1pt}\cmark}}%
\hspace{-2.5pt}}
\newcommand{\wontfix}{\rlap{$\square$}{\large\hspace{1pt}\xmark}}

% allow underscores in words
\chardef\_=`_% https://tex.stackexchange.com/a/301984/135718 

%%Citations
 
%The command \makeandletter turns the ampersand into a printable character, rather than a special alignment tab \makeandletter

\begin{document}
\singlespacing%

\citecase[Antisdale]{Antisdale v City of Galesburg, 420 Mich 265, 362 NW2d 632 (1984)}
\citecase[Bennett]{Bennett v Bennett, 197 Mich. App. 497; 496 N.W.2d 353 (1992)}
\citecase[Berger]{Berger v Berger, 747 N.W.2d 336; 277 Mich. App. 700 (2008)}
\citecase[Briggs]{Briggs Tax Service, LLC v Detroit Pub. Schools, 485 Mich 69; 780 NW2d 753 (2010)}
\citecase[CAF v STC]{CAF Investment Company v State Tax Commission, 392 Mich. 442; 221 N.W.2d 588 (1974)}
\citecase[CAF v Saginaw]{CAF Investment Co. v. Saginaw Twp., 410 Mich. 428, 454, 302 N.W.2d 164 (1981)}
\citecase[Clark]{Clark Equipment Company v Township Of Leoni, 113 Mich. App. 778; 318 N.W.2d 586 (1982)}
\citecase[Kar]{Kar v Hogan, 399 Mich. 529; 251 N.W.2d 77  (1976)}
\citecase[Fisher]{Fisher v. Sunfield Township, 163 Mich App 735; 415 NW2nd 297 (1987)}
\citecase[Jones & Laughlin]{Jones & Laughlin Steel Corporation v City of Warren, 193 Mich App 348; 483 NW2nd 416 (1992)}
\citecase{Lenawee Co v Wagley, 301 Mich App 134; 836 NW2d 193 (2013)} %
\citecase[Menard]{Menard, Inc. v Escanaba, 315 Mich. App. 512; 891 N.W.2d 1 (2016)}
%\citecase[Mich Ed]{Mich Ed Ass'n v Secretary of State (On Rehearing), 489 Mich 194; 801 NW2d 35 (2011)}
\newcase{Mich Ed}{Mich Ed Ass'n v Secretary of State (On Rehearing)}{489 Mich; 801 NW2d}{194; 35}{(2011)}
% (stating that nothing will be read into a clear statute that is not within the manifest intention of the Legislature as derived from the language of the statute itself).

\citecase[Patru 1]{Patru v City of Wayne, unpublished per curiam opinion of the Court of Appeals, issued May 8, 2018 (Docket No. 337547)}%
\addReference{Patru 1}{patruvwayne}% Associate appendix with case
\citecase[Patru 2]{Patru v City of Wayne, unpublished per curiam opinion of the Court of Appeals, issued February 18, 2020 (Docket No. 346894)}%

\citecase[Pontiac Country Club]{Pontiac Country Club v Waterford Twp, 299 Mich App 427; 830 NW2d 785 (2013)}
\citecase[Luckow]{Luckow Estate v Luckow, 291 Mich App 417; 805 NW2d 453 (2011)}
\citecase[Stamp]{Stamp v Mill Street Inn, 152 Mich App 290; 393 NW2d 614 (1986)}
\citecase[Macomb County Department of Human Services]{Macomb County Department of Human Services v Anderson, 304 Mich App 750; 849 NW2d 408 (2014)}
\citecase{Toll Northville LTD v Twp of Northville, 480 Mich 6; 743 NW2d 902 (2008)}
\citecase[Walters]{People v Walters, 266 Mich App 341; 700 NW2d 424 (2005)}

{\makeatletter % needed for optional argument to newstatute.
  % \newstatute[1@]{MCL}{}% place MCL first
  % \newstatute[2@]{MCR}{}
  % \newstatute[3@]{TTR}{}
  % \newstatute[4@]{Dearborn Ordinance}{}% place this fourth
  % \newstatute[5@]{Wayne Ordinance}{}
  \newstatute{MCL 211.2}{}
  \newstatute{MCL 211.2(2)}{}
  %   (1) For the purpose of taxation, real property includes all of the following:
  % (a) All land within this state, all buildings and fixtures on the land, and all appurtenances to the land, except as expressly exempted by law.
  % (b) All real property owned by this state or purchased or condemned for public highway purposes by any board, officer, commission, or department of this state and sold on land contract, notwithstanding the fact that the deed has not been executed transferring title.
  % (c) For taxes levied after December 31, 2002, buildings and improvements located upon leased real property, except buildings and improvements exempt under section 9f or improvements assessable under section 8(h), if the value of the buildings or improvements is not otherwise included in the assessment of the real property. However, buildings and improvements located on leased real property shall not be treated as real property unless they would be treated as real property if they were located on real property owned by the taxpayer.
  % (2) The taxable status of persons and real and personal property for a tax year shall be determined as of each December 31 of the immediately preceding year, which is considered the tax day, any provisions in the charter of any city or village to the contrary notwithstanding. An assessing officer is not restricted to any particular period in the preparation of the assessment roll but may survey, examine, or review property at any time before or after the tax day.

  \newstatute{MCL 211.27}{}% True Cash Value
  \newstatute{MCL 211.27(1)}{}% True Cash Value
  \newstatute{MCL 211.27(2)}{}% MathieuGast
%   (2) The assessor shall not consider the increase in true cash value that is a result of expenditures for normal repairs, replacement, and maintenance in determining the true cash value of property for assessment purposes until the property is sold. For the purpose of implementing this subsection, the assessor shall not increase the construction quality classification or reduce the effective age for depreciation purposes, except if the appraisal of the property was erroneous before nonconsideration of the normal repair, replacement, or maintenance, and shall not assign an economic condition factor to the property that differs from the economic condition factor assigned to similar properties as defined by appraisal procedures applied in the jurisdiction. The increase in value attributable to the items included in subdivisions (a) to (o) that is known to the assessor and excluded from true cash value shall be indicated on the assessment roll. This subsection applies only to residential property. The following repairs are considered normal maintenance if they are not part of a structural addition or completion:
% (a) Outside painting.
% (b) Repairing or replacing siding, roof, porches, steps, sidewalks, or drives.
% (c) Repainting, repairing, or replacing existing masonry.
% (d) Replacing awnings.
% (e) Adding or replacing gutters and downspouts.
% (f) Replacing storm windows or doors.
% (g) Insulating or weatherstripping.
% (h) Complete rewiring.
% (i) Replacing plumbing and light fixtures.
% (j) Replacing a furnace with a new furnace of the same type or replacing an oil or gas burner.
% (k) Repairing plaster, inside painting, or other redecorating.
% (l) New ceiling, wall, or floor surfacing.
% (m) Removing partitions to enlarge rooms.
% (n) Replacing an automatic hot water heater.
% (o) Replacing dated interior woodwork.

  \newstatute{MCL 211.27(3)}{}%

  \newstatute{MCL 211.27(6)}{}%
% (6) Except as otherwise provided in subsection (7), the purchase price paid in a transfer of property is not the presumptive true cash value of the property transferred. In determining the true cash value of transferred property, an assessing officer shall assess that property using the same valuation method used to value all other property of that same classification in the assessing jurisdiction. As used in this subsection and subsection (7), "purchase price" means the total consideration agreed to in an arms-length transaction and not at a forced sale paid by the purchaser of the property, stated in dollars, whether or not paid in dollars.
  
  \newstatute{MCL 211.27a}{}%
  \newstatute{MCL 211.27a(1)}{}%
  \newstatute{MCL 211.27a(2)}{}%
  \newstatute{MCL 211.27a(2)(b)}{}%
  \newstatute{MCL 211.27a(3)}{}%
  
  \newstatute{MCL 211.10d(7)}{}%
%  (7) Every lawful assessment roll shall have a certificate attached signed by the certified assessor who prepared or supervised the preparation of the roll. The certificate shall be in the form prescribed by the state tax commission. If after completing the assessment roll the certified assessor for the local assessing district dies or otherwise becomes incapable of certifying the assessment roll, the county equalization director or the state tax commission shall certify the completed assessment roll at no cost to the local assessing district.


  
  \newstatute{MCL 205.753(2)}{}% allows appeals from a final order of the Tax Tribunal
\newstatute{MCL 2.119(F)(3)}{}% motion for reconsideration must show a palpable error
  \newstatute{MCR 7.204(A)(1)(b)}{}% allows appeals within 21 days of an order on a motion for reconsideration
  
} % end makeatletter block

\def\mathieuGast{\cite[s]{MCL 211.27(2)}}%


%\newmisc{STC Bulletin 6 of 2007}{Michigan State Tax Commission (STC) Bulletin No. 6 of 2007 (Foreclosure Guidelines)}
\newmisc{STC Bulletin}{Michigan State Tax Commission (STC) Bulletin No. 7 of 2014 (Mathieu Gast Act)}
\addReference{STC Bulletin}{bulletin7}% Associate appendix with case

\newbook{The Appraisal of Real Estate}{Appraisal Institute}{The Appraisal of Real Estate}{(14th ed, Chicago: Appraisal Institute, 2013)}
\newbook{Appraising the Tough Ones}{Harrison}{Appraising the Tough Ones : Creative Ways to Value Residential Properties}{(Chicago: Appraisal Institute, 1996)}
\newbook{McCormick}{McCormick}{Evidence}{(2d ed)}
\newbook{Assessor's Manual}{Michigan State Tax Commission}{\href{https://www.michigan.gov/documents/treasury/1.__2014_Michigan_Assessors_Manual_Volume_I_Introduction_575738_7.pdf}{Assessor's Manual}}{(Vol 1 Residential, 2014)} 
\newbook{Guide to Basic Assessing}{Michigan State Tax Commission}{Guide to Basic Assessing}{published May 2018, retrieved February 27, 2020 from \href{https://www.michigan.gov/documents/treasury/Guide_to_Basic_Assessing_1-16_511508_7.pdf}{https://www.michigan.gov/documents/treasury/Guide\_to\_Basic\_Assessing\_1-16\_511508\_7.pdf}}




\newbook{Wex}{Legal Information Institute}{\href{https://www.law.cornell.edu/wex}{Wex}}{<https://www.law.cornell.edu/wex> (accessed October 7, 2019)}

\newmisc{Respondent's Evidence}{Respondent's Evidence (entry line 16)}
\SetIndexType{Respondent's Evidence}{}
\newmisc{Chart of Median Price}{Chart of Ann Arbor Township Median Price Per Square Foot, Petitioner's Evidence (entry line 15)}
\SetIndexType{Chart of Median Price}{}
\newmisc{June 24 letter}{Petitioner's June 24 letter (entry line 11)}
\SetIndexType{June 24 letter}{}
\newmisc{July 10 letter}{Petitioner's July 10 letter (entry line 21)}
\SetIndexType{July 10 letter}{}
\newmisc{Form 865}{STC Form 865 (Mathieu Gast Nonconsideration)}
\newmisc{Affidavit}{Affidavit of Carolyn Lepard (entry line 10)}
\SetIndexType{Affidavit}{}

\newmisc{POJ}{Proposed Opinion and Judgment (POJ)}
\SetIndexType{POJ}{}
\newmisc{Exceptions}{Exceptions (entry line 25)}
\SetIndexType{Exceptions}{}
\newmisc{FOJ}{Final Opinion and Judgment (FOJ)}
\SetIndexType{FOJ}{}
\newcommand{\makeAbbreviation}[3]{% ensure that the frsit time an abbreviated word is used, it is presented in long form, and after that in short form. 1: command name, 2: short name, 3: long name
  \IfBeginWith{#3}{#2}{%
    \newcommand{#1}[0]{#3\renewcommand{#1}[0]{#2}}}{%
    \newcommand{#1}[0]{#3 (#2)\renewcommand{#1}[0]{#2}}}}

\makeAbbreviation{\MLS}{MLS}{Multiple Listing Service}
\makeAbbreviation{\MTT}{MTT}{Michigan Tax Tribunal}
\makeAbbreviation{\STC}{STC}{State Tax Commission}
  
\newcommand{\makeAbbreviationToRecord}[3]{% 1: command/handle, 2: shortname, 3: longname
  % makeAbbreviationToRecord: #1, #2, #3\par%
  \expandafter\makeAbbreviation\csname #1Abbr\endcsname{#2}{#3}%
  \expandafter\newcommand\csname #1\endcsname[1][]{%
    \Call{#1Abbr}%
    \if\relax##1\relax\empty\ (Appendix at \pageref{#1})%
    \else, p ##1 (Appendix at %
    % Check for page range
    \IfSubStr{##1}{-}{%
      \def\pageRefRange####1-####2XXX{\pageref{#1.####1} -- \pageref{#1.####2}}%
      \pageRefRange##1XXX}%
    {\pageref{#1.##1}}%
    )\fi%
  }%
}%


\makeAbbreviationToRecord{explanatoryLetter}{Explanatory Letter}{Explanatory Letter (MTT Docket Line 38)}
% explanatory Letter: (abbr) \explanatoryLetterAbbr\ (to record) \explanatoryLetter[2] \par

\makeAbbreviationToRecord{foj}{FOJ}{Second Final Opinion and Judgment (MTT Docket Line 48)}
% FOJ: (abbr) \fojAbbr (to record) \foj[]\par

\makeAbbreviationToRecord{reconsiderationDenied}{Order Denying Reconsideration}{Order Denying Reconsideration (MTT Docket Line 51)}
% reconsiderationDenied: (abbr) \reconsiderationDeniedAbbr[] (to record) \reconsiderationDenied[] \par

\makeAbbreviationToRecord{repairs}{List of Repairs}{List of Repairs (MTT Docket Line 36)}
% \par repairs: (abbr) \repairsAbbr\ (to record) \repairs[] (to appendix) \pageref{repairs}\par

\makeAbbreviationToRecord{stcform}{STC Form 865}{STC Form 865 Request for Nonconsideration (MTT Docket Line 35)}

\makeAbbreviationToRecord{mlsListing}{MLS Listing}{MLS Listing (MTT Docket Line 32)}
% mlsListing: (abbr) \mlsListingAbbr\ (to record) \mlsListing[]\par

\makeAbbreviationToRecord{mlsHistory}{MLS History}{MLS History (MTT Docket Line 33)}
% mlsHistory: (abbr) \mlsHistoryAbbr\ (to record) \mlsHistory[]\par

\makeAbbreviationToRecord{boardOfReviewDecision}{Board of Review Decision}{Board of Review Decision (MTT Docket Line 2)}

\makeAbbreviationToRecord{cityEvidence}{City's Evidence}{City's Evidence (MTT Docket Line 11)}

\makeAbbreviationToRecord{motionForReconsideration}{Motion for Reconsideration}{Motion for Reconsideration (MTT Docket Line 52)}
%  \newstatute{MCL 211.10d(7)}{}%
%  (7) Every lawful assessment roll shall have a certificate attached signed by the certified assessor who prepared or supervised the preparation of the roll. The certificate shall be in the form prescribed by the state tax commission. If after completing the assessment roll the certified assessor for the local assessing district dies or otherwise becomes incapable of certifying the assessment roll, the county equalization director or the state tax commission shall certify the completed assessment roll at no cost to the local assessing district.


  
  % \newstatute{MCL 205.753(2)}{}% allows appeals from a final order of the Tax Tribunal

  % \newstatute{MCR 7.204(A)(1)(b)}{}% allows appeals within 21 days of an order on a motion for reconsideration
  



%\newmisc{STC Bulletin 6 of 2007}{Michigan State Tax Commission (STC) Bulletin No. 6 of 2007 (Foreclosure Guidelines)}
% \newmisc{STC Bulletin}{Michigan State Tax Commission (STC) Bulletin No. 7 of 2014 (Mathieu Gast Act)}
% \addReference{STC Bulletin}{bulletin7}% Associate appendix with case

% \newcommand{\makeAbbreviation}[3]{% ensure that the frsit time an abbreviated word is used, it is presented in long form, and after that in short form. 1: command name, 2: short name, 3: long name
%   \IfBeginWith{#3}{#2}{%
%     \newcommand{#1}[0]{#3\renewcommand{#1}[0]{#2}}}{%
%     \newcommand{#1}[0]{#3 (#2)\renewcommand{#1}[0]{#2}}}}

% \makeAbbreviation{\MLS}{MLS}{Multiple Listing Service}
% \makeAbbreviation{\MTT}{MTT}{Michigan Tax Tribunal}
% \makeAbbreviation{\STC}{STC}{State Tax Commission}
%\makeAbbreviation{\FOJ}{FOJ}{First Final Opinion and Judgment (2017)}
% \makeAbbreviation{\explanatoryLetterAbbr}{Explanatory Letter}{Explanatory Letter submitted by Appellant to the Tax Tribunal on 9/6/2018}
% \newcommand{\explanatoryLetter}[1][]{\explanatoryLetterAbbr\if\relax#1\relax\empty, Appendix at \pageref{explanatoryLetter}\else, page #1, Appendix at \pageref{explanatoryLetter.#1}\fi}

% Note that the command/handle must match the appendix
% labels. Minimize the variations of names.  \makeAbbreviationToRecord
% creates a simple abbreviation that ends in Abbr if you don't want to
% refer to the record.
% \newcommand{\makeAbbreviationToRecord}[3]{% 1: command/handle, 2: shortname, 3: longname
%   % makeAbbreviationToRecord: #1, #2, #3\par%
%   \expandafter\makeAbbreviation\csname #1Abbr\endcsname{#2}{#3}%
%   \expandafter\newcommand\csname #1\endcsname[1][]{%
%     \Call{#1Abbr}%
%     \if\relax##1\relax\empty\ (Appendix at \pageref{#1})%
%     \else, p ##1 (Appendix at %
%     % Check for page range
%     \IfSubStr{##1}{-}{%
%       \def\pageRefRange####1-####2XXX{\pageref{#1.####1}--\pageref{#1.####2}}%
%       \pageRefRange##1XXX}%
%     {\pageref{#1.##1}}%
%     )\fi%
%   }%
% }%

% \makeAbbreviationToRecord{explanatoryLetter}{Explanatory Letter}{Explanatory Letter (MTT Docket Line 38)}
% % explanatory Letter: (abbr) \explanatoryLetterAbbr\ (to record) \explanatoryLetter[2] \par

% \makeAbbreviationToRecord{foj}{FOJ}{Second Final Opinion and Judgment (MTT Docket Line 48)}
% % FOJ: (abbr) \fojAbbr (to record) \foj[]\par

% \makeAbbreviationToRecord{reconsiderationDenied}{Order Denying Reconsideration}{Order Denying Reconsideration (MTT Docket Line 51)}
% % reconsiderationDenied: (abbr) \reconsiderationDeniedAbbr[] (to record) \reconsiderationDenied[] \par

% \makeAbbreviationToRecord{repairs}{List of Repairs}{List of Repairs (MTT Docket Line 36)}
% % \par repairs: (abbr) \repairsAbbr\ (to record) \repairs[] (to appendix) \pageref{repairs}\par

% \makeAbbreviationToRecord{stcform}{STC Form 865}{STC Form 865 Request for Nonconsideration (MTT Docket Line 35)}

% \makeAbbreviationToRecord{mlsListing}{MLS Listing}{MLS Listing (MTT Docket Line 32)}
% % mlsListing: (abbr) \mlsListingAbbr\ (to record) \mlsListing[]\par

% \makeAbbreviationToRecord{mlsHistory}{MLS History}{MLS History (MTT Docket Line 33)}
% % mlsHistory: (abbr) \mlsHistoryAbbr\ (to record) \mlsHistory[]\par

% \makeAbbreviationToRecord{boardOfReviewDecision}{Board of Review Decision}{Board of Review Decision (MTT Docket Line 2)}

% \makeAbbreviationToRecord{cityEvidence}{City's Evidence}{City's Evidence (MTT Docket Line 11)}

% \makeAbbreviationToRecord{motionForReconsideration}{Motion for Reconsideration}{Motion for Reconsideration (MTT Docket Line 52)}

% \makeAbbreviationToRecord{explanatoryLetter}{Explanatory Letter}{Explanatory Letter (MTT Docket Line 38)}
% % explanatory Letter: (abbr) \explanatoryLetterAbbr\ (to record) \explanatoryLetter[2] \par

% \makeAbbreviationToRecord{foj}{FOJ}{Second Final Opinion and Judgment (MTT Docket Line 48)}
% % FOJ: (abbr) \fojAbbr (to record) \foj[]\par

% \makeAbbreviationToRecord{reconsiderationDenied}{Order Denying Reconsideration}{Order Denying Reconsideration (MTT Docket Line 51)}
% % reconsiderationDenied: (abbr) \reconsiderationDeniedAbbr[] (to record) \reconsiderationDenied[] \par

% \makeAbbreviationToRecord{repairs}{List of Repairs}{List of Repairs (MTT Docket Line 36)}
% % \par repairs: (abbr) \repairsAbbr\ (to record) \repairs[] (to appendix) \pageref{repairs}\par

% \makeAbbreviationToRecord{stcform}{STC Form 865}{STC Form 865 Request for Nonconsideration (MTT Docket Line 35)}

% \makeAbbreviationToRecord{mlsListing}{MLS Listing}{MLS Listing (MTT Docket Line 32)}
% % mlsListing: (abbr) \mlsListingAbbr\ (to record) \mlsListing[]\par

% \makeAbbreviationToRecord{mlsHistory}{MLS History}{MLS History (MTT Docket Line 33)}
% % mlsHistory: (abbr) \mlsHistoryAbbr\ (to record) \mlsHistory[]\par

% \makeAbbreviationToRecord{boardOfReviewDecision}{Board of Review Decision}{Board of Review Decision (MTT Docket Line 2)}

% \makeAbbreviationToRecord{cityEvidence}{City's Evidence}{City's Evidence (MTT Docket Line 11)}

% \makeAbbreviationToRecord{motionForReconsideration}{Motion for Reconsideration}{Motion for Reconsideration (MTT Docket Line 52)}


\begin{centering}
\bf\scshape State of Michigan\\In the Court of Appeals\\Detroit Office\\~\\ 
\rm 

\makeandtab
\setlength{\tabcolsep}{20pt}%
\begin{tabular}{p{.4\textwidth} p{.4\textwidth}}
\cline{1-2}
  {~

  \raggedright Daniel Patru,\par
  \hfill\textit{Petitioner/Appellant,}
  \vspace{.5\baselineskip}\par
  vs\par
  \vspace{.5\baselineskip}
  \raggedright City of Wayne,\par
  \hfill\textit{Respondent/Appellee.}
  
  ~} &  {~
       \par\par
       \hfill Court of Appeals No. 346894\par
       \hfill Lower Court No. 16-001828-TT\par\vspace{\baselineskip}

       \hfill \raggedleft\textbf{Motion and Brief for Reconsideration}\vspace{.5\baselineskip}\par
%       \hfill \textbf{Proof of Service}\newline      
  ~}
  \\ \cline{1-2}\vspace{2mm}
  {~ \par
  Daniel Patru, P74387, \newline%
  3309 Solway\newline%
  Knoxville, TN 37931\newline%
  (734) 274-9624\newline%
  dpatru@gmail.com\newline\newline%
  ~} & {~ \par~\par
       
Stephen J. Hitchcock, P15005, \newline
John C. Clark, P51356, \newline
Attorneys for Respondent/Appellee\newline
Giarmarco, Mullins \& Horton, P.C.\newline
101 W. Big Beaver Road, 10 th floor\newline
Troy, MI 48084\newline
(248) 457-7024\newline
sjh@gmhlaw.com\newline
~}
\end{tabular}
\makeandletter
\par\vspace{\baselineskip}\vspace{\baselineskip}\vspace{\baselineskip}
%\textbf{ORAL ARGUMENT NOT REQUESTED}


\end{centering}

\pagestyle{romanparen}
\pagenumbering{roman}

% \section{Checklist}
% \begin{itemize}
%   \item Prep
%   \begin{todolist}
%   \item[\done] Copy files in new directory
%   \item[\done] Clean up files
%   \item Read the rules, copying them and applying them
%   \item[\done] Reconsideration in Court of Appeals.
%   \item What are the required sections? In what order?
%   \item Where is SEV value defined?
%   \item What questions were presented in the original appeal?
      
%   \end{todolist}

%   \item Write
%     \begin{todolist}
%     \item mistake
%     \item Write solution
%   \end{todolist}

%   \item Proof
%   \begin{todolist}
%   \item A motion for reconsideration may be filed within 21 days after the date of the order or the date stamped on an opinion.
%   \item %The motion shall include all facts, arguments, and citations to authorities in a single document and
%     shall not exceed 10 double-spaced pages.
%   \item A copy of the order or opinion of which reconsideration is sought must be included with the motion.
%   \item Motions for reconsideration are subject to the restrictions contained in MCR 2.119(F)(3).

%   \end{todolist}

  
% \item example
%   \begin{todolist}
%   \item[\done] Frame the problem
%   \item Write solution
%   \item[\wontfix] profit
%   \end{todolist}
% \end{itemize}


%\newpage 

\section*{Table of Contents}

\tableofcontents


\newpage
\tableofauthorities

\pagestyle{plain}
\pagenumbering{arabic}

%Sets the formatting for the entire document after the front matter
\parindent=2.5em
% \setlength{\parskip}{1.25ex plus 2ex minus .5ex} 
% \setstretch{1.45}
\doublespacing
% \linenumbers


% \section{Mathieu Gast Statute -- MCL 211.27(2)}
% \begin{quotation}
% The assessor shall not consider the increase in true cash value that is a result of expenditures for normal repairs, replacement, and maintenance in determining the true cash value of property for assessment purposes until the property is sold.

% For the purpose of implementing this subsection, the assessor shall not increase the construction quality classification or reduce the effective age for depreciation purposes, except if the appraisal of the property was erroneous before nonconsideration of the normal repair, replacement, or maintenance, and shall not assign an economic condition factor to the property that differs from the economic condition factor assigned to similar properties as defined by appraisal procedures applied in the jurisdiction.

% The increase in value attributable to the items included in subdivisions (a) to (o) that is known to the assessor and excluded from true cash value shall be indicated on the assessment roll.

% This subsection applies only to residential property.

% The following repairs are considered normal maintenance if they are not part of a structural addition or completion: [repairs (a)-(o) omitted]

% % (a) Outside painting.
% % (b) Repairing or replacing siding, roof, porches, steps, sidewalks, or drives.
% % (c) Repainting, repairing, or replacing existing masonry.
% % (d) Replacing awnings.
% % (e) Adding or replacing gutters and downspouts.
% % (f) Replacing storm windows or doors.
% % (g) Insulating or weatherstripping.
% % (h) Complete rewiring.
% % (i) Replacing plumbing and light fixtures.
% % (j) Replacing a furnace with a new furnace of the same type or replacing an oil or gas burner.
% % (k) Repairing plaster, inside painting, or other redecorating.
% % (l) New ceiling, wall, or floor surfacing.
% % (m) Removing partitions to enlarge rooms.
% % (n) Replacing an automatic hot water heater.
% % (o) Replacing dated interior woodwork.
% \end{quotation}

\section{Motion}

% \begin{quotation}
%   Rule 7.215 Opinions, Orders, Judgments, and Final Process for Court of Appeals
% (I) Reconsideration.

% (1) A motion for reconsideration may be filed within 21 days after the date of the order or the date stamped on an opinion. The motion shall include all facts, arguments, and citations to authorities in a single document and shall not exceed 10 double-spaced pages. A copy of the order or opinion of which reconsideration is sought must be included with the motion. Motions for reconsideration are subject to the restrictions contained in MCR 2.119(F)(3).

% (2) A party may answer a motion for reconsideration within 14 days after the motion is served on the party. An answer to a motion for reconsideration shall be a single document and shall not exceed 7 double-spaced pages.

% (3) The clerk will not accept for filing a motion for reconsideration of an order denying a motion for reconsideration.

% (4) The clerk will not accept for filing a late motion for reconsideration.
% \end{quotation}

Petioner moves for reconsideration of this Court's opinion of February 18, 2020, under MCR 7.215(I).

\section{Issues}

Did this Court make palpable errors in holding that:
\begin{enumerate}
\item \mathieuGast\ does not cover repairs by a new owner in the year of a sale,
\item the tribunal did not violate the law of the case, and 
\item the tribunal properly found that the repairs did not affect the property's value,
\item the tribunal's rejection of the purchase price was proper?
\end{enumerate}

% \begin{enumerate}
  
% \item Did this Court make a palpable error when it ruled that normal repairs under \cite[s]{MCL 211.27(2)}\ should be considered if they were done in the year of a transfer, even if the repairs were done by the new owner and there was no sale after the repairs?
 
% \item Is it palpable error for this Court to affirm the tribunal's ruling because the repairs did not affect the property's value?

% \item Does this Court's decision palpably violate the law of the case?
  
% \item Did this Court make a palpable error when it affirmed the tribunal's rejection of the purchase price?
% \end{enumerate}

\section{Standard of Review}

% MCL 2.119(F)(3) says:

% \begin{quote}
%   Generally, and without restricting the discretion of the court, a motion for rehearing or reconsideration which merely presents the same issues ruled on by the court, either expressly or by reasonable implication, will not be granted. The moving party must demonstrate a palpable error by which the court and the parties have been misled and show that a different disposition of the motion must result from correction of the error.
% \end{quote}

A motion for reconsideration ``must demonstrate a palpable error by which the court and the parties have been misled and show that a different disposition of the motion must result from correction of the error.''
\cite{MCL 2.119(F)(3)}.
 A palpable error is a clear error ``easily perceptible, plain, obvious, readily
 visible, noticeable, patent, distinct, manifest.''
 \pincite{Luckow}{426; 453}\ (cleaned up).
 \cite[s]{MCL 2.119(F)(3)}\ ``does not categorically prevent a trial court from revisiting an issue even when the motion for reconsideration presents the same issue already ruled upon; in fact, it allows considerable discretion to correct mistakes.''
\pincite{Macomb County Department of Human Services}{754}. 

 
% ``The palpable error provision in MCL 2.119(F)(3)  is not mandatory and only provides guidance to a court about when it may be appropriate to consider a motion for rehearing or reconsideration.'' \pincite{Walters}{350; 430}.

 % \begin{quote}
 %   The rule [MCL 2.119(F)(3)] does not categorically prevent a trial court from revisiting an issue even when the motion for reconsideration presents the same issue already ruled upon; in fact, it allows considerable discretion to correct mistakes.
 % \end{quote}
 % \pincite{Macomb County Department of Human Services}{754}. 

\section{Facts}
\label{facts}

Before it was bought by Appellant, the subject house was owned by HUD.
% (U.S. Dept. of Housing and Urban Development).
\mlsListing[]. 
The house had been listed on the MLS on and off since April 2013;
its initial asking price was \$29,900,
but by the time Appellant bought the house, the asking price was \$32,000,
which was the price Appellant paid in August 2015. \mlsHistory[].

The City of Wayne (Respondent/Appellee) required repairs before it allowed occupancy. %, it required repairs.
\repairs.

\begin{quote}
  It is undisputed that, when he purchased the property, it was in substandard condition and required numerous repairs to make it livable. Patru completed the required repairs on the property as of December 31, 2015.
\end{quote}
\pincite{Patru 1}{1}.

\subsection{First Appeal}

Appellee assessed the subject property for 2016 at a true cash value of \$50,400. Appellant appealed, arguing that Appellee's assessment was the after-repair value as of December 31, 2015.
Appellant contended that the before-repair value should be used. 
``Petitioner claims that under \mathieuGast\ we should not increase the subject's true cash value for normal repairs and maintenance until the subject is sold.'' \pincite{Patru 1}{2}, quoting the tribunal's POJ in the first appeal.

The Tax Tribunal referee ruled that \mathieuGast\ nonconsideration did not apply because the property was in substandard condition.
This Court reversed:

\begin{quote}
  The hearing referee incorrectly interpreted \mathieuGast\ by concluding that because repairs were done to a property in substandard condition, they did not constitute normal repairs. .~.~. This was improper .~.~. Nothing in \mathieuGast\ provides that the repairs .~.~. are not normal repairs in the event that they are performed on a substandard property. Thus by reading a requirement into the statute that was not stated by the legislature, the trial erred .~.~. See \pincite{Mich Ed}{218}\ (stating that nothing will be read into a clear statute that is not within the manifest intention of the Legislature as derived from the language of the statute itself).
\end{quote}
\pincite{Patru 1}{5}.

This Court noted that Appellant ``submitted a spreadsheet detailing the repairs he completed, which . . . included repairs that, under \mathieuGast, constitute normal repairs.'' \pincite{Patru 1}{4}.
However, the Tribunal had not considered this evidence because it was submitted after the hearing.

Because this Court could not determine ``whether the evidence Patru provided at the hearing was reflective of the information on the spreadsheet'' this Court remanded the case back to the tribunal. \pincite{Patru 1}{5}.

% the referee, not understanding how to properly apply \mathieuGast,
%``did not fully evaluate the evidence'' and ``the hearing was not transcribed''),
% This Court could not fully resolve the case because the tax tribunal had not ruled that the repairs met the specific criteria of \mathieuGast.
% The referee had misunderstood the statute and so had not determined whether Appellant's testimony about the repairs fell under the statute. 
% Petitioner had submitted a spreadsheet which did list the repairs and their specific \mathieuGast\ criteria, but the tribunal had refused to consider it because it was submitted after the hearing.
%So this Court remanded the case back to the tribunal. \pincite{Patru 1}{5-6}.

% \begin{quote}
%   \label{firstHolding}
% If the testimony provided was an oral recitation of the
% information included on the spreadsheet, then Patru presented testimony sufficient to establish
% that at least some of the repairs constituted normal repairs under MCL 211.27(2), and so the
% increase in TCV attributed to those repairs should not be considered in the property's TCV for
% assessment purposes until such time as Patru sells the property. However, if Patru merely
% testified that he did some carpentry, electrical, and masonry repairs and no further explanation of
% the work that was provided, then he would have arguably failed to support his claim. Either way,
% on the record before this Court, we cannot evaluate the sufficiency of the evidence presented at
% the hearing. Thus, we conclude that further proceedings are necessary in order to determine
% whether the repairs were normal repairs within the meaning of MCL 211.27(2). Accordingly, we
% remand to the Tax Tribunal for a rehearing. Further, because the existing record is insufficient to
% resolve whether the repairs are normal repairs within the meaning of the statute, the parties shall
% be afforded further opportunity to submit additional proofs. See \pincite{Fisher}{743}
% (requiring rehearing when it was not clear whether the
% proofs submitted were sufficient to establish that repair expenditures were normal repairs).
% \end{quote}
%\pincite{Patru 1}{5-6}.

\subsection{Second Appeal}

On remand, the tribunal ruled that the repairs were normal, but it again refused to apply \mathieuGast\ nonconsideration, this time because
the repairs were done in a year of a transfer. Opinion at 1. Also, the tribunal changed its previously undisputed finding that the property's before-repair condition was substandard. It now ruled that the repairs ``did not affect the assessed [true cash value] of the property .~.~.~.'' \Id. This Court has now affirmed. \Id.

\section{Argument}

Logically, the tribunal's decision rests on two independent grounds:

\begin{enumerate}
\item \mathieuGast\ does not apply in transfer years and the tribunal is not prohibited from considering this exception by the law of the case.
\item Even if nonconsideration is applied, the repairs did not affect value. Considering inflation, the before-repair value was essentially equal to the after-repair value. In particular, the subject's purchase price was not the before-repair value. 
\end{enumerate}

All of these propositions are proven false below. There is no exception to nonconsideration for transfer years. And even if there were such an exception, the law of the case requires that this should have been brought up on the first appeal. Second, the ``fact'' that the repairs did not affect value is not supported by the required evidence. And in particular, the tribunal did not properly consider the subject's purchase price. 

% Appellant asks that this Court reconsider this opinion, because this Court's answer to the first question contradicts the law of the case and also contradicts both the plain language of the statute and the tax act's overall scheme.

\subsection{The exception to MCL 211.27(2) is unsupported}

This court reviews the Tax Tribunal's statutory interpretation de novo. \pincite{Briggs}{75}.

This Court previously faulted the Tax Tribunal for violating the rule that ``nothing will be read into a clear statute that is not within the manifest intention of the Legislature as derived from the language of the statute itself.'' \pincite{Patru 1}{5}, quoting \pincite{Mich Ed}{218}. But this Court now makes the same error by reading into \mathieuGast\ an exception for repairs done in the year of a transfer. 

% The statute specifies that repairs should not considered ``\emph{until} the property is sold.'' This clearly indicates that nonconsideration ends only if there is a sale after the repairs. Here there was no sale after Appellant's repairs. Instead this Court would use a sale before the repairs to end the nonconsideration period. This violates the plain words of the statute.

The statute requires nonconsideration ``\emph{until} the property is sold.'' The word until means that there must be a sale after the repairs for nonconsideration to end. This Court's refusal to apply nonconsideration where there is no sale after the repairs is a violation of the plain words of the statute.

This Court also frustrates the intent of the statute which is to promote repairs.
In direct violation of the statute's plain language, this Court uses the interaction of at least two other statutes to exempt from nonconsideration the repairs of diligent buyers who start repairs right away.
Under the Court's view, they should wait until the year after the purchase so that the before-repair value can be determined on tax day.

% This Court would require that purchasers delay repairs until the year after the sale, to establish the unrepaired value as of December 31.
% Diligent buyers who start repairs too early
% % cannot rely on the plain text of statute, but
% are exempted from nonconsideration not by the plain language of the statute itself but via the complex interplay of at least two other statutes in the tax act.
% It seems odd that this was the legislature's intent.

% 
% By reading into the statute an unwritten exclusion for repairs in the year of purchase,
% this Court transforms Mathieu Gast
% from a pure incentive for homeowners to make repairs
% into a trap which punishes diligent purchasers who,
% relying on the plain words of the statute,
% repair right away. % rather than waiting until the year after purchase.
% Under this Court's interpretation, purchasers who wish to benefit from nonconsideration must delay repairs until the year after the sale, to establish the unrepaired value as of December 31.

%There is no evidence that in a situation like this, the legislature intended to incentivize the Appellant to delay repairs to establish the unrepaired value at the end of the transfer year, and then make the normal repairs in the following year. 

Where did this Court err? This Court is correct up to the proposition 
that the uncapped taxable value is based on the true cash value of the property in the year after the transfer.\footnote{The uncapped taxable value is the equalized assessed value (state equalized value). \cite{MCL 211.27a(3)}. The assessed value is one half of the true cash value. \cite{MCL 211.27a(1)}. The true cash value is always calculated as of tax day. ```true cash value' means the usual selling price .~.~. at the time of assessment,'' \cite{MCL 211.27(1)}. ``The taxable status of .~.~. real .~.~. property for a tax year shall be determined as of each December 31 of the immediately preceding year, which is considered the tax day,'' \cite{MCL 211.2(2)}.}
But it errs in equating true cash value with fair market value ``regardless of any ``normal repairs'' made by petitioner in [the year of purchase].'' Opinion at 5.

True cash value is defined in \cite{MCL 211.27}%
\footnote{Internally \cite[s]{MCL 211.27}\ may define true cash value in terms of itself. For example, \mathieuGast\ uses ``true cash value for assessment purposes'' to distinguish from the ``true cash value'' that increased due to normal repairs. But outside of \cite[s]{MCL 211.27}, in the rest of the tax act, there is only one ``true cash value'' which incorporates all of \cite[s]{MCL 211.27}.
  For example, assessed value is defined in \cite{MCL 211.27a(1)}\ as 50\% of the true cash value. Obviously with respect to \mathieuGast\ this means the ``true cash value for assessment purposes.'' But, as it is outside of \cite[s]{MCL 211.27}, \cite{MCL 211.27a(1)}\ uses the term ``true cash value.''}, 
which has eight subsections, which if reproduced here would span five pages. Courts have summarized true cash value as ``fair market value,'' based on the term ``usual selling price'' in subsection one. But this shorthand is used only in cases where a more complex definition is not needed. There is no case, except this Court's opinion here, that holds that the caselaw-derived term ``fair market value'' overrides or invalidates an applicable subsection of \cite[s]{MCL 211.27}.
% which holds that   defines true cash value as the ``usual selling price'' which is a good working definition for many cases. 
% in this brief. spanning several pages. which would span several pages if reproduced here. All of these subsections refine the meaning of true cash value. Subsection 1 uses the term ``usual selling price'' which caselaw says means fair market value.%
% \footnote{This Court's opinion at 4 cites \pincite{Pontiac Country Club}{434-435}\ for the proposition that ``TCV is synonymous with fair market value.'' There are many such cases because most cases do not require a more specific definition of true cash value. This Court has not cited any case which teaches that a relevant subsection of \cite{MCL 211.27}\ should be ignored in favor of the simple definition that true cash value is fair market value.}
% This is a good general definition but cannot always replace every aspect of a multiple page statute.
% In this case, subsection 2 adds a relevant refinement to the general definition.
% % Here this Court should consider subsection 2 which specifies that the value due to normal repairs is not part of true cash value until the property is sold.
% For this case, true cash value should mean ``the usual selling price .~.~. at the time of assessment,'' \cite{MCL 211.27(1)}, but not considering the value added by normal repairs, \mathieuGast.

% \footnote{\cite{\mathieuGast\ uses the term, ``true cash value for assessment purposes'' to distinguish from the ``true cash value'' of \cite{MCL 211.27(1)}\ which presumably may include the value of normal repairs and is thus not always suitable for assessment purposes. But read as a whole \cite{MCL 211.27}\ as a whole as defining ``true cash value for assessment purposes'' or just ``true cash value'' for short. Obviously the assessed value of \cite{MCL 211.27a(1)}\ uses this ``true cash value for assessment purposes'' because it is for assessment purposes.}

\mathieuGast, as part of the definition of true cash value, is not dependent on the capped or uncapped status of the taxable value defined in the next section, \cite{MCL 211.27a}. Rather, the dependency goes the other way. \mathieuGast\ always
operates to exclude the value of normal repairs no matter if true cash value is used to establish the assessed value of \cite{MCL 211.27a(1)}\ or the additions of \cite{MCL 211.27a(2)(b)}.

This Court's interpretation of Mathieu Gast is unsupported by the State Tax Commission. Its opinion on page 6 cites, with its own emphasis, \pincite{STC Bulletin}{3}:

\begin{quote}
The exemption for normal repairs, replacements and maintenance ends in the year after the
owner who made the repairs, replacements and maintenance sells the property. \emph{In the year
following a sale, the assessed value shall be based on the true cash value of the entire property.}
The amount of assessment increase attributable to the value of formerly exempt property
returning to the assessment roll is new for equalization purposes.
\end{quote}

This Court reads just the second, emphasized, sentence out of its context. The first sentence of the paragraph gives the topic: ``The exemption .~.~. ends in the year after \emph{the owner who made the repairs} .~.~. sells the property.'' In this context, the second sentence's ``entire property'' means the property with the seller's repairs considered.
The buyer's repairs are not in the paragraph's scope.%
\footnote{The opinion at 6 uses this out-of-context quote to justify its holding that before-repair and after-repairs appraisals were not required. As this passage does not say what this Court thinks it says, this Court's holding on this point is without support.}

\subsection{The law of the case requires reversal}

\begin{quote}
  Under the doctrine of the law of the case, if an appellate court has passed on a legal question and remanded the case for further proceedings, the legal question will not be differently determined in a subsequent appeal in the same case were the facts remain materially the same. The primary purpose of the law-of-the-case doctrine is to maintain consistency and avoid reconsideration of matters once decided during the course of a single  % 500*500
  continuing lawsuit. .~.~.  [T]he doctrine is discretionary, rather than mandatory.
\end{quote}
\pincite{Bennett}{499-500}.

The opinion at 5 gives two reasons why the law of the case does not apply. First,
this Court had not resolved whether ``the repairs were normal repairs within the meaning of \mathieuGast.'' This was not resolved because the existing record was insufficient. \pincite{Patru 1}{5}. But the opinion specified that if the Tribunal found that the repairs listed on the spreadsheet had been performed, these would constitute normal repairs and should not be considered as part of the true cash value:

\begin{quote}
  If the testimony provided was an oral recitation of the
information included on the spreadsheet, then Patru presented testimony sufficient to establish
that at least some of the repairs constituted normal repairs under MCL 211.27(2), and so the
increase in TCV attributed to those repairs should not be considered in the property's TCV for
assessment purposes until such time as Patru sells the property.
\end{quote}
\pincite{Patru 1}{5}. So on remand, the tribunal should have applied nonconsideration after it found that the repairs as listed on the spreadsheet were normal repairs.

Instead, despite finding that the repairs were normal, the tribunal did not apply nonconsideration because of the second reason: it now thought that the transfer of ownership and uncapping created an exception. Opinion at 5. %The teaching of \cite[s]{Bennett}\ is helpful here.

In \cite[s]{Bennett}, this Court was asked to reverse its earlier ruling based on a law which it had not explicitly considered the first time. This Court refused, reasoning that even though its first decision was wrong, by substituting its judgment for the prior appeals court panel, it would be doing ``the very activity the law-of-the-case doctrine is designed to discourage.'' \pincite[s]{Bennett}{501}. If the prior panel was not aware of the new law, the parties were at fault because ``ultimately it is the responsibility of the parties to bring to this Court's attention that case law and those statutes that the parties wish the Court to consider in deciding the matter.'' \pincite{Bennett}{501}.

% In \cite[s]{Bennett},
% the lower court had denied defendant's motion to terminate his child support early.
% While the case was on appeal, a law was enacted which supported the lower court's decision.
% The law took effect before the appeals court called the case.
% The appeals court reversed the lower court, but did not mention the new law.
% On remand, plaintiff asked the lower court to deny defendant's motion a second time,
% based on the new law which the appeals court had not addressed.
% The lower court refused because of the law of the case. Plaintiff appealed.

% The appeals court affirmed, reasoning that even though it thought that its first decision was wrong, by substituting its judgment for the prior appeals court panel, it would be doing ``the very activity the law-of-the-case doctrine is designed to discourage.'' \pincite{Bennett}{501}. If the prior panel was not aware of the new law, the parties were at fault because ``ultimately it is the responsibility of the parties to bring to this Court's attention that case law and those statutes that the parties wish the Court to consider in deciding the matter.'' \pincite{Bennett}{501}.

Thus \cite[s]{Bennett}\ teaches that the law of the case precludes even correct arguments if they could have been brought up in the first appeal. This is so here. Thus even if this Court agrees with the tribunal's latest decision,
%that \mathieuGast\ does not cover this case,
it should still reverse and affirm its earlier ruling.



% \cite[s]{Patru 1}\ because

% \begin{quote}
%   this Court did not resolve whether petitioner's 2015 repairs to the property could or could not be considered in determining the property's TCV for the 2016 tax year, but instead
% determined that ``further proceedings are necessary to determine whether the repairs were normal
% repairs within the meaning of MCL 211.27(2).'' .~.~. More significantly, this Court
% did not address the effect of the property's transfer of ownership in 2015 on the tribunal's
% consideration of ``normal repairs'' under MCL 211.27(2) for purposes of the 2016 tax year.
% Because this issue was not actually addressed and decided in the prior appeal, the law-of-the-case
% doctrine does not apply.
% \end{quote}
% Opinion at 5.
% % This Court does not fairly characterize its previous ruling.
% A more complete quote from this Court's prior opinion
% % (shown above at page \pageref{firstHolding})
% shows that this Court had decided that \mathieuGast\ applied to this case, but
% lacked a finding by the tribunal that the repairs listed in the spreadsheet were performed. 
% %, based on the incomplete record, could not decide if Appellant's repairs conformed to the specific categories of repairs listed in the statute.

% \begin{quote}
%   If the testimony provided was an oral recitation of the
% information included on the spreadsheet, then Patru presented testimony sufficient to establish
% that at least some of the repairs constituted normal repairs under MCL 211.27(2), and so the
% increase in TCV attributed to those repairs should not be considered in the property's TCV for
% assessment purposes until such time as Patru sells the property.
% \end{quote}
% \pincite{Patru 1}{5}. On remand, this Court expected the tribunal to determine which of the repairs fit the criteria of \mathieuGast, and then apply nonconsideration to those repairs, using before-repair and after-repair appraisals.

% \begin{quote}
%  We note that, on reconsideration, the Tribunal faulted Patru for failing to establish a pre-repair
% TCV. However, as the Tribunal must make its own, independent determination of TCV, Great
% Lakes Div of Nat'l Steel Corp v City of Ecorse, 227 Mich App 379, 389; 576 NW2d 667 (1998),
% we conclude that Patru's failure to persuade the Tribunal that the property's purchase price
% reflected the pre-repair TCV is irrelevant. The Tribunal independently had to evaluate all the
% evidence presented and, properly applying MCL 211.27(2), arrive at the property's TCV.)
% \end{quote}
% \pincite{Patru 1}{6}, footnote 3.

% This Court explained the law of the case in \pincite{Bennett}{499-500}:

% \begin{quote}
%   Under the doctrine of the law of the case, if an appellate court has passed on a legal question and remanded the case for further proceedings, the legal question will not be differently determined in a subsequent appeal in the same case were the facts remain materially the same. The primary purpose of the law-of-the-case doctrine is to maintain consistency and avoid reconsideration of matters once decided during the course of a single  % 500*500
%   continuing lawsuit. .~.~.  [T]he doctrine is discretionary, rather than mandatory.
% \end{quote}

% The law of the case precludes even correct arguments that could have been brought up in the first appeal.
% In \cite[s]{Bennett},
% the lower court had denied defendant's motion to terminate his child support early.
% While the case was on appeal, a law was enacted which supported the lower court's decision.
% The law took effect before the appeals court called the case.
% The appeals court reversed the lower court, but did not mention the new law.
% On remand, plaintiff asked the lower court to deny defendant's motion a second time,
% based on the new law which the appeals court had not addressed.
% The lower court refused because of the law of the case. Plaintiff appealed.

% The appeals court affirmed, reasoning that even though it thought that its first decision was wrong, by substituting its judgment for the prior appeals court panel, it would be doing ``the very activity the law-of-the-case doctrine is designed to discourage.'' \pincite{Bennett}{501}. If the prior panel was not aware of the new law, the parties were at fault because ``ultimately it is the responsibility of the parties to bring to this Court's attention that case law and those statutes that the parties wish the Court to consider in deciding the matter.'' \pincite{Bennett}{501}.

% Thus even if this Court agrees with the tribunal's latest decision,
% %that \mathieuGast\ does not cover this case,
% it should still reverse and affirm its earlier ruling. As in Bennett, Appellee could have made its arguments (that \mathieuGast\ does not apply if the repairs are made in the year of a transfer and that the property was in average condition before the repairs) before this Court on the first appeal.

\subsection{The ruling that the repairs did not affect the value is unsupported}

The tribunal's factual findings must be supported by ``competent, material, and substantial evidence on the whole record.'' \pincite{Pontiac Country Club}{434}. ``Substantial evidence supports the Tribunal's findings if a reasonable person would accept the evidence as sufficient to support the conclusion.'' \pincite{Pontiac Country Club}{434}. 

Besides ruling that \mathieuGast\ did not apply, % because of uncapping,
the tribunal also ruled that the repairs did not affect the property's true cash value, or equivalently, that the before-repair value was the same as the after-repair value. The opinion says at 5: ``The
tribunal did not credit petitioner's argument that the property was in substandard condition when
he purchased it.''

But the property's substandard condition is not merely Appellant's argument. Throughout the first trip to this Court until the tribunal's last Final Opinion, all the actors in this case, the parties, the tribunal and this Court relied on the well-established fact that the subject property was in substandard condition before the repairs.

\begin{quote}
  It is undisputed that, when he purchased the property, it was in substandard condition and required numerous repairs to make it livable.
\end{quote}
\pincite{Patru 1}{1}.
This case was first appealed to this Court because:

\begin{quote}
  [t]he hearing referee incorrectly interpreted \mathieuGast\ by concluding that because the repairs were done to a property in substandard condition, they did not constitute normal repairs.
\end{quote}
\pincite{Patru 1}{5}. 
% At the second hearing before the tribunal, neither Petitioner/Appellant nor Respondent/Appellee presented evidence or took the position that the house was not in substandard condition before repairs.
The tribunal's reversal of a fact which was foundational to its first ruling was not based on new evidence: neither party argued that the house was not in substandard condition before repairs. Therefore this Court should take a careful look at the tribunal's reasoning for this drastic reinterpretation of the evidence.

The tribunal relied on MLS photographs and the prior year's assessed value:

\begin{quotation}
The tribunal found that
petitioner's MLS listing for the subject property showed a property in ``average'' condition, and
that petitioner's photographs of the property, before any repairs, showed ``a property that is livable
and habitable with reasonable marketability and appeal.'' .~.~.
%The tribunal noted that the purpose of petitioner's repairs was ``to ready the property as a tenant rental.''
It is undisputed that the assessed
TCV of the property for the 2015 tax year was \$48,000.
\end{quotation}
Opinion at 5. The tribunal misused the MLS evidence. While an MLS sale may be used to support or contest a valuation, no principle of appraisal allows what the tribunal does here: reject an MLS sale as evidence of value, but then use the MLS sale for its pictures to argue that the sale price should have been higher. The MLS sale on its face supports a value \$32,000, but the tribunal says it supports a value of \$48,000. Thus, as used by the tribunal, the MLS photographs are not competent, material, and substantial evidence. 

Here, there is direct, uncontested evidence that contradicts the tribunal's opinion of the MLS photographs. After having actually examined the house, the city's inspectors did not think that the house was livable and habitable, but instead, they required repairs before they would allow occupancy.

Regarding the prior year's assessed value, Respondent/Appellee testified:

\begin{quote}
  [M]ass appraisal does not account for properties one-by-one. In other words, properties are assessed uniformly and \emph{Respondent assumes properties are in ``average'' condition.}
\end{quote}
\foj[4], emphasis added. The assessed value here is just the output of a mass appraisal computer program that assumes average condition. It is not evidence that the house was in average condition.

Also, by presuming that the assessed value was correct,  the tribunal violates the teaching of \pincite{Pontiac Country Club}{435-36}: ``The Tribunal may adopt the assessed valuation on the tax rolls as its independent finding of true cash value when competent and substantial evidence supports doing so, \emph{as long as it does not afford the original assessment presumptive validity.}'' Emphasis added.

\subsection{This Court wrongly affirms the tribunal's rejection of the purchase price}

When reviewing Tax Tribunal cases, this Court looks for misapplication of the law or adoption of a wrong principle. Factual findings must be supported by competent, material, and substantial evidence on the whole record. Statutory interpretation is reviewed de novo. \pincite{Briggs}{75; 757-758}.
%The Tribunal's factual findings are accepted as final by this Court ``provided they are supported by competent, material, and substantial evidence. Substantial evidence must be more than a scintilla of evidence, although it may be substantially less than a preponderance of the evidence.'' \pincite[s]{Jones & Laughlin}{352-53}\ (cleaned up).

% Regarding the tribunal's rejection of the sale price, Appellant is not complaining that ``the tribunal erred by rejecting his 2015 purchase price of the
% property as determinative of its TCV.'' Opinion at 6. Rather, Appellant's brief at 18-19 complains that the tribunal gave the sale price ``no weight or credibility,'' \reconsiderationDenied[2], contrary to the clear teaching of \pincite[s]{Jones & Laughlin}{353-354}: ``the price at which an item of
% property actually sold is most certainly relevant evidence'' and ``the tribunal's opinion that the evidence ``has little or no
% bearing'' on the property's earlier value suggests that the evidence was rejected out of hand. Such cursory rejection would be erroneous.''
% Appellant just wants the tribunal to consider the sale price and not reject it out of hand.

Contrary to this Court characterization,
% mischaracterizes Appellant's complaint about the tribunal's treatment of the subject's sale price.
Appellant is not complaining that ``the tribunal erred by rejecting his 2015 purchase price of the
property as determinative of its TCV.'' Opinion at 6. Rather, Appellant's brief at 18-19 complains that the tribunal gave the sale price ``no weight or credibility,'' \reconsiderationDenied[2], contrary to the teaching of \pincite[s]{Jones & Laughlin}{353-354}:

\begin{quotation}
  [T]he price at which an item of
  property actually sold is most certainly relevant evidence .~.~.
  the tribunal's opinion that the evidence ``has little or \emph{no}
bearing'' on the property's earlier value suggests that the evidence was rejected out of hand. Such cursory rejection would be erroneous.
\end{quotation}
Emphasis in original. Appellant just wants the tribunal to consider the sale price and not reject it out of hand.

The opinion at 7 asserts that the tribunal ``considered petitioner's evidence of the 2015 purchase price'' and also ``considered the nature of the sale,'' but there is no record of such consideration \emph{involving the particular facts of this case.}\footnote{In this case the tribunal could have considered, but did not, that the subject house (1) had been listed by an independent, licensed real estate broker on the MLS, (2) had sold more than two years after it was first listed, (3) had had several offers on it, (4) was not reduced in price to make it sell faster, (5) sold at a price consistent with the repairs that were required.}
The Tribunal gave the sale price ``no weight or credibility'' because the seller was HUD.%
\footnote{In support of its rejection of the sale price, the tribunal also cites \cite[s]{MCL 211.27(6)}, prohibiting assessors from presuming that the selling price is the true cash value. But \pincite[s]{Jones & Laughlin}{354}, specifically considered and rejected this as justifying a cursory rejection.}
This amounts to a per se rule which excludes from consideration the sale price of all HUD-owned homes. This is the kind of cursory rejection that \cite[s]{Jones & Laughlin}\ forbids. Especially troubling is that the tribunal justifies this per se rule with bare speculation: ``Because the subject was being sold by a government entity, that entity's motivation may
not have been to receive market value for the property.'' \reconsiderationDenied[2].

\section{Relief}

Therefore because this Court's opinion contains palpable errors which completely undermine its logic, Appellant respectfully asks this Court to reconsider this case.


\needspace{6\baselineskip}
% \section{Proof of Service}

% I certify that I served a copy of these Exceptions on Respondent's representative, Emily Pizzo, by email on the same day I emailed them to the tribunal.

\vspace{1\baselineskip}

{ \setlength{\leftskip}{3.5in}
%\singlespacing
  Respectfully Submitted,

  /s/ Daniel Patru, P74387

\today

  \setlength{\leftskip}{0pt}}


\newpage\empty% we need a new page so that the index entries on the last
        % page get written out to the right file.
\end{document}


%%% Local Variables:
%%% mode: latex
%%% TeX-master: t
%%% End:
