%
% This is one of the samples from the lawtex package:
% http://lawtex.sourceforge.net/
% LawTeX is licensed under the GNU General Public License 
%
\providecommand{\documentclassflag}{}
\documentclass[12pt,\documentclassflag]{michiganCourtOfAppealsBrief}
%\documentclass{article}


% for striking a row in a table, see https://tex.stackexchange.com/a/265728/135718
\usepackage{tikz}
\usetikzlibrary{tikzmark}

\makeandletter% use \makeandtab to turn off

% Use this to show a line grid-
% \usepackage[fontsize=12pt,baseline=24pt,lines=27]{grid}
% \usepackage{atbegshi,picture,xcolor} % https://tex.stackexchange.com/a/191004/135718
% \AtBeginShipout{%
%   \AtBeginShipoutUpperLeft{%
%     {\color{red}%
%     \put(\dimexpr -1in-\oddsidemargin,%
%          -\dimexpr 1in+\topmargin+\headheight+\headsep+\topskip)%
%       {%
%        \vtop to\dimexpr\vsize+\baselineskip{%
%          \hrule%
%          \leaders\vbox to\baselineskip{\hrule width\hsize\vfill}\vfill%
%        }%
%       }%
%   }}%
% }
%   \linespread{1}

\usepackage[modulo]{lineno}% use \linenumbers to show line numbers, see https://texblog.org/2012/02/08/adding-line-numbers-to-documents/

% allow underscores in words
\chardef\_=`_% https://tex.stackexchange.com/a/301984/135718 

%%Citations
 
%The command \makeandletter turns the ampersand into a printable character, rather than a special alignment tab \makeandletter

\begin{document}
\singlespacing%

\citecase[Helin]{Helin v Grosse Pointe Township, 329 Mich. 396; 45 N.W.2d 338 (1951)}
\citecase[Mich Ed]{Mich Ed Ass'n v Secretary of State
  (On Rehearing), 489 Mich 194, 218; 801 NW2d 35 (2011)}
% (stating that nothing will be read into a clear statute that is not within the manifest intention of the Legislature as derived from the language of the statute itself).

\citecase[Patru]{Patru v City of Wayne, unpublished per curiam opinion of the Court of Appeals, issued May 8, 2018 (Docket No. 337547)}%
\addReference{Patru}{patruvwayne}% Associate appendix with case

\citecase[Antisdale]{Antisdale v City of Galesburg, 420 Mich 265, 362 NW2d 632 (1984)}
\citecase[Briggs]{Briggs Tax Service, LLC v Detroit Pub. Schools, 485 Mich 69; 780 NW2d 753 (2010)}
\citecase[Jones & Laughlin]{Jones & Laughlin Steel Corporation v. City of Warren, 193 Mich App 348; 483 NW2nd 416 (1992)}

\citecase[Pontiac Country Club]{Pontiac Country Club v Waterford Twp, 299 Mich App 427; 830 NW2d 785 (2013)}
\citecase[Kar]{Kar v. Hogan, 399 Mich. 529; 251 N.W.2d 77  (1976)}
\citecase[CAF]{CAF Investment Company v. State Tax Commission, 392 Mich. 442; 221 N.W.2d 588 (1974)}
\citecase[Clark]{Clark Equipment Company v. Township Of Leoni, 113 Mich. App. 778; 318 N.W.2d 586 (1982)}
\citecase[Menard]{Menard, Inc. v Escanaba, 315 Mich. App. 512; 891 N.W.2d 1 (2016)}
\citecase[Walters]{People v Walters, 266 Mich App 341; 700 NW2d 424 (2005)}
\citecase[Luckow]{Luckow Estate v Luckow, 291 Mich App 417; 805 NW2d 453 (2011)}
\citecase[Stamp]{Stamp v Mill Street Inn, 152 Mich App 290; 393 NW2d 614 (1986)}
\citecase[Macomb County Department of Human Services]{Macomb County Department of Human Services v Anderson, 304 Mich App 750; 849 NW2d 408 (2014)}
\citecase[Berger]{Berger v. Berger, 747 N.W.2d 336; 277 Mich. App. 700 (2008)}

\citecase[Signature Villas]{Signature Villas, LLC v. City of Ann Arbor, 714 NW 2d 392 (Mich: Court of Appeals 2006)}
% the MTT rules of practice and procedure provide that "[i]f an applicable entire tribunal rule does not exist, the 1995 Michigan Rules of Court, as amended, . . . shall govern." 1999 AC, R 205.1111(4). Therefore, provisions of the Michigan Court Rules apply, where applicable, to MTT dispositions.

\citecase[FMB]{FMB -- First Mich. Bank v. Bailey, 232 Mich App 711; 591 NW 2d 676 (1998)}


% Rather, the apparent objective of MCR 2.114(E) and (F) is to deter parties and attorneys from filing documents or asserting claims and defenses that have not been sufficiently investigated and researched or that are intended to serve an improper purpose.

%The purpose of imposing sanctions under MCR 2.114, however, is to "deter parties and attorneys from filing documents or asserting claims and defenses that have not been sufficiently investigated and researched or that are intended to serve an improper purpose." FMB-First Michigan Bank v Bailey, 232 Mich App 711, 723; 591 NW2d 676, 682 (1998). 


\citecase[Bay City Yacht Club]{Bay City Yacht Club Inc v. Township of Bangor, unpublished per curiam opinion of the Court of Appeals, issued
  January 11th, 2018 (Docket No. 335551)}
\addReference{Bay City Yacht Club}{Bay City Yacht Club}% Associate appendix with c
% However, unlike the March 2016 letter, the August 2016 letter was received after the
% proofs were closed, and the Yacht Club was given no opportunity to rebut it because the Tribunal
% concluded that the letter was irrelevant to the issue at hand. On appeal, the Yacht Club
% challenges the decision to reopen the proofs, contending that the decision deprived it of due
% process because it was given no opportunity to respond to the evidence and was prejudiced by its
% admission. We agree that the admission of the letter without allowing the Yacht Club an
% opportunity to rebut it constituted a due process violation. When deciding whether to reopen
% proofs, a court should consider whether reopening the proofs would cause unfair surprise or
% prejudice to the opposing party. People v Collier, 168 Mich. App. 687, 694-695; 425 NW2d 118
% (1988). Here, there is no question that the admission of the August 2016 letter is prejudicial to
% the Yacht Club. The August letter concludes that, contrary to the March 2016 determination, the
% primary purpose of the Club’s bulkheads or seawalls is not for the prevention or control of
% erosion, which means that the Club cannot receive an exemption for the full value of the seawalls
% or bulkheads.2 Because the Club was plainly prejudiced by the reopening of the proofs to admit
% the letter, the Club should have been permitted to submit further proofs in rebuttal. See Reed v
% Reed, 265 Mich. App. 131, 159; 693 NW2d 825 (2005) (indicating that due process may not be
% satisfied when a party lacks a meaningful opportunity to be heard). See also Kok v Cascade
% Charter Twp, 255 Mich. App. 535, 544; 660 NW2d 389 (2003) (stating that the Tribunal did not
% err in excluding untimely evidence because the opposing party did not have an opportunity to
% evaluate it before the hearing).



\citecase[Lanzo]{Lanzo Construction v. City of Southfield, unpublished per curiam opinion of the Court of Appeals, issued June 28, 2007 (Docket No. 268567)}
\addReference{Lanzo}{Lanzo}% Associate appendix with c

%Petitioner argues that MCR 2.114 does not apply to proceedings before the Tax Tribunal because the Tribunal has its own provision, TTR 205.1145, regarding the awarding of costs to a prevailing party. The Tax Tribunal Rules "govern the practice and procedure in all cases and proceedings before the tribunal." TTR 205.1111(1). However, "[i]f an applicable entire tribunal rule does not exist, the 1995 Michigan Rules of Court, as amended, . . . shall govern." TTR 205.1111(4); Signature Villas, LLC v Ann Arbor, 269 Mich App 694, 705; 714 NW2d 392 (2006). TTR 205.1145, like MCR 2.625, addresses the awarding of costs to a prevailing party. The purpose of awarding costs is to reimburse the prevailing party the costs it paid during the course of the litigation. Wells v Dep't of Corrections, 447 Mich 415, 419; 523 NW2d 217 (1994). The purpose of imposing sanctions under MCR 2.114, however, is to "deter parties and attorneys from filing documents or asserting claims and defenses that have not been sufficiently investigated and researched or that are intended to serve an improper purpose." FMB-First Michigan Bank v Bailey, 232 Mich App 711, 723; 591 NW2d 676 (1998). Nothing in TTR 205.1145 or any other Tax Tribunal Rule addresses sanctions. Therefore, because no applicable Tax Tribunal Rule exists regarding sanctions, MCR 2.114 applies to proceedings before the Tax Tribunal. TTR 205.1111(4). Accordingly, because the Tax Tribunal found that petitioner's petition and motion for reconsideration were filed in violation of MCR 2.114(D), the Tax Tribunal erred when it failed to sanction petitioner, its counsel, or both. In re Forfeiture of Cash & Gambling Paraphernalia, supra at 73. We reverse the Tax Tribunal's September 10, 2004 order and all subsequent orders denying respondent's request for costs and attorney fees and remand for a hearing to determine an appropriate sanction.

{\makeatletter % needed for optional argument to newstatute.
  % \newstatute[1@]{MCL}{}% place MCL first
  % \newstatute[2@]{MCR}{}
  % \newstatute[3@]{TTR}{}
  % \newstatute[4@]{Dearborn Ordinance}{}% place this fourth
  % \newstatute[5@]{Wayne Ordinance}{}

  \newstatute{MCL 205.731}{}
  %Tax tribunal; jurisdiction.
%  The tribunal has exclusive and original jurisdiction over all of the following:
  % (a) A proceeding for direct review of a final decision, finding, ruling, determination, or order of an agency relating to assessment, valuation, rates, special assessments, allocation, or equalization, under the property tax laws of this state.

%  \newstatute{MCL 205.732}{}%
% 205.732 Tax tribunal; powers.
% Sec. 32.
%   The tribunal's powers include, but are not limited to, all of the following:
%   (a) Affirming, reversing, modifying, or remanding a final decision, finding, ruling, determination, or order of an agency.
%   (b) Ordering the payment or refund of taxes in a matter over which it may acquire jurisdiction.
%   (c) Granting other relief or issuing writs, orders, or directives that it deems necessary or appropriate in the process of disposition of a matter over which it may acquire jurisdiction.
%   (d) Promulgating rules for the implementation of this act, including rules for practice and procedure before the tribunal and for mediation as provided in section 47, under the administrative procedures act of 1969, 1969 PA 306, MCL 24.201 to 24.328.
%   (e) Mediating a proceeding before the tribunal.
%   (f) Certifying mediators to facilitate claims in the court of claims and in the tribunal.


  \newstatute{MCL 205.732(c)}{} % Powers of the tribunal.

  \newstatute{MCL 205.753(2)}{}% allows appeals from a final order of the Tax Tribunal


  
  \newstatute{MCL 211.10d(7)}{}%
%  (7) Every lawful assessment roll shall have a certificate attached signed by the certified assessor who prepared or supervised the preparation of the roll. The certificate shall be in the form prescribed by the state tax commission. If after completing the assessment roll the certified assessor for the local assessing district dies or otherwise becomes incapable of certifying the assessment roll, the county equalization director or the state tax commission shall certify the completed assessment roll at no cost to the local assessing district.

  \newstatute{MCL 211.24(1)}{}

    % 211.24 Property tax assessment roll; time; use of computerized database system.
% Sec. 24.

%   (1) On or before the first Monday in March in each year, the assessor shall make and complete an assessment roll, upon which he or she shall set down all of the following:
%   (a) The name and address of every person liable to be taxed in the local tax collecting unit with a full description of all the real property liable to be taxed. If the name of the owner or occupant of any tract or parcel of real property is known, the assessor shall enter the name and address of the owner or occupant opposite to the description of the property. If unknown, the real property described upon the roll shall be assessed as "owner unknown". All contiguous subdivisions of any section that are owned by 1 person, firm, corporation, or other legal entity and all unimproved lots in any block that are contiguous and owned by 1 person, firm, corporation, or other legal entity shall be assessed as 1 parcel, unless demand in writing is made by the owner or occupant to have each subdivision of the section or each lot assessed separately. However, failure to assess contiguous parcels as entireties does not invalidate the assessment as made. Each description shall show as near as possible the number of acres contained in it, as determined by the assessor. It is not necessary for the assessment roll to specify the quantity of land comprised in any town, city, or village lot.
%   (b) The assessor shall estimate, according to his or her best information and judgment, the true cash value and assessed value of every parcel of real property and set the assessed value down opposite the parcel.
%   (c) The assessor shall calculate the tentative taxable value of every parcel of real property and set that value down opposite the parcel.
%   (d) The assessor shall determine the percentage of value of every parcel of real property that is exempt from the tax levied by a local school district for school operating purposes to the extent provided under section 1211 of the revised school code, 1976 PA 451, MCL 380.1211, and set that percentage of value down opposite the parcel.
%   (e) The assessor shall determine the date of the last transfer of ownership of every parcel of real property occurring after December 31, 1994 and set that date down opposite the parcel.
%   (f) The assessor shall estimate the true cash value of all the personal property of each person, and set the assessed value and tentative taxable value down opposite the name of the person. In determining the property to be assessed and in estimating the value of that property, the assessor is not bound to follow the statements of any person, but shall exercise his or her best judgment. For taxes levied after December 31, 2003, the assessor shall separately state the assessed value and tentative taxable value of any leasehold improvements.
%   (g) Property assessed to a person other than the owner shall be assessed separately from the owner's property and shall show in what capacity it is assessed to that person, whether as agent, guardian, or otherwise. Two or more persons not being copartners, owning personal property in common, may each be assessed severally for each person's portion. Undivided interests in lands owned by tenants in common, or joint tenants not being copartners, may be assessed to the owners.
%   (2) Subject to this section, a local tax collecting unit may use a computerized database system as the assessment roll described in subsection (1) if the local tax collecting unit and the assessor certify in a form and manner prescribed by the state tax commission that the proposed system has the capacity to enable a local tax collecting unit to comply and the local tax collecting unit complies with all of the following requirements:
%   (a) The assessor shall certify the assessment roll and maintain a computer printed format or a disk, external drive, or other electronic data processing format compatible with the computer system used by the local tax collecting unit. The affidavit attached to or included with the assessment roll shall include documentation that the assessment roll has been backed up through a computer backup system and a sworn statement by the assessor that the backup system contains a true and complete record of the assessment roll. The affidavit attached to or included with the assessment roll shall include documentation that authorizes and reports all changes in the assessment roll as certified by the assessor.
%   (b) The local tax collecting unit shall certify and maintain a retention policy that complies with the requirements of section 11 of the Michigan history center act, 2016 PA 470, MCL 399.11, and section 491 of the Michigan penal code, 1931 PA 328, MCL 750.491.
%   (c) The local tax collecting unit shall certify that the computerized database system has internal and external security procedures sufficient to assure the integrity of the system.
%   (d) Not later than May 1 of the third year following the year in which a local tax collecting unit begins using a computerized database system as the assessment roll in accordance with this subsection and every 3 years thereafter, the local tax collecting unit shall certify to the state tax commission that the requirements of this subsection are being met.
%   (e) An assessor or local tax collecting unit that provides a computer terminal for public viewing of the assessment roll is considered as having the assessment roll available for public inspection.
%   (f) If at any time the state tax commission believes that a local tax collecting unit is no longer in compliance with this subsection, the state tax commission shall provide written notice to the local tax collecting unit. The notice shall specify the reasons that use of the computerized database system as the original assessment roll is no longer in compliance with this subsection. The local tax collecting unit has 60 days to provide evidence that the local tax collecting unit is in compliance with this subsection or that action to correct noncompliance has been implemented. If, after the expiration of 60 days, the state tax commission believes that the local tax collecting unit is not taking satisfactory steps to correct a condition of noncompliance, the state tax commission upon its own motion may withdraw approval of the use of the computerized database system as the original assessment roll. Proceedings of the state tax commission under this subsection shall be in accordance with rules for other proceedings for the commission promulgated under the administrative procedures act of 1969, 1969 PA 306, MCL 24.201 to 24.328, and shall not be considered a contested case.

  
  \newstatute]{MCL 211.27(1)}{}
  % 211.27 "True cash value" defined; considerations in determining value; indicating exclusions from true cash value on assessment roll; subsection (2) applicable only to residential property; repairs considered normal repairs, replacement, and maintenance; exclusions from real estate sales data; classification as agricultural real property; "present economic income" defined; applicability of subsection (5); "nonprofit cooperative housing corporation" defined; value of transferred property; “purchase price” defined; additional definitions; "standard tool" defined.
% Sec. 27.

%   (1) As used in this act, "true cash value" means the usual selling price at the place where the property to which the term is applied is at the time of assessment, being the price that could be obtained for the property at private sale, and not at auction sale except as otherwise provided in this section, or at forced sale. The usual selling price may include sales at public auction held by a nongovernmental agency or person if those sales have become a common method of acquisition in the jurisdiction for the class of property being valued. The usual selling price does not include sales at public auction if the sale is part of a liquidation of the seller's assets in a bankruptcy proceeding or if the seller is unable to use common marketing techniques to obtain the usual selling price for the property. A sale or other disposition by this state or an agency or political subdivision of this state of land acquired for delinquent taxes or an appraisal made in connection with the sale or other disposition or the value attributed to the property of regulated public utilities by a governmental regulatory agency for rate-making purposes is not controlling evidence of true cash value for assessment purposes. In determining the true cash value, the assessor shall also consider the advantages and disadvantages of location; quality of soil; zoning; existing use; present economic income of structures, including farm structures; present economic income of land if the land is being farmed or otherwise put to income producing use; quantity and value of standing timber; water power and privileges; minerals, quarries, or other valuable deposits not otherwise exempt under this act known to be available in the land and their value. In determining the true cash value of personal property owned by an electric utility cooperative, the assessor shall consider the number of kilowatt hours of electricity sold per mile of distribution line compared to the average number of kilowatt hours of electricity sold per mile of distribution line for all electric utilities. 
 
  \newstatute{MCL 211.27(2)}{}% MathieuGast
%   (2) The assessor shall not consider the increase in true cash value that is a result of expenditures for normal repairs, replacement, and maintenance in determining the true cash value of property for assessment purposes until the property is sold. For the purpose of implementing this subsection, the assessor shall not increase the construction quality classification or reduce the effective age for depreciation purposes, except if the appraisal of the property was erroneous before nonconsideration of the normal repair, replacement, or maintenance, and shall not assign an economic condition factor to the property that differs from the economic condition factor assigned to similar properties as defined by appraisal procedures applied in the jurisdiction. The increase in value attributable to the items included in subdivisions (a) to (o) that is known to the assessor and excluded from true cash value shall be indicated on the assessment roll. This subsection applies only to residential property. The following repairs are considered normal maintenance if they are not part of a structural addition or completion:
% (a) Outside painting.
% (b) Repairing or replacing siding, roof, porches, steps, sidewalks, or drives.
% (c) Repainting, repairing, or replacing existing masonry.
% (d) Replacing awnings.
% (e) Adding or replacing gutters and downspouts.
% (f) Replacing storm windows or doors.
% (g) Insulating or weatherstripping.
% (h) Complete rewiring.
% (i) Replacing plumbing and light fixtures.
% (j) Replacing a furnace with a new furnace of the same type or replacing an oil or gas burner.
% (k) Repairing plaster, inside painting, or other redecorating.
% (l) New ceiling, wall, or floor surfacing.
% (m) Removing partitions to enlarge rooms.
% (n) Replacing an automatic hot water heater.
% (o) Replacing dated interior woodwork.

  \newstatute{MCL 211.27(6)}{}%
% (6) Except as otherwise provided in subsection (7), the purchase price paid in a transfer of property is not the presumptive true cash value of the property transferred. In determining the true cash value of transferred property, an assessing officer shall assess that property using the same valuation method used to value all other property of that same classification in the assessing jurisdiction. As used in this subsection and subsection (7), "purchase price" means the total consideration agreed to in an arms-length transaction and not at a forced sale paid by the purchaser of the property, stated in dollars, whether or not paid in dollars.

   \newstatute{MCL 211.29(2)}{}%
%http://www.legislature.mi.gov/(S(mksbxq2iizpprava5c52qmqd))/mileg.aspx?page=getObject&objectName=mcl-206-1893-BOARD-OF-REVIEW. %  211.29 Board of review of township; meeting; submission, examination, and review of assessment roll; additions to roll; correction of errors; compliance with act; review of roll on tax day; prohibitions; entering valuations in separate columns; approval and adoption of roll; conducting business at public meeting; notice of meeting; notice of change in roll.
% Sec. 29.

%   (1) On the Tuesday immediately following the first Monday in March, the board of review of each township shall meet at the office of the supervisor, at which time the supervisor shall submit to the board the assessment roll for the current year, as prepared by the supervisor, and the board shall proceed to examine and review the assessment roll.
%   (2) During that day, and the day following, if necessary, the board, of its own motion, or on sufficient cause being shown by a person, shall add to the roll the names of persons, the value of personal property, and the description and value of real property liable to assessment in the township, omitted from the assessment roll. The board shall correct errors in the names of persons, in the descriptions of property upon the roll, and in the assessment and valuation of property. The board shall do whatever else is necessary to make the roll comply with this act.
%   (3) The roll shall be reviewed according to the facts existing on the tax day. The board shall not add to the roll property not subject to taxation on the tax day, and the board shall not remove from the roll property subject to taxation on that day regardless of a change in the taxable status of the property since that day.
%   (4) The board shall pass upon each valuation and each interest, and shall enter the valuation of each, as fixed by the board, in a separate column.
%   (5) The roll as prepared by the supervisor shall stand as approved and adopted as the act of the board of review, except as changed by a vote of the board. If for any cause a quorum does not assemble during the days above mentioned, the roll as prepared by the supervisor shall stand as if approved by the board of review.
%   (6) The business which the board may perform shall be conducted at a public meeting of the board held in compliance with Act No. 267 of the Public Acts of 1976, being sections 15.261 to 15.275 of the Michigan Compiled Laws. Public notice of the time, date, and place of the meeting shall be given in the manner required by Act No. 267 of the Public Acts of 1976. Notice of the date, time, and place of the meeting of the board of review shall be given at least 1 week before the meeting by publication in a generally circulated newspaper serving the area. The notice shall appear in 3 successive issues of the newspaper where available; otherwise, by the posting of the notice in 5 conspicuous places in the township.
%   (7) When the board of review makes a change in the assessment of property or adds property to the assessment roll, the person chargeable with the assessment shall be promptly notified in such a manner as will assure the person opportunity to attend the second meeting of the board of review provided in section 30.

 %\newstatute{MCL 211.34d}{}% defining additions, losses, and adjustments

  \newstatute{MCL 211.34d(1)(b)(i)}{}% defines omitted property

  \newstatute{MCL 211.154}{}% taxes for omitted property
  
  \newstatute{MCL 600.2591}{}
%   600.2591 Frivolous civil action or defense to civil action; awarding costs and fees to prevailing party; definitions.
% Sec. 2591.

%   (1) Upon motion of any party, if a court finds that a civil action or defense to a civil action was frivolous, the court that conducts the civil action shall award to the prevailing party the costs and fees incurred by that party in connection with the civil action by assessing the costs and fees against the nonprevailing party and their attorney.
%   (2) The amount of costs and fees awarded under this section shall include all reasonable costs actually incurred by the prevailing party and any costs allowed by law or by court rule, including court costs and reasonable attorney fees.
%   (3) As used in this section:
%   (a) “Frivolous” means that at least 1 of the following conditions is met:
%   (i) The party's primary purpose in initiating the action or asserting the defense was to harass, embarrass, or injure the prevailing party.
%   (ii) The party had no reasonable basis to believe that the facts underlying that party's legal position were in fact true.
%   (iii) The party's legal position was devoid of arguable legal merit.
%   (b) “Prevailing party” means a party who wins on the entire record.

  \newstatute{MCR 2.119(F)(1)}{}% Motions for reconsideration - 21 days
  \newstatute{MCR 2.119(F)(3)}{}% Motions for reconsideration - palpable error
  \newstatute{MCR 2.625}{}
%   Rule 2.625 Taxation of Costs
% (A) Right to Costs.

% (1) In General. Costs will be allowed to the prevailing party in an action, unless prohibited by statute or by these rules or unless the court directs otherwise, for reasons stated in writing and filed in the action.

% (2) Frivolous Claims and Defenses. In an action filed on or after October 1, 1986, if the court finds on motion of a party that an action or defense was frivolous, costs shall be awarded as provided by MCL 600.2591.

% (B) Rules for Determining Prevailing Party.

% (1) Actions With Several Judgments. If separate judgments are entered under MCR 2.116 or 2.505(A) and the plaintiff prevails in one judgment in an amount and under circumstances which would entitle the plaintiff to costs, he or she is deemed the prevailing party. Costs common to more than one judgment may be allowed only once.

% (2) Actions With Several Issues or Counts. In an action involving several issues or counts that state different causes of action or different defenses, the party prevailing on each issue or count may be allowed costs for that issue or count. If there is a single cause of action alleged, the party who prevails on the entire record is deemed the prevailing party.

% (3) Actions With Several Defendants. If there are several defendants in one action, and judgment for or dismissal of one or more of them is entered, those defendants are deemed prevailing parties, even though the plaintiff ultimately prevails over the remaining defendants.

% (4) Costs on Review in Circuit Court. An appellant in the circuit court who improves his or her position on appeal is deemed the prevailing party.

% (C) Costs in Certain Trivial Actions. In an action brought for damages in contract or tort in which the plaintiff recovers less than $100 (unless the recovery is reduced below $100 by a counterclaim), the plaintiff may recover costs no greater than the amount of damages.

% (D) Costs When Default or Default Judgment Set Aside. The following provisions apply to an order setting aside a default or a default judgment:

% (1) If personal jurisdiction was acquired over the defendant, the order must be conditioned on the defendant's paying or securing payment to the party seeking affirmative relief the taxable costs incurred in procuring the default or the default judgment and acting in reliance on it;

% (2) If jurisdiction was acquired by publication, the order may be conditioned on the defendant's paying or securing payment to the party seeking affirmative relief all or a part of the costs as the court may direct;

% (3) If jurisdiction was in fact not acquired, costs may not be imposed.

% (E) Costs in Garnishment Proceedings Brought Pursuant to 3.101(M). Costs in garnishment proceedings to resolve the dispute between a plaintiff and a garnishee regarding the garnishee's liability are allowed as in civil actions. Costs may be awarded to the garnishee defendant as follows:

% (1) The court may award the garnishee defendant as costs against the plaintiff reasonable attorney fees and other necessary expenses the garnishee defendant incurred in filing the disclosure, if the issue of the garnishee defendant's liability to the principal defendant is not brought to trial.

% (2) The court may award the garnishee defendant, against the plaintiff, the total costs of the garnishee defendant's defense, including all necessary expenses and reasonable attorney fees, if the issue of the garnishee defendant's liability to the principal defendant is tried and

% (a) the garnishee defendant is held liable in a sum no greater than that admitted in disclosure, or

% (b) the plaintiff fails to recover judgment against the principal defendant.

% In either (a) or (b), the garnishee defendant may withhold from the amount due the principal defendant the sum awarded for costs, and is chargeable only for the balance.

% (F) Procedure for Taxing Costs at the Time of Judgment.

% (1) Costs may be taxed by the court on signing the judgment, or may be taxed by the clerk as provided in this subrule.

% (2)When costs are to be taxed by the clerk, the party entitled to costs must present to the clerk, within 28 days after the judgment is signed, or within 28 days after entry of an order denying a motion for new trial, a motion to set aside the judgment, a motion for rehearing or reconsideration, or a motion for other postjudgment relief except a motion under MCR 2.612(C),

% (a) a bill of costs conforming to subrule (G),

% (b) a copy of the bill of costs for each other party, and

% (c) a list of the names and addresses of the attorneys for each party or of parties not represented by attorneys.

% In addition, the party presenting the bill of costs shall immediately serve a copy of the bill and any accompanying affidavits on the other parties. Failure to present a bill of costs within the time prescribed constitutes a waiver of the right to costs.

% (3) Within 14 days after service of the bill of costs, another party may file objections to it, accompanied by affidavits if appropriate. After the time for filing objections, the clerk must promptly examine the bill and any objections or affidavits submitted and allow only those items that appear to be correct, striking all charges for services that in the clerk's judgment were not necessary. The clerk shall notify the parties in the manner provided in MCR 2.107.

% (4) The action of the clerk is reviewable by the court on motion of any affected party filed within 7 days from the date that notice of the taxing of costs was sent, but on review only those affidavits or objections that were presented to the clerk may be considered by the court.

% (G) Bill of Costs; Supporting Affidavits.

% (1) Each item claimed in the bill of costs, except fees of officers for services rendered, must be specified particularly.

% (2) The bill of costs must be verified and must contain a statement that

% (a) each item of cost or disbursement claimed is correct and has been necessarily incurred in the action, and

% (b) the services for which fees have been charged were actually performed.

% (3) If witness fees are claimed, an affidavit in support of the bill of costs must state the distance traveled and the days actually attended. If fees are claimed for a party as a witness, the affidavit must state that the party actually testified as a witness on the days listed.

% (H) Taxation of Fees on Settlement. Unless otherwise specified a settlement is deemed to include the payment of any costs that might have been taxable.

% (I) Special Costs or Damages.

% (1) In an action in which the plaintiff's claim is reduced by a counterclaim, or another fact appears that would entitle either party to costs, to multiple costs, or to special damages for delay or otherwise, the court shall, on the application of either party, have that fact entered in the records of the court. A taxing officer may receive no evidence of the matter other than a certified copy of the court records or the certificate of the judge who entered the judgment.

% (2) Whenever multiple costs are awarded to a party, they belong to the party. Officers, witnesses, jurors, or other persons claiming fees for services rendered in the action are entitled only to the amount prescribed by law.

% (3) A judgment for multiple damages under a statute entitles the prevailing party to single costs only, except as otherwise specially provided by statute or by these rules.

% (J)    Costs in Headlee Amendment Suits. A plaintiff who prevails in an action brought pursuant to Const 1963, art 9, § 32 shall receive from the defendant the costs incurred by the plaintiff in maintaining the action as authorized by MCL 600.308a(1) and (6). Costs include a reasonable attorney fee.

% (K) Procedure for Taxing Costs and Fees After Judgment.

% (1) A judgment creditor considered a prevailing party to the action under subrule (B) may recover from the judgment debtor(s) the taxable costs and fees expended after a judgment is entered, including all taxable filing fees, service fees, certification fees, and any other costs, fees, and disbursements associated with postjudgment actions as allowed by MCL 600.2405.

% (2) Until the judgment is satisfied, the judgment debtor may serve on the judgment creditor a request to review postjudgment taxable costs and fees.

% (a) Within 28 days of receipt from a judgment debtor of a request to review postjudgment taxable costs and fees, the judgment creditor shall file with the court a memorandum of postjudgment taxable costs and fees and serve the same upon the judgment debtor. A memorandum of postjudgment taxable costs and fees shall include an itemized list of postjudgment taxable costs and fees. The memorandum must be verified by oath under MCR 1.109(D)(3).

% (b) Within 28 days after receiving the memorandum of postjudgment taxable costs and fees from the judgment creditor, the judgment debtor may file a motion to review postjudgment taxable costs and fees. Upon receipt of a timely motion, the court shall review the memorandum filed by the judgment creditor and issue an order allowing or disallowing the postjudgment costs and fees. The review may be conducted at a hearing at the court's discretion. If the court disallows the postjudgment costs and fees or otherwise amends them in favor of the judgment debtor, the court may order the judgment creditor to deduct from the judgment balance the amount of the motion fee paid by the judgment debtor under this rule.

% (c) The judgment creditor shall deduct any costs or fees disallowed by the court within 28 days after receipt of an order from the court disallowing the same.

% (d) Any error in adding costs or fees to the judgment balance by the judgment creditor or its attorney is not actionable unless there is an affirmative finding by the court that the costs and fees were added in bad faith.

  \newstatute{MCR 7.204(A)(1)(b)}{}% allows appeals within 21 days of an order on a motion for reconsideration
  \newstatute{TTR 219(5)}{}% Submissions by mail




 
}% end makeatletter block
% \def\mathieuGast{\pincite[l]{MCL}{211.27(2)}}
\def\mathieuGast{\cite[s]{MCL 211.27(2)}}%
%\def\ttr287{\pincite[s]{TTR}{287}}
%\def\inspectionOrdinance{\pincite{Wayne Ordinance}{\S1484.04}}
% \long\def\inspectionOrdinanceText{\begin{quote}
% 1484.04  CERTIFICATE REQUIRED PRIOR TO SALE. 
%    It shall be unlawful to sell, convey or transfer an ownership interest, or act as a broker or agent for the sale, conveyance or transfer of an ownership interest, in any residential dwelling unless and until a valid Certificate of Compliance is first issued. 
% (Ord. 1991-10.  Passed 7-16-91.) 
% \end{quote}}


%\newmisc{STC Bulletin 6 of 2007}{Michigan State Tax Commission (STC) Bulletin No. 6 of 2007 (Foreclosure Guidelines)}
\newmisc{STC Bulletin}{Michigan State Tax Commission (STC) Bulletin No. 7 of 2014 (Mathieu Gast Act)}
\addReference{STC Bulletin}{bulletin7}% Associate appendix with case

\newbook{The Appraisal of Real Estate}{Appraisal Institute}{The Appraisal of Real Estate}{(14th ed, Chicago: Appraisal Institute, 2013)}
\newbook{Appraising the Tough Ones}{Harrison}{Appraising the Tough Ones : Creative Ways to Value Residential Properties}{(Chicago: Appraisal Institute, 1996)}
\newbook{McCormick}{McCormick}{Evidence}{(2d ed)}
\newbook{Assessor's Manual}{Michigan State Tax Commission}{\href{https://www.michigan.gov/documents/treasury/1.__2014_Michigan_Assessors_Manual_Volume_I_Introduction_575738_7.pdf}{Assessor's Manual}}{(Vol 1 Residential, 2014)} 




\newbook{Wex}{Legal Information Institute}{\href{https://www.law.cornell.edu/wex}{Wex}}{<https://www.law.cornell.edu/wex> (accessed October 7, 2019)}

\newmisc{Respondent's Evidence}{Respondent's Evidence (entry line 16)}
\SetIndexType{Respondent's Evidence}{}
\newmisc{Chart of Median Price}{Chart of Ann Arbor Township Median Price Per Square Foot, Petitioner's Evidence (entry line 15)}
\SetIndexType{Chart of Median Price}{}
\newmisc{June 24 letter}{Petitioner's June 24 letter (entry line 11)}
\SetIndexType{June 24 letter}{}
\newmisc{July 10 letter}{Petitioner's July 10 letter (entry line 21)}
\SetIndexType{July 10 letter}{}
\newmisc{Form 865}{STC Form 865 (Mathieu-Gast Nonconsideration)}
\newmisc{Affidavit}{Affidavit of Carolyn Lepard (entry line 10)}
\SetIndexType{Affidavit}{}

\newmisc{POJ}{Proposed Opinion and Judgment (POJ)}
\SetIndexType{POJ}{}
\newmisc{Exceptions}{Exceptions (entry line 25)}
\SetIndexType{Exceptions}{}
\newmisc{FOJ1}{Final Opinion and Judgment (FOJ1)}
\newmisc{FOJ2}{Corrected Final Opinion and Judgment (FOJ2)}
\SetIndexType{FOJ1}{}
\SetIndexType{FOJ2}{}
\newmisc{Motion}{Motion to Correct Record Card (Motion) (entry line 32)}
\SetIndexType{Motion}{}
\newmisc{Reply}{Respondent's Reply to the Motion to Correct Record Card (Reply) (entry line 33)}
\SetIndexType{Reply}{}
\newmisc{Denial}{Order Denying Petitioner's Motion to Correct Record Card (Denial) (entry line 34)}
\SetIndexType{Denial}{}

\newcommand{\makeAbbreviation}[3]{% ensure that the frsit time an abbreviated word is used, it is presented in long form, and after that in short form. 1: command name, 2: short name, 3: long name
  \IfBeginWith{#3}{#2}{%
    \newcommand{#1}[0]{#3\renewcommand{#1}[0]{#2}}}{%
    \newcommand{#1}[0]{#3 (#2)\renewcommand{#1}[0]{#2}}}}

\makeAbbreviation{\MLS}{MLS}{Multiple Listing Service}
\makeAbbreviation{\MTT}{MTT}{Michigan Tax Tribunal}
\makeAbbreviation{\STC}{STC}{State Tax Commission}

% Note that the command/handle must match the appendix
% labels. Minimize the variations of names.  \makeAbbreviationToRecordq
% creates a simple abbreviation that ends in Abbr if you don't want to
% refer to the record.
\newcommand{\makeAbbreviationToRecord}[3]{% 1: command/handle, 2: shortname, 3: longname
  % makeAbbreviationToRecord: #1, #2, #3\par%
  \expandafter\makeAbbreviation\csname #1Abbr\endcsname{#2}{#3}%
  \expandafter\newcommand\csname #1\endcsname[1][]{%
    \Call{#1Abbr}%
    \if\relax##1\relax\empty\ (Appendix at \pageref{#1})%
    \else, p ##1 (Appendix at %
    % Check for page range
    \IfSubStr{##1}{-}{%
     \def\pageRefRange####1-####2XXX{\pageref{#1.####1}--\pageref{#1.####2}}%
      \pageRefRange##1XXX}%
    {\pageref{#1.##1}}%
    )\fi%
  }%
}%

% \makeAbbreviationToRecord{explanatoryLetter}{Explanatory Letter}{Explanatory Letter (MTT Docket Line 38)}
% % explanatory Letter: (abbr) \explanatoryLetterAbbr\ (to record) \explanatoryLetter[2] \par

% \makeAbbreviationToRecord{foj}{FOJ}{Second Final Opinion and Judgment (MTT Docket Line 48)}
% % FOJ: (abbr) \fojAbbr (to record) \foj[]\par

% \makeAbbreviationToRecord{reconsiderationDenied}{Order Denying Reconsideration}{Order Denying Reconsideration (MTT Docket Line 51)}
% % reconsiderationDenied: (abbr) \reconsiderationDeniedAbbr[] (to record) \reconsiderationDenied[] \par

% \makeAbbreviationToRecord{repairs}{List of Repairs}{List of Repairs (MTT Docket Line 36)}
% % \par repairs: (abbr) \repairsAbbr\ (to record) \repairs[] (to appendix) \pageref{repairs}\par

% \makeAbbreviationToRecord{stcform}{STC Form 865}{STC Form 865 Request for Nonconsideration (MTT Docket Line 35)}

% \makeAbbreviationToRecord{mlsListing}{MLS Listing}{MLS Listing (MTT Docket Line 9)}
% % mlsListing: (abbr) \mlsListingAbbr\ (to record) \mlsListing[]\par

% \makeAbbreviationToRecord{boardOfReviewDecision}{Board of Review Decision}{Board of Review Decision (MTT Docket Line 2)}

% \makeAbbreviationToRecord{cityEvidence}{City's Evidence}{City's Evidence (MTT Docket Line 11)}

% \makeAbbreviationToRecord{motionForReconsideration}{Motion for Reconsideration}{Motion for Reconsideration (MTT Docket Line 52)}


\begin{centering}
\bf\scshape State of Michigan\\Tax Tribunal Small Claims\\~\\ 
\rm 

\makeandtab
\setlength{\tabcolsep}{10pt}%
\begin{tabular}{p{.4\textwidth} p{.4\textwidth}}
\cline{1-2}
  {~

  \raggedright Andre Haerian,\par
  % \hfill
  \hspace{.1\textwidth}\textit{Petitioner/Appellant,}
  \vspace{.4\baselineskip}\par
  vs\par
  \vspace{.4\baselineskip}
  \raggedright Ann Arbor Charter Township,\par
  % \hfill
  \hspace{.1\textwidth}\textit{Respondent/Appellee.}
  
  ~} &  {~
       \par\par%
       %\hfill%
       \noindent Tax Tribunal No. 19-000445  \vspace{.5\baselineskip}\par
       % \hfill%
       %\raggedleft
       \textbf{Motion to reconsider the order denying the motion to correct the record card}\vspace{.5\baselineskip}\par
       % \hfill%
       % \raggedleft
       \textbf{Brief }\vspace{.5\baselineskip}\par
       % \hfill
       \textbf{Proof of Service}\newline      
  ~}
  \\ \cline{1-2}\vspace{2mm}
  {~ \par
  Andre Haerian, Petitioner\newline
  390 Meadow Creek,\newline
  Ann Arbor Twp, MI 48105\newline \newline
  
  Daniel Patru, P74387, \newline%
  Attorney for Petitioner\newline%
  3309 Solway\newline%
  Knoxville, TN 37931\newline%
  (734) 274-9624\newline%
  dpatru@gmail.com\newline\newline%
  ~} & {~ \par~\par
       
       Ann Arbor Charter Twp, Appellee\newline%
       3792 Pontiac Trail,\newline%
       Ann Arbor, MI 48105-9656\newline%
       (734) 663-3418\newline\newline%

       Emily Pizzo\newline%
       Ann Arbor Twp Deputy Assessor\newline%
       3792 Pontiac Trail,\newline%
       Ann Arbor, MI 48105-9656\newline%
  (734) 663-8540\newline
  assessor@aatwp.org
       
  ~}
\end{tabular}
\makeandletter
\par\vspace{\baselineskip}\vspace{\baselineskip}\vspace{\baselineskip}
%\textbf{ORAL ARGUMENT NOT REQUESTED}

\end{centering}

\pagestyle{romanparen}
\pagenumbering{roman}
\newpage 

\section*{Table of Contents}

\tableofcontents


%\newpage
\tableofauthorities

\pagestyle{plain}
\pagenumbering{arabic}

%Sets the formatting for the entire document after the front matter
\parindent=2.5em
% \setlength{\parskip}{1.25ex plus 2ex minus .5ex} 
% \setstretch{1.45}
\doublespacing
% \linenumbers


% \section{Mathieu-Gast Statute -- MCL 211.27(2)}
% \begin{quotation}
% The assessor shall not consider the increase in true cash value that is a result of expenditures for normal repairs, replacement, and maintenance in determining the true cash value of property for assessment purposes until the property is sold.

% For the purpose of implementing this subsection, the assessor shall not increase the construction quality classification or reduce the effective age for depreciation purposes, except if the appraisal of the property was erroneous before nonconsideration of the normal repair, replacement, or maintenance, and shall not assign an economic condition factor to the property that differs from the economic condition factor assigned to similar properties as defined by appraisal procedures applied in the jurisdiction.

% The increase in value attributable to the items included in subdivisions (a) to (o) that is known to the assessor and excluded from true cash value shall be indicated on the assessment roll.

% This subsection applies only to residential property.

% The following repairs are considered normal maintenance if they are not part of a structural addition or completion: [repairs (a)-(o) omitted]

% % (a) Outside painting.
% % (b) Repairing or replacing siding, roof, porches, steps, sidewalks, or drives.
% % (c) Repainting, repairing, or replacing existing masonry.
% % (d) Replacing awnings.
% % (e) Adding or replacing gutters and downspouts.
% % (f) Replacing storm windows or doors.
% % (g) Insulating or weatherstripping.
% % (h) Complete rewiring.
% % (i) Replacing plumbing and light fixtures.
% % (j) Replacing a furnace with a new furnace of the same type or replacing an oil or gas burner.
% % (k) Repairing plaster, inside painting, or other redecorating.
% % (l) New ceiling, wall, or floor surfacing.
% % (m) Removing partitions to enlarge rooms.
% % (n) Replacing an automatic hot water heater.
% % (o) Replacing dated interior woodwork.
% \end{quotation}

% \subsection{mcl 211.34}

% (2) . . . ``. . . The county board of commissioners shall also make alterations in the description of any land on the rolls as is necessary to render the descriptions conformable to the requirements of this act. . . .''

% (3) The county board of commissioners of a county shall establish and maintain a department to survey assessments and assist the board of commissioners in the matter of equalization of assessments, and may employ in that department technical and clerical personnel which in its judgment are considered necessary. The personnel of the department shall be under the direct supervision and control of a director of the tax or equalization department who may designate an employee of the department as his or her deputy. The director of the county tax or equalization department shall be appointed by the county board of commissioners. The county board of commissioners, through the department, may furnish assistance to local assessing officers in the performance of duties imposed upon those officers by this act, including the development and maintenance of accurate property descriptions, the discovery, listing, and valuation of properties for tax purposes, and the development and use of uniform valuation standards and techniques for the assessment of property.

% 211.34a Tabular statement of tentative equalization ratios and estimated multipliers; preparation; publication; copies, notices; effect on equalization procedures; appeal.
% Sec. 34a.

% (1) The equalization director of each county shall prepare a tabular statement each year, by the several cities and townships of the county, showing the tentative recommended equalization ratios and estimated multipliers necessary to compute individual state equalized valuation of real property and of personal property. The county shall publish the tabulation in a newspaper of general circulation within the county on or before the third Monday in February each year and furnish a copy to each assessor and to each of the boards of review in the county and to the state tax commission. All notices of meetings of the boards of review shall give the tentative ratios and estimated multipliers pertaining to their jurisdiction. The tentative recommended equalization ratios and multiplying figures shall not prejudice the equalization procedures of the county board of commissioners or the state tax commission.


% \subsection{MCL 211.34d Additions, Losses, and Adjustments}
% 211.34d Definitions; tabulation of tentative taxable value; computation of amounts; calculation of millage reduction fraction; transmittal of computations; delivery of signed statement; certification; tax levy; limitation on number of mills; application of millage reduction fraction or limitation; voter approval of tax levy; incorrect millage reduction fraction; recalculation and rounding of fractions; publication of inflation rate; permanent reduction in maximum rates.
% Sec. 34d.

%   (1) As used in this section or section 27a, or section 3 or 31 of article IX of the state constitution of 1963:
%   (a) For taxes levied before 1995, "additions" means all increases in value caused by new construction or a physical addition of equipment or furnishings, and the value of property that was exempt from taxes or not included on the assessment unit's immediately preceding year's assessment roll.
%   (b) For taxes levied after 1994, "additions" means, except as provided in subdivision (c), all of the following:
%   (i) Omitted real property. As used in this subparagraph, "omitted real property" means previously existing tangible real property not included in the assessment. Omitted real property shall not increase taxable value as an addition unless the assessing jurisdiction has a property record card or other documentation showing that the omitted real property was not previously included in the assessment. The assessing jurisdiction has the burden of proof in establishing whether the omitted real property is included in the assessment. Omitted real property for the current and the 2 immediately preceding years, discovered after the assessment roll has been completed, shall be added to the tax roll pursuant to the procedures established in section 154. For purposes of determining the taxable value of real property under section 27a, the value of omitted real property is based on the value and the ratio of taxable value to true cash value the omitted real property would have had if the property had not been omitted.
%   (ii) Omitted personal property. As used in this subparagraph, "omitted personal property" means previously existing tangible personal property not included in the assessment. Omitted personal property shall be added to the tax roll pursuant to section 154.
%   (iii) New construction. As used in this subparagraph, "new construction" means property not in existence on the immediately preceding tax day and not replacement construction. New construction includes the physical addition of equipment or furnishings, subject to the provisions set forth in section 27(2)(a) to (p). For purposes of determining the taxable value of property under section 27a, the value of new construction is the true cash value of the new construction multiplied by 0.50.
%   (iv) Previously exempt property. As used in this subparagraph, "previously exempt property" means property that was exempt from ad valorem taxation under this act on the immediately preceding tax day but is subject to ad valorem taxation on the current tax day under this act. For purposes of determining the taxable value of real property under section 27a:
%   (A) The value of property previously exempt under section 7u is the taxable value the entire parcel of property would have had if that property had not been exempt, minus the product of the entire parcel's taxable value in the immediately preceding year and the lesser of 1.05 or the inflation rate.
%   (B) The taxable value of property that is a facility as that term is defined in section 2 of 1974 PA 198, MCL 207.552, that was previously exempt under section 7k is the taxable value that property would have had under this act if it had not been exempt.
%   (C) The value of property previously exempt under any other section of law is the true cash value of the previously exempt property multiplied by 0.50.
%   (v) Replacement construction. As used in this subparagraph, "replacement construction" means construction that replaced property damaged or destroyed by accident or act of God and that occurred after the immediately preceding tax day to the extent the construction's true cash value does not exceed the true cash value of property that was damaged or destroyed by accident or act of God in the immediately preceding 3 years. Except as otherwise provided in this subparagraph, for purposes of determining the taxable value of property under section 27a, the value of the replacement construction is the true cash value of the replacement construction multiplied by a fraction, the numerator of which is the taxable value of the property to which the construction was added in the immediately preceding year and the denominator of which is the true cash value of the property to which the construction was added in the immediately preceding year, and then multiplied by the lesser of 1.05 or the inflation rate. However, after December 31, 2011, for purposes of determining the taxable value of property under section 27a, if the property's replacement construction is of substantially the same materials as determined by the state tax commission, if the square footage is not more than 5\% greater than the property that was damaged or destroyed, and if the replacement construction is completed not later than December 31 in the year 3 years after the accident or act of God occurred, the replacement construction's taxable value shall be equal to the taxable value of the property in the year immediately preceding the year in which the property was damaged or destroyed, adjusted annually as provided in section 27a(2). Any construction materials required to bring the property into compliance with any applicable health, sanitary, zoning, safety, fire, or construction codes or ordinances shall be considered to be substantially the same materials by the state tax commission for the sake of replacement construction under this section.
%   (vi) An increase in taxable value attributable to the complete or partial remediation of environmental contamination existing on the immediately preceding tax day. The department of environmental quality shall determine the degree of remediation based on information available in existing department of environmental quality records or information made available to the department of environmental quality if the appropriate assessing officer for a local tax collecting unit requests that determination. The increase in taxable value attributable to the remediation is the increase in true cash value attributable to the remediation multiplied by a fraction, the numerator of which is the taxable value of the property had it not been contaminated and the denominator of which is the true cash value of the property had it not been contaminated.
%   (vii) Public services. As used in this subparagraph, "public services" means water service, sewer service, a primary access road, natural gas service, electrical service, telephone service, sidewalks, or street lighting. For purposes of determining the taxable value of real property under section 27a, the value of public services is the amount of increase in true cash value of the property attributable to the available public services multiplied by 0.50, and shall be added in the calendar year following the calendar year when those public services are initially available.
%   (c) For taxes levied after 1994, additions do not include increased value attributable to any of the following:
%   (i) Platting, splits, or combinations of property.
%   (ii) A change in the zoning of property.
%   (iii) For the purposes of the calculation of the millage reduction fraction under subsection (7) only, increased taxable value under section 27a(3) after a transfer of ownership of property.
%   (d) "Assessed valuation of property as finally equalized" means taxable value under section 27a.
%   (e) "Financial officer" means the officer responsible for preparing the budget of a unit of local government.
%   (f) "General price level" means the annual average of the 12 monthly values for the United States Consumer Price Index for all urban consumers as defined and officially reported by the United States Department of Labor, Bureau of Labor Statistics.
%   (g) For taxes levied before 1995, "losses" means a decrease in value caused by the removal or destruction of real or personal property and the value of property taxed in the immediately preceding year that has been exempted or removed from the assessment unit's assessment roll.
%   (h) For taxes levied after 1994, "losses" means, except as provided in subdivision (i), all of the following:
%   (i) Property that has been destroyed or removed. For purposes of determining the taxable value of property under section 27a, the value of property destroyed or removed is the product of the true cash value of that property multiplied by a fraction, the numerator of which is the taxable value of that property in the immediately preceding year and the denominator of which is the true cash value of that property in the immediately preceding year.
%   (ii) Property that was subject to ad valorem taxation under this act in the immediately preceding year that is now exempt from ad valorem taxation under this act. For purposes of determining the taxable value of property under section 27a, the value of property exempted from ad valorem taxation under this act is the amount exempted.
%   (iii) Prior to December 31, 2013, an adjustment in value, if any, because of a decrease in the property's occupancy rate, to the extent provided by law. For purposes of determining the taxable value of real property under section 27a, the value of a loss for a decrease in the property's occupancy rate is the product of the decrease in the true cash value of the property attributable to the decreased occupancy rate multiplied by a fraction, the numerator of which is the taxable value of the property in the immediately preceding year and the denominator of which is the true cash value of the property in the immediately preceding year.
%   (iv) A decrease in taxable value attributable to environmental contamination existing on the immediately preceding tax day. The department of environmental quality shall determine the degree to which environmental contamination limits the use of property based on information available in existing department of environmental quality records or information made available to the department of environmental quality if the appropriate assessing officer for a local tax collecting unit requests that determination. The department of environmental quality's determination of the degree to which environmental contamination limits the use of property shall be based on the criteria established for the categories set forth in section 20120a(1) of the natural resources and environmental protection act, 1994 PA 451, MCL 324.20120a. The decrease in taxable value attributable to the contamination is the decrease in true cash value attributable to the contamination multiplied by a fraction, the numerator of which is the taxable value of the property had it not been contaminated and the denominator of which is the true cash value of the property had it not been contaminated.
%   (i) For taxes levied after 1994, losses do not include decreased value attributable to either of the following:
%   (i) Platting, splits, or combinations of property.
%   (ii) A change in the zoning of property.
%   (j) "New construction and improvements" means additions less losses.
%   (k) "Current year" means the year for which the millage limitation is being calculated.
%   (l) "Inflation rate" means the ratio of the general price level for the state fiscal year ending in the calendar year immediately preceding the current year divided by the general price level for the state fiscal year ending in the calendar year before the year immediately preceding the current year.
%   (2) On or before the first Monday in May of each year, the assessing officer of each township or city shall tabulate the tentative taxable value as approved by the local board of review and as modified by county equalization for each classification of property that is separately equalized for each unit of local government and provide the tabulated tentative taxable values to the county equalization director. The tabulation by the assessing officer shall contain additions and losses for each classification of property that is separately equalized for each unit of local government or part of a unit of local government in the township or city. If as a result of state equalization the taxable value of property changes, the assessing officer of each township or city shall revise the calculations required by this subsection on or before the Friday following the fourth Monday in May. The county equalization director shall compute these amounts and the current and immediately preceding year's taxable values for each classification of property that is separately equalized for each unit of local government that levies taxes under this act within the boundary of the county. The county equalization director shall cooperate with equalization directors of neighboring counties, as necessary, to make the computation for units of local government located in more than 1 county. The county equalization director shall calculate the millage reduction fraction for each unit of local government in the county for the current year. The financial officer for each taxing jurisdiction shall calculate the compounded millage reduction fractions beginning in 1980 resulting from the multiplication of successive millage reduction fractions and shall recognize a local voter action to increase the compounded millage reduction fraction to a maximum of 1 as a new beginning fraction. Upon request of the superintendent of the intermediate school district, the county equalization director shall transmit the complete computations of the taxable values to the superintendent of the intermediate school district within that county. At the request of the presidents of community colleges, the county equalization director shall transmit the complete computations of the taxable values to the presidents of community colleges within the county.
%   (3) On or before the first Monday in June of each year, the county equalization director shall deliver the statement of the computations signed by the county equalization director to the county treasurer.
%   (4) On or before the second Monday in June of each year, the treasurer of each county shall certify the immediately preceding year's taxable values, the current year's taxable values, the amount of additions and losses for the current year, and the current year's millage reduction fraction for each unit of local government that levies a property tax in the county.
%   (5) The financial officer of each unit of local government shall make the computation of the tax rate using the data certified by the county treasurer and the state tax commission. At the annual session in October, or, for a county or local tax collecting unit that approves under section 44a(2) the accelerated collection in a summer property tax levy of a millage that had been previously billed and collected as in a preceding tax year as part of the winter property tax levy, before a special meeting held before the annual levy on July 1, the county board of commissioners shall not authorize the levy of a tax unless the governing body of the taxing jurisdiction has certified that the requested millage has been reduced, if necessary, in compliance with section 31 of article IX of the state constitution of 1963.
%   (6) The number of mills permitted to be levied in a tax year is limited as provided in this section pursuant to section 31 of article IX of the state constitution of 1963. A unit of local government shall not levy a tax rate greater than the rate determined by reducing its maximum rate or rates authorized by law or charter by a millage reduction fraction as provided in this section without voter approval.
%   (7) A millage reduction fraction shall be determined for each year for each local unit of government. For ad valorem property taxes that became a lien before January 1, 1983, the numerator of the fraction shall be the total state equalized valuation for the immediately preceding year multiplied by the inflation rate and the denominator of the fraction shall be the total state equalized valuation for the current year minus new construction and improvements. For ad valorem property taxes that become a lien after December 31, 1982 and through December 31, 1994, the numerator of the fraction shall be the product of the difference between the total state equalized valuation for the immediately preceding year minus losses multiplied by the inflation rate and the denominator of the fraction shall be the total state equalized valuation for the current year minus additions. For ad valorem property taxes that are levied after December 31, 1994, the numerator of the fraction shall be the product of the difference between the total taxable value for the immediately preceding year minus losses multiplied by the inflation rate and the denominator of the fraction shall be the total taxable value for the current year minus additions. For each year after 1993, a millage reduction fraction shall not exceed 1.
%   (8) The compounded millage reduction fraction shall be calculated by multiplying the local unit's previous year's compounded millage reduction fraction by the current year's millage reduction fraction. The compounded millage reduction fraction for the year shall be multiplied by the maximum millage rate authorized by law or charter for the unit of local government for the year, except as provided by subsection (9). A compounded millage reduction fraction shall not exceed 1.
%   (9) The millage reduction shall be determined separately for authorized millage approved by the voters. The limitation on millage authorized by the voters on or before April 30 of a year shall be calculated beginning with the millage reduction fraction for that year. Millage authorized by the voters after April 30 shall not be subject to a millage reduction until the year following the voter authorization which shall be calculated beginning with the millage reduction fraction for the year following the authorization. The first millage reduction fraction used in calculating the limitation on millage approved by the voters after January 1, 1979 shall not exceed 1.
%   (10) A millage reduction fraction shall be applied separately to the aggregate maximum millage rate authorized by a charter and to each maximum millage rate authorized by state law for a specific purpose.
%   (11) A unit of local government may submit to the voters for their approval the levy in that year of a tax rate in excess of the limit set by this section. The ballot question shall ask the voters to approve the levy of a specific number of mills in excess of the limit. The provisions of this section do not allow the levy of a millage rate in excess of the maximum rate authorized by law or charter. If the authorization to levy millage expires after 1993 and a local governmental unit is asking voters to renew the authorization to levy the millage, the ballot question shall ask for renewed authorization for the number of expiring mills as reduced by the millage reduction required by this section. If the election occurs before June 1 of a year, the millage reduction is based on the immediately preceding year's millage reduction applicable to that millage. If the election occurs after May 31 of a year, the millage reduction shall be based on that year's millage reduction applicable to that millage had it not expired.
%   (12) A reduction or limitation under this section shall not be applied to taxes imposed for the payment of principal and interest on bonds or other evidence of indebtedness or for the payment of assessments or contract obligations in anticipation of which bonds are issued that were authorized before December 23, 1978, as provided by section 4 of chapter I of former 1943 PA 202, or to taxes imposed for the payment of principal and interest on bonds or other evidence of indebtedness or for the payment of assessments or contract obligations in anticipation of which bonds are issued that are approved by the voters after December 22, 1978.
%   (13) If it is determined subsequent to the levy of a tax that an incorrect millage reduction fraction has been applied, the amount of additional tax revenue or the shortage of tax revenue shall be deducted from or added to the next regular tax levy for that unit of local government after the determination of the authorized rate pursuant to this section.
%   (14) If as a result of an appeal of county equalization or state equalization the taxable value of a unit of local government changes, the millage reduction fraction for the year shall be recalculated. The financial officer shall effectuate an addition or reduction of tax revenue in the same manner as prescribed in subsection (13).
%   (15) The fractions calculated pursuant to this section shall be rounded to 4 decimal places, except that the inflation rate shall be computed by the state tax commission and shall be rounded to 3 decimal places. The state tax commission shall publish the inflation rate before March 1 of each year.
%   (16) Beginning with taxes levied in 1994, the millage reduction required by section 31 of article IX of the state constitution of 1963 shall permanently reduce the maximum rate or rates authorized by law or charter. The reduced maximum authorized rate or rates for 1994 shall equal the product of the maximum rate or rates authorized by law or charter before application of this section multiplied by the compounded millage reduction applicable to that millage in 1994 pursuant to subsections (8) to (12). The reduced maximum authorized rate or rates for 1995 and each year after 1995 shall equal the product of the immediately preceding year's reduced maximum authorized rate or rates multiplied by the current year's millage reduction fraction and shall be adjusted for millage for which authorization has expired and new authorized millage approved by the voters pursuant to subsections (8) to (12).

  

\section{Introduction}

Petitioner has already won on the main issue in this case, determining the true cash value of the subject. 
This motion for reconsideration concerns a collateral issue of this case:

\begin{quote} Whether Respondent should be free to leave the subject's record card in a state inconsistent with the Tribunal's ruling and even to add false information to the record card so that it is even less reflective of the Tribunal's ruling than when the case began.
\end{quote}

On November 8, 2019, Petitioner filed a motion to order Respondent to correct the 2019 record card to conform to the findings of fact and ruling in this case.
On November 25, 2019, Tribunal Judge Steven M. Bieda denied the motion for two reasons:

\begin{enumerate}
\item The case had been closed with respect to the 2019 tax year with issuance of the Corrected Final Opinion and Judgement, issued October 17, 2019, and
\item The Tribunal did not have jurisdiction over the 2020 tax year.
\end{enumerate}

Petitioner has filed this motion for reconsideration because he believes that the Tribunal's denial was palpably wrong. The fact that the main issue of the case was decided and that the case was ``closed'' is not a good reason for ignoring Respondent's actions which contradict and undermine the Tribunal's ruling in this case. Nor would the Tribunal be doing something unusual here because it regularly hears and grants motions after a case is closed.

Furthermore, although Petitioner has filed the motion to prevent future tax appeals, the motion concerns the subject's 2019 record card, over which the Tribunal already has jurisdiction. The Tribunal has not cited any authority allowing it to deny a motion because the applicant may also obtain a remedy by filing a motion in a future case.

Finally, Respondent's reply, which was written after Petitioner filed his motion, provides additional reason for granting the motion. First, Respondent intends to double count missing items already valued by the Tribunal. And second,  Respondent falsified the record card so as to corrupt the results of the ECF study.


% This case was initially filed because Petitioner disagreed Respondent's 2019 TCV which was \$373,900 (35.6\%) higher than the subject's market sale of \$1,050,000 ten months before tax day. That issue has been resolved for Petitioner. In a Corrected Final Opinion and Judgement, issued October 17, 2019, this Tribunal ruled that the true cash value was its sale price. Moreover, the Tribunal ruled that Respondent's evidence, its sales comparison analysis, should be given no weight. \pincite{FOJ2}{3}.

% However, a new issue has arisen since this Tribunal's resolution of the main issue in this case: Respondent is trying to avoid the consequences of this Tribunal's decision by both failing to update the subject's record card and updating it with false information.


% Respondent filed a reply on November 12, 2019. The reply did not contradict Petitioner's legal arguments, but instead explained that its changes 1) would not add additional value to the taxable value and 2) would exclude the subject from the Economic Condition Factor (ECF) study.

% On November 25, 2019, Tribunal Judge Steven M. Bieda denied the motion for two reasons:

% \begin{enumerate}
% \item The case had been closed with respect to the 2019 tax year with issuance of the Corrected Final Opinion and Judgement, issued October 17, 2019, and
% \item The Tribunal did not have jurisdiction over the 2020 tax year.
% \end{enumerate}

% Petitioner has filed this motion for reconsideration because he believes that the Tribunal's denial was palpably wrong. The fact that the main issue of the case was decided and that the case was ``closed'' is not a good reason for ignoring Respondent's actions which contradict and undermine the Tribunal's ruling in this case. Nor would the Tribunal be doing something unusual here because it regularly hears and grants motions after a case is closed.

% Furthermore, although Petitioner has filed the motion to prevent future tax appeals, the motion concerns the subject's 2019 record card, over which the Tribunal already has jurisdiction. The Tribunal has not cited any authority allowing it to deny a motion because the applicant may also obtain a remedy by filing a motion in a future case.

% Finally, Petitioner would point out that Respondent's reply does not only not contradict the arguments in Petitioner's motion, but provides reasons for granting Petitioner's motion. Here are two of them:

% \begin{enumerate}
% \item Respondent's short memory is certain to cause either more litigation or result in a wrong assessment. Respondent's reply assumes that features of the property which were not on the record card were not included in the true cash value determined by this Tribunal. This despite the fact that the true cash value was based on the subject's sale, the MLS listing of which was the source from which Respondent determined that the record card was incomplete.
% \item Respondent's reply admits that the record card's terms of sale were changed to a false value to avoid inclusion of the subject's sale in the ECF study. This corrupted the ECF study. Thus Respondent's failure to maintain a true record card is corrupting other aspects of property tax system, beyond just the subject.
% \end{enumerate}


% Respondent uses the term ``adjustments'' incorrectly. \cite{MCL 211.34d}\ defines the terms ``addition,'' ``loss,'' and ``adjustment,'' 

% additions = omitted real property, omitted personal property, new construction, previously exempt property, replacement property, remediation of contaminated property, public services,

% losses = destroyed or removed property, property newly exempt, decrease in value due to environmental contamination,

%   (b) For taxes levied after 1994, "additions" means, except as provided in subdivision (c), all of the following:
%   (i) Omitted real property. As used in this subparagraph, "omitted real property" means previously existing tangible real property not included in the assessment. Omitted real property shall not increase taxable value as an addition unless the assessing jurisdiction has a property record card or other documentation showing that the omitted real property was not previously included in the assessment. The assessing jurisdiction has the burden of proof in establishing whether the omitted real property is included in the assessment. Omitted real property for the current and the 2 immediately preceding years, discovered after the assessment roll has been completed, shall be added to the tax roll pursuant to the procedures established in section 154. For purposes of determining the taxable value of real property under section 27a, the value of omitted real property is based on the value and the ratio of taxable value to true cash value the omitted real property would have had if the property had not been omitted.
%   (ii) Omitted personal property. As used in this subparagraph, "omitted personal property" means previously existing tangible personal property not included in the assessment. Omitted personal property shall be added to the tax roll pursuant to section 154.
%   (iii) New construction. As used in this subparagraph, "new construction" means property not in existence on the immediately preceding tax day and not replacement construction. New construction includes the physical addition of equipment or furnishings, subject to the provisions set forth in section 27(2)(a) to (p). For purposes of determining the taxable value of property under section 27a, the value of new construction is the true cash value of the new construction multiplied by 0.50.
%   (iv) Previously exempt property. As used in this subparagraph, "previously exempt property" means property that was exempt from ad valorem taxation under this act on the immediately preceding tax day but is subject to ad valorem taxation on the current tax day under this act. For purposes of determining the taxable value of real property under section 27a:
%   (A) The value of property previously exempt under section 7u is the taxable value the entire parcel of property would have had if that property had not been exempt, minus the product of the entire parcel's taxable value in the immediately preceding year and the lesser of 1.05 or the inflation rate.
%   (B) The taxable value of property that is a facility as that term is defined in section 2 of 1974 PA 198, MCL 207.552, that was previously exempt under section 7k is the taxable value that property would have had under this act if it had not been exempt.
%   (C) The value of property previously exempt under any other section of law is the true cash value of the previously exempt property multiplied by 0.50.
%   (v) Replacement construction. As used in this subparagraph, "replacement construction" means construction that replaced property damaged or destroyed by accident or act of God and that occurred after the immediately preceding tax day to the extent the construction's true cash value does not exceed the true cash value of property that was damaged or destroyed by accident or act of God in the immediately preceding 3 years. Except as otherwise provided in this subparagraph, for purposes of determining the taxable value of property under section 27a, the value of the replacement construction is the true cash value of the replacement construction multiplied by a fraction, the numerator of which is the taxable value of the property to which the construction was added in the immediately preceding year and the denominator of which is the true cash value of the property to which the construction was added in the immediately preceding year, and then multiplied by the lesser of 1.05 or the inflation rate. However, after December 31, 2011, for purposes of determining the taxable value of property under section 27a, if the property's replacement construction is of substantially the same materials as determined by the state tax commission, if the square footage is not more than 5\% greater than the property that was damaged or destroyed, and if the replacement construction is completed not later than December 31 in the year 3 years after the accident or act of God occurred, the replacement construction's taxable value shall be equal to the taxable value of the property in the year immediately preceding the year in which the property was damaged or destroyed, adjusted annually as provided in section 27a(2). Any construction materials required to bring the property into compliance with any applicable health, sanitary, zoning, safety, fire, or construction codes or ordinances shall be considered to be substantially the same materials by the state tax commission for the sake of replacement construction under this section.
%   (vi) An increase in taxable value attributable to the complete or partial remediation of environmental contamination existing on the immediately preceding tax day. The department of environmental quality shall determine the degree of remediation based on information available in existing department of environmental quality records or information made available to the department of environmental quality if the appropriate assessing officer for a local tax collecting unit requests that determination. The increase in taxable value attributable to the remediation is the increase in true cash value attributable to the remediation multiplied by a fraction, the numerator of which is the taxable value of the property had it not been contaminated and the denominator of which is the true cash value of the property had it not been contaminated.
%   (vii) Public services. As used in this subparagraph, "public services" means water service, sewer service, a primary access road, natural gas service, electrical service, telephone service, sidewalks, or street lighting. For purposes of determining the taxable value of real property under section 27a, the value of public services is the amount of increase in true cash value of the property attributable to the available public services multiplied by 0.50, and shall be added in the calendar year following the calendar year when those public services are initially available.
%   (c) For taxes levied after 1994, additions do not include increased value attributable to any of the following:
%   (i) Platting, splits, or combinations of property.
%   (ii) A change in the zoning of property.
%   (iii) For the purposes of the calculation of the millage reduction fraction under subsection (7) only, increased taxable value under section 27a(3) after a transfer of ownership of property.
%   (d) "Assessed valuation of property as finally equalized" means taxable value under section 27a.
%   (e) "Financial officer" means the officer responsible for preparing the budget of a unit of local government.
%   (f) "General price level" means the annual average of the 12 monthly values for the United States Consumer Price Index for all urban consumers as defined and officially reported by the United States Department of Labor, Bureau of Labor Statistics.
%   (g) For taxes levied before 1995, "losses" means a decrease in value caused by the removal or destruction of real or personal property and the value of property taxed in the immediately preceding year that has been exempted or removed from the assessment unit's assessment roll.
%   (h) For taxes levied after 1994, "losses" means, except as provided in subdivision (i), all of the following:
%   (i) Property that has been destroyed or removed. For purposes of determining the taxable value of property under section 27a, the value of property destroyed or removed is the product of the true cash value of that property multiplied by a fraction, the numerator of which is the taxable value of that property in the immediately preceding year and the denominator of which is the true cash value of that property in the immediately preceding year.
%   (ii) Property that was subject to ad valorem taxation under this act in the immediately preceding year that is now exempt from ad valorem taxation under this act. For purposes of determining the taxable value of property under section 27a, the value of property exempted from ad valorem taxation under this act is the amount exempted.
%   (iii) Prior to December 31, 2013, an adjustment in value, if any, because of a decrease in the property's occupancy rate, to the extent provided by law. For purposes of determining the taxable value of real property under section 27a, the value of a loss for a decrease in the property's occupancy rate is the product of the decrease in the true cash value of the property attributable to the decreased occupancy rate multiplied by a fraction, the numerator of which is the taxable value of the property in the immediately preceding year and the denominator of which is the true cash value of the property in the immediately preceding year.
%   (iv) A decrease in taxable value attributable to environmental contamination existing on the immediately preceding tax day. The department of environmental quality shall determine the degree to which environmental contamination limits the use of property based on information available in existing department of environmental quality records or information made available to the department of environmental quality if the appropriate assessing officer for a local tax collecting unit requests that determination. The department of environmental quality's determination of the degree to which environmental contamination limits the use of property shall be based on the criteria established for the categories set forth in section 20120a(1) of the natural resources and environmental protection act, 1994 PA 451, MCL 324.20120a. The decrease in taxable value attributable to the contamination is the decrease in true cash value attributable to the contamination multiplied by a fraction, the numerator of which is the taxable value of the property had it not been contaminated and the denominator of which is the true cash value of the property had it not been contaminated.
%   (i) For taxes levied after 1994, losses do not include decreased value attributable to either of the following:
%   (i) Platting, splits, or combinations of property.
%   (ii) A change in the zoning of property.
%   (j) "New construction and improvements" means additions less losses.
%   (k) "Current year" means the year for which the millage limitation is being calculated.
%   (l) "Inflation rate" means the ratio of the general price level for the state fiscal year ending in the calendar year immediately preceding the current year divided by the general price level for the state fiscal year ending in the calendar year before the year immediately preceding the current year.
%   (2) On or before the first Monday in May of each year, the assessing officer of each township or city shall tabulate the tentative taxable value as approved by the local board of review and as modified by county equalization for each classification of property that is separately equalized for each unit of local government and provide the tabulated tentative taxable values to the county equalization director. The tabulation by the assessing officer shall contain additions and losses for each classification of property that is separately equalized for each unit of local government or part of a unit of local government in the township or city. If as a result of state equalization the taxable value of property changes, the assessing officer of each township or city shall revise the calculations required by this subsection on or before the Friday following the fourth Monday in May. The county equalization director shall compute these amounts and the current and immediately preceding year's taxable values for each classification of property that is separately equalized for each unit of local government that levies taxes under this act within the boundary of the county. The county equalization director shall cooperate with equalization directors of neighboring counties, as necessary, to make the computation for units of local government located in more than 1 county. The county equalization director shall calculate the millage reduction fraction for each unit of local government in the county for the current year. The financial officer for each taxing jurisdiction shall calculate the compounded millage reduction fractions beginning in 1980 resulting from the multiplication of successive millage reduction fractions and shall recognize a local voter action to increase the compounded millage reduction fraction to a maximum of 1 as a new beginning fraction. Upon request of the superintendent of the intermediate school district, the county equalization director shall transmit the complete computations of the taxable values to the superintendent of the intermediate school district within that county. At the request of the presidents of community colleges, the county equalization director shall transmit the complete computations of the taxable values to the presidents of community colleges within the county.
%   (3) On or before the first Monday in June of each year, the county equalization director shall deliver the statement of the computations signed by the county equalization director to the county treasurer.
%   (4) On or before the second Monday in June of each year, the treasurer of each county shall certify the immediately preceding year's taxable values, the current year's taxable values, the amount of additions and losses for the current year, and the current year's millage reduction fraction for each unit of local government that levies a property tax in the county.
%   (5) The financial officer of each unit of local government shall make the computation of the tax rate using the data certified by the county treasurer and the state tax commission. At the annual session in October, or, for a county or local tax collecting unit that approves under section 44a(2) the accelerated collection in a summer property tax levy of a millage that had been previously billed and collected as in a preceding tax year as part of the winter property tax levy, before a special meeting held before the annual levy on July 1, the county board of commissioners shall not authorize the levy of a tax unless the governing body of the taxing jurisdiction has certified that the requested millage has been reduced, if necessary, in compliance with section 31 of article IX of the state constitution of 1963.
%   (6) The number of mills permitted to be levied in a tax year is limited as provided in this section pursuant to section 31 of article IX of the state constitution of 1963. A unit of local government shall not levy a tax rate greater than the rate determined by reducing its maximum rate or rates authorized by law or charter by a millage reduction fraction as provided in this section without voter approval.
%   (7) A millage reduction fraction shall be determined for each year for each local unit of government. For ad valorem property taxes that became a lien before January 1, 1983, the numerator of the fraction shall be the total state equalized valuation for the immediately preceding year multiplied by the inflation rate and the denominator of the fraction shall be the total state equalized valuation for the current year minus new construction and improvements. For ad valorem property taxes that become a lien after December 31, 1982 and through December 31, 1994, the numerator of the fraction shall be the product of the difference between the total state equalized valuation for the immediately preceding year minus losses multiplied by the inflation rate and the denominator of the fraction shall be the total state equalized valuation for the current year minus additions. For ad valorem property taxes that are levied after December 31, 1994, the numerator of the fraction shall be the product of the difference between the total taxable value for the immediately preceding year minus losses multiplied by the inflation rate and the denominator of the fraction shall be the total taxable value for the current year minus additions. For each year after 1993, a millage reduction fraction shall not exceed 1.
%   (8) The compounded millage reduction fraction shall be calculated by multiplying the local unit's previous year's compounded millage reduction fraction by the current year's millage reduction fraction. The compounded millage reduction fraction for the year shall be multiplied by the maximum millage rate authorized by law or charter for the unit of local government for the year, except as provided by subsection (9). A compounded millage reduction fraction shall not exceed 1.
%   (9) The millage reduction shall be determined separately for authorized millage approved by the voters. The limitation on millage authorized by the voters on or before April 30 of a year shall be calculated beginning with the millage reduction fraction for that year. Millage authorized by the voters after April 30 shall not be subject to a millage reduction until the year following the voter authorization which shall be calculated beginning with the millage reduction fraction for the year following the authorization. The first millage reduction fraction used in calculating the limitation on millage approved by the voters after January 1, 1979 shall not exceed 1.
%   (10) A millage reduction fraction shall be applied separately to the aggregate maximum millage rate authorized by a charter and to each maximum millage rate authorized by state law for a specific purpose.
%   (11) A unit of local government may submit to the voters for their approval the levy in that year of a tax rate in excess of the limit set by this section. The ballot question shall ask the voters to approve the levy of a specific number of mills in excess of the limit. The provisions of this section do not allow the levy of a millage rate in excess of the maximum rate authorized by law or charter. If the authorization to levy millage expires after 1993 and a local governmental unit is asking voters to renew the authorization to levy the millage, the ballot question shall ask for renewed authorization for the number of expiring mills as reduced by the millage reduction required by this section. If the election occurs before June 1 of a year, the millage reduction is based on the immediately preceding year's millage reduction applicable to that millage. If the election occurs after May 31 of a year, the millage reduction shall be based on that year's millage reduction applicable to that millage had it not expired.
%   (12) A reduction or limitation under this section shall not be applied to taxes imposed for the payment of principal and interest on bonds or other evidence of indebtedness or for the payment of assessments or contract obligations in anticipation of which bonds are issued that were authorized before December 23, 1978, as provided by section 4 of chapter I of former 1943 PA 202, or to taxes imposed for the payment of principal and interest on bonds or other evidence of indebtedness or for the payment of assessments or contract obligations in anticipation of which bonds are issued that are approved by the voters after December 22, 1978.
%   (13) If it is determined subsequent to the levy of a tax that an incorrect millage reduction fraction has been applied, the amount of additional tax revenue or the shortage of tax revenue shall be deducted from or added to the next regular tax levy for that unit of local government after the determination of the authorized rate pursuant to this section.
%   (14) If as a result of an appeal of county equalization or state equalization the taxable value of a unit of local government changes, the millage reduction fraction for the year shall be recalculated. The financial officer shall effectuate an addition or reduction of tax revenue in the same manner as prescribed in subsection (13).
%   (15) The fractions calculated pursuant to this section shall be rounded to 4 decimal places, except that the inflation rate shall be computed by the state tax commission and shall be rounded to 3 decimal places. The state tax commission shall publish the inflation rate before March 1 of each year.
%   (16) Beginning with taxes levied in 1994, the millage reduction required by section 31 of article IX of the state constitution of 1963 shall permanently reduce the maximum rate or rates authorized by law or charter. The reduced maximum authorized rate or rates for 1994 shall equal the product of the maximum rate or rates authorized by law or charter before application of this section multiplied by the compounded millage reduction applicable to that millage in 1994 pursuant to subsections (8) to (12). The reduced maximum authorized rate or rates for 1995 and each year after 1995 shall equal the product of the immediately preceding year's reduced maximum authorized rate or rates multiplied by the current year's millage reduction fraction and shall be adjusted for millage for which authorization has expired and new authorized millage approved by the voters pursuant to subsections (8) to (12).










% Petitioner moves for reconsideration of his \cite{Motion}, filed November 8, 2019, and denied on November 25, 2019. 

% This motion is a response to Respondent's disregard of MCL 211.24(1) and this Tribunal's decision in this case. MCL 211.24(1) requires that ``the assessor shall make and complete an assessment roll, upon which he or she shall set down . . . a full description of all real property liable to be taxed.'' This Tribunal's final decision in this case, issued in a Corrected Final Opinion and Judgement on October 17, 2019, valued the subject based on its sale price, which this Tribunal found to be the subject's market price at the time of sale and on tax day. The Tribunal's  decision ordered that ``the officer charged with maintaining the assessment
% rolls for the tax year(s) at issue shall correct or cause the assessment rolls to be
% corrected to reflect the property's true cash and taxable values as provided in this
% Corrected Final Opinion and Judgment within 20 days of entry of this Corrected Final
% Opinion and Judgment, subject to the processes of equalization.''

% Instead of complying with the statute and this Tribunal's order, Respondent decided not to update the subject's record card to include a full description of the property. Respondent left out items (e.g., an elevator) that were present in the subject at its sale but not in the record card. Also, Respondent, without informing this Tribunal or Petitioner, changed the ``Terms of Sale'' from ``CONVENTIONAL SALE'' to ``NO CONSIDERATION.'' Thus the subject's record card is now wrong and misleading. Rather than showing that the sale price was the market value for the full property, the record card leaves out expensive items from the record card and shows that the sale price was unrelated to the property's market value. 

% As a result of Respondent's disregard for statute and this Tribunal's order, the subject's record card for this year is in a state that would allow Respondent to add back the value of the excluded items next year and inclease both the Assessed Value and Taxable Values. (Respondent has admitted in writing that this is what it intends to do. ``Respondent's agent . . . notified [Petitioner's Agent] that the items which were excluded from the record card were to be added to the property for the 2020 years''

\section{Motion}

Petitioner moves for reconsideration of his \cite[s]{Motion}, filed November 8, 2019, and denied on November 25, 2019. 

This motion for reconsideration is proper under \cite[s]{MCR 2.119(F)(1)}\ which provides, ``Unless another rule provides a different procedure for reconsideration of a decision . . . a motion for rehearing or reconsideration of the decision on a motion must be served and filed not later than 21 days after entry of an order deciding the motion.''
\cite{TTR 219(5)} says ``Submissions by mail are considered filed on the date indicated by the U.S. postal service postmark on the envelope containing the submissions.'' 


% Petitioner asks 
% On November 8, 2019, Petitioner filed a motion requesting that the Tribunal correct the subject's property record card.

% On November 12, 2019, Respondent filed a response to the motion.

% On November 25, 2019, Tribunal Judge Steven M. Bieda denied the motion because this case, 19-000445, had been closed in a Corrected Final Opinion and Judgment on October 17, 2019.

% Petitioner now asks this Tribunal to reopen this case and order the 



% Petitioner moves to reopen the case and for reconsideration of the \cite{Order denying Petitioner's motion to correct record card}\ under MCR 2.119(F).

% In his \cite{Exceptions}, Petitioner has listed several problems Respondent's Deputy Assessor has told Petitioner's representative that she intends to update the subject's property record card next year to reflect missing features that she discovered this year, with the purpose of raising Petitioner's taxes.

% Petitioner seeks protection from this Tribunal and asks that the Tribunal require the assessor to correct the property record card now.




% \section{Issue}

% When the assessor knows of features of the subject which are not in its record card, but refuses to update the record card during this year's appeal process so that she can raise the subject's taxes next year, should the Tribunal order the assessor to update the subject's record card as part of this year's tax appeal?

% \section{Issues}

% \begin{enumerate}
% \item Does the subject's sale comform to the sales comparision approach, one of the three recognized methods of valuation?
  
% \item Does \cite[s]{MCL 211.27(6)}\ bar the Tribnual from using the subject's sale to set its true cash value?
  
% \item Where the Tribunal based its valuation on the subject's sale along with two other comparable sales, did the Tribunal err when did not discount the subject's sale price because of market decline even though it discounted the comparables?

% \item Did the Tribunal err in relying on admittedly flawed analysis to reach its valuation?
% % \item
% %   Must the Tribunal show how it arrived at its valuation beyond stating that it is between the parties' contentions?
% \end{enumerate}  

  
\section{Facts}
\subsection{Background}
% the sale
Petitioner bought the subject property on 2/28/2018 for \$1,050,000.
The subject had been on the market for five years prior to being bought by the Petitioner. The listing broker was The Charles Reinhart Company, the leading real estate broker in the Ann Arbor area. 

Respondent assessed the property at \$1,423,900 true cash value (TCV).

Believing that the TCV should be the subject's sale price, Petitioner appealed to the board of review, which affirmed Respondent's TCV. Petitioner then appealed to this Tax Tribunal.

ALJ Wesley Margeson issued a \cite{POJ}\ proposing a TCV of \$1,100,000, relying in part on Respondent's sale comparison analysis.
Petitioner filed exceptions, but Tribunal Judge Steven Bieda issued a \cite{FOJ1}\ which adopted the FOJ's TCV.
Petitioner filed a motion for reconsideration.

Tribunal Judge Preeti Gadola issued a \cite{FOJ2}\ on Thursday, October 17, 2019, which gave no weight to Respondent's evidence and set the TCV to the subject's sale price of \$1,050,000.

Among several reasons for giving no weight to Respondent's evidence, Judge Gadola cited the fact that ``the subject property has a central vacuum, in floor heating, and an elevator that are not accounted for in Respondent's analysis, which begs the question what other features of the subject and/or comparables are not accounted for.'' \pincite[s]{FOJ2}{2}.

The features called out by Judge Gadola (central vacuum, in floor heating, and elevator) were shown in the subject's MLS listing, submitted as evidence on 5/14/2019, line 9. Respondent was aware of these features because Respondent's representative cited them at the hearing. ``It also has a central vacuum, in floor heating and an elevator.'' \pincite{POJ}{5}. Petitioner also presented the MLS listing at the board of review in March.

On Friday, October 18, 2019, the day after the corrected FOJ was issued, I, Petitioner's representative, called Emily Pizzo, Respondent's Deputy Assessor, to ask about how she had calculated the market's depreciation.

Ms. Pizzo was very upset. She said that I should tell my client that his taxes will increase next year because she (Ms. Pizzo) was planning to add the central vacuum, in floor heating, and elevator to the property record card as additions next year. This would add to the property's taxable value and result in higher taxes. And because this would be for the 2020 tax year, the subject's sale in 2018 could not be used as a comparable in an appeal.

On Thursday, November 7, 2019, I used the BS&A website\footnote{https://bsaonline.com/?uid=284, accessible from   Respondent's website (http://aatwp.org/township-government/assessment-department/}
to see how the property record card for the subject has changed since this case began.
% \footnote{I also printed copies of the current record cards for the comparable properties as checks. All four record cards are attached. The old record cards are in Respondent's evidence filed 6/25/2019, line 16.}
I found:

\begin{itemize}
\item Central vacuum had been included in the old record card and the current one. (Ms. Pizzo was therefore mistaken in thinking that central vaccuum was not indicated in the record card.)
\item In contrast, there was no record of in floor heating or an elevator in the record cards. Respondent had not updated the record card to include these items.
\item Moreover, the ``Terms of Sale'' had been changed from ``CONVENTIONAL SALE'' to ``NO CONSIDERATION.''
  The terms of sale for the three comparable properties %(100 Underdown, 1303 Towsley, and 310 Corrie)
  had not been changed. Thus it appears that Respondent had already manipulated the subject's record card to indicate going forward that the subject's sale was not a market sale. This directly contradicts the Tribunal's finding of fact ``4. The subject sold on February 28, 2018 for \$1,050,000 in an arm's-length transaction.'' \pincite[s]{POJ}{6}.
\item The property record online is in a different form and apparently less complete than record card submitted by Respondent. Therefore there may be additional changes to the subject's record in Respondent's database which are not visible online.
\end{itemize}

\subsection{Motion to correct the record card}

On November 8, 2019, Petitioner filed a motion to order Respondent to correct the current, 2019, record card to conform to the findings of fact and ruling in this case. ``Petitioner . . . asks that the Tribunal require
the assessor to correct the property record card now.'' \pincite{Motion}{1}. 

Two arguments supported the motion. First, an accurate record card is required by statute. \cite{MCL 211.24(1)}\ requires the assessor to include a ``full description'' of the real property on the assessment roll. \cite{MCL 211.29(2)}\ requires the board of review to ``correct errors . . . in the descriptions of property upon the roll . . .'' The Tribunal, as the reviewing court, has the power to correct the record card under \cite{MCL 205.732(c)}\ (Tribunal powers include: ``Granting other relief or issuing writs, orders, or directives that it deems necessary or appropriate in the process of disposition of a matter over which it may acquire jurisdiction.'').

Second, an accurate record card now promotes efficiency. Addressing this issue now saves both Petitioner and this Tribunal the trouble of addressing this issue in another appeal.

\subsection{Respondent's Reply}

Respondent filed a reply on November 12, 2019. The reply did not contradict Petitioner's legal arguments, but instead sought to explain how it intends to add the missing items to the record card and why it changed the terms of sale to indicate that the sale was not a market sale.

Respondent intends to add the missing items to the record card as ``adjustments'' without adding to the taxable value. ``All items will be added as adjustments and no additinal value would be added to the taxable value.'' \pincite{Reply}{1}. The items will be added not as new items, but ``as if they were completed at the time of construction, the the necessary depreciation for the age of the home. Those adjustment [sic] may increase the value of the SEV based on the value determined by the State Tax Commission manual.'' \pincite{Reply}{1}. Respondent does not cite to a statute or rule to support its explanation.

Respondent explains that it changed the terms of sale of the subject to exclude it from the ECF study. ``The changing of the terms was done soley for Non-use in our study based on the vacancy issues and real market value of comparable properties.'' \pincite{Reply}{2}. Respondent does not cite any authority for falsifying the terms of sale, which were determined by a Tribunal's finding of fact, so as to exclude a property from the ECF study. 

\subsection{Motion Denied}

On November 25, 2019, Tribunal Judge Steven M. Bieda denied the motion for two reasons. First, the case had been closed with respect to the 2019 tax year with issuance of the Corrected Final Opinion and Judgement, issued October 17, 2019. ``This case is thus closed with respect to the 2019 tax year.'' \pincite{Denial}{1}. Second, ``Petitioner's Motion asks the Tribunal to order a correction to the property record card that would have an effect on the 2020 assessment. . . . The Tribunal has no jurisdiction over the subject's 2020 assessment . . .'' \pincite{Denial}{1}.

% Petitioner has filed this motion for reconsideration because he believes that the Tribunal's denial was palpably wrong. The fact that the main issue of the case was decided and that the case was ``closed'' is not a good reason for ignoring Respondent's actions which contradict and undermine the Tribunal's ruling in this case. Nor would the Tribunal be doing something unusual here because it regularly hears and grants motions after a case is closed.

% Furthermore, although Petitioner has filed the motion to prevent future tax appeals, the motion concerns the subject's 2019 record card, over which the Tribunal already has jurisdiction. The Tribunal has not cited any authority allowing it to deny a motion because the applicant may also obtain a remedy by filing a motion in a future case.

% Finally, Petitioner would point out that Respondent's reply does not only not contradict the arguments in Petitioner's motion, but provides reasons for granting Petitioner's motion. Here are two of them:

% \begin{enumerate}
% \item Respondent's short memory is certain to cause either more litigation or result in a wrong assessment. Respondent's reply assumes that features of the property which were not on the record card were not included in the true cash value determined by this Tribunal. This despite the fact that the true cash value was based on the subject's sale, the MLS listing of which was the source from which Respondent determined that the record card was incomplete.
% \item Respondent's reply admits that the record card's terms of sale were changed to a false value to avoid inclusion of the subject's sale in the ECF study. This corrupted the ECF study. Thus Respondent's failure to maintain a true record card is corrupting other aspects of property tax system, beyond just the subject.
% \end{enumerate}


% \begin{itemize}
% \item In a phone call on October 18, 2019, Respondent's Deputy Assessor told Petitioner's representative that she intended to raise Petitioner's taxes in 2020 by adding to the subject's record card an elevator and in-floor heating, even though these features had been present in the property for years, were featured in the MLS listing information for the property, and were included the sale which was used to set the true cash value. 
% \item On November 7, 2019, Petitioner's representative looked up the subject's record card online and discovered that Respondent had changed the subject's record card so that it was inconsistent with the Tribunal's ruling and findings in this case: without informing this Tribunal and Petitioner, Respondent had recharacterized the subject's sale on the subject's record card from ``CONVENTIONAL SALE'' to ``NO CONSIDERATION.'' Also, Respondent had not corrected the record card to show that the subject had had an elevator and in floor heating.
% \end{itemize}

% This seems to be a case of omitted real property. \cite{MCL 211.34d(1)(b)(i)}\ says:

% \begin{quote}
%   As used in this subparagraph, "omitted real property" means previously existing tangible real property not included in the assessment. Omitted real property shall not increase taxable value as an addition unless the assessing jurisdiction has a property record card or other documentation showing that the omitted real property was not previously included in the assessment. The assessing jurisdiction has the burden of proof in establishing whether the omitted real property is included in the assessment. Omitted real property for the current and the 2 immediately preceding years, discovered after the assessment roll has been completed, shall be added to the tax roll pursuant to the procedures established in section 154. For purposes of determining the taxable value of real property under section 27a, the value of omitted real property is based on the value and the ratio of taxable value to true cash value the omitted real property would have had if the property had not been omitted.
% \end{quote}

% \cite{MCL 211.154} 
% \begin{quotation}
%   211.154 Incorrect reporting or omission of property liable to taxation; placement of corrected assessment value on assessment roll; certification of taxes due; change in assessment; collection of additional taxes; penalty and interest; refund of excess tax payments; appeal.
% Sec. 154.

%   (1) If the state tax commission determines that property subject to the collection of taxes under this act, including property subject to taxation under 1974 PA 198, MCL 207.551 to 207.572, 1905 PA 282, MCL 207.1 to 207.21, 1953 PA 189, MCL 211.181 to 211.182, and the commercial redevelopment act, 1978 PA 255, MCL 207.651 to 207.668, has been incorrectly reported or omitted for any previous year, but not to exceed the current assessment year and 2 years immediately preceding the date the incorrect reporting or omission was discovered and disclosed to the state tax commission, the state tax commission shall place the corrected assessment value for the appropriate years on the appropriate assessment roll. The state tax commission shall issue an order certifying to the treasurer of the local tax collecting unit if the local tax collecting unit has possession of a tax roll for a year for which an assessment change is made or the county treasurer if the county has possession of a tax roll for a year for which an assessment change is made the amount of taxes due as computed by the correct annual rate of taxation for each year except the current year. Taxes computed under this section shall not be spread against the property for a period before the last change of ownership of the property.
%   (2) If an assessment change made under this section results in increased property taxes, the additional taxes shall be collected by the treasurer of the local tax collecting unit if the local tax collecting unit has possession of a tax roll for a year for which an assessment change is made or by the county treasurer if the county has possession of a tax roll for a year for which an assessment change is made. Not later than 20 days after receiving the order certifying the amount of taxes due under subsection (1), the treasurer of the local tax collecting unit if the local tax collecting unit has possession of a tax roll for a year for which an assessment change is made or the county treasurer if the county has possession of a tax roll for a year for which an assessment change is made shall submit a corrected tax bill, itemized by taxing jurisdiction, to each person identified in the order and to the owner of the property on which the additional taxes are assessed, if different than a person named in the order, by first-class mail, address correction requested. Except for real property subject to taxation under 1974 PA 198, MCL 207.551 to 207.572, 1905 PA 282, MCL 207.1 to 207.21, 1953 PA 189, MCL 211.181 to 211.182, and the commercial redevelopment act, 1978 PA 255, MCL 207.651 to 207.668, and for real property only, if the additional taxes remain unpaid on the March 1 in the year immediately succeeding the year in which the state tax commission issued the order certifying the additional taxes under subsection (1), the real property on which the additional taxes are due shall be returned as delinquent to the county treasurer. Real property returned for delinquent taxes under this section, and upon which taxes, interest, penalties, and fees remain unpaid after the property is returned as delinquent to the county treasurer, is subject to forfeiture, foreclosure, and sale for the enforcement and collection of the delinquent taxes as provided in sections 78 to 79a.
%   (3) Except as otherwise provided in subsection (4), a corrected tax bill based on an assessment roll corrected for incorrectly reported or omitted personal property that is issued after the effective date of the amendatory act that added this subsection shall include penalty and interest at the rate of 1.25% per month or fraction of a month from the date the taxes originally could have been paid without interest or penalty. If the tax bill has not been paid within 60 days after the corrected tax bill is issued, interest shall again begin to accrue at the rate of 1.25% per month or fraction of a month.
%   (4) If a person requests that an increased assessment due to incorrectly reported or omitted personal property be added to the assessment roll under this section before March 1, 2004 with respect to statements filed or required to be filed under section 19 for taxes levied before January 1, 2004, and the corrected tax bill issued under this subsection is paid within 30 days after the corrected tax bill is issued, that person is not liable for any penalty or interest on that portion of the additional tax attributable to the increased assessment resulting from that request. However, a person who pays a corrected tax bill issued under this subsection more than 30 days after the corrected tax bill is issued is liable for the penalties and interest imposed under subsection (3).
%   (5) Except as otherwise provided in this section, the treasurer of the local tax collecting unit or the county treasurer shall disburse the payments of interest received to this state and to a city, township, village, school district, county, and authority, in the same proportion as required for the disbursement of taxes collected under this act. The amount to be disbursed to a local school district, except for that amount of interest attributable to mills levied under section 1211(2) or 1211c of the revised school code, 1976 PA 451, MCL 380.1211 and 380.1211c, and mills that are not included as mills levied for school operating purposes under section 1211 of the revised school code, 1976 PA 451, MCL 380.1211, shall be paid to the state treasury and credited to the state school aid fund established by section 11 of article IX of the state constitution of 1963. For an intermediate school district receiving state aid under section 56, 62, or 81 of the state school aid act of 1979, 1979 PA 94, MCL 388.1656, 388.1662, and 388.1681, of the interest that would otherwise be disbursed to or retained by the intermediate school district, all or a portion, to be determined on the basis of the tax rates being utilized to compute the amount of the state school aid, shall be paid instead to the state treasury and credited to the state school aid fund established by section 11 of article IX of the state constitution of 1963.
% \end{quotation}



  
%   inselling price, found Petitioner asked this Tribunal to correct the record card based on \cite{MCL 211.24(1)}, which requires the assessor to include a ``full description'' of the real property on the assessment roll. Respondent's reply does not address the statute. Rather Respondent's reply explains that Respondent intends to correct the record card in 2020 by adding the items as adjustments. ``All items will be added as adjustments and no additional value would be added to the taxable value.'' \pincite{Reply}{1}. Respondent cites to no supporting statute or rule which would allow it to add 


%   \subsection{old argument}
% \subsubsection{An accurate record card is required by statute}

% \cite{MCL 211.24(1)}\ requires the assessor to include a ``full description'' of the real property on the assessment roll. \cite{MCL 211.29(2)}\ requires the board of review to ``correct errors . . . in the descriptions of property upon the roll . . .'' Both the assessor and the board of review failed to ensure that the assessment roll contains a full and accurate description of the subject property.

% It now falls on this Tribunal, as the reviewing court, to correct the record card using the powers granted to it in \cite{MCL 205.732(c)}.

% \subsubsection{An accurate record card now promotes efficiency}

% The TCV determined by this Tribunal is based on the subject's actual sale price which includes everything in the subject including the in floor heating and elevator. Adding the value of any existing feature to next year's assessment would be double counting.

% The assessor's scheme, if carried out, would be intentional over-taxation. The Michigan Supreme Court has characterized intentional over-taxation as fraudulent.

% \begin{quotation}
%   "A valuation is necessarily fraudulent where it is so unreasonable that the assessor must have known that it was wrong. If the valuation is purposely made too high through prejudice or a reckless disregard of duty in opposition to what must necessarily be the judgment of all competent persons, or through the adoption of a rule which is designed to operate unequally upon a class and to violate the constitutional rule of uniformity, the case is a plain one for the equitable remedy by injunction." 4 Cooley on Taxation (4th ed), %§
%   \S 1645.
% \end{quotation}
% \pincite{Helin}{406-407}. 

% Petitioner and this Tribunal need not wait in suspense to see if the assessor will actually carry out her threat. The Tribunal can order the assessor to correct the record card now. This would help to prevent the assessor from fraudulently adding to the record card in the future and save both Petitioner and this Tribunal the trouble of addressing this issue in another appeal.


\section{Standard of Review}

\cite{MCR 2.119(F)(3)}\ says:

\begin{quote}
  Generally, and without restricting the discretion of the court, a motion for rehearing or reconsideration which merely presents the same issues ruled on by the court, either expressly or by reasonable implication, will not be granted. The moving party must demonstrate a palpable error by which the court and the parties have been misled and show that a different disposition of the motion must result from correction of the error.
\end{quote}

 A palpable error is a clear error ``easily perceptible, plain, obvious, readily
 visible, noticeable, patent, distinct, manifest.'' \pincite{Luckow}{426; 453}\ (cleaned up).
 
``The palpable error provision in \cite{MCR 2.119(F)(3)}\ is not mandatory and only provides guidance to
a court about when it may be appropriate to consider a motion for rehearing or reconsideration.''
\pincite{Walters}{350; 430}.

``The rule [\cite{MCR 2.119(F)(3)}] does not categorically prevent a trial court from revisiting an issue even when the motion for reconsideration presents the same issue already ruled upon; in fact, it allows considerable discretion to correct mistakes.'' \pincite{Macomb County Department of Human Services}{754}. 

\section{Relevant Law}

\cite{MCL 211.24(1)}\ requires that ``each year, the assessor shall make and complete an assessment roll, upon which he or she shall set down all of the following: . . . a full description of all the real property liable to be taxed.''

\cite{MCL 211.29(2)}\ requires that the board of review ``shall correct errors . . . in the descriptions of property upon the roll . . . . The board shall do whatever else is necessary to make the roll comply with this act.''

\cite{MCL 205.731}\ gives the tax tribunal jurisdiction ``over . . . [a] proceeding for direct review of a final decision, finding, ruling, determination, or order of an agency relating to assessment, valuation, . . .
% rates, special assessments, allocation, or equalization,
under the property tax laws of this state.'' This includes the board of review.

\cite{MCL 205.732(c)}\ gives the tax tribunal powers including: ``Granting other relief or issuing writs, orders, or directives that it deems necessary or appropriate in the process of disposition of a matter over which it may acquire jurisdiction.'' 

\section{Argument}

\subsection{The Tribunal's reasoning would render MCL 205.732(c) illusory}

Respondent has undermined the ruling of this Tribunal by failing to correctly update, even actively falsifying the subject's record card so that it does not accurately reflect the findings and ruling of the Tribunal. Specifically: Respondent has identified items in the subject, present at its sale, which were included in the true cash value as set by the Tribunal, which Respondent refuses to include in this year's record card. This overvalues the existing items in the record card. This is inconsistent with the Tribunal's ruling.

Furthermore, Respondent has written that it intends to add these missing items to the record card next year, increasing the assessed value by double counting items which have already been valued. This also is inconsistent with the Tribunal's ruling.

Furthermore, Respondent has falsified the record card's terms of sale for the subject's sale. This Tribunal specifically found that the sale was a market sale. Respondent has changed it from ``CONVENTIONAL SALE'' to ``NO CONSIDERATION,'' falsely indicating that the sale was not a market sale. This is inconsistent with the Tribunal's ruling.

Furthermore, Respondent has falsified the terms of sale to exclude the subject from the ECF study. Had the terms of sale been left correct, the subject would have been in the ECF study. Thus Respondent has corrupted the ECF study. This is inconsistent with the Tribunal's ruling. 

\cite[s]{MCL 205.732(c)}\ gives the Tribunal powers including, ``Granting other relief or issuing writs, orders, or directives that it deems necessary or appropriate in the process of disposition of a matter over which it may acquire jurisdiction.'' These powers are largely illusory if a taxing jurisdiction need only wait until it has lost, and the case has been closed or disposed, to partially nullify the Tribunal's ruling with inconsistent actions or inaction.

\subsection{The Tribunal can accept motions after a case is closed}

This case was first disposed on 9/17/2019, after the first FOJ was issued, line 26. Because the case was disposed, Petitioner had to file his Motion for Reconsideration by US mail instead of the efiling system. But the motion was received and granted, despite the case being ``closed.'' 
Similarly, the corrected FOJ also closed the case, but explicitly allowed a motion for reconsideration.

So the fact that a case is closed does not mean that the Tribunal cannot or will not accept motions.

This motion was brought on by Respondent's bad actions which Petitioner discovered after the case was closed.

Nor does this motion come so late as to present a burden on the Tribunal. The motion was filed just a day after the closing of the window for reconsideration for the final order in this case, the order granting reconsideration of the FOJ. This was the day after Petitioner's representative discovered that Respondent has falsified the record card. 

\subsection{This issue concerns this year's appeal, the 2019 record card}

The Tribunal's asserts that Petitioner's request is outside of the authority of the Tribunal because it concerns the subject's future 2020 assessment. Petitioner is not asking this Tribunal to change future assessments. Instead, Petitioner is asking this Tribunal to order Respondent to change this year's 2019 record card so that it complies with statute and this Tribunal's own opinion in this case. This Tribunal does have jurisdiction regarding the 2019 record card.

This Tribunal's failure to act now would allow the subject's 2019 record card to be inconsistent with the Tribunal's decision in this case. This may result in a future case before this Tribunal especially if the Deputy Assessor carries through with her threat to double-count some of the subject's features. But right now, Petitioner's motion only asks that this Tribunal correct the 2019 record card.

\subsection{Respondent's Reply provides further reasons for granting the motion}

Finally, Respondent's reply, which was written after Petitioner filed his motion, provides at least two additional reasons for granting the motion. First, despite Petitioner's explanation that items missing from the record card but revealed in the MLS listing are already included in the subject's true cash value, Respondent has confirmed that it still intends to adjust the SEV (and therefore the assessed value) when it adds the items next year. ``Those adjustment [sic] may increase the value of the SEV based on the value determined by the State Tax Commission manual.'' \pincite{Reply}{1}.

Thus even in this very case, right after this Tribunal has ruled that the subject's true cash value was its sale price, Respondent is treating the true cash value as a valuation of the record card instead of the subject, and admiting in a writing filed with the Tribunal that it intends raise the assessed value when it adds to the record card items that were valued in the sale.

Second, Respondent's reply admits that the record card's terms of sale were changed to a false value to avoid inclusion of the subject's sale in the ECF study. ``The changing of the terms was done soley for Non-use in our study based on the vacancy issues and real market value of comparable properties.'' \pincite{Reply}{2}. This corrupted the ECF study. Thus Respondent's failure to maintain a true record card is corrupting other aspects of property tax system, beyond just the subject.


% Respondent did not provide a good argument for denying the motion as shown that the Tribunal did not reject the motion on ground's advocated by Respondent. Indeed Respondent did not give any grounds for rejecting the motion. Instead Respondent's reply merely provides a misleading characterization of the issue, which even if it were true does not contradict or even eddress Petitioner's arguments.

% Respondent's reply brief says:
% \begin{enumerate}
% \item Respondent's actions will not affect the taxable value.
% \item Missing items will be added back next year as an adjustment.
% \end{enumerate}

% None of Respondent's points address, much less contradict, Petitioner's assertion that MCL 211.24(1) requires the assessor to correct the subject's record card this year.

% Worse, Respondent's reply brief appears to misrepresent the law. Respondent says that the ommitted property will be added back as adjustments and that this will not affect the taxable value. This seems to contradict MCL 211.34(d) which defines the terms \emph{additions}, \emph{losses}, and \emph{adjustments}.




% Regarding the closed case, this issue results from Respondent's actions:
%   \begin{enumerate}
%   \item Respondent failed to obey MCL 211.24(1) which requires that the assessor include a full description of the real property on the assessment roll.
%   \item Respondent failed to follow this Tribunal's order to ``correct . . . the assessement rolls . . . to reflect the property's true cash and taxable values as provided in [the] Corrected Final Opinion and Judgment . . . .'' \pincite{FOJ2}{3}.
%   \item Respondent changed the terms of sale of the subject to indicate that the sale was not a market sale, contrary to Respondent own testimony at the hearing (``This is
% the only contemporary home in the subdivision, and thus it is being used in the whole
% sales study, but not the neighborhood study.'' \pincite{POJ}{5}) and the Tribunal's finding of fact 4 (``The subject sold on February 28, 2018 for \$1,050,000 in an arm's-length transaction.'' \pincite{POJ}{6}). Respondent made this change secretly, without informing the Tribunal or Petitioner.
% \end{enumerate}
% Thus this issue is before the Tribunal because of Respondent's actions.



\section{Relief}

Therefore because:

\begin{itemize}
\item the powers granted to the Tribunal in \cite{MCL 205.732(c)}\ are not illusory,
\item the Tribunal can grant motions after a case has been closed,
\item this issue concerns the 2019 record card, and 
\item Respondent's reply has indicated that its actions have been done to double count items and corrupt the ECF study,
\end{itemize}
Petitioner asks that this Tribunal reconsider its order denying Petitioner's motion.


\needspace{6\baselineskip}
\section{Proof of Service}

I certify that I served a copy of these Exceptions on Respondent's representative, Emily Pizzo, by email on the same day I mailed them to the Tribunal.

I certify that on this date I served a copy of this motion and order on the parties or their attorneys by first-class mail addressed to
their last-known addresses as defined in MCR 2.107(C)(3).

\vspace{1\baselineskip}

{ \setlength{\leftskip}{3.5in}

  Respectfully Submitted,

  /s/ Daniel Patru, P74387

\today

  \setlength{\leftskip}{0pt}}

\newpage\empty% we need a new page so that the index entries on the last
        % page get written out to the right file.
\end{document}


%%% Local Variables:
%%% mode: latex
%%% TeX-master: t
%%% End:
