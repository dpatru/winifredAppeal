\section{Checklist}
\begin{todolist}
\item Drafts
  \begin{todolist}
  \item structure
  \item language flow
  \item word choice
  \item references
  \item technical form
  \item proof
  \end{todolist}

\item Continue from the Motion for reconsideration filed with COA. Address the arguments made in the answer.

  Re the \mathieuGast\ exception for repairs done in the year of sale, contrast uncapping under MCL 211.27a with nonconsideration under \mathieuGast. MCL 211.27a uncapping involves the AV and TV. But \mathieuGast\ nonconsideration involves determining before-repair and after-repair appraisals, tracking the value of normal repairs in the assessment roll, and adding the value back in the TCV after the owner who made the repairs sells the property.
  The two statutes are in different statutes. They protect different taxpayers. MCL 211.27a protects all property owners from rapid increases in property taxes due to market appreciation after the purchase a property. \mathieuGast\ protects only residential property owners, homeowners, from increased property taxes due to their own normal repairs. They function differently. MCL 211.27a uses AV and capped or uncapped TV while \mathieuGast\ uses the definition of TCV and tracks the value added by normal repairs on the assessment roll, not ``considering'' it for assessment purposes until the property is sold.

  

  This Court should take this case in part because it is important that tax law be clear. As it is, the STC and the plain language of the statute appear to be in conflict with the Tax Tribunal and the Court of Appeals in Detroit which covers Wayne County.

  Appellant has five cases before the Tax Tribunal which are being held in abeyance pending the resolution of this case.

  Every year in Michigan thousands of homes sell significantly below their expected market value, as indicated by their assessed TCV.\footnote{Assessors use computer-driven mass appraisal which computes the TCV based on an average sale price of similar homes. I assume here that the TCV thus computed is generally close to the market value, \emph{if the home is in average condition.}  When a home sells for significantly less than its TCV, either the home's features were not accurately determined by the assessor, or, the home is in disrepair and not in average condition.} Many of these homes sell below their market value because they need the kind of ``normal repairs'' which are protected by \mathieuGast. Under the COA's ruling, buyers of such homes would get the significant benefit of the statute only if they dely making the repairs until the year after purchase. 

  Re law of the case, note that \cite{Bennett}, teaches that arguments which could have been brought in the earlier proceeding cannot be used in a later proceeding. Here the issue of a \mathieuGast\ exception (whether there is an unwritten exception to \mathieuGast\ which allows consideration of normal repairs as long as they are done in the same year as the purchase) could have been brought in the earlier proceeding and would have obviated consideration of repairs. In other words, the legal rule relied on by the Tribunal and the COA in the second appeal solve the case without requiring any of the analysis in the first COA's decision. \cite{Bennett}, clearly covers a case such as this.

  Re the assertion that the repairs had no bearing on Respondent's assessment, this is wrong in several ways:
  \begin{enumerate}
    \item It ignores the clear teaching of Patru 1 that by valuing the considering the repaired value of the  if the repairs are normal, their value cannot be included in the TCV for assessment purpsoses.

    \item The statute's words that the ``assessor must not consider'' is characterized by the STC as ``nonconsideration treatment''. Nonconsideration treatment requires an accounting of the effect of the repairs on the market value. The STC requires that the assessor make a before-repair appraisal and an after-repair appraisal. The before-repair value is used as the true cash value while the difference between the before-repair value and the after-repair value is noted in the assessment roll.\footnote{The Tribunal recognized that it's interpretation was contrary to the STC, but asserted that the STC's interpretation was not binding on the Tribunal. Quote. In contrast, the  COA misquoted the STC to make it appear that there was no conflict.}

    \item In contrast, the COA in \cite{Patru 2}, and the Tribunal equate the statute's words ``not consider'' with ``ignore'' or ``take no notice or account of''. (In its Answer to the motion for reconsideration, Appellee says that ``the Tribunal -- and, subsequently, \emph{this} Court -- made it a point to explain that Petitioner's repairs had no bearing, whatsoever, on Respondent's 2016 assessment.'' \emph{Answer} at 5.)  So, because the property had been overvalued by the assessor in 2015,\footnote{The assessor had valued the subject property in 2015 using mass appraisal computer software which assumed that the property was in average condition, despite the fact that another department in the City had deemed the property uninhabitable until numerous repairs had been made.} an assessment which simply increased the value by inflation without regard to any repairs was deemed to satisfy the statute. 

    \item In relying on the 2015 computer-derived assessment set the before-repair value, the Tribunal and the COA violated the principle of \cite{Jones & Laughlin} that the Tribunal must make its own independent determination of TCV. This is an error of law. The Tribunal may not start with the assumption that the prior year's computer-derived TCV calculation is correct and then simply adjust for inflation. Cite.
      
    \item To the extent that the Tribunal relied on MLS pictures and the prior year's TCV to determine the value of the repairs, this is an error of fact. 

    \end{enumerate}
  \item Follow the example of
  \begin{todolist}
  \item \href{https://www.naacpldf.org/files/about-us/2017-11-1%20MorningSide%20v.%20Sabree%20Leave%20Application%20-%20Final.pdf}{Morningside Community}
      \item \href{https://www.mackinac.org/archives/2010/ApplicationforLeavetoAppealtoMSC.pdf}{Mackinac Center Legal Foundation Application for Leave to Appeal} and
      \item  \href{https://courts.michigan.gov/Courts/MichiganSupremeCourt/Clerks/ClerksOfficeDocuments/Pro-Per_MI-Sup-Ct_Civil-Application_05-2017_FillableForm.pdf}{Pro Se Application}.
      \end{todolist}
      % \item Google these and then ask Clerk of Supreme Court:
%   \begin{todolist}
%   \item Notice of appeal to Tax Tribunal and Court of Appeals
%   \item Form for Application for leave to appeal MCR 7.305(A)(1) or just style it as a brief per MCL 7.212(B).
%   \end{todolist}
% \item Rule 7.305 Application for Leave to Appeal

% (A) What to File. To apply for leave to appeal, a party must file:

% (1) 1 signed copy of an application for leave to appeal prepared in conformity with MCR 7.212(B) and consisting of the following:

% (a) a statement identifying the judgment or order appealed and the date of its entry;

% (b) the questions presented for review related in concise terms to the facts of the case;

% (c) a table of contents and index of authorities conforming to MCR 7.212(C)(2) and (3);

% (d) a concise statement of the material proceedings and facts conforming to MCR 7.212(C)(6);

% (e) a concise argument, conforming to MCR 7.212(C)(7), in support of the appellant's position on each of the stated questions and establishing a ground for the application as required by subrule (B); and

% (f) a statement of the relief sought.

% (2) 1 copy of any opinion, findings, or judgment of the trial court or tribunal relevant to the question as to which leave to appeal is sought and 1 copy of the opinion or order of the Court of Appeals, unless review of a pending case is being sought;

% (3) proof that a copy of the application was served on all other parties, and that a notice of the filing of the application was served on the clerks of the Court of Appeals and the trial court or tribunal; and

% (4) the fee provided by MCR 7.319(C)(1).

% (B)   Grounds. The application must show that

% (1)   the issue involves a substantial question about the validity of a legislative act;

% (2)   the issue has significant public interest and the case is one by or against the state or one of its agencies or subdivisions or by or against an officer of the state or one of its agencies or subdivisions in the officer's official capacity;

% (3)   the issue involves a legal principle of major significance to the state's jurisprudence;

% (4)   in an appeal before a decision of the Court of Appeals,

% (a)   delay in final adjudication is likely to cause substantial harm, or

% (b)   the appeal is from a ruling that a provision of the Michigan Constitution, a Michigan statute, a rule or regulation included in the Michigan Administrative Code, or any other action of the legislative or executive branches of state government is invalid;

% (5)   in an appeal of a decision of the Court of Appeals,

% (a)   the decision is clearly erroneous and will cause material injustice, or

% (b)   the decision conflicts with a Supreme Court decision or another decision of the Court of Appeals; or

% (6)   in an appeal from the Attorney Discipline Board, the decision is clearly erroneous and will cause material injustice.

% (C) When to File.

% (1) By pass Application. In an appeal before the Court of Appeals decision, the application must be filed within 42 days after:

% (a) a claim of appeal is filed in the Court of Appeals;

% (b) an application for leave to appeal is filed in the Court of Appeals; or

% (c) an original action is filed in the Court of Appeals.

% (2) Application After Court of Appeals Decision. Except as provided in subrule (C)(4), the application must be filed within 28 days in termination of parental rights cases, within 42 days in other civil cases, or within 56 days in criminal cases, after:

% (a) the Court of Appeals order or opinion resolving an appeal or original action, including an order denying an application for leave to appeal,

% (b) the Court of Appeals order or opinion remanding the case to the lower court or Tribunal for further proceedings while retaining jurisdiction,

% (c) the Court of Appeals order denying a timely filed motion for reconsideration, or

% (d) the Court of Appeals order granting a motion to publish an opinion that was originally released as unpublished.

% (3) Interlocutory Application from the Court of Appeals. Except as provided in subrules (C)(1) and (C)(2), the application must be filed within 28 days after a Court of Appeals order that does not resolve the appeal or original action, including an order granting an application for leave to appeal.

% (4) Attorney Discipline Board Decision. In an appeal from an order of discipline or dismissal entered by the Attorney Discipline Board, the application must be filed within the time provided in MCR 9.122(A)(1).

% (5) Late Application, Exception. Late applications will not be accepted except as allowed under this subrule. If an application for leave to appeal in a criminal case is not received within the time periods provided in subrules (C)(1) or (2), and the appellant is an inmate in the custody of the Michigan Department of Corrections and has submitted the application as a pro se party, the application shall be deemed presented for filing on the date of deposit of the application in the outgoing mail at the correctional institution in which the inmate is housed. Timely filing may be shown by a sworn statement, which must set forth the date of deposit and state that first-class postage was prepaid. The exception applies to applications from decisions of the Court of Appeals rendered on or after March 1, 2010. This exception also applies to an inmate housed in a federal or other state correctional institution who is acting pro se in a criminal appeal from a Michigan court.

% (6) Decisions Remanding for Further Proceedings. If the decision of the Court of Appeals remands the case to a lower court for further proceedings, an application for leave to appeal may be filed within 28 days in termination of parental rights cases, 42 days in other civil cases, and 56 days in criminal cases, after the date of

% (a)   the Court of Appeals order or opinion remanding the case,

% (b)   the Court of Appeals order denying a timely filed motion for reconsideration of a decision remanding the case, or

% (c)   the Court of Appeals order or opinion disposing of the case following the remand procedure, in which case an application may be made on all issues raised initially in the Court of Appeals, as well as those related to the remand proceedings.

% (7) Effect of Appeal on Decision Remanding Case. If a party appeals a decision that remands for further proceedings as provided in subrule (C)(6)(a), the following provisions apply:

% (a)   If the Court of Appeals decision is a judgment under MCR 7.215(E)(1), an application for leave to appeal stays proceedings on remand unless the Court of Appeals or the Supreme Court orders otherwise.

% (b)   If the Court of Appeals decision is an order other than a judgment under MCR 7.215(E)(1), the proceedings on remand are not stayed by an application for leave to appeal unless so ordered by the Court of Appeals or the Supreme Court.

% (8) Orders Denying Motions to Remand. If the Court of Appeals has denied a motion to remand, the appellant may raise issues relating to that denial in an application for leave to appeal the decision on the merits.

% (D) Answer. A responding party may file 1 signed copy of an answer within 28 days after service of the application. The party must file proof that a copy of the answer was served on all other parties.

% (E) Reply. The appellant may file 1 signed copy of a reply within 21 days after service of the answer, along with proof of its service on all other parties. The reply must:

% (1) contain only a rebuttal of the arguments in the answer;

% (2) include a table of contents and an index of authorities; and

% (3) be no longer than 10 pages, exclusive of tables, indexes, and appendixes.

% (F)   Nonconforming Pleading. On its own initiative or on a party's motion, the Court may order a party who filed a pleading that does not substantially comply with the requirements of this rule to file a conforming pleading within a specified time or else it may strike the nonconforming pleading. The submission to the clerk of a nonconforming pleading does not satisfy the time limitation for filing the pleading if it has not been corrected within the specified time.

% (G)   Submission and Argument. Applications for leave to appeal may be submitted for a decision after the reply brief has been filed or the time for filing such has expired, whichever occurs first. There is no oral argument on an application for leave to appeal unless ordered by the Court under subrule (H)(1).

% (H) Decision.

% (1) Possible Court Actions. The Court may grant or deny the application for leave to appeal, enter a final decision, direct argument on the application, or issue a peremptory order. The clerk shall issue the order entered and provide either a paper copy or access to an electronic version to each party and to the Court of Appeals clerk.

% (2)   Appeal Before Court of Appeals Decision. If leave to appeal is granted before a decision of the Court of Appeals, the appeal is thereafter pending in the Supreme Court only, and subchapter 7.300 applies.

% (3)   Appeal After Court of Appeals Decision. If leave to appeal is denied after a decision of the Court of Appeals, the Court of Appeals decision becomes the final adjudication and may be enforced in accordance with its terms. If leave to appeal is granted, jurisdiction over the case is vested in the Supreme Court, and subchapter 7.300 applies.

% (4) Issues on Appeal.

% (a) Unless otherwise ordered by the Court, an appeal shall be limited to the issues raised in the application for leave to appeal.

% (b) On motion of any party establishing good cause, the Court may grant a request to add additional issues not raised in the application for leave to appeal or not identified in the order granting leave to appeal. Permission to brief and argue additional issues does not extend the time for filing the brief and appendixes.

% (I) Stay of Proceedings. MCR 7.209 applies to appeals in the Supreme Court. When a stay bond has been filed on appeal to the Court of Appeals under MCR 7.209 or a stay has been entered or takes effect pursuant to MCR 7.209(E)(7), it operates to stay proceedings pending disposition of the appeal in the Supreme Court unless otherwise ordered by the Supreme Court or the Court of Appeals.

%   \begin{todolist}
%   \item[\done] Copy files in new directory
%   \item[\done] Clean up files
%   \item Read the rules, copying them and applying them
%   \item[\done] Reconsideration in Court of Appeals.
%   \item What are the required sections? In what order?
%   \item Where is SEV value defined?
%   \item What questions were presented in the original appeal?
      
%   \end{todolist}

%   \item Write
%     \begin{todolist}
%     \item mistake
%     \item Write solution
%   \end{todolist}

%   \item Proof
%   \begin{todolist}
%   \item A motion for reconsideration may be filed within 21 days after the date of the order or the date stamped on an opinion.
%   \item %The motion shall include all facts, arguments, and citations to authorities in a single document and
%     shall not exceed 10 double-spaced pages.
%   \item A copy of the order or opinion of which reconsideration is sought must be included with the motion.
%   \item Motions for reconsideration are subject to the restrictions contained in MCR 2.119(F)(3).

%   \end{todolist}

  
% \item example
%   \begin{todolist}
%   \item[\done] Frame the problem
%   \item Write solution
%   \item[\wontfix] profit
%   \end{todolist}
      \item
      (6) A statement of facts that must be a clear, concise, and chronological narrative. All material facts, both favorable and unfavorable, must be fairly stated without argument or bias. The statement must contain, with specific page references to the transcript, the pleadings, or other document or paper filed with the trial court,

      \item (a) the nature of the action;

      \item (b) the character of pleadings and proceedings;

      \item (c) the substance of proof in sufficient detail to make it intelligible, indicating the facts that are in controversy and those that are not;

      \item (d) the dates of important instruments and events;

      \item (e) the rulings and orders of the trial court;

      \item (f) the verdict and judgment; and

      \item (g) any other matters necessary to an understanding of the controversy and the questions involved;

      \item (7) The arguments, each portion of which must be prefaced by the principal point stated in capital letters or boldface type. As to each issue, the argument must include a statement of the applicable standard or standards of review and supporting authorities, and must comply with the provisions of MCR 7.215(C) regarding citation of unpublished Court of Appeals opinions. Facts stated must be supported by specific page references to the transcript, the pleadings, or other document or paper filed with the trial court. Page references to the transcript, the pleadings, or other document or paper filed with the trial court must also be given to show whether the issue was preserved for appeal by appropriate objection or by other means. If determination of the issues presented requires the study of a constitution, statute, ordinance, administrative rule, court rule, rule of evidence, judgment, order, written instrument, or document, or relevant part thereof, this material must be reproduced in the brief or in an addendum to the brief. If an argument is presented concerning the sentence imposed in a criminal case, the appellant's attorney must send a copy of the presentence report to the court at the time the brief is filed;

        \item fix appendix references
        \item Make sure to show how each issue was preserved.
\end{todolist}





\section{This Court should grant leave to appeal}

correct errors

delineate a clear standard, both legal and factual,
for the state tax commission, the Tax Tribunal, and the Court of Appeals.

To do justice in this case.




But the issues here potentially involve a lot of taxpayers.
Appellant himself has five cases being held in abeyance at the Tax Tribunal pending resolution of this case.
Appellant has identified more than 3000 properties in michigan that sold in 2019 for significantly less than their assessed (true cash) value. Many of these properties may be in the same situation as Apellant, so this decision will likely have larger application than the instant case.



% Appellant appologizes for the number of issues presented here, but they are made necessary by the arguments made in the opinions of the Tax Tribunal and the Court of Appeals. The issues As the Sixth Circuit has said, ``When a party comes to us with nine grounds for reversing the district court, that usually means there are none.'' \pincite{Fifth Third}{509}

% In 2015, Appellant bought a house, owned by the \HUD, fixed it up, and rented it, all before the year was up. That was the easy part. In the five years since then, Appellant has been battling with the City of Wayne to assess the house at its before-repaired value per the Mathieu-Gast Statute, MCL 211.27(2), which requires the assessor to ``not consider the value of normal repairs until the property is sold.''


% \section{Facts and Material Proceedings}
% \label{facts}

% \subsection{Background}

% Before it was bought by Appellant, the subject house was owned by HUD.
% % (U.S. Dept. of Housing and Urban Development).
% \mlsListing[]. 
% The house had been listed on the MLS on and off since April 2013;
% its initial asking price was \$29,900,
% but by the time Appellant bought the house, the asking price was \$32,000. \mlsHistory[].


% Appellant bought the house for the asking price in August 2015. 

% The City of Wayne (Respondent/Appellee) required repairs before it allowed occupancy. %, it required repairs.
% \repairs.

% Appellee assessed the subject property for 2016 at a true cash value of \$50,400. Appellant appealed, arguing that Appellee's assessment was the after-repair value as of December 31, 2015.
% Appellant believed that the before-repair value should be used. 
% ``Petitioner claims that under \mathieuGast\ we should not increase the subject's true cash value for normal repairs and maintenance until the subject is sold.'' \pincite{Patru 1}{2}, quoting the tribunal's POJ in the first appeal.

% Appellant appealed to the Board of Review and then to the Tax Tribunal.

% \subsection{The First Appeal to the Tax Tribunal}

% The Tax Tribunal's Referee ruled that \mathieuGast\ nonconsideration did not apply because the property was in substandard condition.
% In Exceptions and a Motion for Reconsideration, Appellant objected to the Tribunal's ruling that \mathieuGast\ did not apply to houses in substandard condition. In addition to his argument, Appellant submitted a spreadsheet clearly showing that his repairs fit under the explicit terms of the statute.\footnote{\mathieuGast\ had fifteen specific categories (a-o) of repairs that were normal repairs under the statute.} \repairs. The Tribunal's Judges refused to consider the spreadsheet and affirmed the Referee's decision in their final opinion and in the order denying reconsideration. 

% Appellant appealed for the first time to the Court of Appeals.

% \subsection{The First Appeal to the Court of Appeals}

% In an unpublished decision, the Court of Appeals Court reversed. The Court recognized that the house had been in substandard condition. ``It is undisputed that, when he purchased the property, [the house] was in substandard condition and required numerous repairs to make it livable. Patru completed the required repairs on the property as of December 31, 2015.''
% \pincite{Patru 1}{1}.

% The Court further noted that the repairs as listed in the spreadsheet were normal repairs under the Statute: ``Patru submitted a spreadsheet detailing the repairs he completed, which . . . included repairs that, under \mathieuGast, constitute normal repairs .~.~.'' \pincite{Patru 1}{4}.

% The Court ruled that the tribunal referee had erred in failing to apply \mathieuGast\ to a house in substandard condition.

% \begin{quote}
%   The hearing referee incorrectly interpreted \mathieuGast\ by concluding that because repairs were done to a property in substandard condition, they did not constitute normal repairs. .~.~. This was improper .~.~. Nothing in \mathieuGast\ provides that the repairs .~.~. are not normal repairs in the event that they are performed on a substandard property. Thus by reading a requirement into the statute that was not stated by the legislature, the trial erred .~.~. See \pincite{Mich Ed}{218}\ (stating that nothing will be read into a clear statute that is not within the manifest intention of the Legislature as derived from the language of the statute itself).
%   [\pincite{Patru 1}{5}.]
% \end{quote}

% The Court went on to rule that in light of the referee's error of law, the Tribunal erred in not allowing additional proof.

% \begin{quote} In its final opinion and judgement, the Tribunal recognized that the referee erred in its
% interpretation of MCL 211.27(2); however, it nevertheless upheld the determination of TCV.
% The Tribunal reasoned that because the spreadsheet detailing the repairs completed on the
% property had not been submitted before the hearing, it had no obligation to consider that
% evidence, so it concluded that Patru failed to establish that the repairs constituted normal repairs.
% However, as stated above, Patru did present evidence at the hearing in support of his claim that
% MCL 211.27(2) applied. The referee did not fully evaluate that evidence—which included
% testimony—because it misapprehended how to properly apply MCL 211.27(2).
%   [\pincite{Patru 1}{5}.]
% \end{quote}

% The Court explained that had the testimony at the hearing matched the contents of the spreadsheet, then the value added by the repairs should not be considered in the TCV:

% \begin{quote}If the testimony provided was an oral recitation of the
% information included on the spreadsheet, then Patru presented testimony sufficient to establish
% that at least some of the repairs constituted normal repairs under MCL 211.27(2), and so the
% increase in TCV attributed to those repairs should not be considered in the property's TCV for
% assessment purposes until such time as Patru sells the property. However, if Patru merely
% testified that he did some carpentry, electrical, and masonry repairs and no further explanation of
% the work that was provided, then he would have arguably failed to support his claim. [\pincite{Patru 1}{5}]
% \end{quote}

% Because the spreadsheet itself had not been admitted and because the actual evidence presented at the hearing was not in the record, the Court remanded for a rehearing where additional proofs could be submitted to determine if the repairs were normal under the statute. \pincite{Patru 1}{5}.



% % had not been properly 
% % Because the spreadsheet had not been accepted by the Tribunal, and because ``on the record before this Court, we cannot evaluate the sufficiency of the evidence presented at the hearing .~.~. we conclude that further proceedings are necessary .~.~. to determine whether the repairs were normal repairs within the meaning of MCL 211.27(2).''


% % The Tribunal had not made a finding of fact that the repairs that Appellant had done were normal repairs under the statute. It had refused to accept the spreadsheet because it was submitted after the hearing. \pincite{Patru 1}{5}. And it had not understood the evidence at the hearing. ``The referee did not fully evaluate that evidence—which included
% % testimony—because it misapprehended how to properly apply MCL 211.27(2).'' \pincite{Patru 1}{5}.

% % Therefore, the Court remanded for findings of fact:

% % \begin{quote}
% %   Thus, we conclude that further proceedings are necessary in order to determine
% % whether the repairs were normal repairs within the meaning of MCL 211.27(2). Accordingly, we
% % remand to the Tax Tribunal for a rehearing. Further, because the existing record is insufficient to
% % resolve whether the repairs are normal repairs within the meaning of the statute, the parties shall
% % be afforded further opportunity to submit additional proofs. [\pincite{Patru 1}{5}.]
% % \end{quote}

% \subsection{The Second Appeal to the Tax Tribunal}

% On remand, the tribunal ruled that the repairs were normal, but it again refused to apply \mathieuGast\ nonconsideration, this time because
% the repairs were done in a year of a transfer. \pincite{Patru 2}{1}.

% Also, the tribunal changed its previously undisputed finding that the property's before-repair condition was substandard. It now ruled that the repairs ``did not affect the assessed [true cash value] of the property .~.~.~.'' \pincite{Patru 2}{1}.

% Appellant filed a motion for reconsideration, but the tribunal affirmed.

% Appellant appealed for the second time to the Court of Appeals.

% \subsection{The Second Appeal to the Court of Appeals}

% In an unpublished decision, the Court of Appeals affirmed, agreeing with the Tax Tribunal on all the issues in dispute. (The details of the Court's ruling will be discussed in the Argument section below.) Appellant filed a motion for reconsideration, which was answered by Appellee, but the Court denied the motion without comment on March 31, 2020.

% Appellant now asks this Court to review the case.

% % \begin{enumerate}

% % \item The Court ruled ``that because there was a transfer of ownership of the subject property in 2015, MCL 211.27(2) did not prohibit the assessor from considering the impact of any ``normal repairs'' on the property's TCV for purposes of the 2016 tax year.'' \pincite{Patru 2{3}. 

% % \item The Court refused to apply the law of the case doctrine because:

% % \begin{quote}
% %   In \emph{Patru I}, this Court did not resolve whether petitioner's 2015 repairs to the property could or could not be considered in determining the property's TCV for the 2016 tax year, but instead determined that ``further proceedings are necessary to determine whether the repairs were normal repairs within the meaning of MCL 211.27(2).'' \emph{Id.}, unpub op at 5.  More significantly, this Court did not address the effect of the property's transfer of ownership in 2015 on the tribunal's consideration of ``normal repairs'' under MCL  211.27(2) for purposes  of the 2016 tax  year.  Because this issue was not actually addressed and decided in the prior appeal, the law-of-the-case doctrine does not apply. [\pincite{Patru 2}{5}.]
% % \end{quote}

% % \item The Court ruled that the tribunal did not err in finding that ``petitioner's repairs did not have any bearing on the property's TCV'' because the tribunal 
% % The 

\section{Argument: Grounds for Granting the Application}

\subsection{Introduction}

\mathieuGast\ exempts from the true cash value the value added by normal repairs until the property is sold. The Tax Tribunal and the Court of Appeals in this case attack this tax exemption
% of \mathieuGast\
in two ways. First, they exclude from the statute repairs done by a purchaser in the same year of the purchase. The statute requires nonconsideration treatment%
\footnote{\label{nonconsiderationTreatmentFn}The term ``nonconsideration treatment'' is not in the statute itself, but is used to explain the statute in \cite{STC Bulletin}. The term is a shorthand for the statute's requirement that the value added by normal repairs be ``indicated on the assessment roll'' but not considered ``in determining the true cash value of the property for assessment purposes until the property is sold.''}
for normal repairs ``until the property is sold.'' The word ``until'' means that there must be a sale \emph{after} the repairs to cause nonconsideration treatment to end. But the Tribunal and the Court of Appeals ignore the word ``until'' and refuse to require nonconsideration treatment because of a sale that occurs \emph{before} the repairs but in the same year. This is violates the plain language of the statute and makes it inconsistent and illogical. 

The second way that the Tax Tribunal and the Court of Appeals attack the tax exemption is by using the assessor's true cash value for the prior year. The prior year's assessment was based on the assumption%
\footnote{The assessor testified at the hearing that mass appraisal does not account for properties one-by-one. In other words, properties are assessed uniformly and \emph{Respondent assumes properties are in ``average'' condition.} \foj[4].}
that the property was in average condition and did not need extensive repairs.%
In other words, the prior year's assessment was based on the assumption that the repairs were not needed or had already been made. Of course this was not actually true because after the sale the Appellee itself required the repairs before it would grant a certificate of occupancy.

Despite the fact that the prior year's true cash value was based on untrue assumption, the Tribunal and the Court of Appeals used it to deny the exemption on two alternative grounds. First, the courts argued that the assessor had not considered the repairs in determining the true cash value: the assessor had simply adjusted on the prior year's assessment for inflation. Alternatively, the courts argued that the value added by the repairs was zero, because, adjusting for inflation, the property's true cash value was the same in the years before and after the repairs. The first ground misinterprets the statute's ``not consider'' as ``to ignore'' rather than indicating nonconsideration treatment.\footnoteref{nonconsiderationTreatmentFn} The second ground is an error of fact and also violates the requirement that the Tribunal must make an independent determination of true cash value.


esthis fact in two ways to deny the exemption. First, they said that 

provides that ``the assessor shall not consider Second, they exclude from the statute properties that were over-assessed in the preceding year, either by saying that the assessor did not need to consider the repairs because the property's TCV was derived from the previous year's TCV, or by setting the before-repair value to the previous-year's TCV and thus determining that the repairs did not add to the TCV.

%   (2) The assessor shall not consider the increase in true cash value that is a result of expenditures for normal repairs, replacement, and maintenance in determining the true cash value of property for assessment purposes until the property is sold. For the purpose of implementing this subsection, the assessor shall not increase the construction quality classification or reduce the effective age for depreciation purposes, except if the appraisal of the property was erroneous before nonconsideration of the normal repair, replacement, or maintenance, and shall not assign an economic condition factor to the property that differs from the economic condition factor assigned to similar properties as defined by appraisal procedures applied in the jurisdiction. The increase in value attributable to the items included in subdivisions (a) to (o) that is known to the assessor and excluded from true cash value shall be indicated on the assessment roll. This subsection applies only to residential property. The following repairs are considered normal maintenance if they are not part of a structural addition or completion:

already  . Third, they repeat the second attack in a different way by using the alternatively int


create an unwritten exception for repairs by a purchaser done in the same year as the purchase. Second, they create another unwritten exception for properties that have been over-valued by the assessor in the previous year.\footnote{In this case, the second exception is not only an error of law, but can also be viewed as an error of fact. The Tribunal and Court of Appeals found that the before-repair value was the previous year's assessed TCV. This error is addressed in the third Question Presented.} 
The stakes of this case amount to on the order of a thousand dollars a year.

In terms of the dollars directly at issue, this case is one of the smallest cases this Court will see. The tax liability at issue is on the order of a thousand dollars a year. But the principles deciding the issues in this case have wide applicability.

\begin{enumerate}
\item Mathieu Gast nonconsideration:
  \begin{enumerate}
  \item Is there an unwritten exception for repairs completed by a buyer in the year of purchase?
  \item Can the taxpayer rely on the plain text of the tax code? Can the taxpayer rely on published STC bulletins? Or will the taxpayer be surprised by intricate and difficult intepretations of the law by the Tax Tribunal and the Court of Appeals? 
  \end{enumerate}

\item the law of the case:
  \begin{enumerate}
  \item Can the taxpayer rely on the fact that after he wins at the Court of Appeals, that on remand the case will move forward? Or can findings of fact and legal questions established during the first appeal be revisited so as to give the taxing authority another chance to win?
  \item Or will legal issues that could have been brought up during the first proceeding be barred in the second preceeding?
  \end{enumerate}
\end{enumerate}

\subsection{Granting question one (re. the law of the case) is proper under MCR 7.305(B)(2, 3, 5a, and 5b)}

Per \cite{MCR 7.305(B)(2)}\ the first issue has significant public interest and the case is against one of the state's subdivisions.

The big-picture issue of this case from the view of the law-of-the-case doctrine and the public is whether taxing authorities and tribunal judges\footnote{The City, appellant, has not advanced any written legal arguments at the tribunal level. Throughout this case, the conflict has been primarily between the taxpayer and the Tribunal Judges. All of the issues presented before this Court originate from the tribunal judges. This does not mean that the tribunal judges are biased or acting improperly, because they have a duty to make their own decisions, independent of positions put forth by the parties.} should be permitted to try again with a different theory after they have lost a dispute with a taxpayer at the Court of Appeals.

In this case, the Tax Tribunal first ruled that the house was in such \emph{poor} condition before the repairs that the repairs were not normal repairs under the statute. Then, after the Court of Appeals rejected this theory, the Tax Tribunal tried a couple more theories. This time, on the exact same evidence, the Tribunal ruled that the house was in such \emph{good} condition before the repairs that that the repairs, worth \$10,000, did not affect the value at all. In addition, the Tribunal ruled that because the repairs were done in the year when the property was purchased, that \mathieuGast\ nonconsideration did not apply at all, a theory that could have been brought up in the first appeal. This necessitated another appeal to the Court of Appeals. The process so far has lasted more four years since the initial appeal at the 2016 March Board of Review. The process has required Appellant to file numerous lengthy documents. The only way Appellant could afford to do these appeals is because he did the appeals himself for ``free.'' Appellant could not have afforded to pay a lawyer to do it for him.

Especially in cases like this, where the lone taxpayer is up against the unlimited resources of the government, the ordinary tax-paying public has an interest in the efficiency promoted by the law-of-the-case doctrine. The best arguments should be brought on the first appeal and arguments that are not brought, but could have been brought, should be barred. Likewise, facts determined in the first appeal should not changed. The taxing authority should not be allowed to keep trying new theories until it succeeds either on the merits or by exhausting the taxpayer. 

Per \cite{MCR 7.305(B)(3)}\ the first issue involves a legal principle of major significance to the state's jurisprudence. The law-of-the-case doctrine has been around for a long time and serves an important interest not only 

Per \cite{MCR 7.305(B)(5a)}\ the Court of Appeals' handling of the first issue is clearly erroneous and will cause material injustice.

Per \cite{MCR 7.305(B)(5b)}\ the Court of Appeals' decision conflicts with a Supreme Court decision or another decision of the Court of Appeals. 

\subsection{Granting question two (whether there is a first-year exception to \mathieuGast) is proper under MCR 7.305(B)(2, 3, 5a, and 5b)}

Per \cite{MCR 7.305(B)(2)}\ the second issue has significant public interest and the case is against one of the state's subdivisions.

There are two big-picture issues of this case from the point of \mathieuGast\ nonconsideration and the public. Will industrious buyers of distressed homes be penalized for making immediate repairs rather than waiting until the year after the sale? And, can ordinary members of the public rely on the plain language of the tax code and the guidance of the state tax commission, or must they take their chances that a Tax Tribunal Judge will make a novel interpretation of the tax code at odds with both the code's plain language and the STC?

Per \cite{MCR 7.305(B)(3)}\ the second issue involves a legal principle of major significance to the state's jurisprudence, namely, plain language interpretation of statutes. 

Per \cite{MCR 7.305(B)(5a)}\ the Court of Appeals' handling of the second issue is clearly erroneous and will cause material injustice.

Per \cite{MCR 7.305(B)(5b)}\ the Court of Appeals' decision conflicts with a Supreme Court decision or another decision of the Court of Appeals. 

\subsection{Granting question three (rejecting a sale based on speculation) is proper under MCR 7.305(B)(2, 3, 5a, and 5b)}

Per \cite{MCR 7.305(B)(2)}\ the third issue has significant public interest and the case is against one of the state's subdivisions.

The big-picture issue of this case from the point of view of question three and the public. Will all relevant evidence be considered fairly or will some evidence by excluded out of hand based on arbitrary per-se rules? Here the subject's actual sale was replaced with mls marketing pictures taken out of context and the previous year's assessed value which by the city's own admission assumed the point at issue.\footnote{The Tribunal Judge himself in his final opinion says that the Appellant's representative testified that properties are assessed under the assumption that they are in average condition. Thus the assessed value cannot be used to prove the actual condition of the property.}

Per \cite{MCR 7.305(B)(3)}\ the third issue involves a legal principle of major significance to the state's jurisprudence, namely, a correct determination of the true cash value of properties. 

Per \cite{MCR 7.305(B)(5a)}\ the Court of Appeals' handling of the third issue is clearly erroneous and will cause material injustice. Violates \cite{Jones & Laughlin}\ and prevent Appellant from receiving a fair hearing.

Per \cite{MCR 7.305(B)(5b)}\ the Court of Appeals' decision conflicts with a Supreme Court decision or another decision of the Court of Appeals. \cite{Jones & Laughlin}.

\subsection{Granting question four (determination of before-repair value) is proper under MCR 7.305(B)(2, 3, 5a, and 5b)}

Per \cite{MCR 7.305(B)(2)}\ the fourth issue has significant public interest and the case is against one of the state's subdivisions.

The big-picture issue of this case from the point of view of question four and the public. Will STC's guidance by followed, or can assessors and tribunal judges just do their own thing?

Is this just another way to create a first-year exception? If the previous year's assessed value is used for the before-repair value, then in many cases in which the city has not noted the below-average condition of the property, the result will be indistinguishable from an explicit first-year exception. 

Per \cite{MCR 7.305(B)(3)}\ the fourth issue involves a legal principle of major significance to the state's jurisprudence, namely, independent determination by tax tribunal judges. tk where does this requirement come from?

Per \cite{MCR 7.305(B)(5a)}\ the Court of Appeals' handling of the fourth  issue is clearly erroneous and will cause material injustice. Violates \cite{Jones & Laughlin}\ and is a back-door way to implement an year-of-sale exception.

Per \cite{MCR 7.305(B)(5b)}\ the Court of Appeals' decision conflicts with a Supreme Court decision or another decision of the Court of Appeals. \cite{Jones & Laughlin}. \cite{Jones & Laughlin}.

\subsection{Granting question five (no basis for finding no effect of repairs on value) is proper under MCR 7.305(B)(2, 3, 5a, and 5b)}

Per \cite{MCR 7.305(B)(2)}\ the fifth issue has significant public interest and the case is against one of the state's subdivisions.

The big-picture issue of this case from the point of view of question five and the public: must Judges be disciplined in their factual findings or can they cherry-pick evidence, use ``evidence'' that begs the question, ignore evidence they don't like, etc?

Per \cite{MCR 7.305(B)(3)}\ the fifth issue involves a legal principle of major significance to the state's jurisprudence, namely, correct even-handed evaluation of the evidence. No case will have exactly this evidence, but the manner in which evidence is handled can be the same. The Court needs to correct the Tribunal before this becomes a habit.

Per \cite{MCR 7.305(B)(5a)}\ the Court of Appeals' handling of the fourth  issue is clearly erroneous and will cause material injustice. This is clearly wrong here and results in higher taxes.

Per \cite{MCR 7.305(B)(5b)}\ the Court of Appeals' decision conflicts with a Supreme Court decision or another decision of the Court of Appeals. Violates the rule for evidence which is cited in many appellate cases.



% \cite{MCR 7.305(B)} lists six grounds under which this Court may grant leave to appeal. 

% (B)   Grounds. The application must show that

% (1)   the issue involves a substantial question about the validity of a legislative act;

% (2)   the issue has significant public interest and the case is one by or against the state or one of its agencies or subdivisions or by or against an officer of the state or one of its agencies or subdivisions in the officer's official capacity;

% (3)   the issue involves a legal principle of major significance to the state's jurisprudence;

% (4)   in an appeal before a decision of the Court of Appeals,

% (a)   delay in final adjudication is likely to cause substantial harm, or

% (b)   the appeal is from a ruling that a provision of the Michigan Constitution, a Michigan statute, a rule or regulation included in the Michigan Administrative Code, or any other action of the legislative or executive branches of state government is invalid;

% (5)   in an appeal of a decision of the Court of Appeals,

% (a)   the decision is clearly erroneous and will cause material injustice, or

% (b)   the decision conflicts with a Supreme Court decision or another decision of the Court of Appeals; or

% (6)   in an appeal from the Attorney Discipline Board, the decision is clearly erroneous and will cause material injustice.


\subsection{The fact that this case is not published should not cause the Court to think that its bad practices and teaching won't spread if unchecked now}

\subsubsection{This is a case of first impression regarding the first year's exception to \mathieuGast, the per-se rule against bank sales or sales by government institutions, the obligation to independently determine the before-repair value, and the improper use of MLS pictures and previous-year assessed value. It is a precedent even if nonbinding.}

Other cases will follow. The tendency will be follow the lead of this case, especially if the Supreme Court has refused to correct it. Why pay a lawyer to craft an appeal, when someone has already tried and been shot down?

Appellant himself is counsel for five pending cases which are on hold pending the resolution of the issues in this case.

For cases involving residential fixer-uppers, it unlikely that an owner will have the resources, know-how, and willingness to appeal even to the Court of Appeals. It is likely that many more abuses like in the present case before this Court has an opportunity to correct.


\section{Argument}


1. The issue raises a serious question about the legality of a law passed by the legislature.
2. The issue raises a legal principle that is very important to Michigan law. MCL 7.305(B)(3)
3. The Court of Appeals decision is clearly wrong and will cause material injustice to me. MCL 7.305(B)(5)(a)
4. The decision conflicts with a Supreme Court decision or another decision of the Court of Appeals. MCL 7.305(B)(5)(b)


Issue 1, concerning the law of the case, is important because in this case both the Tax Tribunal and the Court of Appeals ignored the doctirine of the law of the case. , the law of the case was ignored, by both the Tax Tribunal and the Court of Appeals. S
\subsection{Standard of Review}



\subsection{This Court should grant leave to appeal on question one under MCR 7.305(B)(2, 3, 5a, and 5b)}

standard of review and preservation

\subsubsection{Enforcing the doctrine of the law of the case serves a significant public interest in reducing the cost and length of tax appeals}

Per \cite{MCR 7.305(B)(2)}\ the first issue has significant public interest and the case is against one of the state's subdivisions.

The big-picture issue of this case from the view of the law-of-the-case doctrine and the public is whether taxing authorities and tribunal judges\footnote{The City, appellant, has not advanced any written legal arguments at the tribunal level. Throughout this case, the conflict has been primarily between the taxpayer and the Tribunal Judges. All of the issues presented before this Court originate from the tribunal judges. This does not mean that the tribunal judges are biased or acting improperly, because they have a duty to make their own decisions, independent of positions put forth by the parties.} should be permitted to try again with a different theory after they have lost a dispute with a taxpayer at the Court of Appeals.

In this case, the Tax Tribunal first ruled that the house was in such \emph{poor} condition before the repairs that the repairs were not normal repairs under the statute. Then, after the Court of Appeals rejected this theory, the Tax Tribunal tried a couple more theories. This time, on exact same evidence, the Tribunal ruled that the house was in such \emph{good} condition before the repairs that that the repairs, worth \$10,000, did not affect the value at all. In addition, the Tribunal ruled that because the repairs were done in the year when the property was purchased, that \mathieuGast\ nonconsideration did not apply at all, a theory that could have been brought up in the first appeal. This necessitated another appeal to the Court of Appeals. The process so far has lasted more four years since the initial appeal at the 2016 March Board of Review. The process has required Appellant to file numerous lengthy documents. The only way Appellant could afford to do these appeals is because he did the appeals himself for ``free.'' Appellant could not have afforded to pay a lawyer to do it for him.

Especially in cases like this, where the lone taxpayer is up against the unlimited resources of the government, the ordinary tax-paying public has an interest in the efficiency promoted by the law-of-the-case doctrine. The best arguments should be brought on the first appeal and arguments that are not brought, but could have been brought, should be barred. Likewise, facts determined in the first appeal should not changed. The taxing authority should not be allowed to keep trying new theories until it succeeds either on the merits or by exhausting the taxpayer. 

\subsubsection{The doctrine of the law of the case is a legal principle with major significance to the state's jurisprudence because it helps make sure that the state's limited judicial resources are used efficiently.}

Per \cite{MCR 7.305(B)(3)}\ the first issue involves a legal principle of major significance to the state's jurisprudence. tk the law-of-the-case doctrine has been around for a long time and serves an important interest.

\subsubsection{The Court of Appeals' decision clearly contradicts Michigan caselaw and has already caused material injustice}

Per \cite{MCR 7.305(B)(5a)}\ the Court of Appeals' handling of the first issue is clearly erroneous and will cause material injustice.
Per \cite{MCR 7.305(B)(5b)}\ the Court of Appeals' decision conflicts with a Supreme Court decision or another decision of the Court of Appeals. 

\subsection{This Court should grant leave to appeal on question two under MCR 7.305(B)(2, 3, 5a, and 5b)}

standard of review and preservation

%(whether there is a first-year exception to \mathieuGast) is proper
\subsubsection{The public has a significant interest in the plain language application of \mathieuGast\ in particular and the tax law generally.}

Per \cite{MCR 7.305(B)(2)}\ the second issue has significant public interest and the case is against one of the state's subdivisions.

There are two big-picture issues of this case from the point of \mathieuGast\ nonconsideration and the public. Will industrious buyers of distressed homes be penalized for making immediate repairs rather than waiting until the year after the sale? And, can ordinary members of the public rely on the plain language of the tax code and the guidance of the state tax commission, or must they take their chances that a Tax Tribunal Judge will make a novel interpretation of the tax code at odds with both the code's plain language and the STC?


Per \cite{MCR 7.305(B)(3)}\ the second issue involves a legal principle of major significance to the state's jurisprudence, namely, plain language interpretation of statutes. 

Per \cite{MCR 7.305(B)(5a)}\ the Court of Appeals' handling of the second issue is clearly erroneous and will cause material injustice.

Per \cite{MCR 7.305(B)(5b)}\ the Court of Appeals' decision conflicts with a Supreme Court decision or another decision of the Court of Appeals. 

\subsubsection{The exception to MCL 211.27(2) is unsupported}

This court reviews the Tax Tribunal's statutory interpretation de novo. \pincite{Briggs}{75}.

This Court previously faulted the Tax Tribunal for violating the rule that ``nothing will be read into a clear statute that is not within the manifest intention of the Legislature as derived from the language of the statute itself.'' \pincite{Patru 1}{5}, quoting \pincite{Mich Ed}{218}. But this Court now makes the same error by reading into \mathieuGast\ an exception for repairs done in the year of a transfer. 

% The statute specifies that repairs should not considered ``\emph{until} the property is sold.'' This clearly indicates that nonconsideration ends only if there is a sale after the repairs. Here there was no sale after Appellant's repairs. Instead this Court would use a sale before the repairs to end the nonconsideration period. This violates the plain words of the statute.

The statute requires nonconsideration ``\emph{until} the property is sold.'' The word until means that there must be a sale after the repairs for nonconsideration to end. This Court's refusal to apply nonconsideration where there is no sale after the repairs is a violation of the plain words of the statute.

This Court also frustrates the intent of the statute which is to promote repairs.
In direct violation of the statute's plain language, this Court uses the interaction of at least two other statutes to exempt from nonconsideration the repairs of diligent buyers who start repairs right away.
Under the Court's view, they should wait until the year after the purchase so that the before-repair value can be determined on tax day.

% This Court would require that purchasers delay repairs until the year after the sale, to establish the unrepaired value as of December 31.
% Diligent buyers who start repairs too early
% % cannot rely on the plain text of statute, but
% are exempted from nonconsideration not by the plain language of the statute itself but via the complex interplay of at least two other statutes in the tax act.
% It seems odd that this was the legislature's intent.

% 
% By reading into the statute an unwritten exclusion for repairs in the year of purchase,
% this Court transforms Mathieu Gast
% from a pure incentive for homeowners to make repairs
% into a trap which punishes diligent purchasers who,
% relying on the plain words of the statute,
% repair right away. % rather than waiting until the year after purchase.
% Under this Court's interpretation, purchasers who wish to benefit from nonconsideration must delay repairs until the year after the sale, to establish the unrepaired value as of December 31.

%There is no evidence that in a situation like this, the legislature intended to incentivize the Appellant to delay repairs to establish the unrepaired value at the end of the transfer year, and then make the normal repairs in the following year. 

Where did this Court err? This Court is correct up to the proposition 
that the uncapped taxable value is based on the true cash value of the property in the year after the transfer.\footnote{The uncapped taxable value is the equalized assessed value (state equalized value). \cite{MCL 211.27a(3)}. The assessed value is one half of the true cash value. \cite{MCL 211.27a(1)}. The true cash value is always calculated as of tax day. ```true cash value' means the usual selling price .~.~. at the time of assessment,'' \cite{MCL 211.27(1)}. ``The taxable status of .~.~. real .~.~. property for a tax year shall be determined as of each December 31 of the immediately preceding year, which is considered the tax day,'' \cite{MCL 211.2(2)}.}
But it errs in equating true cash value with fair market value ``regardless of any ``normal repairs'' made by petitioner in [the year of purchase].'' Opinion at 5.

True cash value is defined in \cite{MCL 211.27}%
\footnote{Internally \cite[s]{MCL 211.27}\ may define true cash value in terms of itself. For example, \mathieuGast\ uses ``true cash value for assessment purposes'' to distinguish from the ``true cash value'' that increased due to normal repairs. But outside of \cite[s]{MCL 211.27}, in the rest of the tax act, there is only one ``true cash value'' which incorporates all of \cite[s]{MCL 211.27}.
  For example, assessed value is defined in \cite{MCL 211.27a(1)}\ as 50\% of the true cash value. Obviously with respect to \mathieuGast\ this means the ``true cash value for assessment purposes.'' But, as it is outside of \cite[s]{MCL 211.27}, \cite{MCL 211.27a(1)}\ uses the term ``true cash value.''}, 
which has eight subsections, which if reproduced here would span five pages. Courts have summarized true cash value as ``fair market value,'' based on the term ``usual selling price'' in subsection one. But this shorthand is used only in cases where a more complex definition is not needed. There is no case, except this Court's opinion here, that holds that the caselaw-derived term ``fair market value'' overrides or invalidates an applicable subsection of \cite[s]{MCL 211.27}.
% which holds that   defines true cash value as the ``usual selling price'' which is a good working definition for many cases. 
% in this brief. spanning several pages. which would span several pages if reproduced here. All of these subsections refine the meaning of true cash value. Subsection 1 uses the term ``usual selling price'' which caselaw says means fair market value.%
% \footnote{This Court's opinion at 4 cites \pincite{Pontiac Country Club}{434-435}\ for the proposition that ``TCV is synonymous with fair market value.'' There are many such cases because most cases do not require a more specific definition of true cash value. This Court has not cited any case which teaches that a relevant subsection of \cite{MCL 211.27}\ should be ignored in favor of the simple definition that true cash value is fair market value.}
% This is a good general definition but cannot always replace every aspect of a multiple page statute.
% In this case, subsection 2 adds a relevant refinement to the general definition.
% % Here this Court should consider subsection 2 which specifies that the value due to normal repairs is not part of true cash value until the property is sold.
% For this case, true cash value should mean ``the usual selling price .~.~. at the time of assessment,'' \cite{MCL 211.27(1)}, but not considering the value added by normal repairs, \mathieuGast.

% \footnote{\cite{\mathieuGast\ uses the term, ``true cash value for assessment purposes'' to distinguish from the ``true cash value'' of \cite{MCL 211.27(1)}\ which presumably may include the value of normal repairs and is thus not always suitable for assessment purposes. But read as a whole \cite{MCL 211.27}\ as a whole as defining ``true cash value for assessment purposes'' or just ``true cash value'' for short. Obviously the assessed value of \cite{MCL 211.27a(1)}\ uses this ``true cash value for assessment purposes'' because it is for assessment purposes.}

\mathieuGast, as part of the definition of true cash value, is not dependent on the capped or uncapped status of the taxable value defined in the next section, \cite{MCL 211.27a}. Rather, the dependency goes the other way. \mathieuGast\ always
operates to exclude the value of normal repairs no matter if true cash value is used to establish the assessed value of \cite{MCL 211.27a(1)}\ or the additions of \cite{MCL 211.27a(2)(b)}.

This Court's interpretation of Mathieu Gast is unsupported by the State Tax Commission. Its opinion on page 6 cites, with its own emphasis, \pincite{STC Bulletin}{3}:

\begin{quote}
The exemption for normal repairs, replacements and maintenance ends in the year after the
owner who made the repairs, replacements and maintenance sells the property. \emph{In the year
following a sale, the assessed value shall be based on the true cash value of the entire property.}
The amount of assessment increase attributable to the value of formerly exempt property
returning to the assessment roll is new for equalization purposes.
\end{quote}

This Court reads just the second, emphasized, sentence out of its context. The first sentence of the paragraph gives the topic: ``The exemption .~.~. ends in the year after \emph{the owner who made the repairs} .~.~. sells the property.'' In this context, the second sentence's ``entire property'' means the property with the seller's repairs considered.
The buyer's repairs are not in the paragraph's scope.%
\footnote{The opinion at 6 uses this out-of-context quote to justify its holding that before-repair and after-repairs appraisals were not required. As this passage does not say what this Court thinks it says, this Court's holding on this point is without support.}

\subsubsection{The law of the case requires reversal}

\begin{quote}
  Under the doctrine of the law of the case, if an appellate court has passed on a legal question and remanded the case for further proceedings, the legal question will not be differently determined in a subsequent appeal in the same case were the facts remain materially the same. The primary purpose of the law-of-the-case doctrine is to maintain consistency and avoid reconsideration of matters once decided during the course of a single  % 500*500
  continuing lawsuit. .~.~.  [T]he doctrine is discretionary, rather than mandatory.
\end{quote}
\pincite{Bennett}{499-500}.

The opinion at 5 gives two reasons why the law of the case does not apply. First,
this Court had not resolved whether ``the repairs were normal repairs within the meaning of \mathieuGast.'' This was not resolved because the existing record was insufficient. \pincite{Patru 1}{5}. But the opinion specified that if the Tribunal found that the repairs listed on the spreadsheet had been performed, these would constitute normal repairs and should not be considered as part of the true cash value:

\begin{quote}
  If the testimony provided was an oral recitation of the
information included on the spreadsheet, then Patru presented testimony sufficient to establish
that at least some of the repairs constituted normal repairs under MCL 211.27(2), and so the
increase in TCV attributed to those repairs should not be considered in the property's TCV for
assessment purposes until such time as Patru sells the property.
\end{quote}
\pincite{Patru 1}{5}. So on remand, the tribunal should have applied nonconsideration after it found that the repairs as listed on the spreadsheet were normal repairs.

Instead, despite finding that the repairs were normal, the tribunal did not apply nonconsideration because of the second reason: it now thought that the transfer of ownership and uncapping created an exception. Opinion at 5. %The teaching of \cite[s]{Bennett}\ is helpful here.

In \cite[s]{Bennett}, this Court was asked to reverse its earlier ruling based on a law which it had not explicitly considered the first time. This Court refused, reasoning that even though its first decision was wrong, by substituting its judgment for the prior appeals court panel, it would be doing ``the very activity the law-of-the-case doctrine is designed to discourage.'' \pincite[s]{Bennett}{501}. If the prior panel was not aware of the new law, the parties were at fault because ``ultimately it is the responsibility of the parties to bring to this Court's attention that case law and those statutes that the parties wish the Court to consider in deciding the matter.'' \pincite{Bennett}{501}.

% In \cite[s]{Bennett},
% the lower court had denied defendant's motion to terminate his child support early.
% While the case was on appeal, a law was enacted which supported the lower court's decision.
% The law took effect before the appeals court called the case.
% The appeals court reversed the lower court, but did not mention the new law.
% On remand, plaintiff asked the lower court to deny defendant's motion a second time,
% based on the new law which the appeals court had not addressed.
% The lower court refused because of the law of the case. Plaintiff appealed.

% The appeals court affirmed, reasoning that even though it thought that its first decision was wrong, by substituting its judgment for the prior appeals court panel, it would be doing ``the very activity the law-of-the-case doctrine is designed to discourage.'' \pincite{Bennett}{501}. If the prior panel was not aware of the new law, the parties were at fault because ``ultimately it is the responsibility of the parties to bring to this Court's attention that case law and those statutes that the parties wish the Court to consider in deciding the matter.'' \pincite{Bennett}{501}.

Thus \cite[s]{Bennett}\ teaches that the law of the case precludes even correct arguments if they could have been brought up in the first appeal. This is so here. Thus even if this Court agrees with the tribunal's latest decision,
%that \mathieuGast\ does not cover this case,
it should still reverse and affirm its earlier ruling.



% \cite[s]{Patru 1}\ because

% \begin{quote}
%   this Court did not resolve whether petitioner's 2015 repairs to the property could or could not be considered in determining the property's TCV for the 2016 tax year, but instead
% determined that ``further proceedings are necessary to determine whether the repairs were normal
% repairs within the meaning of MCL 211.27(2).'' .~.~. More significantly, this Court
% did not address the effect of the property's transfer of ownership in 2015 on the tribunal's
% consideration of ``normal repairs'' under MCL 211.27(2) for purposes of the 2016 tax year.
% Because this issue was not actually addressed and decided in the prior appeal, the law-of-the-case
% doctrine does not apply.
% \end{quote}
% Opinion at 5.
% % This Court does not fairly characterize its previous ruling.
% A more complete quote from this Court's prior opinion
% % (shown above at page \pageref{firstHolding})
% shows that this Court had decided that \mathieuGast\ applied to this case, but
% lacked a finding by the tribunal that the repairs listed in the spreadsheet were performed. 
% %, based on the incomplete record, could not decide if Appellant's repairs conformed to the specific categories of repairs listed in the statute.

% \begin{quote}
%   If the testimony provided was an oral recitation of the
% information included on the spreadsheet, then Patru presented testimony sufficient to establish
% that at least some of the repairs constituted normal repairs under MCL 211.27(2), and so the
% increase in TCV attributed to those repairs should not be considered in the property's TCV for
% assessment purposes until such time as Patru sells the property.
% \end{quote}
% \pincite{Patru 1}{5}. On remand, this Court expected the tribunal to determine which of the repairs fit the criteria of \mathieuGast, and then apply nonconsideration to those repairs, using before-repair and after-repair appraisals.

% \begin{quote}
%  We note that, on reconsideration, the Tribunal faulted Patru for failing to establish a pre-repair
% TCV. However, as the Tribunal must make its own, independent determination of TCV, Great
% Lakes Div of Nat'l Steel Corp v City of Ecorse, 227 Mich App 379, 389; 576 NW2d 667 (1998),
% we conclude that Patru's failure to persuade the Tribunal that the property's purchase price
% reflected the pre-repair TCV is irrelevant. The Tribunal independently had to evaluate all the
% evidence presented and, properly applying MCL 211.27(2), arrive at the property's TCV.)
% \end{quote}
% \pincite{Patru 1}{6}, footnote 3.

% This Court explained the law of the case in \pincite{Bennett}{499-500}:

% \begin{quote}
%   Under the doctrine of the law of the case, if an appellate court has passed on a legal question and remanded the case for further proceedings, the legal question will not be differently determined in a subsequent appeal in the same case were the facts remain materially the same. The primary purpose of the law-of-the-case doctrine is to maintain consistency and avoid reconsideration of matters once decided during the course of a single  % 500*500
%   continuing lawsuit. .~.~.  [T]he doctrine is discretionary, rather than mandatory.
% \end{quote}

% The law of the case precludes even correct arguments that could have been brought up in the first appeal.
% In \cite[s]{Bennett},
% the lower court had denied defendant's motion to terminate his child support early.
% While the case was on appeal, a law was enacted which supported the lower court's decision.
% The law took effect before the appeals court called the case.
% The appeals court reversed the lower court, but did not mention the new law.
% On remand, plaintiff asked the lower court to deny defendant's motion a second time,
% based on the new law which the appeals court had not addressed.
% The lower court refused because of the law of the case. Plaintiff appealed.

% The appeals court affirmed, reasoning that even though it thought that its first decision was wrong, by substituting its judgment for the prior appeals court panel, it would be doing ``the very activity the law-of-the-case doctrine is designed to discourage.'' \pincite{Bennett}{501}. If the prior panel was not aware of the new law, the parties were at fault because ``ultimately it is the responsibility of the parties to bring to this Court's attention that case law and those statutes that the parties wish the Court to consider in deciding the matter.'' \pincite{Bennett}{501}.

% Thus even if this Court agrees with the tribunal's latest decision,
% %that \mathieuGast\ does not cover this case,
% it should still reverse and affirm its earlier ruling. As in Bennett, Appellee could have made its arguments (that \mathieuGast\ does not apply if the repairs are made in the year of a transfer and that the property was in average condition before the repairs) before this Court on the first appeal.

% \ section{Standard of Review}

% % MCL 2.119(F)(3) says:

% % \begin{quote}
% %   Generally, and without restricting the discretion of the court, a motion for rehearing or reconsideration which merely presents the same issues ruled on by the court, either expressly or by reasonable implication, will not be granted. The moving party must demonstrate a palpable error by which the court and the parties have been misled and show that a different disposition of the motion must result from correction of the error.
% % \end{quote}

% A motion for reconsideration ``must demonstrate a palpable error by which the court and the parties have been misled and show that a different disposition of the motion must result from correction of the error.''
% \cite{MCL 2.119(F)(3)}.
%  A palpable error is a clear error ``easily perceptible, plain, obvious, readily
%  visible, noticeable, patent, distinct, manifest.''
%  \pincite{Luckow}{426; 453}\ (cleaned up).
%  \cite[s]{MCL 2.119(F)(3)}\ ``does not categorically prevent a trial court from revisiting an issue even when the motion for reconsideration presents the same issue already ruled upon; in fact, it allows considerable discretion to correct mistakes.''
% \pincite{Macomb County Department of Human Services}{754}. 

 
% % ``The palpable error provision in MCL 2.119(F)(3)  is not mandatory and only provides guidance to a court about when it may be appropriate to consider a motion for rehearing or reconsideration.'' \pincite{Walters}{350; 430}.

%  % \begin{quote}
%  %   The rule [MCL 2.119(F)(3)] does not categorically prevent a trial court from revisiting an issue even when the motion for reconsideration presents the same issue already ruled upon; in fact, it allows considerable discretion to correct mistakes.
%  % \end{quote}
%  % \pincite{Macomb County Department of Human Services}{754}. 
%  % 


Logically, the tribunal's decision rests on two independent grounds:

\begin{enumerate}
\item \mathieuGast\ does not apply in transfer years and the tribunal is not prohibited from considering this exception by the law of the case.
\item Even if nonconsideration is applied, the repairs did not affect value. Considering inflation, the before-repair value was essentially equal to the after-repair value. In particular, the subject's purchase price was not the before-repair value. 
\end{enumerate}

All of these propositions are proven false below. There is no exception to nonconsideration for transfer years. And even if there were such an exception, the law of the case requires that this should have been brought up on the first appeal. Second, the ``fact'' that the repairs did not affect value is not supported by the required evidence. And in particular, the tribunal did not properly consider the subject's purchase price. 

% Appellant asks that this Court reconsider this opinion, because this Court's answer to the first question contradicts the law of the case and also contradicts both the plain language of the statute and the tax act's overall scheme.

\subsection{Granting question three (rejecting a sale based on speculation) is proper under MCR 7.305(B)(2, 3, 5a, and 5b)}

standard of review and preservation

Per \cite{MCR 7.305(B)(2)}\ the third issue has significant public interest and the case is against one of the state's subdivisions.

The big-picture issue of this case from the point of view of question three and the public. Will all relevant evidence be considered fairly or will some evidence by excluded out of hand based on arbitrary per-se rules? Here the subject's actual sale was replaced with mls marketing pictures taken out of context and the previous year's assessed value which by the city's own admission assumed the point at issue.\footnote{The Tribunal Judge himself in his final opinion says that the Appellant's representative testified that properties are assessed under the assumption that they are in average condition. Thus the assessed value cannot be used to prove the actual condition of the property.}

Per \cite{MCR 7.305(B)(3)}\ the third issue involves a legal principle of major significance to the state's jurisprudence, namely, a correct determination of the true cash value of properties. 

Per \cite{MCR 7.305(B)(5a)}\ the Court of Appeals' handling of the third issue is clearly erroneous and will cause material injustice. Violates \cite{Jones & Laughlin}\ and prevent Appellant from receiving a fair hearing.

Per \cite{MCR 7.305(B)(5b)}\ the Court of Appeals' decision conflicts with a Supreme Court decision or another decision of the Court of Appeals. \cite{Jones & Laughlin}.


\subsubsection{This Court wrongly affirms the tribunal's rejection of the purchase price}

When reviewing Tax Tribunal cases, this Court looks for misapplication of the law or adoption of a wrong principle. Factual findings must be supported by competent, material, and substantial evidence on the whole record. Statutory interpretation is reviewed de novo. \pincite{Briggs}{75; 757-758}.
%The Tribunal's factual findings are accepted as final by this Court ``provided they are supported by competent, material, and substantial evidence. Substantial evidence must be more than a scintilla of evidence, although it may be substantially less than a preponderance of the evidence.'' \pincite[s]{Jones & Laughlin}{352-53}\ (cleaned up).

% Regarding the tribunal's rejection of the sale price, Appellant is not complaining that ``the tribunal erred by rejecting his 2015 purchase price of the
% property as determinative of its TCV.'' Opinion at 6. Rather, Appellant's brief at 18-19 complains that the tribunal gave the sale price ``no weight or credibility,'' \pincite{Tribunal's Denial of Reconsideration 2}{2}, contrary to the clear teaching of \pincite[s]{Jones & Laughlin}{353-354}: ``the price at which an item of
% property actually sold is most certainly relevant evidence'' and ``the tribunal's opinion that the evidence ``has little or no
% bearing'' on the property's earlier value suggests that the evidence was rejected out of hand. Such cursory rejection would be erroneous.''
% Appellant just wants the tribunal to consider the sale price and not reject it out of hand.

Contrary to this Court characterization,
% mischaracterizes Appellant's complaint about the tribunal's treatment of the subject's sale price.
Appellant is not complaining that ``the tribunal erred by rejecting his 2015 purchase price of the
property as determinative of its TCV.'' Opinion at 6. Rather, Appellant's brief at 18-19 complains that the tribunal gave the sale price ``no weight or credibility,'' \pincite{Tribunal's Denial of Reconsideration 2}{2}, contrary to the teaching of \pincite[s]{Jones & Laughlin}{353-354}:

\begin{quotation}
  [T]he price at which an item of
  property actually sold is most certainly relevant evidence .~.~.
  the tribunal's opinion that the evidence ``has little or \emph{no}
bearing'' on the property's earlier value suggests that the evidence was rejected out of hand. Such cursory rejection would be erroneous.
\end{quotation}
Emphasis in original. Appellant just wants the tribunal to consider the sale price and not reject it out of hand.

The opinion at 7 asserts that the tribunal ``considered petitioner's evidence of the 2015 purchase price'' and also ``considered the nature of the sale,'' but there is no record of such consideration \emph{involving the particular facts of this case.}\footnote{In this case the tribunal could have considered, but did not, that the subject house (1) had been listed by an independent, licensed real estate broker on the MLS, (2) had sold more than two years after it was first listed, (3) had had several offers on it, (4) was not reduced in price to make it sell faster, (5) sold at a price consistent with the repairs that were required.}
The Tribunal gave the sale price ``no weight or credibility'' because the seller was HUD.%
\footnote{In support of its rejection of the sale price, the tribunal also cites \cite[s]{MCL 211.27(6)}, prohibiting assessors from presuming that the selling price is the true cash value. But \pincite[s]{Jones & Laughlin}{354}, specifically considered and rejected this as justifying a cursory rejection.}
This amounts to a per se rule which excludes from consideration the sale price of all HUD-owned homes. This is the kind of cursory rejection that \cite[s]{Jones & Laughlin}\ forbids. Especially troubling is that the tribunal justifies this per se rule with bare speculation: ``Because the subject was being sold by a government entity, that entity's motivation may
not have been to receive market value for the property.'' \pincite{Tribunal's Denial of Reconsideration 2}{2}.

\subsection{Granting question four (necessity of determinating the before-repair value) is proper under MCR 7.305(B)(2, 3, 5a, and 5b)}

standard of review and preservation

Per \cite{MCR 7.305(B)(2)}\ the fourth issue has significant public interest and the case is against one of the state's subdivisions.

The big-picture issue of this case from the point of view of question four and the public. Will STC's guidance by followed, or can assessors and tribunal judges just do their own thing?

Is this just another way to create a first-year exception? If the previous year's assessed value is used for the before-repair value, then in many cases in which the city has not noted the below-average condition of the property, the result will be indistinguishable from an explicit first-year exception. 

Per \cite{MCR 7.305(B)(3)}\ the fourth issue involves a legal principle of major significance to the state's jurisprudence, namely, independent determination by tax tribunal judges. tk where does this requirement come from?

Per \cite{MCR 7.305(B)(5a)}\ the Court of Appeals' handling of the fourth  issue is clearly erroneous and will cause material injustice. Violates \cite{Jones & Laughlin}\ and is a back-door way to implement an year-of-sale exception.

Per \cite{MCR 7.305(B)(5b)}\ the Court of Appeals' decision conflicts with a Supreme Court decision or another decision of the Court of Appeals. \cite{Jones & Laughlin}. \cite{Jones & Laughlin}.

\subsubsection{If the Tribunal finds that normal repairs were made, it must determine the before-repair value.}

The STC requires it.

The Court of Appeals misread the STC Bulletin


\subsection{Granting question five (no basis for finding no effect of repairs on value) is proper under MCR 7.305(B)(2, 3, 5a, and 5b)}

standard of review and preservation

Per \cite{MCR 7.305(B)(2)}\ the fifth issue has significant public interest and the case is against one of the state's subdivisions.

The big-picture issue of this case from the point of view of question five and the public: must Judges be disciplined in their factual findings or can they cherry-pick evidence, use ``evidence'' that begs the question, ignore evidence they don't like, etc?

Per \cite{MCR 7.305(B)(3)}\ the fifth issue involves a legal principle of major significance to the state's jurisprudence, namely, correct even-handed evaluation of the evidence. No case will have exactly this evidence, but the manner in which evidence is handled can be the same. The Court needs to correct the Tribunal before this becomes a habit.

Per \cite{MCR 7.305(B)(5a)}\ the Court of Appeals' handling of the fourth  issue is clearly erroneous and will cause material injustice. This is clearly wrong here and results in higher taxes.

Per \cite{MCR 7.305(B)(5b)}\ the Court of Appeals' decision conflicts with a Supreme Court decision or another decision of the Court of Appeals. Violates the rule for evidence which is cited in many appellate cases.


\subsubsection{The ruling that the repairs did not affect the value  is unsupported}

The tribunal's factual findings must be supported by ``competent, material, and substantial evidence on the whole record.'' \pincite{Pontiac Country Club}{434}. ``Substantial evidence supports the Tribunal's findings if a reasonable person would accept the evidence as sufficient to support the conclusion.'' \pincite{Pontiac Country Club}{434}. 

Besides ruling that \mathieuGast\ did not apply, % because of uncapping,
the tribunal also ruled that the repairs did not affect the property's true cash value, or equivalently, that the before-repair value was the same as the after-repair value. The opinion says at 5: ``The
tribunal did not credit petitioner's argument that the property was in substandard condition when
he purchased it.''

But the property's substandard condition is not merely Appellant's argument. Throughout the first trip to this Court until the tribunal's last Final Opinion, all the actors in this case, the parties, the tribunal and this Court relied on the well-established fact that the subject property was in substandard condition before the repairs.

\begin{quote}
  It is undisputed that, when he purchased the property, it was in substandard condition and required numerous repairs to make it livable.
\end{quote}
\pincite{Patru 1}{1}.
This case was first appealed to this Court because:

\begin{quote}
  [t]he hearing referee incorrectly interpreted \mathieuGast\ by concluding that because the repairs were done to a property in substandard condition, they did not constitute normal repairs.
\end{quote}
\pincite{Patru 1}{5}. 
% At the second hearing before the tribunal, neither Petitioner/Appellant nor Respondent/Appellee presented evidence or took the position that the house was not in substandard condition before repairs.
The tribunal's reversal of a fact which was foundational to its first ruling was not based on new evidence: neither party argued that the house was not in substandard condition before repairs. Therefore this Court should take a careful look at the tribunal's reasoning for this drastic reinterpretation of the evidence.

The tribunal relied on MLS photographs and the prior year's assessed value:

\begin{quotation}
The tribunal found that
petitioner's MLS listing for the subject property showed a property in ``average'' condition, and
that petitioner's photographs of the property, before any repairs, showed ``a property that is livable
and habitable with reasonable marketability and appeal.'' .~.~.
%The tribunal noted that the purpose of petitioner's repairs was ``to ready the property as a tenant rental.''
It is undisputed that the assessed
TCV of the property for the 2015 tax year was \$48,000.
\end{quotation}
Opinion at 5. The tribunal misused the MLS evidence. While an MLS sale may be used to support or contest a valuation, no principle of appraisal allows what the tribunal does here: reject an MLS sale as evidence of value, but then use the MLS sale for its pictures to argue that the sale price should have been higher. The MLS sale on its face supports a value \$32,000, but the tribunal says it supports a value of \$48,000. Thus, as used by the tribunal, the MLS photographs are not competent, material, and substantial evidence. 

Here, there is direct, uncontested evidence that contradicts the tribunal's opinion of the MLS photographs. After having actually examined the house, the city's inspectors did not think that the house was livable and habitable, but instead, they required repairs before they would allow occupancy.

Regarding the prior year's assessed value, Respondent/Appellee testified:

\begin{quote}
  [M]ass appraisal does not account for properties one-by-one. In other words, properties are assessed uniformly and \emph{Respondent assumes properties are in ``average'' condition.}
\end{quote}
\foj[4], emphasis added. The assessed value here is just the output of a mass appraisal computer program that assumes average condition. It is not evidence that the house was in average condition.

Also, by presuming that the assessed value was correct,  the tribunal violates the teaching of \pincite{Pontiac Country Club}{435-36}: ``The Tribunal may adopt the assessed valuation on the tax rolls as its independent finding of true cash value when competent and substantial evidence supports doing so, \emph{as long as it does not afford the original assessment presumptive validity.}'' Emphasis added.




\subsection{The fact that this case is not published should not cause the Court to think that its bad practices and teaching won't spread if unchecked now}

\subsubsection{This is a case of first impression regarding the first year's exception to \mathieuGast, the per-se rule against bank sales or sales by government institutions, the obligation to independently determine the before-repair value, and the improper use of MLS pictures and previous-year assessed value. It is a precedent even if nonbinding.}

Other cases will follow. The tendency will be follow the lead of this case, especially if the Supreme Court has refused to correct it. Why pay a lawyer to craft an appeal, when someone has already tried and been shot down?

Appellant himself is counsel for five pending cases which are on hold pending the resolution of the issues in this case.

For cases involving residential fixer-uppers, it unlikely that an owner will have the resources, know-how, and willingness to appeal even to the Court of Appeals. It is likely that many more abuses like in the present case before this Court has an opportunity to correct.


Appellant in this case is perhaps unusual in being able to sustain an appellate review about a relatively small matter. There are many cases where the amount in controversy in terms of annual tax is too small to justify the cost of an effective advocate. 

A long time may pass before this Court sees again the issues raised in this case. In the interim, the abuses represented by the issues raised here (assuming Appellant is correct on the issues) will continue and many injustices will be finalized before an appellate court, much less this Court has a chance to correct them. Therefore Appellant respectfully urges this Court to correct the issues in this case.



\subsection{Question}
\subsubsection{Ground for Application}

Rule 7.305

(B)   Grounds. The application must show that

(1)   the issue involves a substantial question about the validity of a legislative act;

(2)   the issue has significant public interest and the case is one by or against the state or one of its agencies or subdivisions or by or against an officer of the state or one of its agencies or subdivisions in the officer's official capacity;

(3)   the issue involves a legal principle of major significance to the state's jurisprudence;

(4)   in an appeal before a decision of the Court of Appeals,

(a)   delay in final adjudication is likely to cause substantial harm, or

(b)   the appeal is from a ruling that a provision of the Michigan Constitution, a Michigan statute, a rule or regulation included in the Michigan Administrative Code, or any other action of the legislative or executive branches of state government is invalid;

(5)   in an appeal of a decision of the Court of Appeals,

(a)   the decision is clearly erroneous and will cause material injustice, or

(b)   the decision conflicts with a Supreme Court decision or another decision of the Court of Appeals; or

(6)   in an appeal from the Attorney Discipline Board, the decision is clearly erroneous and will cause material injustice.

\section{Standard of Review}

% MCL 2.119(F)(3) says:

% \begin{quote}
%   Generally, and without restricting the discretion of the court, a motion for rehearing or reconsideration which merely presents the same issues ruled on by the court, either expressly or by reasonable implication, will not be granted. The moving party must demonstrate a palpable error by which the court and the parties have been misled and show that a different disposition of the motion must result from correction of the error.
% \end{quote}

A motion for reconsideration ``must demonstrate a palpable error by which the court and the parties have been misled and show that a different disposition of the motion must result from correction of the error.''
\cite{MCL 2.119(F)(3)}.
 A palpable error is a clear error ``easily perceptible, plain, obvious, readily
 visible, noticeable, patent, distinct, manifest.''
 \pincite{Luckow}{426; 453}\ (cleaned up).
 \cite[s]{MCL 2.119(F)(3)}\ ``does not categorically prevent a trial court from revisiting an issue even when the motion for reconsideration presents the same issue already ruled upon; in fact, it allows considerable discretion to correct mistakes.''
\pincite{Macomb County Department of Human Services}{754}. 

 
% ``The palpable error provision in MCL 2.119(F)(3)  is not mandatory and only provides guidance to a court about when it may be appropriate to consider a motion for rehearing or reconsideration.'' \pincite{Walters}{350; 430}.

 % \begin{quote}
 %   The rule [MCL 2.119(F)(3)] does not categorically prevent a trial court from revisiting an issue even when the motion for reconsideration presents the same issue already ruled upon; in fact, it allows considerable discretion to correct mistakes.
 % \end{quote}
 % \pincite{Macomb County Department of Human Services}{754}. 

\subsubsection{Legal Argument}
