
\subsection{keep}
Winifred answer

This is a normal tax case

Five other cases are being held in abeyance.

It’s special because it highlights how the tribunal and coa mishandle mg

Its commonness is an argument for hearing the case and ensuring that the law is applied correctly in this case and in the many others like it.

P’s question involves the correct application of mg. If this is understood, then the COA’s mistakes will be apparent

Fn: r does not address this directly

First note the context: mcl 200.27 defining TCV

Mg temporarily removes from TCV the value from normal repairs

To ensure that the repairs are accounted for and not ignored, the statute requires that the value contributed by the repairs be noted in the assessment roll

Otherwise, mg does not change how TCV is determined, specifically mg does not excuse the tribunal from making its own independent determination of TCV per Jones . Also, if the house was over assessed before the repairs, the repairs should not prevent a correction of the over assessment.

Note that the stc bulletins agree with before and after appraisals and the tax tribunal explicitly claimed that it was not obligated to follow the stc’s guidance. The coa misquoted the bulletin to make it appear that there was no conflict.

Applying this standard to this case, the errors become apparent

The COA’s uncapping exception is a straight up violation of mg. It counts normal repairs because of a prior sale whereas mg explicitly requires non consideration until the property is sold.

Besides violations the plain language of the statute, the coa errs in failing to apply mg to help define TCV, essentially creating a custom definition which applied only m uncapping years

Secondly, the coa errs in ignoring normal repairs because of the prior years assessment. This violates mg’s requirement to account for the value of the repairs in the assessment roll. It also violates Jones’s requirement of independent determination. Mg does not change jones. Just as it would violate Jones if the tribunal began its analysis by assuming the correctness of the prior years TCV in a regular case, so in a mg case, assuming the correctness of the prior year’s TCV to determine the before repair value violates Jones.

Regarding law of the case, appellant has chosen not to allege the error to focus the case on clarifying the law regarding mg. But see motion for reconsideration which cites Bennett.

\subsection{750 words}
Respondent in its brief characterized this case as a "garden-variety tax case." This is true. The situation in this case is quite common. In fact, Petitioner is counsel in five other cases before the Tribunal which are being held in abeyance pending the resolution of this case. The situation in these cases is this: a buyer buys a house at a discount because the house needs repairs. The house is worth less than its assessed value because the mass appraisal method used by assessors assumes that all the houses are in average condition.

Respondent, the Tax Tribunal, and the Court of Appeals believe that to gain the benefit of Mathieu-Gast, such a buyer must keep the house in disrepair, possibly vacant, until the New Year so that the buyer can obtain a lower assessment based on the poor condition of the house on Tax Day, December 31. After the New Year, the buyer can make normal repairs which, per Mathieu-Gast, will not be considered for assessment purposes until after the buyer sells the property.

The Tax Tribunal gave three reasons for this view. First, MG does not apply to a house in substandard condition. This was reversed by the first COA to hear the case. Second, MG contains a hidden, uncapping exception, which requires that normal repairs be considered on the first tax day after a sale. Third, if the property was assessed in normal condition in the year before the repairs, then an assessment valuing the property in normal condition satisfies MG because the assessor did not need to consider the repairs for the assessment. The second COA upheld these last two reasons.

On the other hand, Petitioner believes that Mathieu-Gast allows such a buyer to make normal repairs right away and have those repairs "not considered" for assessment purposes in a tax appeal. He bases this first on the plain language of the statute which require nonconsideration "until the property is sold."

 


\section{Relief}

Therefore, under \cite{MCR 7.305(B)(3)}, \cite{MCR 7.305(B)(5)(a)}, and \cite{MCR 7.305(B)(5)(b)}, this Court should take up this case on appeal.
