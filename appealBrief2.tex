%
% This is one of the samples from the lawtex package:
% http://lawtex.sourceforge.net/
% LawTeX is licensed under the GNU General Public License 
%
\providecommand{\documentclassflag}{}
\documentclass[12pt,\documentclassflag]{michiganCourtOfAppealsBrief} 
\usepackage{etoolbox}
\usepackage[margin=1in]{geometry}
\usepackage{newcent,microtype}
\usepackage{setspace,xcolor}
\usepackage{trace}
\usepackage[T1]{fontenc}

% From tex.stackexchange.com (https://tex.stackexchange.com/a/142258/135718)
\usepackage{xcolor}
\usepackage{todonotes}
% horizontal rule with text, https://tex.stackexchange.com/a/15122/135718
\newcommand*\ruleline[1]{%
  \par\noindent%
  \raisebox{.8ex}{\makebox[\linewidth]{\hrulefill\hspace{1ex}%
      \raisebox{-.8ex}{#1}%
      \hspace{1ex}\hrulefill}}}%

% https://www.overleaf.com/learn/latex/Environments
\newenvironment{draft}{%
  \color{blue}%
  \ruleline{Begin Draft}
}{%
  \ruleline{End Draft}
  \color{black}%
}
% 
% 

% \usepackage{xr-hyper}
\usepackage{xr}
\externaldocument{appendix}
\usepackage[hyperindex=false,linkbordercolor=white]{hyperref}


\makeandletter% use \makeandtab to turn off

% Use this to show a line grid-
% \usepackage[fontsize=12pt,baseline=24pt,lines=27]{grid}
% \usepackage{atbegshi,picture,xcolor} % https://tex.stackexchange.com/a/191004/135718
% \AtBeginShipout{%
%   \AtBeginShipoutUpperLeft{%
%     {\color{red}%
%     \put(\dimexpr -1in-\oddsidemargin,%
%          -\dimexpr 1in+\topmargin+\headheight+\headsep+\topskip)%
%       {%
%        \vtop to\dimexpr\vsize+\baselineskip{%
%          \hrule%
%          \leaders\vbox to\baselineskip{\hrule width\hsize\vfill}\vfill%
%        }%
%       }%
%   }}%
% }
%   \linespread{1}

\usepackage[modulo]{lineno}% use \linenumbers to show line numbers, see https://texblog.org/2012/02/08/adding-line-numbers-to-documents/

\chardef\_=`_% https://tex.stackexchange.com/a/301984/135718 

%%Citations
 
%The command \makeandletter turns the ampersand into a printable character, rather than a special alignment tab \makeandletter

% %Set the information for the title page (later produced by \makefrontmatter)
% \docket{No. 10-553} 
% \appellant{Daniel Patru}
% \appellee{City of Wayne}
% \court{Michigan Court of Appeals}
% \circuit{Sixth}
% \brieffor{Appellant}
% \author{Daniel Patru , P74387\\{\em Petitioner}}
% \address{3309 Solway \\ Knoxville, TN 37931\\ (734) 274-9624}

\begin{document}
\singlespacing%

\citecase[Mich Ed]{Mich Ed Ass’n v Secretary of State
  (On Rehearing), 489 Mich 194, 218; 801 NW2d 35 (2011)}
% (stating that nothing will be read into a clear statute that is not within the manifest intention of the Legislature as derived from the language of the statute itself).

\citecase[Patru]{Patru v City of Wayne, unpublished per curiam opinion of the Court of Appeals, issued May 8, 2018 (Docket No. 337547)}%
\addReference{Patru}{patruvwayne}% Associate appendix with case

\citecase[Antisdale]{Antisdale v City of Galesburg, 420 Mich 265, 362 NW2d 632 (1984)}
\citecase[Briggs]{Briggs Tax Service, LLC v Detroit Pub. Schools, 485 Mich 69; 780 NW2d 753 (2010)}
\citecase[Jones & Laughlin]{Jones & Laughlin Steel Corporation v. City of Warren, 193 Mich App 348; 483 NW2nd 416 (1992)}
\citecase[Rovas]{In re Complaint of Rovas Against SBC Michigan, 482 Mich 90; 754 NW2d 259 (2008)}

% \citecase[Fisher]{Fisher v. Sunfield Township, 163 Mich App 735; 415 NW2nd 297 (1987)}
% \citecase{Arbaugh v. Y&H Corp., 546 U.S. 500 (2006)}
% \citecase[CAF]{CAF Investment Co. v Saginaw Twp, 410 Mich 428; 302 NW2d 164 (1981)}
% \citecase[Coyne]{Coyne v Highland Twp, 169 Mich. App. 401; 425 NW2d 567 (1988)}
% \citecase[Malpass]{Malpass v Dep't of Treasury, 494 Mich. 237; 833 NW2d 272 (2013)}
% \citecase[Meadowlanes]{Meadowlanes Ltd Dividend Housing Ass'n v City of Holland, 437 Mich 473; 473 NW2d 636 (1991)}
% \citecase[Plymouth]{Plymouth Twp. v Wayne County Board of Commissioners, 137 Mich App 738; 359 NW2d 547 (1984)}
% \citecase[Professional Plaza]{Professional Plaza v City of Detroit, 647 NW2d 529; 250 Mich. App. 473  (2002)}
% \citecase[Randolf]{Dep't of Transportation v Randolph, 461 Mich. 757; 610 NW2d 893 (2000)}
% %\citecase[SBC Health Midwest]{SBC Health Midwest, Inc. v City of Kentwood,  \_\_  Mich \_\_ (decided May 1, 2017)}
% \citecase[Stevens]{Stevens v Bangor Twp, 150 Mich App 756; 389 NW2d 176 (1986)}
 
{\makeatletter % needed for optional argument to newstatute.
  % \newstatute[1@]{MCL}{}% place MCL first
  % \newstatute[2@]{MCR}{}
  % \newstatute[3@]{TTR}{}
  % \newstatute[4@]{Dearborn Ordinance}{}% place this fourth
  % \newstatute[5@]{Wayne Ordinance}{}
  \newstatute{MCL 211.27(2)}{}% MathieuGast
%   (2) The assessor shall not consider the increase in true cash value that is a result of expenditures for normal repairs, replacement, and maintenance in determining the true cash value of property for assessment purposes until the property is sold. For the purpose of implementing this subsection, the assessor shall not increase the construction quality classification or reduce the effective age for depreciation purposes, except if the appraisal of the property was erroneous before nonconsideration of the normal repair, replacement, or maintenance, and shall not assign an economic condition factor to the property that differs from the economic condition factor assigned to similar properties as defined by appraisal procedures applied in the jurisdiction. The increase in value attributable to the items included in subdivisions (a) to (o) that is known to the assessor and excluded from true cash value shall be indicated on the assessment roll. This subsection applies only to residential property. The following repairs are considered normal maintenance if they are not part of a structural addition or completion:
% (a) Outside painting.
% (b) Repairing or replacing siding, roof, porches, steps, sidewalks, or drives.
% (c) Repainting, repairing, or replacing existing masonry.
% (d) Replacing awnings.
% (e) Adding or replacing gutters and downspouts.
% (f) Replacing storm windows or doors.
% (g) Insulating or weatherstripping.
% (h) Complete rewiring.
% (i) Replacing plumbing and light fixtures.
% (j) Replacing a furnace with a new furnace of the same type or replacing an oil or gas burner.
% (k) Repairing plaster, inside painting, or other redecorating.
% (l) New ceiling, wall, or floor surfacing.
% (m) Removing partitions to enlarge rooms.
% (n) Replacing an automatic hot water heater.
% (o) Replacing dated interior woodwork.
  
  \newstatute{MCL 211.10d(7)}{}%
%  (7) Every lawful assessment roll shall have a certificate attached signed by the certified assessor who prepared or supervised the preparation of the roll. The certificate shall be in the form prescribed by the state tax commission. If after completing the assessment roll the certified assessor for the local assessing district dies or otherwise becomes incapable of certifying the assessment roll, the county equalization director or the state tax commission shall certify the completed assessment roll at no cost to the local assessing district.


  
  \newstatute{MCL 205.753(2)}{}% allows appeals from a final order of the Tax Tribunal

  \newstatute{MCR 7.204(A)(1)(b)}{}% allows appeals within 21 days of an order on a motion for reconsideration
  
}% end makeatletter block
% \def\mathieuGast{\pincite[l]{MCL}{211.27(2)}}
\def\mathieuGast{\cite[s]{MCL 211.27(2)}}%
%\def\ttr287{\pincite[s]{TTR}{287}}
%\def\inspectionOrdinance{\pincite{Wayne Ordinance}{\S1484.04}}
% \long\def\inspectionOrdinanceText{\begin{quote}
% 1484.04  CERTIFICATE REQUIRED PRIOR TO SALE. 
%    It shall be unlawful to sell, convey or transfer an ownership interest, or act as a broker or agent for the sale, conveyance or transfer of an ownership interest, in any residential dwelling unless and until a valid Certificate of Compliance is first issued. 
% (Ord. 1991-10.  Passed 7-16-91.) 
% \end{quote}}


%\newmisc{STC Bulletin 6 of 2007}{Michigan State Tax Commission (STC) Bulletin No. 6 of 2007 (Foreclosure Guidelines)}
\newmisc{STC Bulletin}{Michigan State Tax Commission (STC) Bulletin No. 7 of 2014 (Mathieu Gast Act)}
\addReference{STC Bulletin}{bulletin7}% Associate appendix with case

\newcommand{\makeAbbreviation}[3]{% ensure that the frsit time an abbreviated word is used, it is presented in long form, and after that in short form. 1: command name, 2: short name, 3: long name
  \IfBeginWith{#3}{#2}{%
    \newcommand{#1}[0]{#3\renewcommand{#1}[0]{#2}}}{%
    \newcommand{#1}[0]{#3 (#2)\renewcommand{#1}[0]{#2}}}}

\makeAbbreviation{\MLS}{MLS}{Multiple Listing Service}
\makeAbbreviation{\MTT}{MTT}{Michigan Tax Tribunal}
\makeAbbreviation{\STC}{STC}{State Tax Commission}
%\makeAbbreviation{\FOJ}{FOJ}{First Final Opinion and Judgment (2017)}
% \makeAbbreviation{\explanatoryLetterAbbr}{Explanatory Letter}{Explanatory Letter submitted by Appellant to the Tax Tribunal on 9/6/2018}
% \newcommand{\explanatoryLetter}[1][]{\explanatoryLetterAbbr\if\relax#1\relax\empty, Appendix at \pageref{explanatoryLetter}\else, page #1, Appendix at \pageref{explanatoryLetter.#1}\fi}

% Note that the command/handle must match the appendix
% labels. Minimize the variations of names.  \makeAbbreviationToRecord
% creates a simple abbreviation that ends in Abbr if you don't want to
% refer to the record.
\newcommand{\makeAbbreviationToRecord}[3]{% 1: command/handle, 2: shortname, 3: longname
  % makeAbbreviationToRecord: #1, #2, #3\par%
  \expandafter\makeAbbreviation\csname #1Abbr\endcsname{#2}{#3}%
  \expandafter\newcommand\csname #1\endcsname[1][]{%
    \Call{#1Abbr}%
    \if\relax##1\relax\empty\ (Appendix at \pageref{#1})%
    \else, p ##1 (Appendix at %
    % Check for page range
    \IfSubStr{##1}{-}{%
      \def\pageRefRange####1-####2XXX{\pageref{#1.####1}--\pageref{#1.####2}}%
      \pageRefRange##1XXX}%
    {\pageref{#1.##1}}%
    )\fi%
  }%
}%

\makeAbbreviationToRecord{explanatoryLetter}{Explanatory Letter}{Explanatory Letter (MTT Docket Line 38)}
% explanatory Letter: (abbr) \explanatoryLetterAbbr\ (to record) \explanatoryLetter[2] \par

\makeAbbreviationToRecord{foj}{FOJ}{Second Final Opinion and Judgment (MTT Docket Line 48)}
% FOJ: (abbr) \fojAbbr (to record) \foj[]\par

\makeAbbreviationToRecord{reconsiderationDenied}{Order Denying Reconsideration}{Order Denying Reconsideration (MTT Docket Line 51)}
% reconsiderationDenied: (abbr) \reconsiderationDeniedAbbr[] (to record) \reconsiderationDenied[] \par

\makeAbbreviationToRecord{repairs}{List of Repairs}{List of Repairs (MTT Docket Line 36)}
% \par repairs: (abbr) \repairsAbbr\ (to record) \repairs[] (to appendix) \pageref{repairs}\par

\makeAbbreviationToRecord{stcform}{STC Form 865}{STC Form 865 Request for Nonconsideration (MTT Docket Line 35)}

\makeAbbreviationToRecord{mlsListing}{MLS Listing}{MLS Listing (MTT Docket Line 32)}
% mlsListing: (abbr) \mlsListingAbbr\ (to record) \mlsListing[]\par

\makeAbbreviationToRecord{mlsHistory}{MLS History}{MLS History (MTT Docket Line 33)}
% mlsHistory: (abbr) \mlsHistoryAbbr\ (to record) \mlsHistory[]\par

\makeAbbreviationToRecord{boardOfReviewDecision}{Board of Review Decision}{Board of Review Decision (MTT Docket Line 2)}

\makeAbbreviationToRecord{cityEvidence}{City's Evidence}{City's Evidence (MTT Docket Line 11)}

\makeAbbreviationToRecord{motionForReconsideration}{Motion for Reconsideration}{Motion for Reconsideration (MTT Docket Line 52)}

\begin{centering}
\bf\scshape State of Michigan\\In the Court of Appeals\\Detroit Office\\~\\ 
\rm 

\makeandtab
\setlength{\tabcolsep}{20pt}%
\begin{tabular}{p{.4\textwidth} p{.4\textwidth}}
\cline{1-2}
  {~

  \raggedright Daniel Patru,\par
  \hfill\textit{Petitioner/Appellant,}
  \vspace{.5\baselineskip}\par
  vs\par
  \vspace{.5\baselineskip}
  \raggedright City of Wayne,\par
  \hfill\textit{Respondant/Appellee.}
  
  ~} &  {~
       \par\par
       \hfill Court of Appeals No. 346894\par
       \hfill Lower Court No. 16-001828-TT\par\vspace{\baselineskip}
       \hfill \textbf{Appellant's Brief}\par
       \hfill \textbf{Proof of Service}       
  ~}
  \\ \cline{1-2}\vspace{2mm}
  {~ \par
  Daniel Patru, P74387, Appellant\newline%
  3309 Solway\newline%
  Knoxville, TN 37931\newline%
  (734) 274-9624\newline%
  dpatru@gmail.com\newline\newline%
  ~} & {~ \par~\par
       
       City of Wayne, Appellee\newline%
       3355 South Wayne Rd,\newline%
       Wayne, MI 48184\newline%
       (734) 722-2000\newline\newline%

       Stephen J. Hitchcock (P15005)\newline%
       John C. Clark (P51356)\newline%
       Attorneys for Respondent/Appellee\newline%
       Giarmarco, Mullins \& Horton, P.C.\newline%
       101 W. Big Beaver Road, 10th floor\newline%
       Troy, MI 48084\newline%
       (248) 457-7024\newline%
       sjh@gmhlaw.com
  ~}
\end{tabular}
\makeandletter
\par\vspace{\baselineskip}\vspace{\baselineskip}\vspace{\baselineskip}
\textbf{ORAL ARGUMENT NOT REQUESTED}

\end{centering}

\pagestyle{romanparen}
\pagenumbering{roman}
\newpage 

\section*{Table of Contents}

\tableofcontents

\newpage
\tableofauthorities

\pagestyle{plain}
\pagenumbering{arabic}

%This commands creates the title page, table of contents, and table of authorities
% \makefrontmatter{Brief\\Proof of Service}


%Sets the formatting for the entire document after the front matter
\parindent=2.5em
% \setlength{\parskip}{1.25ex plus 2ex minus .5ex} 
% \setstretch{1.45}
\doublespacing
% \linenumbers

\section{Jurisdictional Statement}
 
The Court of Appeals has jurisdiction over this claim of appeal under \cite{MCL 205.753(2)}\ (allowing appeals from a final order of the Tax Tribunal) and \cite{MCR 7.204(A)(1)(b)}\ (requiring appeals to be made within 21 days after the entry of an order deciding a motion for reconsideration). The \fojAbbr\ from the Tax Tribunal was entered 12/04/2018. A motion for reconsideration was filed 12/05/2018. The order denying reconsideration was entered 12/14/2018. The claim of appeal for this case was filed 12/18/2018, within the 21 days required.

\newpage 
\section{Introduction}

This case concerns the application of the Mathieu-Gast Home Improvement Act, \cite[s]{MCL 211.27(2)}, which instructs assessors to ``not consider'' the value of normal repairs when determining the true cash value of residential property for assessment purposes, but rather, to separately indicate the value of the repairs on the assessment roll.
The State Tax Commission (STC) requires that the value of the repairs be determined by before-repair and after-repair appraisals.

In 2015, Appellant bought a house for \$32,000 and made about \$10,000 in repairs to it. The house was repaired by tax day (12/31/2015) when the City of Wayne assessed the house at \$50,400 true cash value.

The City's assessment did not apply Mathieu-Gast nonconsideration: it assessed the house at its after-repair value and did not exclude the value of the repairs from the true cash value nor indicate the value of the repairs on the assessment roll.

The Appellant has asked the Tax Tribunal to apply nonconsideration, but so far the Tribunal has refused. The Tribunal first ruled that the repairs were not normal repairs because the house was in substandard condition when Appellant bought it.

This Court reversed because the statute does not exclude repairs to a house in substandard condition.
% This Court reversed, ruling that the condition of the house was statutorily irrelevant to whether repairs were normal repairs.
This Court remanded for a rehearing because the hearing referee's misunderstanding of the statute made it impossible to determine whether Appellant had proved at the hearing that the repairs were normal.

After the rehearing, the Tribunal accepted the repairs as normal but still refused to apply nonconsideration because it believed that:
1) the assessor had not considered the repairs in the assessment,
2) a before-repair appraisal was not required, and
3) the selling price of the property was entitled to no weight or credibility.

% After the rehearing, the Tribunal accepted the repairs as normal but still refused to apply nonconsideration because it believed that:

% \begin{enumerate}
% \item The Tribunal was not required to determine the before-repair value if it disagreed with Appellant's evidence on that point; and
% \item The assessor had increased the true cash value because of inflation, not repairs. There was some evidence to believe that the property had been in average condition all along: a) the city's assessment before Appellant purchased the house and after it was repaired differed by just 5\% and b) the MLS marketing photos showed that the house was in average condition. 
% \end{enumerate}


% Appellant bought a house that needed repairs. He repaired it. He wants to be taxed on the before-repair value of the house because \mathieuGast\ instructs the assessor to ``not consider'' the value of normal repairs for assessment purposes.

% Appellee, the City of Wayne, wants to tax based on the value of the house after repairs. In other words, the City wants to include whatever value was added by the repairs in the true cash value for assessment purposes.

% After the first hearing, the Tribunal agreed with Appellee. It refused to offer Appellant the benefits of \mathieuGast in part because it thought that the repairs were not ``normal repairs'' under statute (i.e. the repairs did not qualify for nonconsideration) because the house was in substandard condition. It set the true cash value to the value of the house after repairs.

% This Court reversed, ruling that the poor condition of the house is statutorily irrelevent to whether repairs were normal repairs. It remanded the case back to the Tribunal for a rehearing to determine whether the repairs had been normal repairs under the statute. 

% After the rehearing, the Tribunal found that the repairs had been normal repairs, but that the true cash value should still be set to the value of the house after repairs for two reasons:


\newpage 
\section{Questions Involved} 

% \noindent I. After finding normal repairs, did the Tribunal violate \mathieuGast\ when it refused to apply nonconsideration treatment?

\noindent I. Did the Tribunal err when it refused to apply nonconsideration to normal repairs?

Appellant answers yes. The Tribunal answers no. The Appellee's answer is unknown. 
\vspace{\baselineskip}

% \noindent II. After finding normal repairs, did the Tribunal violate \mathieuGast\ by refusing to determine the before-repair value?

\noindent II. Did the Tribunal err when it refused to determine the before-repair value?

Appellant answers yes. The Tribunal answers no. The Appellee's answer is unknown. 
\vspace{\baselineskip}

\noindent III. Did the Tribunal err when it gave the selling price of the property before-repair ``no weight or credibility'' based solely on the fact that the seller was HUD and speculation that the seller may not have been motivated to receive market value?

Appellant answers yes. The Tribunal answers. The Appellee's answer is unknown. 


% \noindent I. Did the Tribunal fail in its duty to independently determine the true cash value when it refused to determine the before-repair value because a) Appellant had the burden of proof and 2) the Tribunal disagreed with Appellant's proof?

% Appellant answers yes. The Tribunal answers no. The Appellee's answer is unknown. 
% \vspace{\baselineskip}

% \noindent II. Did the Tribunal misinterpret \mathieuGast\ when it ignored normal repairs bec`ause there was some evidence that the assessor did not need to consider repairs to justify the assessed value?

% Appellant answers yes. The Tribunal answers no. The Appellee's answer is unknown.

% \vspace{\baselineskip}

% \noindent III. Did the Tribunal err when it gave the selling price of the property ``no weight or credibility'' as to the market value of the property based solely on a) the fact that the seller was a government entity and b) speculation that the seller may not have been motivated to receive market value?

% Appellant answers yes. The Tribunal answers. The Appellee's answer is unknown. 
 
\newpage

\section{Facts}
\label{facts}
\subsection{Appellant buys the house, repairs it, and appeals its assessment}

Appellant purchased the subject house in August 2015 for \$32,000. The house was sold by the Department of Housing and Urban Development (HUD) through a real estate broker who had listed the house on the MLS. \mlsListing[]. At the time of purchase, the house needed numerous repairs, most of which were required by the City to obtain a Certificate of Occupancy. \repairs[]. By tax day, 12/31/2015, Appellant had repaired the house and rented it. \foj[4-5].

The City of Wayne assessed the house on tax day at \$50,400 true cash value, rather than the \$32,000 purchase price. \boardOfReviewDecision.

Appellant appealed to the Board of Review and then to the Tax Tribunal. Appellant does not dispute that the house was worth \$50,400 on tax day in its repaired condition. But he contends that under \mathieuGast\ the repairs were normal repairs and that the true cash value for assessment purposes cannot include the value of the repairs. He contends that the correct true cash value is therefore the before-repair value. \explanatoryLetter[2].

Appellant contends that the best evidence of the house's before-repair value is its sale price of \$32,000 when it was unrepaired. Id. The house was marketed in the normal way and for a sufficient time. Licensed real estate brokers listed the house on the MLS, initially for \$29,900 on 4/3/2013 and later for \$32,000 on 6/17/2015. Before Appellant bought the property there had been at least two accepted offers on the property that failed to close. \mlsHistory[]. 

The City contends that the true cash value should be the after-repair value. \cityEvidence.

\subsection{The Tribunal refuses to apply nonconsideration but this Court reverses}

The Referee who first heard the case at the Tribunal refused to apply nonconsideration. She held that the repairs were not normal repairs because the house was in substandard condition. ``Thus, the referee determined that if a property is purchased in substandard condition, any repairs done on the property to bring it into good repair do not constitute normal repairs, maintenance, or replacement within the meaning of MCL 211.27(2), so the increase in TCV resulting from those repairs can be immediately considered in determining the TCV for assessment purposes.'' \pincite{Patru}{2}.

This Court reversed. ``Nothing in MCL 211.27(2) provides that the repairs .~.~. are not normal repairs in the event that they are performed on a substandard property. Thus, by reading a requirement into the statute that was not stated by the legislature, the trial court erred .~.~.'' \pincite{Patru}{5}.

This Court remanded for a rehearing to determine if the repairs were normal repairs because ``[t]he referee did not fully evaluate that evidence---which included testimony---because [she] misapprehended how to properly apply MCL 211.27(2).'' \pincite{Patru}{5}.

\subsection{On rehearing, the Tribunal again refuses to apply nonconsideration}

The rehearing was held in December 2018 before Tribunal Judge Marcus L. Abood. He ruled that the repairs were normal repairs, worth approximately \$10,000. \foj[4]. However, he went on to rule on pages 5 and 6 of the FOJ that ``Petitioner's contentions related to his purchase price and `normal' repairs are given no weight or credibility in the determination of market value'' because:
\begin{enumerate}
\item The property's assessment for 2016 changed based on the sale transaction in August 2015 and not based on Petitioner's repairs. 
\item The assessment considered the property in average condition as it was on 12/31/2015, tax day.
\item The MLS photographs submitted by Petitioner show the property in average condition, not neglected or vandalized.
\item The property's sale was not ``an arm's length sale transaction'' because the Petitioner has not claimed it so and because the seller was HUD.
\end{enumerate}

Judge Abood did not determine the property's before-repair value. Instead, he ruled that the property must be valued in its repaired condition because it was repaired before tax day. \foj[5]. He accepted the City's comparative sales analysis, writing: ``Respondent's sales adjustment grid does not included a line-item entry for repairs. This comparative analysis is devoid of any relationship to Petitioner's ``normal'' repairs to subject property which occurred before the issuance of a certificate of occupancy and the December 31, 2015 tax day.'' \foj[6].

\subsection{Appellant files Motion for Reconsideration}

Appellant responded to the FOJ with a motion for reconsideration. The motion points out that this Court reversed the previous judgment of the Tribunal because  ``the referee's finding that the property's TCV was \$50,400 was based on its assessment of the property's value after it had been repaired.'' \motionForReconsideration[1]. The Tribunal was repeating the mistake, except this time after admitting that the repairs were normal repairs. Id.\ p 2. 

The Motion for Reconsideration also pointed out that the Tribunal had not done before-repair and after-repair appraisals as required by \mathieuGast. Id.\ p 4.

The Motion for Reconsideration also pointed out flaws in the four reasons given in the FOJ for giving no weight or credibility to Appellant's contentions related to his purchase price and normal repairs. Specifically:

\begin{enumerate}
\item The FOJ claimed the property's assessment changed based on the sale transaction. Appellant pointed out that assessments are not based on sales but rather are done yearly. The property \emph{uncapped} as a result of the sale, but Appellant had no issue with uncapping. Id.\ p 4.

\item The FOJ claimed the assessment considered the property in average condition as it was on 12/31/2015, tax day. Appellant pointed out that this does not invalidate \mathieuGast\ which requires the removal of the contribution of normal repairs to the assessed value. Id.\ p 5. 

\item The FOJ claimed that the MLS photographs submitted by Petitioner show the property in average condition, not neglected or vandalized. Appellant pointed out that photos do not excuse performance of \mathieuGast; nor do they contradict the fact that the City itself inspected the house and required repairs; nor do the photos, put in the MLS by the real estate brokers who listed and sold the house, show that the house was listed and sold at a non-market price. Id. p 5.
  
\item The FOJ claimed that the property's sale was not an arm's length sale transaction because the Petitioner has not claimed it so and because the seller was HUD. Appellant pointed out that he had included MLS data to show that the property's sale was a market sale and that merely mentioning that the seller was HUD is not evidence that the sale was not a market sale. Id.\ p 6.
\end{enumerate}

\subsection{The Tribunal denies the Motion for Reconsideration}

Tribunal Judge David B. Marmon, instead of Judge Abood, denied Motion for Reconsideration. He did not specifically rebut Appellant's points made in the motion. Instead he clarified why the Tribunal was not applying \mathieuGast. The \reconsiderationDenied[2] contains the Tribunal's reasoning. In summary:

\begin{enumerate}
  
\item The Tribunal did not determine the before-repair value because:
  \begin{enumerate}
  \item the text of the statute does not plainly require a before-repair appraisal;
  \item the situations in the second sentence of MCL 211.27(2) are not at issue;
  \item the STC guidance requires appraisal if the value of the repairs are on the assessment roll, and here they are not;
  \item STC guidance lacks the force of law; and
  \item Petitioner has the burden of proof and the Tribunal disagrees with Petitioner's evidence (the sale price is given no weight or credibility).
  \end{enumerate}
  
\item The Tribunal gave the property's sale price ``no weight or credibility'' because the seller was a government entity (HUD) who may not have been motivated to receive market value.
  
\item The Tribunal did not give nonconsideration treatment to the normal repairs because the assessment did not consider the repairs. The property's true cash value assessment was \$48,000 in the year before the repairs and \$50,400 after the repairs. The 5\% change was due to inflation not repairs. Also, the pictures on the MLS showed the property in average condition.

\end{enumerate}

% Points 1, 2, and 4 are addressed in order in the Arguments. Point 3 is addressed in the discussion of point 2.  

\section{Mathieu-Gast Statute -- MCL 211.27(2)}
\begin{quotation}
The assessor shall not consider the increase in true cash value that is a result of expenditures for normal repairs, replacement, and maintenance in determining the true cash value of property for assessment purposes until the property is sold.

For the purpose of implementing this subsection, the assessor shall not increase the construction quality classification or reduce the effective age for depreciation purposes, except if the appraisal of the property was erroneous before nonconsideration of the normal repair, replacement, or maintenance, and shall not assign an economic condition factor to the property that differs from the economic condition factor assigned to similar properties as defined by appraisal procedures applied in the jurisdiction.

The increase in value attributable to the items included in subdivisions (a) to (o) that is known to the assessor and excluded from true cash value shall be indicated on the assessment roll.

This subsection applies only to residential property.

The following repairs are considered normal maintenance if they are not part of a structural addition or completion: [repairs (a)-(o) omitted]

% (a) Outside painting.
% (b) Repairing or replacing siding, roof, porches, steps, sidewalks, or drives.
% (c) Repainting, repairing, or replacing existing masonry.
% (d) Replacing awnings.
% (e) Adding or replacing gutters and downspouts.
% (f) Replacing storm windows or doors.
% (g) Insulating or weatherstripping.
% (h) Complete rewiring.
% (i) Replacing plumbing and light fixtures.
% (j) Replacing a furnace with a new furnace of the same type or replacing an oil or gas burner.
% (k) Repairing plaster, inside painting, or other redecorating.
% (l) New ceiling, wall, or floor surfacing.
% (m) Removing partitions to enlarge rooms.
% (n) Replacing an automatic hot water heater.
% (o) Replacing dated interior woodwork.
\end{quotation}

\section{Argument}

\subsection{The Tribunal erred when it refused to apply nonconsideration to normal repairs}

The Tribunal violated \mathieuGast\ when it refused to apply nonconsideration treatment to normal repairs. This is an error of statutory interpretation. In this case, the Tribunal's reasoning also lacks factual support. Both of these are addressed below.

\subsubsection{The Tribunal misinterpreted MCL 211.27(2) in ignoring normal repairs}

When reviewing Tribunal cases, this Court looks for misapplication of the law or adoption of a wrong principle. Statutory interpretation is reviewed de novo. \pincite{Briggs}{75; 757-758}. 

\mathieuGast\ reads in relevant part:

\begin{quote}
The assessor {\em shall} not consider the increase in true cash value that is a result of expenditures for normal repairs .~.~. in determining the true cash value of property for assessment purposes until the property is sold.~.~.~. The increase in value attributable to [normal repairs] that is known to the assessor and excluded from true cash value {\em shall} be indicated on the assessment roll. .~.~. (emphasis supplied)
\end{quote}

The \STC\ says that ``Assessors are required to give non-consideration treatment to known qualifying changes to real property, regardless of whether the taxpayer has filed a form L-4293.'' \pincite{STC Bulletin}{2}.

In this case, the Tribunal found that the repairs were normal under the statute. \foj[4]. Yet the Tribunal goes on to decide the case as if there had been no normal repairs, essentially ignoring them. ``Respondent's sale adjustment grid does not include a line-item entry for `repairs. This comparative analysis is devoid of any relationship to Petition's `normal' repairs to subject property .~.~.'' \foj[6]. The Tribunal's finding of normal repairs has not changed the true cash value from what it would have been had \mathieuGast\ not been part of the case.

The Tribunal appears to believe that if it can cherry-pick some evidence that the house was not in need of repair, the assessor is free to ignore normal repairs.
In this case, that evidence was a) the assessed true cash value before and after the repairs changed by just 5\%, due to inflation, not repairs,
from \$48,000 on 12/31/2014 to \$50,400 on 12/31/2015 and
b) the MLS marketing pictures taken before the house was sold show that the house was in average condition. \reconsiderationDenied[2]. The Tribunal says that simply because it found this evidence, the assessor did not violate \mathieuGast:

\begin{quote}
	Although not necessarily evidence of true cash value, this evidence supports the property's assessment as a property in average condition both at the time Petitioner acquired it and after he completed the normal repairs. In other words, the record evidence supports the conclusion that the assessment did not consider the increase in true cash value that was the result of normal repairs. \reconsiderationDenied[2]. 
\end{quote}
 
The Tribunal sees \mathieuGast\ as being about the mind of the assessor. If the assessor subjectively bases her assessment on the effects of the repairs, then she violates the statute. But if her assessment is based on the prior year's assessment adjusted for inflation or on pictures that show the house well, then she can safely ignore normal repairs.

Under the Tribunal's interpretation, the statute's effect would depend on whether the assessor accurately assessed the property when it needed repair. If the assessor overassessed the property in the year before the repairs, as is the case here, then the home owner would not benefit from repairs taxwise because the assessment would be based not on the repairs, but on the previous year's inflated assessment. Thus, rather than encouraging homeowners to make repairs, the Tribunal's interpretation would have the statute encouraging assessors to overassess houses that need repair.

\mathieuGast\ is not concerned with the minds of assessors.
Instead, the statute is focused on making sure that the contribution of normal repairs are quantified, noted in the assessment rolls, and explicitly removed from the assessed value.

Contrary to the Tribunal's view, the plain language of the statute makes clear that normal repairs must be given non-consideration treatment. The statute uses the word ``shall'' to indicate that the assessor must determine the true cash value of the repairs, indicate it in the assessment roll, and ``not consider'' them in the true cash value.

\mathieuGast\ is violated whenever the value of normal repairs is included in the true cash value for assessment purposes. Whether the assessor subjectively thought about the repairs is irrelevant. The first time this Court heard this case, the Tribunal referee had also determined the true cash value to be the after-repaired value. This Court reversed because the after-repaired value included the value of the repairs:

\begin{quote}
[C]ontrary to MCL 211.27(2), the referee {\em considered} the increase in value attributed to the repairs when determining the property's TCV. Stated differently, the referee's finding that the property's TCV was \$50,400 was based on its assessment of the property's value after it had been repaired. This was improper because MCL 211.27(2) expressly provides that certain repairs constitute normal repairs so long as they are not part of a structural addition or completion. \pincite{Patru}{5}.
\end{quote}

There is no room in the statutory scheme for the Tribunal's idea that normal repairs can be found yet ignored: their value not determined, nor indicated on the assessment roll, nor excluded from the true cash value.

This Court reversed the previous ruling of the Tribunal in part because it had added its own condition to the statute: the Tribunal had ruled that the statute did not apply when the property was in substandard condition. ``Thus, by reading a requirement into the statute that was not stated by the legislature, the trial court erred by interpreting and applying MCL 211.27(2).'' \pincite{Patru}{5}. Now the Tribunal has ruled that the statute does not apply when there is some evidence that house did not need repairs. This condition is not in the statute. This Court should reverse this ruling for the same reason.

\subsubsection{The Tribunal's assertion that the assessor believed that the property did not need repairs is not supported by the evidence}
When reviewing Tribunal cases, this Court looks for misapplication of the law or adoption of a wrong principle. Factual findings must be supported by competent, material, and substantial evidence on the whole record. \pincite{Briggs}{75; 757-758}

The Tribunal asserts that the assessor did not consider normal repairs because she believed that the property was in average condition with a true cash value of \$48,000 before Petitioner purchased the property. \reconsiderationDenied[2]. There is no evidence that the assessor had a fact-based belief that the property was in average condition. The assessor testified that her assessments are done using mass appraisal which ``does not account for properties one-by-one'', but rather she ``{\em assumes} that properties are in `average' condition.'' \foj[4] (emphasis added). Thus the prior year's true cash value assessment of \$48,000 was based on an assumption, not evidence.

Besides the assessor's assumption, the Tribunal's only other support for the \$48,000 true cash value assessment is pictures: ``the interior photographs depict a property in average condition before Petitioner acquired it.'' \reconsiderationDenied[2]. The pictures are from the MLS listing which shows that the property was on the market for \$29,000 and then for \$32,000 and that it sold for \$32,000. \mlsHistory[]. Usually the sale price is used to determine value. Here the Tribunal ignores the sale price and uses the pictures.

The actual evidence in the case shows that the property required repairs. The City's own inspectors had inspected the property and imposed repair requirements before they would issue a Certificate of Occupancy; the repairs were valued at approximately \$10,000. \repairs. 

The first time the Tribunal looked at this case, it ruled that the condition of the house was so substandard that any repairs to it could not be normal. (This Court corrected the legal reasoning but not the fact that the house needed repairs. ``It is undisputed that, when he purchased the property, it was in substandard condition and required numerous repairs to make it livable.'' \pincite{Patru}{1}.)

Finally, there is no record in this case that the assessor claimed that she believed that the property was in average condition at the time of its sale. As noted above the assessor testified that she used mass appraisal and assumed the property was in average condition. Nor is there any record that the assessor relied on the MLS pictures to support a belief that the property did not need repairs. The Tribunal here appears to have gone hunting outside the arguments presented by the parties for evidence to support its conclusion.

Therefore, the Tribunal's argument as to why it can ignore normal repairs is both a wrong interpretation of the statute and also without factual support.

\subsection{The Tribunal erred in not determining the before-repair value}

The Tribunal has refused to determine the before-repair value because:
1) it believed that the STC's requirement of a before-repair appraisal is not required
and
2) it disagreed with Appellant's evidence on the value of the property before repair.
\reconsiderationDenied[2]. Both of these are addressed below.

\subsubsection{The Tribunal overruled the STC's guidance
  % on before-repair appraisal
  without cogent reasons}
  

The Michigan Supreme Court has given the following standard for judicial review of an administrative agency's interpretation of a statute:

\begin{quote}
The construction given to a statute by those charged with the duty of executing it is always entitled to the most respectful consideration and ought not to be overruled without cogent reasons. However, these are not binding on the courts, and while not controlling, the practical construction given to doubtful or obscure laws in their administration by public officers and departments with a duty to perform under them is taken note of by the courts as an aiding element to be given weight in construing such laws and is sometimes deferred to when not in conflict with the indicated spirit and purpose of the legislature. \pincite{Rovas}{103}\ (cleaned up).
\end{quote}

Regarding normal repairs under \mathieuGast\ the \STC\ requires that ``the true cash value of the item shall be calculated by performing `before' and `after' appraisals and then deducting the `before' true cash value from the `after' true cash value.'' \pincite{STC Bulletin}{2}. Also in its ``Instruction to Assessor for Processing Form 865'' the \STC\ writes:
 ``The assessor is required to estimate the true cash value of the property both before and after the expenditures.'' \stcform[2].


% \stcform
% % STC form 865, Request for Nonconsideration,
% allows property owners to list expenditures for repairs qualifying as normal repairs under \mathieuGast. The second page of that form has a section entitled ``Instructions To Assessor for Processing Form 865.'' The first instruction in that section says: ``The assessor is required to estimate the true cash value of the property both before and after the expenditures.''

% The STC has also required before-repair and after-repair appraisals in its Bulletin to assessors: ``the true cash value of the item shall be
% calculated by performing `before' and `after' appraisals and then deducting the `before' true cash value from the `after' true cash value.'' \pincite{STC Bulletin}{2}.

Despite the clear instruction of the STC requiring before-repair and after-repair appraisals, the Tribunal states that ``there is nothing in MCL 211.27(2) requiring `before repair' or `after repair' appraisals when determining whether an assessment includes the true cash value of normal repairs.'' \reconsiderationDenied[2]. To support its interpretation of the statute against the STC's interpretation,  the Tribunal in footnote 7 of the \reconsiderationDenied[2], has three arguments, none of which withstand scrutiny.

The Tribunal's first reason for departing from STC guidelines and refusing to determine the before-repair value is that situations in the second sentence of \mathieuGast\ is not at issue:

\begin{quote}
  {\em footnote 7} MCL 211.27(2) allows an assessor to increase ``construction quality classification or reduce the effective age for depreciation purposes'' if the ``appraisal of the property was erroneous before non-consideration of the normal repair.'' It also prevents an assessor from assigning an economic condition factor to the property that differs from the economic condition factor assigned to similar properties as defined by appraisal procedures applied in the jurisdiction. Neither situation is at issue here. . . . \reconsiderationDenied[2].
\end{quote}

The Tribunal quotes from the second sentence of \mathieuGast, then declares that the situations addressed by that sentence are not at issue. The second sentence of the statute does not prohibit the assessor or the Tribunal from determining the before-repair value. It only prohibits the assessor from increasing construction quality, reducing the effective age, and asigning a different economic condition factor in certain situations. 
The Tribunal does not explain why these prohibitions are relevant to whether the Tribunal must determine the before-repair value.
%Nor does the Tribunal explain why the situations in the second sentence is not at issue or why this is relevant.
% A before-repair appraisal is required because it along with an after-repair appraisal is needed to determine the value of the repairs. The value of the repairs is needed because the first sentence of the statute requires nonconsideration of the value of the repairs in the true cash value and the third sentence of the statute requires the value of the repairs to be indicated on the assessment roll.
% The second sentence of the statute does not explicitly excuse the assessor from performing a before-repair appraisal to obey the first sentence and third sentences of the statute.
% The only way it would be relevant would be if its prohibitions would somehow prevent the assessor from performing the appraisal. Appellee has not argued this nor has the Tribunal alleged this.
Thus the Tribunal's first reason for not determining the before-repair value is unpersuasive. %reference to the second sentence is irrelevant.

The Tribunal's second reason for departing from STC guidelines and refusing to determine the before-repair value is that the STC requires before-repair and after-repair appraisals only if the value of the repairs appear on the assessment roll:

\begin{quote}
  {\em footnote 7} . . .  Although State Tax Commission Bulletin No. 7 of 2014 requires ``before'' and ``after'' appraisals, such appraisals are only required ``[i]f the true cash value of non-consideration items is shown on the assessment roll. .~.~.'' As described herein, the true cash value of the non-consideration items is not shown on the assessment roll and the STC requirement does not apply. . . . \reconsiderationDenied[2].
\end{quote}

The Tribunal has misinterpreted the paragraphs 3 and 4 of \pincite[s]{STC Bulletin}{2}. Below is a fuller quote. The Tribunal's quote is italicized, and the words immediately after are bolded to clarify:

\begin{quotation}
3. {\em If the true cash value of non-consideration items is shown on the assessment roll} \textbf{in the first year after the qualifying change is made}, then the true cash value of the item shall be calculated by performing ``before'' and ``after'' appraisals and then deducting the ``before'' true cash value from the ``after'' true cash value.

4. {\em If the true cash value of non-consideration items is shown on the assessment roll} \textbf{in years subsequent to the first year after the qualifying change}, then the true cash value of the item shall be calculated each year by performing ``before'' and ``after'' appraisals and then deducting the ``before'' true cash value from the ``after'' true cash value to determine the true cash value contribution of the item for that assessment year. The purpose of this approach is to reflect the current contribution, rather than the initial contribution, to true cash value which is provided by the item.
\end{quotation}

Paragraphs 3 and 4 of the \cite[s]{STC Bulletin}\ distinguish between the two possible cases when repairs may be accounted for: in the first year or in years after the first year. The paragraphs detail {\em how} the repairs are to be valued not {\em if} they are to be valued. In both cases before and after appraisals are used. Therefore the \cite[s]{STC Bulletin}\ does not support the Tribunal.

The Tribunal's third and final reason for departing from STC guidelines and refusing to determine the before-repair value is that ``STC guidance lacks the force of law. [\pincite{Rovas}{103; 267}].'' \reconsiderationDenied[2].

\cite[s]{Rovas}\ concerns the respect that judicial-branch courts should give to executive agency interpretations of law. The Tribunal is itself an executive agency as is the State Tax Commission. \cite[s]{Rovas}\ does not address an executive agency vis-a-vis another executive agency.

Although \cite[s]{Rovas}\ does not give the Tribunal authority to overule the STC, it does teach that an executive agency's interpretation ``is always entitled to the most respectful consideration and ought not to be overruled without cogent reasons.'' \pincite[s]{Rovas}{103}. Here the Tribunal relies on irrelevancies and out-of-context quotes to overrule the STC.
Therefore, the Tribunal has not given cogent reasons for disregarding the guidance of the STC on before-repair and after-repair appraisals. This Court should reverse. 

%\subsubsection{Once Appellant has presented evidence, the Tribunal must determine the true cash value even if it is unconvinced by the evidence}

\subsubsection{Failure of petitioner's proofs does not excuse the Tribunal from independently determining the true cash value}

When reviewing Tribunal cases, this Court looks for misapplication of the law or adoption of a wrong principle. Statutory interpretation is reviewed de novo. \pincite{Briggs}{75; 757-758}. 

Once a petitioner has presented evidence, the Tribunal must make an independent determination of the true cash value at issue, even if the evidence is unconvincing. In \cite{Jones & Laughlin}, this Court reversed when the Tribunal violated this rule:

\begin{quotation}
The tribunal further erred in failing to make an independent determination of the true cash value of the property. The tribunal apparently believed that no such determination was necessary after it concluded that petitioner had failed to meet its burden of proof and dismissed petitioner's appeal. The tribunal correctly noted that the burden of proof was on petitioner, This burden encompasses two separate concepts: (1) the burden of persuasion, which does not shift during the course of the hearing; and (2) the burden of going forward with the evidence, which may shift to the opposing party. The tribunal's decision, however, seems analogous to the entry of a directed verdict upon the failure of a plaintiff's proofs. To the extent this analogy may be accurate in this case, the entry of judgment against petitioner for its failure to provide sufficient evidence was erroneous because, while petitioner may not have met its burden of persuasion, it did meet its burden of going forward with evidence.
	
	Even if the tribunal had correctly concluded that petitioner's proofs had failed, the tribunal still would be required to make an independent determination of the true cash value of the property. The tribunal may not automatically accept a respondent's assessment, but must make its own findings of fact and arrive at a legally supportable true cash value. .~.~. On remand, the tribunal shall make an independent determination of true cash value. \pincite[s]{Jones & Laughlin}{354-356}\ (cleaned up).
\end{quotation}

In this case the Tribunal has made the same error as the Tribunal in \cite[s]{Jones & Laughlin}: it declined to determine the true cash value of the property at issue (by determining the before-repair value) because the Appellant bore the burden of proof and the Tribunal disagreed with Appellant's proof:

\begin{quote}
To begin, there is nothing in MCL 211.27(2) requiring ``before repair'' and ``after repair'' appraisals when determining whether an assessment includes the true cash value of the normal repairs. Rather, in a proceeding before the Tribunal, Petitioner bears the burden of proof. Further, the Tribunal cannot conclude that the FOJ erred when it concluded that Petitioner's contentions concerning the purchase price, i.e. the true cash value before repairs, was entitled to no weight or credibility. The selling price of a property is not is presumptive true cash value. Despite Petitioner's assertions that the marketing efforts for the subject show that the sale was a ``market sale,'' the home was being sold by the U.S. Department of Housing and Urban Development (``HUD''). Because the subject was being sold by a government entity, that entity's motivation may not have been to receive market value for the property.
\reconsiderationDenied[2] (footnotes omitted).
\end{quote}

The exact analysis of \cite[s]{Jones & Laughlin}\ quoted above applies. ``Even if the tribunal had correctly concluded that petitioner's proofs had failed, the tribunal still would be required to make an independent determination .~.~.'' \pincite[s]{Jones & Laughlin}{354-356}. Therefore this Court should reverse.

% \subsubsection{The previous-year's assessed value is irrelevant to the Tribunal's Mathieu-Gast analysis}

% Mention assessor's instruction 3 on page 2 of form 865.

% Mention that cases talk about Mathieu Gast nonconsideration for repairs done several years in the past. The previous year's assessed value is irrelevant here.

% Mention that in this case, this was Appellant's first opportunity to appeal. It makes no sense to consider past assessments.

% Mention that modern assessments, including the one in this case, are done using computerized mass appraisals which assume that properties are in average condition. Thus, past assessments should have no evidentiary weight in Tribunal proceedings because they are not based on evidence but rather on assumptions. The Tribunal's factual determinations must be supported by competent, material, and substantial evidence. 

\subsection{The Tribunal erred when it gave the selling price of the property ``no weight or credibility'' solely because the seller was HUD and speculation that the seller may not have been motivated to receive market value}

When reviewing Tax Tribunal cases, this Court looks for misapplication of the law or adoption of a wrong principle. Factual findings must be supported by competent, material, and substantial evidence on the whole record. Statutory interpretation is reviewed de novo. \pincite{Briggs}{75; 757-758}

%The Tribunal's factual findings are accepted as final by this Court ``provided they are supported by competent, material, and substantial evidence. Substantial evidence must be more than a scintilla of evidence, although it may be substantially less than a preponderance of the evidence.'' \pincite[s]{Jones & Laughlin}{352-53}\ (cleaned up).

In \cite[s]{Jones & Laughlin}, this Court reversed the Tribunal for cursorily rejecting the sale price of the subject property:

%had ruled, ``A sale that occurs {\em after} the tax date has little or no bearing on the assessment made prior to the sale.'' \pincite{Jones & Laughlin}{354}\ (emphasis in original). This Court reversed:

\begin{quotation}
The Tax Tribunal, however, erred as a matter of law in its treatment of petitioner's evidence regarding the sale. The tribunal held "A sale that occurs {\em after} the tax date has little or no bearing on the assessment made prior to the sale." (Emphasis in original.)

We disagree. Unlike some situations involving assessments of industrial property for which no ready market exists and a hypothetical buyer must be posited, in this case the equipment was actually sold in a commercial transaction, albeit after the tax date.
We believe that evidence of the price at which an item of property actually sold is most certainly relevant evidence of its value at an earlier time within the meaning of the term ``relevant evidence.'' MRE 401. Although the sale .~.~. occurred approximately nine months after the tax date, the lapse in time is important only with respect to the weight that should be given the evidence, not to the relevance of the evidence. While the tribunal correctly noted that the sale price of a particular piece of property does not control its determination of the value of that property, the tribunal's opinion that the evidence ``has little or no bearing'' on the property's earlier value suggests that the evidence was rejected out of hand. Such cursory rejection would be erroneous. \pincite[s]{Jones & Laughlin}{353-354}\ (cleaned up).
\end{quotation}

In this case the Tribunal has ruled that the sale price ``is entitled to no weight or credibility'' because the seller was a government entity and speculation that the seller may not have been motivated to receive market value. \reconsiderationDenied[2]. The Tribunal does not cite evidence, facts, or authority to suggest that an otherwise market sale is not a market sale because the seller was HUD. Nor does does the Tribunal merely discount the weight of the evidence; it rules that the sale is entitled to {\em no} weight or credibility. Such a cursory dismissal of the sale price based on speculation in contrary to the teaching of \cite[s]{Jones & Laughlin}. This Court should reverse and require that the Tribunal base its rulings on evidence and not cursorily reject the sale price. 

\section{Relief Requested}

Applicant respectfully asks this Court to reverse the ruling of the Tribunal. 

\section{Proof of Service}

On 3/29/2019, I served a copy of this Brief on Appellee's counsel by electronic service.

\vspace{1\baselineskip}

{ \setlength{\leftskip}{3.5in}

  Respectfully Submitted,

  /s/ Daniel Patru, P74387

  3/29/2019

  \setlength{\leftskip}{0pt}}

\newpage\empty% we need a new page so that the index entries on the last
        % page get written out to the right file.
\end{document}


%%% Local Variables:
%%% mode: latex
%%% TeX-master: t
%%% End:
