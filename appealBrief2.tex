%
% This is one of the samples from the lawtex package:
% http://lawtex.sourceforge.net/
% LawTeX is licensed under the GNU General Public License 
%
\providecommand{\documentclassflag}{}
\documentclass[12pt,\documentclassflag]{michiganCourtOfAppealsBrief} 
\usepackage{etoolbox}
\usepackage[margin=1in]{geometry}
\usepackage{newcent,microtype}
\usepackage{setspace,xcolor}
\usepackage{trace}
\usepackage[T1]{fontenc}

% From tex.stackexchange.com (https://tex.stackexchange.com/a/142258/135718)
\usepackage{xcolor}
\usepackage{todonotes}
% horizontal rule with text, https://tex.stackexchange.com/a/15122/135718
\newcommand*\ruleline[1]{%
  \par\noindent%
  \raisebox{.8ex}{\makebox[\linewidth]{\hrulefill\hspace{1ex}%
      \raisebox{-.8ex}{#1}%
      \hspace{1ex}\hrulefill}}}%

% https://www.overleaf.com/learn/latex/Environments
\newenvironment{draft}{%
  \color{blue}%
  \ruleline{Begin Draft}
}{%
  \ruleline{End Draft}
  \color{black}%
}
% 
% 

% \usepackage{xr-hyper}
\usepackage{xr}
\externaldocument{appendix}
\usepackage[hyperindex=false,linkbordercolor=white]{hyperref}


\makeandletter% use \makeandtab to turn off

% Use this to show a line grid-
% \usepackage[fontsize=12pt,baseline=24pt,lines=27]{grid}
% \usepackage{atbegshi,picture,xcolor} % https://tex.stackexchange.com/a/191004/135718
% \AtBeginShipout{%
%   \AtBeginShipoutUpperLeft{%
%     {\color{red}%
%     \put(\dimexpr -1in-\oddsidemargin,%
%          -\dimexpr 1in+\topmargin+\headheight+\headsep+\topskip)%
%       {%
%        \vtop to\dimexpr\vsize+\baselineskip{%
%          \hrule%
%          \leaders\vbox to\baselineskip{\hrule width\hsize\vfill}\vfill%
%        }%
%       }%
%   }}%
% }
%   \linespread{1}

\usepackage[modulo]{lineno}% use \linenumbers to show line numbers, see https://texblog.org/2012/02/08/adding-line-numbers-to-documents/

\chardef\_=`_% https://tex.stackexchange.com/a/301984/135718 

%%Citations
 
%The command \makeandletter turns the ampersand into a printable character, rather than a special alignment tab \makeandletter

% %Set the information for the title page (later produced by \makefrontmatter)
% \docket{No. 10-553} 
% \appellant{Daniel Patru}
% \appellee{City of Wayne}
% \court{Michigan Court of Appeals}
% \circuit{Sixth}
% \brieffor{Appellant}
% \author{Daniel Patru , P74387\\{\em Petitioner}}
% \address{3309 Solway \\ Knoxville, TN 37931\\ (734) 274-9624}

\begin{document}
\singlespacing%

\citecase[Patru]{Patru v City of Wayne, unpublished per curiam opinion of the Court of Appeals, issued May 8, 2018 (Docket No. 337547)}%
\caseReference{Patru}{patruvwayne}% Associate appendix with case

\citecase[Antisdale]{Antisdale v City of Galesburg, 420 Mich 265, 362 NW2d 632 (1984)}
\citecase[Briggs]{Briggs Tax Service, LLC v Detroit Pub. Schools, 485 Mich 69; 780 NW2d 753 (2010)}
\citecase[Jones & Laughlin]{Jones & Laughlin Steel Corporation v. City of Warren, 193 Mich App 348; 483 NW2nd 416 (1992)}
\citecase[Rovas]{In re Complaint of Rovas Against SBC Michigan, 482 Mich 90; 754 NW2d 259 (2008)}

% \citecase[Fisher]{Fisher v. Sunfield Township, 163 Mich App 735; 415 NW2nd 297 (1987)}
% \citecase{Arbaugh v. Y&H Corp., 546 U.S. 500 (2006)}
% \citecase[CAF]{CAF Investment Co. v Saginaw Twp, 410 Mich 428; 302 NW2d 164 (1981)}
% \citecase[Coyne]{Coyne v Highland Twp, 169 Mich. App. 401; 425 NW2d 567 (1988)}
% \citecase[Malpass]{Malpass v Dep't of Treasury, 494 Mich. 237; 833 NW2d 272 (2013)}
% \citecase[Meadowlanes]{Meadowlanes Ltd Dividend Housing Ass'n v City of Holland, 437 Mich 473; 473 NW2d 636 (1991)}
% \citecase[Plymouth]{Plymouth Twp. v Wayne County Board of Commissioners, 137 Mich App 738; 359 NW2d 547 (1984)}
% \citecase[Professional Plaza]{Professional Plaza v City of Detroit, 647 NW2d 529; 250 Mich. App. 473  (2002)}
% \citecase[Randolf]{Dep't of Transportation v Randolph, 461 Mich. 757; 610 NW2d 893 (2000)}
% %\citecase[SBC Health Midwest]{SBC Health Midwest, Inc. v City of Kentwood,  \_\_  Mich \_\_ (decided May 1, 2017)}
% \citecase[Stevens]{Stevens v Bangor Twp, 150 Mich App 756; 389 NW2d 176 (1986)}
 
{\makeatletter % needed for optional argument to newstatute.
  % \newstatute[1@]{MCL}{}% place MCL first
  % \newstatute[2@]{MCR}{}
  % \newstatute[3@]{TTR}{}
  % \newstatute[4@]{Dearborn Ordinance}{}% place this fourth
  % \newstatute[5@]{Wayne Ordinance}{}
  \newstatute{MCL 211.27(2)}{}

  \newstatute{MCL 205.753(2)}{}% allows appeals from a final order of the Tax Tribunal

  \newstatute{MCR 7.204(A)(1)(b)}{}% allows appeals within 21 days of an order on a motion for reconsideration
  
} % end makeatletter block
% \def\mathieuGast{\pincite[l]{MCL}{211.27(2)}}
\def\mathieuGast{\cite[s]{MCL 211.27(2)}}%
%\def\ttr287{\pincite[s]{TTR}{287}}
%\def\inspectionOrdinance{\pincite{Wayne Ordinance}{\S1484.04}}
% \long\def\inspectionOrdinanceText{\begin{quote}
% 1484.04  CERTIFICATE REQUIRED PRIOR TO SALE. 
%    It shall be unlawful to sell, convey or transfer an ownership interest, or act as a broker or agent for the sale, conveyance or transfer of an ownership interest, in any residential dwelling unless and until a valid Certificate of Compliance is first issued. 
% (Ord. 1991-10.  Passed 7-16-91.) 
% \end{quote}}


%\newmisc{STC Bulletin 6 of 2007}{Michigan State Tax Commission (STC) Bulletin No. 6 of 2007 (Foreclosure Guidelines)}
\newmisc{STC Bulletin}{Michigan State Tax Commission (STC) Bulletin No. 7 of 2014 (Mathieu Gast Act)}

\newcommand{\makeAbbreviation}[3]{% ensure that the frsit time an abbreviated word is used, it is presented in long form, and after that in short form. 1: command name, 2: short name, 3: long name
  \IfBeginWith{#3}{#2}{%
    \newcommand{#1}[0]{#3\renewcommand{#1}[0]{#2}}}{%
    \newcommand{#1}[0]{#3 (#2)\renewcommand{#1}[0]{#2}}}}

\makeAbbreviation{\MLS}{MLS}{Multiple Listing Service}
\makeAbbreviation{\MTT}{MTT}{Michigan Tax Tribunal}
\makeAbbreviation{\STC}{STC}{State Tax Commission}
%\makeAbbreviation{\FOJ}{FOJ}{First Final Opinion and Judgment (2017)}
% \makeAbbreviation{\explanatoryLetterAbbr}{Explanatory Letter}{Explanatory Letter submitted by Appellant to the Tax Tribunal on 9/6/2018}
% \newcommand{\explanatoryLetter}[1][]{\explanatoryLetterAbbr\if\relax#1\relax\empty, Appendix at \pageref{explanatoryLetter}\else, page #1, Appendix at \pageref{explanatoryLetter.#1}\fi}

% Note that the command/handle must match the appendix
% labels. Minimize the variations of names.  \makeAbbreviationToRecord
% creates a simple abbreviation that ends in Abbr if you don't want to
% refer to the record.
\newcommand{\makeAbbreviationToRecord}[3]{% 1: command/handle, 2: shortname, 3: longname
  % makeAbbreviationToRecord: #1, #2, #3\par%
  \expandafter\makeAbbreviation\csname #1Abbr\endcsname{#2}{#3}%
  \expandafter\newcommand\csname #1\endcsname[1][]{%
    \Call{#1Abbr}%
    \if\relax##1\relax\empty\ (Appendix at \pageref{#1})%
    \else, p ##1 (Appendix at %
    % Check for page range
    \IfSubStr{##1}{-}{%
      \def\pageRefRange####1-####2XXX{\pageref{#1.####1}-\pageref{#1.####2}}%
      \pageRefRange##1XXX}%
    {\pageref{#1.##1}}%
    )\fi%
  }%
}%

\makeAbbreviationToRecord{explanatoryLetter}{Explanatory Letter}{Explanatory Letter (MTT Docket Line 38)}
% explanatory Letter: (abbr) \explanatoryLetterAbbr\ (to record) \explanatoryLetter[2] \par

\makeAbbreviationToRecord{foj}{FOJ}{Second Final Opinion and Judgment (MTT Docket Line 48)}
% FOJ: (abbr) \fojAbbr (to record) \foj[]\par

\makeAbbreviationToRecord{reconsiderationDenied}{Order Denying Reconsideration}{Order Denying Reconsideration (MTT Docket Line 51)}
% reconsiderationDenied: (abbr) \reconsiderationDeniedAbbr[] (to record) \reconsiderationDenied[] \par

\makeAbbreviationToRecord{repairs}{List of Repairs}{List of Repairs (MTT Docket Line 36)}
% \par repairs: (abbr) \repairsAbbr\ (to record) \repairs[] (to appendix) \pageref{repairs}\par

\makeAbbreviationToRecord{mlsListing}{MLS Listing}{MLS Listing (MTT Docket Line 32)}
% mlsListing: (abbr) \mlsListingAbbr\ (to record) \mlsListing[]\par

\makeAbbreviationToRecord{mlsHistory}{MLS History}{MLS History (MTT Docket Line 33)}
% mlsHistory: (abbr) \mlsHistoryAbbr\ (to record) \mlsHistory[]\par

\makeAbbreviationToRecord{boardOfReviewDecision}{Board of Review Decision}{Board of Review Decision (MTT Docket Line 2)}

\makeAbbreviationToRecord{cityEvidence}{City's Evidence}{City's Evidence (MTT Docket Line 11)}

\makeAbbreviationToRecord{motionForReconsideration}{Motion for Reconsideration}{Motion for Reconsideration (MTT Docket Line 52)}

\begin{centering}
\bf\scshape State of Michigan\\In the Court of Appeals\\Detroit Office\\~\\ 
\rm 

\makeandtab
\begin{tabular}{p{.45\textwidth}|p{.45\textwidth}}
\cline{1-1}
  {~

  \raggedright Daniel Patru,\par
  \hfill\textit{Appellant,}
  \vspace{.5\baselineskip}
  \centerline{v}
  \vspace{.5\baselineskip}
  \raggedright City of Wayne,\par
  \hfill\textit{Appellee.}
  
  ~} &  {
       \hfill Court of Appeals No. 346894\par
       \hfill Lower Court No. 16-001828-TT\par\vspace{\baselineskip}
       \hfill \textbf{Appellant's Brief}\par
       \hfill \textbf{Proof of Service}       
  }
  \\ \cline{1-1}\vspace{2mm}
  Daniel Patru, P74387, Appellant\newline%
  3309 Solway\newline%
  Knoxville, TN 37931\newline%
  (734) 274-9624\newline\newline%
  City of Wayne, Appellee\newline%
  3355 South Wayne Rd,\newline%
  Wayne, MI 48184\newline%
  (734) 722-2000%
  % \end{verbatim}
  & \\ 
\end{tabular}
\makeandletter

\end{centering}

\pagestyle{romanparen}
\pagenumbering{roman}
\newpage 

\section*{Table of Contents}

\tableofcontents

\newpage
\tableofauthorities

\pagestyle{plain}
\pagenumbering{arabic}

%This commands creates the title page, table of contents, and table of authorities
% \makefrontmatter{Brief\\Proof of Service}


%Sets the formatting for the entire document after the front matter
\parindent=2.5em
% \setlength{\parskip}{1.25ex plus 2ex minus .5ex} 
% \setstretch{1.45}
\doublespacing
% \linenumbers

\section{Jurisdictional Statement}
 
The Court of Appeals has jurisdiction over this claim of appeal under \cite{MCL 205.753(2)}\ (allowing appeals from a final order of the Tax Tribunal) and \cite{MCR 7.204(A)(1)(b)}\ (requiring appeals to be made within 21 days after the entry of an order deciding a motion for reconsideration). The \fojAbbr\ from the Tax Tribunal was entered 12/04/2018. A motion for reconsideration was filed 12/05/2018. The order denying reconsideration was entered 12/14/2018. The claim of appeal for this case was filed 12/18/2018, within the 21 days required.

\newpage 
\section{Introduction}

This case concerns the application of the Mathieu-Gast Home Improvement Act, \cite[s]{MCL 211.27(2)}, which instructs assessors to ``not consider'' the value of normal repairs when determining the true cash value of residential property for assessment purposes.

In 2015, Appellant bought a house for \$32,000 and made about \$10,000 in repairs to it. The house was repaired by 12/31/2015 when the City of Wayne assessed the house at \$50,400 true cash value. The City's assessment did not apply Mathieu-Gast nonconsideration: it did not exclude the value of the repairs from the true cash value. The Appellant has petitioned the Tax Tribunal to apply Mathieu-Gast nonconsideration, but so far the Tax Tribunal has refused.

The Tax Tribunal first ruled that the repairs were not normal repairs because the house was in substandard condition when Appellant bought it. This Court reversed, ruling that the condition of the house was statutorily irrelevant to whether repairs were normal repairs. This Court remanded the case for a rehearing because the hearing referee's misunderstanding of the statute made it impossible to determine whether Appellant had proved at the hearing that the repairs were normal.

After the rehearing, the Tax Tribunal accepted the repairs as normal but still refused to apply nonconsideration because it believed that:

\begin{enumerate}
\item The Tribunal was not required to determine the before-repair value if it disagreed with Appellant's evidence on that point; and
\item The assessor had increased the true cash value because of inflation, not repairs. There was some evidence to believe that the property had been in average condition all along: a) the city's assessment before Appellant purchased the house and after it was repaired differed by just 5\% and b) the MLS marketing photos showed that the house was in average condition. 
\end{enumerate}


% Appellant bought a house that needed repairs. He repaired it. He wants to be taxed on the before-repair value of the house because \mathieuGast instructs the assessor to ``not consider'' (or exclude) the value of normal repairs for assessment purposes.

% Appellee, the City of Wayne, wants to tax based on the value of the house after repairs. In other words, the City wants to include whatever value was added by the repairs in the true cash value for assessment purposes.

% After the first hearing, the Tax Tribunal agreed with Appellee. It refused to offer Appellant the benefits of \mathieuGast in part because it thought that the repairs were not ``normal repairs'' under statute (i.e. the repairs did not qualify for nonconsideration) because the house was in substandard condition. It set the true cash value to the value of the house after repairs.

% This Court reversed, ruling that the poor condition of the house is statutorily irrelevent to whether repairs were normal repairs. It remanded the case back to the Tax Tribunal for a rehearing to determine whether the repairs had been normal repairs under the statute. 

% After the rehearing, the Tax Tribunal found that the repairs had been normal repairs, but that the true cash value should still be set to the value of the house after repairs for two reasons:


\newpage 
\section{Questions Involved}

\noindent I. Did the Tax Tribunal fail in its duty to independently determine the true cash value when it refused to determine the before-repair value because a) Appellant had the burden of proof and 2) the Tribunal disagreed with Appellant's proof?

Appellant answers yes. The Tax Tribunal answers no. The Appellee's answer is unknown. 
\vspace{\baselineskip}

\noindent II. Did the Tax Tribunal violate \mathieuGast\ when it ignored normal repairs because there was some evidence that the assessor did not need to consider repairs to justify the assessed value?

Appellant answers yes. The Tax Tribunal answers no. The Appellee's answer is unknown.

\vspace{\baselineskip}

\noindent III. Did the Tax Tribunal err when it gave the selling price of the property ``no weight or credibility'' as to the market value of the property based solely on a) the fact that the seller was a government entity and b) speculation that the seller may not have been motivated to receive market value?

Appellant answers yes. The Tax Tribunal answers. The Appellee's answer is unknown. 
 
\newpage

\section{Facts}
\label{facts}
\subsection{Appellant buys a house, the City assesses it, and Appellant appeals}

Appellant purchased the subject house in August 2015 for \$32,000. The house was sold by the Department of Housing and Urban Development (HUD) through a real estate broker who had listed the house on the MLS. \mlsListing[]. At the time of purchase, the house needed numerous repairs; most of which were required by the City of Wayne as a condition of obtaining a Certificate of Occupancy. \repairs[]. By tax day, 12/31/2015, Appellant had repaired the house and rented it. \foj[4-5].

The City of Wayne determined that the true cash value of the house as of tax day was \$50,400, rather than the \$32,000 purchase price. \boardOfReviewDecision.

Appellant appealed to the Board of Review and then to the Tax Tribunal. Appellant does not dispute that the house was worth \$50,400 on tax day in its repaired condition. But he contends that under \mathieuGast\ the repairs were normal repairs and that the true cash value for assessment purposes cannot include the value of the repairs. He contends that the correct true cash value is therefore the before-repair value. \explanatoryLetter[].

Appellant contends that the house's sale for \$32,000 is evidence of its before-repair value. Licensed real estate brokers listed the house on the \MLS, initially for \$29,900 on 4/3/2013 and later for \$32,000 on 6/17/2015. Before Appellant bought the property there had been at least two accepted offers on the property that failed to close. \mlsHistory[]. 

The City of Wayne contends that the true cash value should be the repaired value of the house as was on tax day with no adjustment for Appellant's repairs. \cityEvidence.

\subsection{The Tax Tribunal sides with the City but the Court of Appeals reverses}

The Referee who first heard the case at the Tax Tribunal ruled that \mathieuGast\ did not apply because the repairs were not normal repairs because the house was in substandard condition. ``Thus, the referee determined that if a property is purchased in substandard condition, any repairs
done on the property to bring it into good repair do not constitute normal repairs, maintenance, or
replacement within the meaning of MCL 211.27(2), so the increase in TCV resulting from those
repairs can be immediately considered in determining the TCV for assessment purposes.'' \pincite{Patru}{2}.

The Court of Appeals reversed. ``Nothing in MCL 211.27(2) provides that the repairs .~.~. are not normal repairs in the event that they are performed on a substandard property. Thus, by reading a requirement into the statute that was not stated by the legislature, the trial court erred .~.~.'' \pincite{Patru}{5}. The Court of Appeals remanded for a rehearing to determine whether the repairs were normal repairs because ``[t]he referee did not fully evaluate that evidence---which included testimony---because it misapprehended how to properly apply MCL 211.27(2).'' \pincite{Patru}{5}.

\subsection{On rehearing, the Tax Tribunal sides with the City again}

The rehearing was held in December 2018. Tax Tribunal Judge Marcus L. Abood ruled that the repairs were normal repairs and that the repairs were worth approximately \$10,000. \foj[4]. However, he went on to rule on pages 5 and 6 of the FOJ that ``Petitioner's contentions related to his purchase price and ``normal'' repairs are given no weight or credibility in the determination of market value'' because:
\begin{enumerate}
\item The property's assessment for 2018 changed based on the sale transaction in August 2015 and not based on Petitioner's repairs. 
\item The assessment considered the property in average condition as it was on December 31, 2015, tax day.
\item The MLS photographs submitted by Petitioner show the property in average condition, not neglected or vandalized.
\item The property's sale was not ``an arm's length sale transaction'' because the Petitioner has not claimed it so and because the seller was HUD.
\end{enumerate}

Judge Abood did not attempt to value the property's before-repair value. Instead, he ruled that the property must be valued in its repaired condition because it was repaired on tax day. He accepted the City's comparative sales analysis, noting that

\begin{quote}
  ``Respondent's sales adjustment grid does not included a line-item entry for repairs. This comparative analysis is devoid of any relationship to Petitioner's ``normal'' repairs to subject property which occurred before the issuance of a certificate of occupancy and the December 31, 2015 tax day.'' \foj[6].
\end{quote}

Judge Abood also noted in footnote 12 that ``Petitioner's insistence that Respondent should have performed before and after valuations around Petitioner's repairs is disingenuous given Petitioner's faint reference to its own raw, unadjusted, unapplied sales data.'' \foj[5].

\subsection{Appellant files motion for reconsideration}

Appellant responded to the FOJ with a motion for reconsideration. The motion raised the following points (relevant to this appeal):
\begin{enumerate}
\item The FOJ erred when it found normal repairs but yet did not apply \mathieuGast. The Court of Appeals, in this very case, said that a true cash value based on the repaired value is contrary to \mathieuGast\ because it ``considered the increase in value attributed to the repairs.'' \motionForReconsideration[1]. When the Court of Appeals was considering the case, there was a question of whether the repairs were normal. But after finding that the repairs were normal, the Tax Tribunal has no more excuse not to apply \mathieuGast. Id.
\item The FOJ erred when it did not determine the value of the house before repairs as required by \mathieuGast. 
\item The FOJ's four reasons for its conclusions of law were wrong and irrelevant, uncontested but besides the point, irrelevant, and not based in evidence.
\item Petitioner/Appellant was not disingenuous for insisting on before-repair and after-repair appraisals.
\end{enumerate}

\subsection{The Tax Tribunal denies the motion for reconsideration}

Tax Tribunal Judge David B. Marmon denied the motion. He did not specifically rebut Appellant's points made in the motion. Instead he clarified why the Tribunal was not applying \mathieuGast. Two paragraphs and a footnote contain the essence of Tax Tribunal's reasoning. They are reproduced below:

\begin{quotation}
To begin, there is nothing in MCL 211.27(2) requiring ``before repair'' and ``after repair'' appraisals when determining whether an assessment includes the true cash value of the normal repairs. [See footnote 7 below] Rather, in a proceeding before the Tribunal, Petitioner bears the burden of proof. [footnote 8 omitted] Further, the Tribunal cannot conclude that the FOJ erred when it concluded that Petitioner's contentions concerning the purchase price, i.e. the true cash value before repairs, was entitled to no weight or credibility. The selling price of a property is not is presumptive true cash value. [footnote 9 omitted] Despite Petitioner's assertions that the marketing efforts for the subject show that the sale was a ``market sale,'' the home was being sold by the U.S. Department of Housing and Urban Development (``HUD''). Because the subject was being sold by a government entity, that entity's motivation may not have been to receive market value for the property.

The Tribunal also concludes that it was not a palpable error to conclude that the assessment did not consider the ``normal repairs.'' This is supported by the fact that the property record card indicates that Respondent believed the true cash value of the subject to be \$48,000 \textit{before Petitioner purchased the property} [footnote 10 omitted] and \$50,400 as of December 31, 2015, after the repairs were complete. An increase of \$2,400 in true cash value (5\%) is easily attributable to inflation and increases in the market. In addition, as stated by the FOJ, the interior photographs depict a property in average condition before Petitioner acquired it. Although not necessarily evidence of true cash value, this evidence supports the property's assessment as a property in average condition both at the time Petitioner acquired it and after he completed the normal repairs. In other words, the record evidence supports the conclusion that the assessment did not consider the increase in true cash value that was the result of normal repairs.

\textit{footnote 7} MCL 211.27(2) allows an assessor to increase ``construction quality classification or reduce the effective age for depreciation purposes'' if the ``appraisal of the property was erroneous before non-consideration of the normal repair.'' It also prevents an assessor from assigning an economic condition factor to the property that differs from the economic condition factor assigned to similar properties as defined by appraisal procedures applied in the jurisdiction. Neither situation is at issue here. Although State Tax Commission Bulletin No. 7 of 2014 requires ``before'' and ``after'' appraisals, such appraisals are only required ``[i]f the true cash value of non-consideration items is shown on the assessment roll. .~.~.'' As described herein, the true cash value of the non-consideration items is not shown on the assessment roll and the STC requirement does not apply. In addition, STC guidance lacks the force of law. \pincite{Rovas}{103; 267}.

\reconsiderationDenied[2].
\end{quotation}

Appellant summarizes the Tribunal's decision as:

\begin{enumerate}
  
\item The Tax Tribunal believes that it did not have to make a before-repair appraisal because Petitioner has the burden of proof and the Tribunal disagrees with Petitioner's proof on this point. (The subject's sale is given no weight or credibility.)
  
\item The Tax Tribunal believes that \mathieuGast\ is not violated if there is some evidence that the assessor did not need to consider repairs to justify the true cash value. Here the true cash value was just 5\% more than the previous year's true cash value. The 5\% change was due to inflation, not repairs. The assessor's property record card indicated that the property had been in average condition before its sale. Also, the pictures on the \MLS\ showed the property in average condition.
  
\item The Tax Tribunal believes that \mathieuGast\ does not require before-repair and after-repair appraisals because:
  \begin{enumerate}
  \item The text of the statute does not plainly require the appraisals.
  \item The STC guidance requires repairs if the value of the repairs are on the assessment roll, and here they are not.
  \item Even if the STC required the repairs, the Tax Tribunal is not bound by STC guidance.
  \end{enumerate}

\item The Tax Tribunal gave the property's sale ``no weight or credibility'' because the seller was a government entity (HUD) who may not have been motivated to receive market value.
  
\end{enumerate}

Points 1, 2, and 4 are addressed in order in the Arguments. Point 3 is addressed in the discussion of point 2.  

% The Appellant believes that these arguments were not put forth by the Appellee. The \foj[4], records Appellee's argument this way:

% \begin{quotation}
% 	Respondent argues that the change in the property assessment was not based on Petitioner's repairs but based on the purchase of the property in 2015 which uncapped the assessment for 2016. Respondent assessed the property as ``average'' condition as of December 31, 2015. The changes were not substantial and were based on the rate of inflation.
	
% 	On rebuttal, Respondent argues mass appraisal does not account for properties on-by-on. In other words, properties are assessed uniformly and Respondent assumes properties are in average condition.
% \end{quotation}

% Appellant believes that the Tax Tribunal's characterization of Appellee's argument in the first paragraph quoted above is a little nonsensical. It appears to confuse the assessed value (which is determined every year independent of sales) with the taxable value (which uncaps based on sales or transfers). What was probably meant in the first paragraph was something like: 

% \begin{quote}
% 	Respondent argues that the increase in property taxes was caused by an uncapping because of the transfer of the property in 2015. The actual assessed value (or true cash value which is twice the assessed value) changed minimally from 2015 to 2016 based on the rate of inflation. The assessed value did not changed because of repairs done to the property. 
% \end{quote}

% In any case, Appellant believes that however Appellee's arguments are characterized, because the repairs had not yet been ruled to be normal repairs, Appellee's arguments were not as wrong as the arguments put forth by the Tax Tribunal in the \fojAbbr\ and \reconsiderationDeniedAbbr. The Tax Tribunal appears to argue that the assessor may simply ignore normal repairs under certain circumstances. Appellee has not asserted this so far in this case.

\section{Argument}
 
Following are three allegations of error, any of which is enough to cause a reversal.

\subsection{The Tax Tribunal failed in its duty to independently determine the true cash value of the property when it refused to make a determination of the before-repair value because Appellant had the burden of proof and the Tribunal disagreed with Appellant's proof 
}

When reviewing Tax Tribunal cases, this Court looks for misapplication of the law or adoption of a wrong principle. Statutory interpretation is reviewed de novo. \pincite{Briggs}{75; 757-758}. 

Once a taxpayer has presented evidence, the Tax Tribunal must make an independent determination of the true cash value at issue, even if it finds the taxpayer's evidence unconvincing. In \cite{Jones & Laughlin}\ this Court reversed when the Tax Tribunal did not follow this rule:

\begin{quotation}
	The tribunal further erred in failing to make an independent determination of the true cash value of the property. The tribunal apparently believed that no such determination was necessary after it concluded that petitioner had failed to meet its burden of proof and dismissed petitioner's appeal. The tribunal correctly noted that the burden of proof was on petitioner, This burden encompasses two separate concepts: (1) the burden of persuasion, which does not shift during the course of the hearing; and (2) the burden of going forward with the evidence, which may shift to the opposing party. The tribunal's decision, however, seems analogous to the entry of a directed verdict upon the failure of a plaintiff's proofs. To the extent this analogy may be accurate in this case, the entry of judgment against petitioner for its failure to provide sufficient evidence was erroneous because, while petitioner may not have met its burden of persuasion, it did meet its burden of going forward with evidence.
	
	Even if the tribunal had correctly concluded that petitioner's proofs had failed, the tribunal still would be required to make an independent determination of the true cash value of the property. The tribunal may not automatically accept a respondent's assessment, but must make its own findings of fact and arrive at a legally supportable true cash value. .~.~. On remand, the tribunal shall make an independent determination of true cash value. \pincite[s]{Jones & Laughlin}{354-356}\ (cleaned up).
\end{quotation}

In this case the Tax Tribunal has made the same error as the Tax Tribunal in \cite[s]{Jones & Laughlin}: it declined to determine the true cash value of the property at issue (by determining the before-repair value) because the Appellant bore the burden of proof and the Tribunal disagreed with Appellant's proof. \reconsiderationDenied[2]. The exact analysis of \cite[s]{Jones & Laughlin}\ quoted above applies. ``Even if the tribunal had correctly concluded that petitioner's proofs had failed, the tribunal still would be required to make an independent determination .~.~.'' For this reason, this Court should reverse.

\subsection{The Tax Tribunal misinterpreted \protect\mathieuGast\ when it ruled that normal repairs may be ignored if there is some evidence that the assessor did not need to consider repairs to justify the assessed value}

When reviewing Tax Tribunal cases, this Court looks for misapplication of the law or adoption of a wrong principle. Statutory interpretation is reviewed de novo. \pincite{Briggs}{75; 757-758}. 

\mathieuGast\ reads in relevant part:

\begin{quote}
The assessor shall not consider the increase in true cash value that is a result of expenditures for normal repairs .~.~. in determining the true cash value of property for assessment purposes until the property is sold.~.~.~. The increase in value attributable to [normal repairs] that is known to the assessor and excluded from true cash value shall be indicated on the assessment roll. .~.~.
\end{quote}

The \STC\ also says that ``Assessors are required to give non-consideration treatment to known qualifying changes to real property, regardless of whether the taxpayer has filed a form L-4293.'' \pincite{STC Bulletin}{2}.

In this case, the Tax Tribunal found that the repairs were normal under the statute. \foj[4]. Despite this, the Tax Tribunal goes on to decide the case as if there had been no normal repairs, essentially ignoring them. The Tax Tribunal appears to believe that the assessor is free to ignore normal repairs without violating \mathieuGast\ as long as she has some evidence to believe that repairs are not necessary to justify the true cash value of the property. In this case, that evidence was a) the assessed value change of 5\% from 12/31/2014 to 12/31/2015 which was due to inflation and b) the MLS marketing pictures taken before the house was sold which show that the house was in average condition. The Tax Tribunal says that simply because this evidence existed, the assessor did not violate \mathieuGast: 

\begin{quote}
	Although not necessarily evidence of true cash value, this evidence supports the property's assessment as a property in average condition both at the time Petitioner acquired it and after he completed the normal repairs. In other words, the record evidence supports the conclusion that the assessment did not consider the increase in true cash value that was the result of normal repairs. \reconsiderationDenied[2]. 
\end{quote}
 
In pointing to some evidence that the property was in average condition, the Tax Tribunal carefully avoids actually determining whether the property really was in average condition. Under the Tax Tribunal's interpretation of \mathieuGast\ the objective facts do not matter. The assessor and the Tax Tribunal may ignore normal repairs as long as they have \textit{some evidence} supporting the true cash value without having to consider (or appeal to) repairs.

Contrary to the Tax Tribunal's view, the plain language of the statute makes clear that normal repairs must be given non-consideration treatment. The statute uses the word ``shall'' to indicate that the assessor must determine the true cash value of the repairs and indicate it in the assessment roll.

Non-consideration treatment requires before-repair and after-repair appraisals. \pincite{STC Bulletin}{2}. Had the assessor attempted a before-repair appraisal, she would have found ample evidence that the property that she had valued at \$48,000, using a computerized, mass appraisal which \textit{assumed} that the property was in average condition, \foj[4], was in fact not correct. The MLS pictures which appear to show the property in average condition come from an MLS listing which shows that the property was on the market for \$29,000 and then for \$32,000 and that it sold for \$32,000. \mlsHistory[]. The City's inspectors had inspected the property and imposed repair requirements before they would issue a Certificate of Occupancy; the repairs were valued at approximately \$10,000. \repairs. The first time the Tax Tribunal looked at this case, it ruled that the condition of the house was so substandard that any repairs to it could not be normal. (This Court corrected the legal reasoning but did not question the fact that the house needed repairs. ``It is undisputed that, when he purchased the property, it was in substandard condition and required numerous repairs to make it livable.'' \pincite{Patru}{1}.)
 
Perhaps the root of the Tax Tribunal's error regarding \mathieuGast\ stems from its misunderstanding of what the statute means by ``not consider''. In everyday usage, ``to consider'' implies activity and to ``to not consider'' implies passivity. But in \mathieuGast, non-consideration is active. The statute says that ``the increase in value attributable to [normal repairs] ... shall be indicated on the assessment roll.'' The \cite[s]{STC Bulletin}\ uses the term ``non-consideration treatment''. Passively ignoring normal repairs is a violation of the statute as this Court noted the first time it heard this case:

\begin{quote}
... contrary to MCL 211.27(2), the referee \textit{considered} the increase in value attributed to the repairs when determining the property's TCV. Stated differently, the referee's finding that the property's TCV was \$50,400 was based on its assessment of the property's value after it had been repaired. This was improper because MCL 211.27(2) expressly provides that certain repairs constitute normal repairs so long as they are not part of a structural addition or completion. \pincite{Patru}{5}.
\end{quote}

In the \reconsiderationDenied[3], footnote 7, the Tax Tribunal attempts to justify its refusal to perform before-repair and after-repair appraisals:

\begin{quote}
	Although State Tax Commission Bulletin No. 7 of 2014 requires ``before'' and ``after'' appraisals, such appraisals are only required ``[i]f the true cash value of non-consideration items is shown on the assessment roll. .~.~.'' As described herein, the true cash value of the non-consideration items is not shown on the assessment roll and the STC requirement does not apply.
\end{quote}

The Tax Tribunal has misinterpreted the \cite[s]{STC Bulletin}. Below is a fuller quote of paragraphs 3 and 4 on page 2, referenced by the Tribunal. The Tribunal's quote is underlined, and I have bolded the words immediately after to clarify:

\begin{quotation}
3. \underline{If the true cash value of non-consideration items is shown on the assessment roll} \textbf{in the first year after the qualifying change is made}, then the true cash value of the item shall be calculated by performing ``before'' and ``after'' appraisals and then deducting the ``before'' true cash value from the ``after'' true cash value.

4. \underline{If the true cash value of non-consideration items is shown on the assessment roll} \textbf{in years subsequent to the first year after the qualifying change}, then the true cash value of the item shall be calculated each year by performing ``before'' and ``after'' appraisals and then deducting the ``before'' true cash value from the ``after'' true cash value to determine the true cash value contribution of the item for that assessment year. The purpose of this approach is to reflect the current contribution, rather than the initial contribution, to true cash value which is provided by the item.
\end{quotation}

Paragraphs 3 and 4 of the Bulletin distinguish between two cases: if the repairs are accounted for in the first year or in years after the first year. The paragraphs detail \textit{how} the repairs are to be valued not \textit{if} they are to be valued. The statute itself and paragraph 2 of the bulletin quoted above makes it clear that once a normal repair is known, its value must be noted on the assessment roll. The fact that the repairs were not on the assessment roll does not excuse before-repair and after-repair appraisals, it is evidence that the statute was violated.

\begin{draft}
 
  \listoftodos
  After misinterpreting the State Tax Commission's guidance, the Tax Tribunal brushes it aside by saying that STC guidance lacks the force of law, citing \cite[s]{Rovas}. \foj[2]. The problem with this is that Appellant was complaining that the assessor has not followed the STC guidance. In this situation, when the MTT says that the STC does not have the force of law, it is excusing the \textit{assessor's} lack of compliance with STC guidance. But the legislature has commanded assessors to follow the STC.

  In \cite[s]{Rovas}, this Court was dealing with a situation where the taxpayer was challenging the STC. Here the taxpayer is relying on STC guidance and the assessor, with the Tax Tribunal's endorsement, is ignoring STC guidance. The assessor in this case and in future cases need not concern herself with obeying the law and following STC guidance regarding \mathieuGast\ knowing that if the taxpayer appeals, the Tax Tribunal will back her up. In this situation, the MTT is not following the teaching of \cite[s]{Rovas}\ but fashioning its own teaching that undermines the legislature's command that assessors should follow STC guidance. This is error.

  It may be that in cases where the STC has been challenged by the taxpayer and been ruled wrong, the assessor may follow the Tax Tribunal's ruling, but \cite[s]{Rovas}\ surely does not stand for the proposition that the assessor may challenge STC guidance herself.

  Also, \cite[s]{Rovas}\ clearly teaches that STC guidance is entitled to respect. But here, the Tax Tribunal does not say that the assessor even mentioned STC guidance, much less sought to argue that the guidance was wrong. Nor does the Tax Tribunal's own argument show that it afforded the STC respect. partial footnote on the subject that cherry-picks a quote out of context does not meet \cite[s]{Rovas}'s standard.

  The implication \todo{show don't tell} is that even if the STC supports Appellant's position, the Tax Tribunal is free to ignore it. This is troubling because here Appellant has submitted to the Assessor a list of normal repairs and the City has ignored them -- \mathieuGast\ has not affected in the least how the City has assessed Appellant's property. The Appellant appeals to the Michigan Tax Tribunal and the Tribunal rules that the City has done no wrong to ignore the repairs, and that the City does not have to obey the guidance of the STC in this matter. Under this reasoning, the STC is never binding on assessors if the Tax Tribunal agrees with the assessors. Even without reading \textit{Rovas}, it is clear that the case could not stand for this proposition. The STC guidelines are written for the assessors, but also for the taxpayers. And when a taxpayer seeks the protection of the tax law, it is unseemly to allow the taxing authority to speak with a divided voice.

What do I want to say here? What is the problem?

The problem is that the MTT's position here would deprive taxpayers of certainty of how the state's tax law will be interpreted. MTT is saying that it will not implement the STC's guidelines when the assessor refuses to do so. Thus the taxpayers of Michigan cannot trust the STC's interpretation of tax law, because their local assessor may not follow it, and the MTT may not enforce the STC's interpretation.

The Tax Tribunal misunderstood \mathieuGast\ and ruled that normal repairs may be ignored as long as the assessor could justify the true cash value on other grounds. The statute

\end{draft}

\subsection{The Tax Tribunal erred when it gave the selling price of the property ``no weight or credibility'' as to the market value of the property based solely on the fact that the seller was a government entity and speculation that the seller may not have been motivated to receive market value}

The Tax Tribunal's factual findings are accepted as final by this Court ``provided they are supported by competent, material, and substantial evidence. Substantial evidence must be more than a scintilla of evidence, although it may be substantially less than a preponderance of the evidence.'' \pincite{Jones & Laughlin}{352-53}\ (cleaned up).

In \cite[s]{Jones & Laughlin}, the Tax Tribunal had ruled, ``A sale that occurs \textit{after} the tax date has little or no bearing on the assessment made prior to the sale.'' \pincite{Jones & Laughlin}{354}\ (emphasis in original). This Court reversed holding: 

\begin{quote}
Evidence of the price at which an item of property actually sold is most certainly relevant evidence of its value at an earlier time within the meaning of the term 'relevant evidence.' MRE 401. Although the sale .~.~. occurred approximately nine months after the tax date, the lapse in time is important only with respect to the weight that should be given the evidence, not to the relevance of the evidence. While the tribunal correctly noted that the sale price of a particular piece of property does not control its determination of the value of that property, the tribunal's opinion that the evidence ``has little or no bearing'' on the property's earlier value suggests that the evidence was rejected out of hand. Such cursory rejection would be erroneous. \pincite{Jones & Laughlin}{354}\ (cleaned up).
\end{quote} 

As in \cite[s]{Jones & Laughlin}, the Tax Tribunal in this case cursorily rejected the sale of the property albeit for a different reason. In the \fojAbbr\ this took just four sentences: 

\begin{quote}
	Lastly, Petitioner did not contend that his purchase of the subject property was an arm's length sale transaction under the definition of \textit{market value}. [footnote cite to Dictionary of Real Estate Appraisal.] The grantor in this bank sale transaction was the the U.S. Department of Housing and Urban Development (HUD). Petitioner's purchase price is not the presumptive determination of market value. Therefore, Petitioner's contentions related to his purchase price and ``normal'' repairs are given no weight or credibility in the determination of market value for the subject property. \foj[6] (emphasis in original).
\end{quote}

The Order Denying Reconsideration is essentially the same except that it adds a speculation: ``Because the subject was being sold by a government entity, that entity's motivation may not have been to receive market value for the property.'' \reconsiderationDenied[2].

Besides being cursory, the Tax Tribunal's rejection of the sale is not based on any evidence or even logic, but rather solely on the fact that the seller was HUD along with the speculation that the seller's ``motivation may not been to receive market value for the property.'' There is no proof offered for why this speculation may be true in the general case or in this instance. This does not meet the evidence standard of competent, material, and substantial.

Appellant complained of this issue the first time this case came before this Court. This Court chose to reverse on other grounds and did not address this issue. The Tax Tribunal has again rejected the sale of the property cursorily based on pure speculation. Appellant fears that the Tax Tribunal will continue to make cursory rejections until it is stopped by this Court. Therefore Appellant respectfully asks this Court to reverse on this ground as well as the others. 

\section{Relief Requested}

Applicant has presented here three independent allegations error. He respectfully asks this Court to reverse the ruling of the Tax Tribunal. 

\section{Proof of Service}

On 2/15/2019, I served a copy of the Brief on the Appellee, the City of Wayne, by first class mail to: 3355 S. Wayne Rd, Wayne, MI 48184. 

\vspace{1\baselineskip}

{ \setlength{\leftskip}{3.5in}

  Respectfully Submitted,

  /s/ Daniel Patru

  2/15/2019

  \setlength{\leftskip}{0pt}}

\newpage\empty% we need a new page so that the index entries on the last
        % page get written out to the right file.
\end{document}


%%% Local Variables:
%%% mode: latex
%%% TeX-master: t
%%% End:
